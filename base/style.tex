% ----------------------------
% ESTILO GENERAL DE LA TESIS
% ----------------------------

% --- Márgenes ---
\usepackage{geometry}
\geometry{
    letterpaper,
    top=2cm,
    bottom=2cm,
    left=3cm,
    right=2cm
}

% --- Títulos de capítulos y secciones ---
\usepackage{titlesec}

\titleformat{\chapter}[block]
  {\normalfont\bfseries\fontsize{12}{18}\selectfont\centering} % Arial-like, mayúsculas, centrado
  {CAPÍTULO \thechapter\ --}{1em}{\MakeUppercase}

\titlespacing*{\chapter}{0pt}{0pt}{20pt}

\titleformat{\section}[block]
  {\normalfont\bfseries\fontsize{11}{17}\selectfont\raggedright} % Arial-like, 11pt, negrita
  {\thesection.}{1em}{}

\titlespacing*{\section}{0pt}{12pt}{6pt}

% --- Configuraciones para subtítulos ---
\titleformat{\subsection}[block]
  {\normalfont\bfseries\fontsize{11}{17}\selectfont\raggedright}
  {\thesubsection.}{1em}{}

\titlespacing*{\subsection}{0pt}{12pt}{6pt}

\titleformat{\subsubsection}[block]
  {\normalfont\bfseries\fontsize{11}{17}\selectfont\raggedright}
  {\thesubsubsection.}{1em}{}

\titlespacing*{\subsubsection}{0pt}{12pt}{6pt}

% --- Interlineado y párrafos ---
\setstretch{1.0}             % interlineado sencillo
\setlength{\parskip}{6pt}    % espacio entre párrafos
\setlength{\parindent}{0pt}  % sin sangría
\sloppy                      % relajar justificación para evitar desbordes

% --- Profundidad del índice ---
\setcounter{tocdepth}{2}

% --- Estilo de listings (código fuente) ---
\usepackage{listings}
\lstset{
  basicstyle=\ttfamily,
  breaklines=true,
}
\renewcommand{\lstlistingname}{Code extraction}
\renewcommand*{\lstlistlistingname}{Code Excerpts Index}

% Definiciones para código y rutas
\newcommand{\code}[1]{\hyphenchar\font=45 \texttt{#1}}
\newcommand{\codepath}[1]{\path{#1}}

% --- Colores especiales (opcional) ---
\definecolor{US_red}{cmyk}{0, 1, 0.65, 0.34}
\definecolor{US_yellow}{cmyk}{0, 0.3, 0.94, 0}

\usepackage[framemethod=tikz]{mdframed}
\mdfdefinestyle{US_style}{
  backgroundcolor=US_yellow!20,
  font=\bfseries,
  hidealllines=true
}

% --- Número de página centrado en pie de página ---
\usepackage{fancyhdr}
\pagestyle{fancy}
\fancyhf{}
\fancyfoot[C]{\thepage}
\renewcommand{\headrulewidth}{0pt}

% --- Numeración global de figuras y tablas ---
\usepackage{chngcntr}
\counterwithout{figure}{chapter}
\counterwithout{table}{chapter}

% --- Etiquetas: usar "Cuadro" en lugar de "Tabla" ---
\renewcommand{\tablename}{Cuadro}
\renewcommand{\listtablename}{Índice de cuadros}

% --- Configuración de índices (tocloft) ---
\renewcommand{\cftfigpresnum}{Figura }
\renewcommand{\cftfigaftersnum}{:}
\setlength{\cftfignumwidth}{2.5cm} % Ajustar ancho para "Figura X.Y"

\renewcommand{\cfttabpresnum}{Cuadro }
\renewcommand{\cfttabaftersnum}{:}
\setlength{\cfttabnumwidth}{2.5cm} % Ajustar ancho para "Cuadro X.Y"

% --- Entorno de fichas para requisitos (RF) ---
% Requiere: array, enumitem (cargados en packages.tex)
\newcolumntype{P}[1]{>{\raggedright\arraybackslash}p{#1}}
\newenvironment{requisitoficha}{%
  \renewcommand{\arraystretch}{1.2}%
  \begin{tabular}{@{} P{3.0cm} | P{0.68\linewidth} @{} }
}{\end{tabular}}

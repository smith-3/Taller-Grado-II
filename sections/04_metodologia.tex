% ===================== CAPÍTULO IV (drop-in replacement) =====================
\chapter{METODOLOGÍA Y DESARROLLO DEL PROYECTO}
\label{chap:metodologia}

% --- Nota de cumplimiento (no teoría, solo aplicación) ---
% (Hint EN: "Applied chapter, no theory dump.")

\noindent Para el desarrollo de este proyecto se utilizó TecnoTime, una aplicación Android diseñada para optimizar la gestión de horarios académicos en la FCyT-UMSS. El desarrollo se estructuró en doce fases iterativas e incrementales, detalladas en el Cuadro~\ref{tab:fases_metodologia}, cada una con objetivos, entregables y criterios de aceptación específicos. Este capítulo documenta la aplicación práctica de la metodología, las decisiones técnicas adoptadas, los artefactos generados y las evidencias verificables de cada fase.

% ---- Opcional: compact helpers (si tus normas lo permiten) ----
\setlength{\emergencystretch}{2em}
% \renewcommand{\arraystretch}{1.08}
% \setlist[itemize]{itemsep=0.25em, topsep=0.25em}
% \setlist[enumerate]{itemsep=0.35em, topsep=0.25em}
% \captionsetup{font=small,skip=6pt}

\providecommand{\Class}[1]{\textsf{#1}}
\providecommand{\ClassRef}[2]{\textsf{#2}}
\newcommand{\cfile}[1]{\texttt{#1}}
\providecommand{\cclass}[2]{\textsf{#2}}

% ----------------------------------------------------------------------
\section{Metodología adoptada: Kanban iterativo-incremental}

Para el desarrollo de TecnoTime se adoptó la metodología Kanban en su variante iterativo-incremental, ajustada a un desarrollador único. Cada fase constituyó un incremento funcional con criterios de aceptación claros y revisión continua mediante tablero digital. Esta aproximación permitió priorizar funcionalidades según la evidencia de la encuesta aplicada y adaptarse a cambios en los requerimientos durante el desarrollo.

\begin{table}[H]
  \centering\small
  \caption[Aplicación de la metodología por fases]{Aplicación de la metodología por fases. \\ \small{Fuente: Elaboración propia.}}
  \label{tab:fases_metodologia}
  \begin{tabularx}{\linewidth}{@{}p{1.2cm}p{3.9cm}>{\RaggedRight\arraybackslash}X@{}}
    \hline
    Fase & Propósito principal & Entregables y evidencias \\
    \hline
    F0 & Identificar necesidades reales mediante estudio de campo & Instrumento de recolección (Google Forms), muestra de 126 estudiantes, análisis cuantitativo con hallazgos clave (94\% Android, 80\% coordina por WhatsApp, 88\% solicita actualización automática) y derivación de requerimientos funcionales trazables. \\
    F1 & Incorporar horarios oficiales al sistema & Canal de descarga controlada de PDF, normalización a esquema Room y verificación de consistencia por carrera. \\
    F2 & Definir el modelo de dominio & Entidades, reglas y agregados consolidados en la capa \texttt{domain}, acompañados de diagramas de referencia. \\
    F3 & Generar horarios viables & Servicio de generación configurable, heurísticas MRV/LCV y pantallas para configurar y aplicar resultados. \\
    F4 & Priorizar el uso sin conexión & Flujos de sincronización diferida, políticas de frescura y mecanismos de recuperación manual. \\
    F5 & Intercambiar horarios & Contrato JSON (formato de intercambio de datos basado en texto) interoperable, flujo de importación/exportación y asistentes para compartir mediante WhatsApp. \\
    F6 & Refinar la experiencia de usuario & Onboarding (pantallas iniciales de orientación), navegación principal, widget (componente visual que se muestra en la pantalla principal del dispositivo) de acceso rápido y guías contextuales. \\
    F7 & Automatizar recordatorios e IA & Notificaciones programadas, plantillas reutilizables y asistente ``Simón'' con activación voluntaria. \\
    F8 & Exportar resultados & Generadores de reporte en PDF, imagen y Excel con validación de formato. \\
    F9 & Realizar aseguramiento de calidad & Batería de pruebas instrumentadas, casos manuales priorizados y checklist de aceptación. \\
    F10 & Endurecer seguridad y privacidad & Gestión de permisos, cifrado de datos sensibles y revisión de políticas de respaldo. \\
    F11 & Preparar liberación y métricas & Empaquetado en release, tablero de indicadores (TTS, adopción) y plan de seguimiento post despliegue. \\
    \hline
  \end{tabularx}
\end{table}

\subsection{Contexto empírico}
El desarrollo partió de un prototipo de ingesta de PDF y normalización local. A partir de esa base se ejecutaron iteraciones sobre selección de materias, generación de horarios y validaciones con usuarios piloto. Las fases posteriores incorporaron intercambio JSON para envío y recepción de horarios, mejoras de experiencia de usuario (pantallas guía, configuración y widget), recordatorios y exportaciones en formatos comunes. Al cierre se integró la asistencia ``Simón'' como componente optativo y se documentaron controles de seguridad.

\subsection{Factores de decisión}
La encuesta aplicada a 125 estudiantes orientó las prioridades del tablero:
\begin{itemize}
  \item Preferencia por Android (aproximadamente 71,7\%): orientación exclusiva a componentes nativos y widget principal.
  \item Coordinación por WhatsApp (aproximadamente 80,3\%): intercambio de horarios mediante JSON y acciones directas de compartir.
  \item Sincronización automática (aproximadamente 88,4\%): verificación de frescura y actualización discreta sin intervención del usuario.
  \item Uso sin conexión (aproximadamente 77,7\%): arquitectura offline-first (prioriza el funcionamiento sin conexión) con caché local prioritaria.
  \item Recordatorios (aproximadamente 56,2\%): programación de notificaciones con plantillas y canales específicos.
  \item Compartir horario (aproximadamente 51,2\%): exportación a PDF, imagen y Excel con parámetros de periodo.
\end{itemize}

\subsection{Decisiones técnicas derivadas}
Los factores anteriores guiaron la priorización de las fases (ver Cuadro~\ref{tab:fases_metodologia}: F4--F8) y condicionaron los artefactos liberados por incremento. Se reforzó el almacenamiento local y la sincronización diferida, se consolidó el contrato JSON con asistentes de importación y envío, se diseñaron recordatorios parametrizables y se establecieron exportaciones multiplataforma. Los artefactos y evidencias se encuentran documentados en \cite{tecnotime_repo}.

\subsection{Arquitectura de la solución}
La solución se apoya en una aplicación Android autosuficiente: la interacción sucede en la interfaz offline (Jetpack Compose), el motor de dominio coordina reglas y recordatorios, la ingesta procesa PDFs y la persistencia local resguarda horarios, configuraciones y políticas de frescura. El asistente IA opera sobre el mismo contexto y la única integración externa es el portal UMSS que publica los PDFs oficiales. La Figura \ref{fig:arquitectura_tecnotime} resume esta arquitectura offline-first.

\begin{figure}[H]
  \centering
  \includegraphics[width=1\linewidth]{images/diagrams/cap4_arq_general.png}
  \caption[Arquitectura general empleada en TecnoTime]{Arquitectura general empleada en TecnoTime. \\ \small{Fuente: Elaboración propia.}}
  \label{fig:arquitectura_tecnotime}
\end{figure}

% ----------------------------------------------------------------------
\section{Estudio de campo y levantamiento de requerimientos}

\subsection{Diseño del instrumento de recolección}

Para fundamentar el desarrollo de TecnoTime en necesidades reales de los estudiantes de la FCyT, se diseñó un cuestionario estructurado orientado a capturar patrones de uso de herramientas actuales, puntos de dolor en la gestión de horarios y expectativas sobre funcionalidades deseadas en una aplicación móvil complementaria.

El instrumento se organizó en cuatro bloques temáticos: caracterización del encuestado (carrera, semestre, dispositivos utilizados), uso de herramientas actuales (Cappuchino, PDFs oficiales, canales de comunicación), problemática percibida (frustración, repetición de pasos, dificultades de coordinación) y demandas específicas (actualización automática, modo sin conexión, recordatorios, compartir horarios).

El cuestionario fue aplicado mediante Google Forms durante el periodo académico 2025-I, con participación voluntaria y anónima. El instrumento completo se presenta en el Anexo \ref{ann:instrumento}, mientras que los resultados tabulados se encuentran disponibles en el Anexo \ref{ann:historias} y en formato digital \cite{encuesta_fcyt_2025}.

\subsection{Aplicación y características de la muestra}

Se aplicó un muestreo por conveniencia dirigido a estudiantes de la Facultad de Ciencias y Tecnología de la UMSS, obteniendo un total de 126 respuestas válidas. La muestra incluyó estudiantes de diferentes carreras, niveles académicos y turnos, reflejando la diversidad de la población objetivo.

El perfil de los encuestados abarca desde estudiantes de primeros semestres hasta niveles avanzados, con predominio de las carreras de Ingeniería Informática e Ingeniería de Sistemas. La participación fue voluntaria y la tasa de respuesta completa alcanzó el 100\% de los formularios iniciados, evidenciando el interés de los estudiantes en la problemática abordada.

\subsection{Resultados cuantitativos de la encuesta}

\subsubsection{Dispositivos utilizados}

El análisis de dispositivos revela una marcada preferencia por plataformas móviles. El 94\% de los encuestados utiliza dispositivos Android como herramienta principal para consultas académicas, mientras que solo el 26\% dispone de computadora portátil de uso regular. Esta distribución confirma la necesidad de priorizar el desarrollo nativo para Android y optimizar la experiencia móvil.

\begin{figure}[H]
  \centering
  \includegraphics[width=0.4\textwidth]{images/diagrams/encuesta_dispositivo.png}
  \caption[Distribución de dispositivos utilizados por los estudiantes (n=126)]{Distribución de dispositivos utilizados por los estudiantes (n=126). \\ \small{Fuente: \cite{encuesta_fcyt_2025}.}}
  \label{fig:encuesta_dispositivo}
\end{figure}

\subsubsection{Uso de Cappuchino}

La herramienta Cappuchino presenta un patrón de uso concentrado en periodos específicos. El 48\% de los estudiantes la utiliza únicamente durante el periodo de inscripción, mientras que el 44\% la consulta varias veces por semestre. Solo el 8\% reporta un uso diario o muy frecuente, evidenciando que la herramienta no cubre necesidades de consulta continua fuera del periodo de inscripción.

\begin{figure}[H]
  \centering
  \includegraphics[width=0.4\textwidth]{images/diagrams/encuesta_uso_capuchino.png}
  \caption[Frecuencia de uso de Cappuchino por los estudiantes (n=126)]{Frecuencia de uso de Cappuchino por los estudiantes (n=126). \\ \small{Fuente: \cite{encuesta_fcyt_2025}.}}
  \label{fig:encuesta_uso_capuchino}
\end{figure}

\subsubsection{Coordinación social mediante WhatsApp}

El 80\% de los encuestados valida información de horarios y coordina cambios mediante grupos de WhatsApp, confirmando que la mensajería instantánea se ha convertido en un canal informal pero crítico para la gestión de información académica. Este hallazgo fundamenta la necesidad de habilitar intercambio estructurado de horarios compatible con plataformas de mensajería.

\begin{figure}[H]
  \centering
  \includegraphics[width=0.4\textwidth]{images/diagrams/encuesta_whatsapp.png}
  \caption[Coordinación social mediante WhatsApp (n=126)]{Coordinación social mediante WhatsApp (n=126). \\ \small{Fuente: \cite{encuesta_fcyt_2025}.}}
  \label{fig:encuesta_whatsapp}
\end{figure}

\subsubsection{Frustración por repetir pasos}

El 60\% de los estudiantes califica como alta su frustración (niveles 1-2-3 en escala de 1 a 5) por tener que repetir pasos para consultar horarios. Este indicador evidencia la necesidad de memoria de contexto y persistencia de selecciones para evitar procesos repetitivos en cada sesión de consulta.

\begin{figure}[H]
  \centering
  \includegraphics[width=0.4\textwidth]{images/diagrams/encuesta_frustracion.png}
  \caption[Nivel de frustración por repetir pasos (n=126)]{Nivel de frustración por repetir pasos (n=126). \\ \small{Fuente: \cite{encuesta_fcyt_2025}.}}
  \label{fig:encuesta_frustracion}
\end{figure}

\subsubsection{Principales demandas identificadas}

Las demandas con mayor valoración por parte de los estudiantes incluyen: actualización automática de horarios (88\%), modo sin conexión (78\%), recordatorios antes de clase (56\%) y funcionalidad de compartir horarios (51\%). Estas prioridades guiaron la definición de las fases correspondientes del desarrollo (ver Cuadro~\ref{tab:fases_metodologia}: F4, F5, F7 y F8).

\begin{figure}[H]
  \centering
  \includegraphics[width=0.7\textwidth]{images/diagrams/encuesta_demandas.png}
  \caption[Principales demandas de los estudiantes (n=126)]{Principales demandas de los estudiantes (n=126). \\ \small{Fuente: \cite{encuesta_fcyt_2025}.}}
  \label{fig:encuesta_demandas}
\end{figure}

\subsection{Derivación de requerimientos funcionales}

Los hallazgos cuantitativos de la encuesta se tradujeron en requerimientos funcionales específicos mediante un proceso de trazabilidad que vincula cada demanda identificada con una funcionalidad concreta de la aplicación y un criterio de aceptación verificable.

\begin{table}[H]
  \centering
  \caption[Trazabilidad de hallazgos a requerimientos funcionales]{Trazabilidad de hallazgos a requerimientos funcionales. \\ \small{Fuente: Elaboración propia basada en \cite{encuesta_fcyt_2025}.}}
  \label{tab:derivacion_req}
  \small
  \begin{tabular}{|p{5.0cm}|p{7.5cm}|p{3.5cm}|}
    \hline
    Hallazgo & Requerimiento derivado & Criterio de aceptación \\
    \hline
    94\% usa Android & Priorizar Android y vista móvil & Flujo usable sin laptop \\
    \hline
    60\% frustración por repetir pasos & Memoria de carrera/nivel/materias & Ingreso sin repetir selección \\
    \hline
    80\% coordina por WhatsApp & Intercambio estructurado en JSON & Compartir/abrir desde mensajería \\
    \hline
    78\% valora modo sin conexión & Operación offline-first & Consulta sin red \\
    \hline
    57\% valora notificaciones & Recordatorios previos a clase & Alertas configurables \\
    \hline
    88\% solicita actualización automática & Actualización desde fuente oficial & Fecha visible de actualización \\
    \hline
  \end{tabular}
\end{table}

\subsection{Proceso propuesto (TO-BE)}

Con base en las demandas identificadas y los requerimientos derivados, se diseñó un proceso mejorado que aborda las limitaciones del flujo actual. El proceso propuesto con TecnoTime simplifica la gestión de horarios mediante: generación automática de combinaciones sin choques, memoria de contexto para evitar repetición de selecciones, intercambio estructurado compatible con mensajería, consulta diaria sin conexión y recordatorios configurables antes de clase.

\begin{figure}[H]
  \centering
  \includegraphics[width=0.35\textwidth]{images/diagrams/area_tobe_proceso.png}
  \caption[Procesos propuestos (TO-BE) a nivel de uso]{Procesos propuestos (TO-BE) a nivel de uso con TecnoTime. \\ \small{Fuente: Elaboración propia.}}
  \label{fig:area_tobe}
\end{figure}

El proceso propuesto reduce significativamente los pasos necesarios para consultar y actualizar horarios, elimina la dispersión de información entre múltiples canales y habilita funcionalidades de coordinación social mediante formatos estructurados.

% ----------------------------------------------------------------------
% (Reubicado al final del capítulo para que 4.2 inicie con F1)
% \section{Mapa de fases (F1-F11)} — contenido movido más adelante

% ============================
\section{F1: Ingestión y normalización (de PDF a BD)}

\subsection{Objetivo}
Transformar horarios oficiales (PDF) a un modelo estructurado local, controlando la carga a la fuente y asegurando la frescura de los datos.

\subsubsection{4.2.1 Verificación de caché y decisión}
Al abrir el generador se valida conectividad y vigencia de datos. Si el contenido local está fresco, se usa cache; de lo contrario, se inicia sincronización automática.
\begin{itemize}
  \item Control en la interfaz: `GenerateScheduleViewModel` ejecuta \textsf{checkConnectivityAndSync()} y bloquea acciones durante sincronización.
  \item Conectividad: `NetworkConnectivityChecker` discrimina entre online/offline.
  \item Frescura: `AutoSyncUseCase` compara \textsf{updatedDate} remoto con \textsf{lastSyncTime} local (por carrera) y decide actualizar.
  \item Preferencias: `SyncPreferences` limita chequeos con una ventana (p. ej., 6 horas) para evitar consultas excesivas.
\end{itemize}

\begin{figure}[H]
  \centering
  \includegraphics[width=0.35\linewidth]{images/diagrams/cap4_c4_1_context.png}
  \caption[C4 Nivel 1 — Contexto del sistema para F1]{C4 Nivel 1 — Contexto del sistema para F1. \\ \small{Fuente: Elaboración propia.}}
  \label{fig:c4_context_f1}
\end{figure}

\subsubsection{4.2.2 Ingesta, análisis y persistencia con frescura}
Cuando se requiere actualización, el flujo descarga, extrae y analiza el PDF oficial; luego normaliza y persiste en Room. Se aplican controles de retroceso exponencial (backoff) y cortacircuito (circuit breaker) para proteger la fuente.
\begin{itemize}
  \item Descubrimiento/descarga: `ScheduleScraper` ubica recursos y `PdfDownloader` obtiene el PDF.
  \item Extracción y análisis: `PdfExtractor` obtiene texto; `PdfParser` (componente que analiza y transforma datos de entrada) aplica expresiones regulares tolerantes para cabeceras, grupos, días y horas, consolidando líneas y docentes.
  \item Normalización/persistencia: `ImportCareerSchedulesUseCase` proyecta entidades y persiste con índices únicos compuestos (Room) y políticas \textsf{ON CONFLICT}.
  \item Frescura: tras actualizar, se registra \textsf{lastSyncTime} por carrera y se guarda el instante del chequeo en `SyncPreferences`.
  \item Robustez: `ExponentialBackoff` y `CircuitBreaker` controlan reintentos y abren/cierran el circuito ante fallas repetidas.
  \item Respaldo: si no hay red, se permite importación manual desde PDF local y se opera totalmente en caché.
\end{itemize}

\begin{figure}[H]
  \centering
  \includegraphics[width=1\linewidth]{images/diagrams/cap4_c4_2_container.png}
  \caption[C4 Nivel 2 — Contenedores involucrados en F1]{C4 Nivel 2 — Contenedores involucrados en F1. \\ \small{Fuente: Elaboración propia.}}
  \label{fig:c4_container_f1}
\end{figure}

\begin{figure}[H]
  \centering
  \includegraphics[width=1\linewidth]{images/diagrams/cap4_c4_3_component_f1.png}
  \caption[C4 Nivel 3 — Componentes de ingesta y normalización (F1)]{C4 Nivel 3 — Componentes de ingesta y normalización (F1). \\ \small{Fuente: Elaboración propia.}}
  \label{fig:c4_component_f1}
\end{figure}

\begin{figure}[H]
  \centering
  \includegraphics[width=1\linewidth]{images/diagrams/cap4_c4_4_code_f1.png}
  \caption[C4 Nivel 4 — Clases y casos de uso relevantes (F1)]{C4 Nivel 4 — Clases y casos de uso relevantes (F1). \\ \small{Fuente: Elaboración propia.}}
  \label{fig:c4_code_f1}
\end{figure}

Criterios de entrada (Entry).
Conectividad disponible o PDF local; sin sincronización en curso.

Criterios de salida (Exit/DoD).
Base de datos consistente por carrera, nivel, materia, grupo y bloques; \textsf{lastSyncTime} actualizado; pruebas de ingestión y nuevo análisis exitosas.

Evidencias (código).
\begin{itemize}
  \item Caso de uso de sincronización automática (\Class{AutoSyncUseCase}) y preferencias de sincronización (\Class{SyncPreferences}) para ventanas y sellos de tiempo.
  \item Rastreador de horarios (\Class{ScheduleScraper}) y componentes de adquisición/procesamiento de PDF (\Class{PdfDownloader}/\Class{PdfExtractor}/\Class{PdfParser}).
  \item Importación por carrera (\Class{ImportCareerSchedulesUseCase}) y componentes de resiliencia (\Class{ExponentialBackoff}, \Class{CircuitBreaker}).
\end{itemize}

KPIs.
Cache hit-rate semanal $\geq 80\%$; reintentos acotados; tiempo de ingesta por PDF dentro del objetivo operacional.

% ============================
\section{F2: Modelo de dominio (entidades, reglas, mapeo)}

\subsection{Objetivo}
Consolidar un modelo de dominio coherente con el área de aplicación, garantizando integridad y trazabilidad entre entidades y sus reglas.

\subsection{Qué se hizo}
Se modelaron entidades y relaciones clave; se definieron índices/llaves para unicidad e integridad; y se implementaron mapeos entre ingreso (PDF/JSON), dominio y almacenamiento.

\subsection{Entidades (derivadas del código)}
\begin{itemize}
  \item Materia: \cclass{domain/model/Subject.kt}{Subject} (\textsf{code}, \textsf{name}, \textsf{isElective}, \textsf{isApproved}, \textsf{isActive}).
  \item Grupo: \Class{Group} (\textsf{id}, \textsf{subjectCode}, \textsf{groupId}, \textsf{levelId}, \textsf{groupName}, \textsf{type}, \textsf{modality}, \textsf{isActive}).
  \item Bloque/Slot: \Class{GroupSchedule} (\textsf{day}, \textsf{startTime}, \textsf{endTime}, aula/docente opcionales).
  \item Carrera/Nivel: \Class{Career}, \Class{Level}; agregados: \Class{CareerWithLevels}, \Class{LevelWithSubjects}.
  \item Docente/Aula: \Class{Teacher}, \Class{Classroom}.
  \item Selección del usuario: \Class{SelectedSubject}, \Class{SelectedSubjectWithGroup}.
  \item Preferencias/Usuario: \Class{UserSettings} (24h, fines de semana, recordatorios, IA opt-in, etc.).
\end{itemize}

\subsection{Reglas e invariantes}
\begin{itemize}
  \item Unicidad de \textsf{Subject.code}: \Class{SubjectEntity} define índice único sobre \textsf{code}.
  \item Unicidad compuesta \textsf{(subjectCode, groupId)}: \Class{GroupEntity} con índice compuesto único; FK a \Class{LevelEntity} y \Class{SubjectEntity}.
  \item Temporal: \Class{GroupScheduleEntity} asegura \textsf{startTime}–\textsf{endTime} para cada \textsf{day} y \textsf{groupId}; validación operativa start<end al normalizar/guardar.
  \item Integridad referencial: FKs en \Class{GroupScheduleEntity} hacia \Class{GroupEntity}, \Class{TeacherEntity} y \Class{ClassroomEntity} (nulables cuando no hay asignación).
\end{itemize}

\subsection{Mapeos del ingreso al dominio y al almacenamiento}
\begin{itemize}
  \item DTO de horario compartible (\cclass{domain/model/ShareableScheduleDto.kt}{ShareableScheduleDto}) y analizadores PDF/JSON para el ingreso.
  \item Mapeadores de dominio (p. ej., \Class{SubjectMapper}, \Class{GroupMapper}, \Class{GroupScheduleMapper}, \Class{UserSettingsMapper}) para transformar entre entidades y dominio.
  \item Casos de uso para interoperabilidad JSON (\cclass{domain/usecase/GenerateShareableScheduleJsonUseCase.kt}{GenerateShareableScheduleJsonUseCase}, \Class{LoadShareableScheduleUseCase}, \Class{ImportSharedScheduleUseCase}).
\end{itemize}

\subsection{Evidencias (código)}
\begin{itemize}
  \item Unicidad: \Class{SubjectEntity}: índice unique(\textsf{code}); \Class{GroupEntity}: índice compuesto unique(\textsf{subject\_code}, \textsf{group\_id}).
  \item Modelos: \Class{Subject}, \Class{Group}, \Class{GroupSchedule}, \Class{UserSettings}.
  \item Mapeo: \Class{GroupMapper} incluye \textsf{levelId}, \textsf{subjectCode} y \textsf{groupId} como claves de relación.
\end{itemize}

Como apoyo visual, la Figura \ref{fig:f2_modelo_dominio_er} ilustra el diagrama entidad–relación del modelo de dominio, destacando entidades operativas, claves e integridad referencial utilizadas en la implementación.

\begin{figure}[H]
  \centering
  \includegraphics[width=1\linewidth]{images/diagram_er.png}
  \caption[Modelo de dominio: diagrama entidad–relación (ER)]{Modelo de dominio: diagrama entidad–relación (ER). \\ \small{Fuente: elaboración propia.}}
  \label{fig:f2_modelo_dominio_er}
\end{figure}

\subsection{Criterios de entrada/salida (Entry/Exit/DoD)}
Entrada: entidades y relaciones definidas, migraciones aplicadas y datos de prueba disponibles.

Salida/DoD: unicidad e integridad referencial verificadas; mapeos ida/vuelta probados; repositorios y casos de uso operativos sobre el modelo.

% ============================
\section{F3: Generador (restricciones, heurísticas, UI)}

\subsection{Objetivo}
Seleccionar un grupo por materia evitando choques por defecto y optimizando los huecos y la afinidad con docentes favoritos.

\subsection{Qué se hizo}
Se parametrizó el problema, se aplicó búsqueda con poda y heurísticas MRV/LCV con scoring multicriterio y se integró UI para configurar y aplicar resultados.

\subsection{Parámetros reales}
\begin{itemize}
  \item Límite de materias por horario: \(\leq\)20 y número de horarios generados: hasta 25. Parametrización en \cclass{domain/model/ScheduleGenerationParams.kt}{ScheduleGenerationParams} (\textsf{totalSubjectsCount}, \textsf{numberOfSchedules}).
  \item Flags principales:
    \begin{itemize}
      \item Priorizar docentes favoritos (\textsf{prioritizeFavoriteTeachers}).
      \item Minimizar huecos (\textsf{minimizeGaps}).
      \item Aceptar choques controlados (\textsf{acceptConflicts}; desactivado por defecto).
      \item Fijar grupos ya elegidos para regeneración parcial (\textsf{fixedGroups}).
    \end{itemize}
\end{itemize}

\subsection{Estrategia operativa}
\begin{itemize}
\item Selección incremental con poda temprana y un orden implícito MRV/LCV; se promueven los niveles bajos mediante \textsf{promoteLowestLevels(...)} y se asegura la progresión de series con \textsf{enforceSeriesProgression(...)} en \texttt{GenerateSchedulesUseCaseImpl}.
\item Priorización de grupos con docentes favoritos mediante \textsf{prioritizeGroupsByFavoriteTeacher(...)}: los favoritos sin conflicto prevalecen y, si \textsf{acceptConflicts} está habilitado, se consideran también los que generan conflicto.
\item Se definió una fábrica de estrategias (\texttt{ScheduleStrategyFactory}), que integra: estrategia de aceptación de choques (\texttt{AcceptConflictsStrategy}, peso 1.0), estrategia para minimizar huecos (\texttt{MinimizeGapsStrategy}, peso 1.0 cuando está activa) y estrategia para priorizar docentes favoritos (\texttt{PrioritizeTeachersStrategy}, peso 2.0 cuando está activa) dentro de un compuesto (\texttt{CompositeStrategy}). Estas ponderaciones priorizan evitar choques (crítico), reducir huecos (confort) y favorecer docentes favoritos (preferencia fuerte cuando se habilita).
\item El generador de horarios (\texttt{ScheduleGenerator}) conserva las mejores combinaciones (Top-N) y penaliza la ausencia de alternativas preferidas.
\end{itemize}
(Rutas completas en el Anexo \ref{ann:trazabilidad-tecnotime}).

\subsection{Interacción de usuario (pantallas reales)}
\begin{itemize}
  \item Pantalla de configuración (\texttt{GenerateScheduleConfigScreen}) y modelo de vista (\texttt{GenerateScheduleViewModel}) para selección de materias, opciones y lanzamiento de la generación.
  \item Pantalla de resultados (\texttt{GenerateScheduleResultsScreen}) y vista previa (\texttt{SchedulePreview}) para aplicar parcial (solo marcadas vía \textsf{fixedGroups}) o completo (todo el horario propuesto).
\end{itemize}
(Rutas completas en el Anexo \ref{ann:trazabilidad-tecnotime}).

\subsection{Diagramas}
A continuación se presentan dos vistas complementarias del proceso de generación: (i) secuencia extremo a extremo desde la UI hasta el evaluador y (ii) decisiones de poda, scoring y relajación.

\begin{figure}[H]
  \centering
  \includegraphics[width=1\linewidth]{images/diagrams/cap4_f3_secuencia_generacion.png}
\caption[Secuencia de generación de horarios]{Secuencia de generación de horarios. \\ \small{Fuente: Elaboración propia.}}
  \label{fig:f3_secuencia_generacion}
\end{figure}

\noindent Para facilitar la lectura, la Figura~\ref{fig:f3_decisiones_generador} se organiza en dos paneles: (I) conformación del conjunto candidato (obligatorias, preferencia por docentes y orden de exploración) y (II) evaluación de combinaciones (penalización de huecos, tratamiento de choques y relajación cuando no hay opciones válidas).

\begin{figure}[H]
  \centering
  \begin{minipage}[t]{0.48\linewidth}
    \centering
    \includegraphics[width=\linewidth]{images/diagrams/cap4_f3_decisiones_generador_part1.png}
  \end{minipage}\hfill
  \begin{minipage}[t]{0.48\linewidth}
    \centering
    \includegraphics[width=\linewidth]{images/diagrams/cap4_f3_decisiones_generador_part2.png}
  \end{minipage}
\caption[Decisiones del generador]{Decisiones del generador. \\ \small{Fuente: Elaboración propia.}}
  \label{fig:f3_decisiones_generador}
\end{figure}

\subsection{Criterios de entrada/salida (Entry/Exit/DoD)}
Entrada: datos frescos (si hay red) y selección de materias/grupos fijos.

Salida/DoD: \(N\) horarios sin solapes por defecto; aplicación parcial/completa funcional; casos TC-(TBD) ejecutados.

\subsection{KPIs}
TTS (Time-To-Schedule) \(\leq\) 3 min para \(N{=}10\); regeneración parcial \(\leq\) 1 s por cambio.

% \begin{figure}[H]
%   \centering
%   \includegraphics[width=0.82\linewidth]{TODO}
%   \caption{Decisiones y mantenimiento de Top-N (vista simplificada).}
%   \label{fig:f3_generador}
% \end{figure}

\subsection{Criterios y pesos verificados}
La evaluación combina tres estrategias y dos penalizaciones fijas. Resumen breve:
\begin{itemize}
  \item Choques: \texttt{AcceptConflictsStrategy} (peso 1.0). Siempre activo. Si \textsf{acceptConflicts=false}, un choque descarta la combinación (infinito); si \textsf{true}, aplica penalización baja por conflicto.
  \item Huecos (gaps): \texttt{MinimizeGapsStrategy} (peso 1.0). Activo solo con \textsf{minimizeGaps=true}. Puntúa la suma de recesos entre bloques; menor es mejor.
  \item Docentes favoritos: \texttt{PrioritizeTeachersStrategy} (peso 2.0). Activo con \textsf{prioritizeFavoriteTeachers=true}. Reduce el puntaje cuando hay bloques con docente favorito.
  \item Penalización por preferidos ausentes: +1000.0 por cada código preferido no satisfecho.
  \item Penalización por grupo no favorito: +1500.0 si, existiendo grupos favoritos para una materia, el elegido no es favorito.
\end{itemize}
(Rutas completas en el Anexo \ref{ann:trazabilidad-tecnotime}).

\subsection{Evidencias (código)}
\begin{itemize}
  \item Generación: \Class{GenerateSchedulesUseCaseImpl}, \Class{ScheduleGenerator}.
  \item Estrategias y evaluación: \Class{ScheduleStrategyFactory}, \Class{CompositeStrategy}, \Class{AcceptConflictsStrategy}, \Class{MinimizeGapsStrategy}, \Class{PrioritizeTeachersStrategy}.
  \item UI y orquestación: \Class{GenerateScheduleConfigScreen}, \Class{GenerateScheduleResultsScreen}, \Class{GenerateScheduleViewModel}.
\end{itemize}

% ============================
\section{F4: Prioridad sin conexión (frescura, políticas, fallback)}

\subsection{Objetivo}
Asegurar disponibilidad sin red priorizando la caché local, controlando frescura y habilitando recuperación manual por PDF.

\subsection{Qué se hizo}
Se implementaron verificaciones de conectividad y frescura, políticas de sincronización (ventanas/horarios), reintentos con backoff y circuit breaker y alternativa de respaldo (fallback) por PDF local.

\subsection{Frescura}
Antes de invocar el generador se valida conectividad y frescura de datos:
\begin{itemize}
  \item \Class{GenerateScheduleViewModel}: ejecuta \textsf{checkConnectivityAndSync()} al cargar; usa \Class{NetworkConnectivityChecker} y, con red disponible, dispara \Class{AutoSyncUseCase}.
  \item \Class{AutoSyncUseCase}: compara \textsf{Career.updatedDate} remoto con \textsf{lastSyncTime} local y actualiza materias mediante \Class{RefreshSubjectsForCareerUseCase}. Actualiza \textsf{SyncPreferences.setLastSyncCheck()}.
  \item \Class{SyncPreferences}: define intervalo de verificación de 6\,h y evita consultas excesivas.
  \item Durante la sincronización, \Class{GenerateScheduleViewModel} marca \textsf{isSyncing=true} y bloquea generate/regenerate para mantener consistencia.
\end{itemize}

\subsection{Políticas}
\(T_{fresh}=48\,h\): si los datos locales superan ese umbral, se prioriza sincronizar antes de generar. \(W_{nosync}=07{:}00\text{–}12{:}30\) (horario académico): se difieren sincronizaciones automáticas en ese rango para no interrumpir uso intensivo; el ajuste es configurable. Reintentos controlados con:
\begin{itemize}
  \item \Class{ExponentialBackoff}: $t_0{=}2$\,s, $t_{max}{=}1$\,h, multiplicador 2.0 con jitter.
  \item \Class{CircuitBreaker}: umbral de 5 fallos y recuperación en 10\,min; integra \Class{AutoSyncUseCase} para bloquear temporalmente el origen tras errores repetidos.
\end{itemize}

\subsection{Fallback}
Sin conectividad se opera desde caché local (Room). Si el servicio cae pero existe PDF oficial actualizado, se habilita reingesta manual vía \Class{SyncCareerFromLocalPdfUseCase}.

\subsection{Criterios de entrada/salida (Entry/Exit/DoD)}
Entrada: conectividad verificada o disponibilidad de PDF local; sin sincronización en curso.

Salida/DoD: datos locales actualizados y consistentes por carrera; generación bloqueada durante sincronización; reintentos espaciados y circuito recuperado tras fallos.

\subsection{Evidencias (código)}
\begin{itemize}
  \item Gate del generador: \Class{GenerateScheduleViewModel} (bandera \textsf{isSyncing}).
  \item Chequeo y sincronización: \Class{NetworkConnectivityChecker}, \Class{AutoSyncUseCase}, \Class{SyncPreferences}.
  \item Fallback manual: \Class{SyncCareerFromLocalPdfUseCase}.
\end{itemize}

\begin{figure}[H]
  \centering
  \includegraphics[width=1\linewidth]{images/diagrams/cap4_f1a_cache_decision.png}
  \caption[Gate de caché y frescura]{Gate de caché y frescura. \\ \small{Fuente: Elaboración propia.}}
  \label{fig:f4_offline}
\end{figure}

\begin{figure}[H]
  \centering
  \includegraphics[width=1\linewidth]{images/diagrams/offline_first_sync.png}
  \caption[Sincronización con backoff y circuit breaker]{Sincronización con backoff y circuit breaker. \\ \small{Fuente: Elaboración propia.}}
  \label{fig:f4_offline_sync}
\end{figure}

\begin{figure}[H]
  \centering
  \includegraphics[width=1\linewidth]{images/diagrams/cap4_f4_pdf_fallback.png}
  \caption[Fallback por PDF local]{Fallback por PDF local. \\ \small{Fuente: Elaboración propia.}}
  \label{fig:f4_offline_pdf}
\end{figure}

% ============================
\section{F5: Interoperabilidad (JSON + WhatsApp)}

\subsection{Objetivo}
Permitir compartir e importar horarios entre estudiantes de forma interoperable y segura.

\subsection{Qué se hizo}
Se añadieron exportaciones (PNG/PDF/Excel/JSON), un contrato JSON auto-contenido y flujos de envío/recepción por mensajería con validaciones e importación idempotente.

\subsection{Exportar}
Acciones: imagen (PNG), PDF, Excel y copia JSON del horario. Se solicita confirmación previa; si no hay materias inscritas, se bloquea la exportación con mensaje. Implementación de UI en \texttt{SettingsSendScheduleScreen} y orquestación en \texttt{SettingsSendScheduleViewModel}. Generación: \texttt{GenerateWeeklyScheduleImageUseCase}, \texttt{GenerateWeeklySchedulePdfUseCase}, \texttt{GenerateScheduleExcelUseCase}, \texttt{GenerateShareableScheduleJsonUseCase}. Compartición mediante \textsf{FileProvider} + \textsf{Intent.ACTION\_SEND}. (Rutas completas en el Anexo \ref{ann:trazabilidad-tecnotime}).

\subsection{Enviar copia (JSON)}
Permite selección parcial en \texttt{SettingsSendScheduleViewModel}. El generador JSON \texttt{ShareableScheduleJsonGenerator} produce claves: \textsf{meta}, \textsf{careers}, \textsf{subjects}, \textsf{groups}, \textsf{teachers}, \textsf{classrooms}, \textsf{entries}. (Rutas completas en el Anexo \ref{ann:trazabilidad-tecnotime}).

\subsection{Importar (desde WhatsApp)}
Al abrir el JSON con TecnoTime: \texttt{LoadShareableScheduleUseCase} valida y carga; \texttt{ImportSharedScheduleUseCase} realiza merge sin duplicados y maneja:
\begin{itemize}
  \item Materias ya inscritas/aprobadas (no duplicar; cambio de grupo si aplica).
  \item Diferencia de carrera (\textsf{autoSyncCareers} para sincronizar primero si corresponde).
  \item Persistencia idempotente en \texttt{SelectedSubjectRepository} y \texttt{GroupScheduleRepository}.
\end{itemize}
(Rutas completas en el Anexo \ref{ann:trazabilidad-tecnotime}).

\begin{itemize}[nosep,leftmargin=*]
  \item \texttt{loadEnrolledSubjects()}
  \item \texttt{toggleSubjectSelection()}
  \item \texttt{selectAllSubjects()}
  \item \texttt{clearSelection()}
\end{itemize}

% Rutas omitidas: se referencian por nombre de clase.

\begin{figure}[H]
  \centering
  \includegraphics[width=1\linewidth]{images/diagrams/cap4_f5_share_import_flow_part1.png}
\caption[Exportación del horario en la aplicación]{Exportación del horario en la aplicación. \\ \small{Fuente: Elaboración propia.}}
  \label{fig:f5_flow_export}
\end{figure}

\begin{figure}[H]
  \centering
  \includegraphics[width=1\linewidth]{images/diagrams/cap4_f5_share_import_flow_part2.png}
\caption[Importación del horario en la aplicación]{Importación del horario en la aplicación. \\ \small{Fuente: Elaboración propia.}}
  \label{fig:f5_flow_import}
\end{figure}

El intercambio define un bloque de metadatos y un cuerpo auto-contenido:
\begin{itemize}
  \item \textsf{meta}: \{\textsf{version}, \textsf{min\_supported}\}. Se usa control semántico (\textsf{1.0.0}). La aplicación acepta archivos cuya \textsf{min\_supported} $\leq$ versión actual del contrato.
  \item \textsf{careers}: carreras implicadas (código y nombre) para habilitar/sincronizar si corresponde.
  \item \textsf{subjects}, \textsf{groups}, \textsf{teachers}, \textsf{classrooms}: catálogos mínimos para enriquecer la vista previa y resolver identificadores.
  \item \textsf{entries}: lista de EnrolledSchedulePreview.
\end{itemize}
Campos mínimos obligatorios por entry para garantizar la importación: \textsf{subject.code}, \textsf{group.groupId} y \textsf{schedule} (día y franja horaria). El resto de atributos enriquecen la experiencia (docente, aula, color/emoji, notificaciones) y se consumen cuando están disponibles.

\subsection{Criterios de entrada/salida (Entry/Exit/DoD)}
\begin{itemize}
  \item Entrada: Materias inscritas disponibles; permisos de almacenamiento/compartir concedidos si aplica; JSON válido al importar.
  \item Salida/DoD: Exportaciones válidas (PNG/PDF/Excel/JSON) compartibles; importación ejecuta validaciones y merge sin duplicados; bloqueos correctos cuando no hay materias.
\end{itemize}

% ============================
\section{F6: UX aplicada: onboarding, navegación, edición y widget}

\subsection{Objetivo}
Optimizar la interacción para tareas frecuentes con mínima fricción y consistencia visual.

\subsection{Qué se hizo}
Se diseñaron flujos guiados (onboarding, agregar/editar), navegación por días y un widget informativo; se consolidaron criterios de entrada/salida.

Para mejorar la legibilidad y evitar secuencias con flechas, el Cuadro \ref{tab:f6_resumen_ux} resume los flujos, comportamientos esperados e implementación de componentes de interfaz asociados a F6.

\subsection{Criterios de entrada/salida (Entry/Exit/DoD)}
\begin{table}[H]
   \centering\small
   \caption[Resumen de UX aplicada (F6)]{Resumen de UX aplicada (F6). \\ \small{Fuente: Elaboración propia.}}\label{tab:f6_resumen_ux}
   \begin{tabularx}{\linewidth}{@{}p{3.0cm}>{\RaggedRight\arraybackslash}X>{\RaggedRight\arraybackslash}X@{}}
     \hline
     Flujo & Pasos / Comportamiento & Implementación (clases principales) \\
     \hline
     Onboarding & Bienvenida; nombre de usuario; ajustes iniciales (formato 24h, fines de semana/días sin clase, tema claro/oscuro); sincronización de carreras; progreso por nivel (A, B, C, ... con porcentaje); activación opcional de Simón (IA). & \texttt{WelcomeScreen}, \texttt{WelcomeViewModel}; pantallas de ajustes y componentes comunes según corresponda. \\
     Agregar materia & Secuencia guiada: seleccionar nivel; seleccionar materia; elegir grupo con preview de días/horas; confirmar. Asignación de color/emoji automática con opción de elección manual. & \texttt{AddSubjectFlowScreen}, \texttt{AddSubjectViewModel}; \texttt{SelectLevelScreen}, \texttt{SelectSubjectScreen}, \texttt{SelectGroupScreen}; \texttt{ColorPickerScreen}, \texttt{EmojiPickerScreen}. \\
     Editar grupo & Desde el card del home: cambiar de grupo (con vista de horarios de destino) o finalizar materia con estado aprobado/abandonado (diálogo de confirmación). Edición de color/emoji disponible. & \texttt{EditGroupScreen}, \texttt{EditGroupSelectScreen}, \texttt{EditGroupViewModel}; \texttt{EditColorScreen}, \texttt{EditEmojiScreen}. \\
     Navegación & Deslizamiento entre días; vista semanal con pestañas animadas; selector de fecha en encabezado para ir a una fecha específica. & \texttt{HomeScreen} (paginación), \texttt{WeekDayTabsAnimated}. \\
     Widget & Muestra el día actual; permite desplazamiento entre días; tarjetas con información esencial (materia, grupo, hora, aula, docente). & \texttt{ScheduleAppWidgetProvider}, \texttt{ScheduleRemoteViewsService}; utilidades en \texttt{WidgetDateUtils}, preferencias en \texttt{WidgetPrefs}. \\
     \hline
   \end{tabularx}
\end{table}

\begin{table}[H]
   \centering\small
   \caption[Criterios de entrada y salida (F6)]{Criterios de entrada y salida (F6). \\ \small{Fuente: Elaboración propia.}}\label{tab:f6_entry_exit}
   \begin{tabularx}{\linewidth}{@{}p{2.8cm}>{\RaggedRight\arraybackslash}X@{}}
     \hline
     Criterio & Descripción \\
     \hline
     Entrada & Interfaz base disponible con datos mínimos para interacción guiada. \\
     Salida / DoD & Reducción de pasos en tareas críticas; preview consistente; widget operativo y sincronizado. \\
     \hline
   \end{tabularx}
\end{table}

% Rutas omitidas para simplificar; se emplean únicamente nombres de clases.

% ============================
\section{F7: Notificaciones e IA (Simón, opt-in)}

\subsection{Objetivo}
Entregar recordatorios puntuales y una IA opcional sin afectar la operación base.

\subsection{Qué se hizo}
Se crearon canales y estilos de notificación, programación en background y un flujo opt-in para IA on-device con control de descarga/uso.

Para estandarizar la presentación y facilitar la lectura, el Cuadro \ref{tab:f7_notif_ai} resume los flujos, el comportamiento esperado y los componentes clave. Se mantiene la activación voluntaria (opt-in) de IA y los canales separados para notificaciones.

\subsection{Criterios de entrada/salida (Entry/Exit/DoD)}
\begin{table}[H]
   \centering\small
   \caption[Resumen de notificaciones e IA (F7)]{Resumen de notificaciones e IA (F7). \\ \small{Fuente: Elaboración propia.}}\label{tab:f7_notif_ai}
   \begin{tabularx}{\linewidth}{@{}p{3.5cm}>{\RaggedRight\arraybackslash}X>{\RaggedRight\arraybackslash}X@{}}
     \hline
     Flujo & Comportamiento / Reglas & Implementación (clases principales) \\
     \hline
     Recordatorios de clase y generales & Canales: \textsf{schedule\_channel} (alta prioridad) y \textsf{default}. Anticipación por defecto: 10 minutos (ajustable). Programación en segundo plano con WorkManager. Estilos enriquecidos (imagen, botones, corazón opcional). & \texttt{TecnoTimeApp} (creación de canales); \texttt{NotificationServiceImpl} (mostrar/programar); \texttt{NotifyWorker} (entrega diferida); \texttt{NotificationStyler} (estilos y acciones); \texttt{ShowNotificationUseCase}, \texttt{ScheduleNotificationUseCase}. \\
     IA "Simón" (opt-in, on-device) & Descarga opcional del modelo; evaluación de estado (up-to-date/needs-download); uso con reserva temporal (lease); generación de cierres para check-in y recordatorios cuando está activa. Sin telemetría obligatoria. & \texttt{ModelInitializationService} (estado); \texttt{ModelDownloader} (DownloadManager, metadatos y política Wi‑Fi/datos vía \texttt{UserSettingsRepository}); \texttt{AiModelUsageManager} (uso del modelo); \texttt{AiManagementViewModel} (UI y toggles); \texttt{NotificationActionReceiver} (acciones en notificación y generación de respuestas). \\
     Red y almacenamiento & Selección de red (Wi‑Fi/datos) para descarga; validación de espacio libre; reintentos controlados. IA desactivada por defecto (\textsf{enableAi=false}); opt-out disponible (borrado del modelo). & \texttt{AiManagementViewModel} (flujos de descarga/actualización/borrado); \texttt{UserSettingsRepository} (preferencias). \\
     \hline
   \end{tabularx}
\end{table}

\begin{figure}[H]
  \centering
  \includegraphics[width=1\linewidth]{images/diagrams/cap4_f7_secuencia_ia_recordatorios_part1.png}
  \caption[Secuencia (1/2): activación y programación del recordatorio]{Secuencia (1/2): activación y programación del recordatorio. \\ \small{Fuente: Elaboración propia.}}
  \label{fig:f7_notifications_ai_part1}
\end{figure}

\begin{figure}[H]
  \centering
  \includegraphics[width=1\linewidth]{images/diagrams/cap4_f7_secuencia_ia_recordatorios_part2.png}
  \caption[Secuencia (2/2): reprogramación y respuestas en notificación (Simón/Like)]{Secuencia (2/2): reprogramación y respuestas en notificación (Simón/Like). \\ \small{Fuente: Elaboración propia.}}
  \label{fig:f7_notifications_ai_part2}
\end{figure}

\begin{table}[H]
   \centering\small
   \caption[Criterios de entrada y salida (F7)]{Criterios de entrada y salida (F7). \\ \small{Fuente: Elaboración propia.}}\label{tab:f7_entry_exit}
   \begin{tabularx}{\linewidth}{@{}p{3.0cm}>{\RaggedRight\arraybackslash}X@{}}
     \hline
     Criterio & Descripción \\
     \hline
     Entrada & Permisos de notificación habilitados; IA desactivada salvo activación explícita; canales creados en inicialización. \\
     Salida / DoD & Notificaciones puntuales y configurables; estilos consistentes; IA descargable y reversible (opt-out), integrada a recordatorios y mensajes motivacionales sin afectar el funcionamiento base. \\
     \hline
   \end{tabularx}
\end{table}

% ============================
\section{F8: Export (imagen, PDF, Excel)}

\subsection{Objetivo}
Facilitar la portabilidad del horario en formatos estándar y su compartición segura.

\subsection{Qué se hizo}
Se implementaron generadores de imagen, PDF y Excel; y la copia JSON interoperable, con validaciones y share sheet del sistema.

Para mantener consistencia con F6–F7, el Cuadro \ref{tab:f8_export} estructura formatos, reglas de exportación y clases implicadas. Se usa la hoja de compartir del sistema y FileProvider para exponer archivos.

\subsection{Criterios de entrada/salida (Entry/Exit/DoD)}
\begin{table}[H]
   \centering\small
   \caption[Resumen de exportación (F8)]{Resumen de exportación (F8). \\ \small{Fuente: Elaboración propia.}}\label{tab:f8_export}
   \begin{tabularx}{\linewidth}{@{}p{3.0cm}>{\RaggedRight\arraybackslash}X>{\RaggedRight\arraybackslash}X@{}}
     \hline
     Formato & Reglas / Flujo & Implementación (clases principales) \\
     \hline
     Imagen (PNG) & Confirmación previa; bloqueo si no hay materias inscritas; generación de imagen semanal; comparte vía share sheet con MIME \textsf{image/png}. & \texttt{SettingsSendScheduleScreen} (UI/confirmación/share); \texttt{SettingsSendScheduleViewModel} (orquestación); \texttt{GenerateWeeklyScheduleImageUseCase}. \\
     PDF & Confirmación previa; bloqueo sin materias; generación de PDF semanal; comparte con MIME \textsf{application/pdf}. & \texttt{SettingsSendScheduleScreen}; \texttt{SettingsSendScheduleViewModel}; \texttt{GenerateWeeklySchedulePdfUseCase}. \\
     Excel & Confirmación previa; bloqueo sin materias; generación de hoja con tabla; comparte con MIME \textsf{application/vnd.ms-excel}. & \texttt{SettingsSendScheduleScreen}; \texttt{SettingsSendScheduleViewModel}; \texttt{GenerateScheduleExcelUseCase}. \\
     Copia (JSON) & Selección opcional de materias a incluir; confirmación; comparte con MIME \textsf{application/json}. Detalle del contrato en F5 (interoperabilidad). & \texttt{SettingsSendScheduleScreen} (diálogo de selección); \texttt{SettingsSendScheduleViewModel}; \texttt{GenerateShareableScheduleJsonUseCase} (ver F5). \\
     \hline
   \end{tabularx}
\end{table}

\begin{table}[H]
  \centering\small
  \caption[Aspectos técnicos de compartición]{Aspectos técnicos de compartición. \\ \small{Fuente: Elaboración propia.}}\label{tab:f8_tecnico}
  \begin{tabularx}{\linewidth}{@{}p{3.2cm}>{\RaggedRight\arraybackslash}X@{}}
    \hline
    Elemento & Descripción \\
    \hline
    Permisos & En Android < 13, se solicita almacenamiento para escritura si aplica; en versiones recientes se comparte vía URI con permisos de lectura temporales. \\
    Share sheet & Se utiliza \textsf{Intent.ACTION\_SEND} con tipo MIME acorde al archivo y bandera \textsf{FLAG\_GRANT\_READ\_URI\_PERMISSION}. \\
    Exposición de archivos & \textsf{FileProvider} otorga URI seguras para archivos generados. \\
    Validaciones previas & Si no existen materias inscritas, no se permite exportar; se informa al usuario. \\
    \hline
  \end{tabularx}
\end{table}

\begin{figure}[H]
  \centering
  \includegraphics[width=1\linewidth]{images/diagrams/cap4_f8_export_flow_part1.png}
  \caption[Flujo (1/2): validación, selección de formato y preparación]{Flujo (1/2): validación, selección de formato y preparación. \\ \small{Fuente: Elaboración propia.}}
  \label{fig:f8_export_flow_part1}
\end{figure}

\begin{figure}[H]
  \centering
  \includegraphics[width=0.45\linewidth]{images/diagrams/cap4_f8_export_flow_part2.png}
  \caption[Flujo (2/2): confirmación, guardado y compartición]{Flujo (2/2): confirmación, guardado y compartición (FileProvider + share sheet). \\ \small{Fuente: Elaboración propia.}}
  \label{fig:f8_export_flow_part2}
\end{figure}

\begin{table}[H]
  \centering\small
  \caption[Criterios de entrada y salida (F8)]{Criterios de entrada y salida (F8). \\ \small{Fuente: Elaboración propia.}}\label{tab:f8_entry_exit}
  \begin{tabularx}{\linewidth}{@{}p{2.8cm}>{\RaggedRight\arraybackslash}X@{}}
    \hline
    Criterio & Descripción \\
    \hline
    Entry & Materias inscritas presentes; permisos concedidos cuando corresponda; espacio de almacenamiento suficiente. \\
    Exit / DoD & Archivos contienen materia, grupo, día, hora, aula y docente; MIME correcto; compartición exitosa; bloqueo adecuado ante horario vacío. \\
    \hline
  \end{tabularx}
\end{table}

% ============================
\section{F9: Pruebas y validación (QA)}

\subsection{Objetivo}
Verificar funcionalidad crítica end-to-end y asegurar calidad en unidades, integración y UI.

\subsection{Qué se hizo}
Se definió una pirámide de pruebas y se implementaron casos críticos E2E; se validaron parsers, repositorios, generador y flujos de export/import y notificaciones.

\subsection{Criterios de entrada/salida (Entry/Exit/DoD)}
\begin{itemize}
  \item Entrada: módulos críticos identificados, datos semilla y fixtures disponibles, entorno de pruebas configurado.
  \item Salida/DoD: cobertura \(\geq 75\%\) en módulos núcleo; flujos E2E críticos en verde; issues bloqueantes cerrados.
\end{itemize}

\subsection{Pirámide de pruebas}
Se aplicó el enfoque unit/integration/UI:E2E, alineado al código real:
  \begin{itemize}
    \item Unitarias: utilitarios de tiempo y solapes (\texttt{Utils}), parsers PDF y mapeos (\textsf{data/remote/pdf/*}, \textsf{data/mapper/*}), heurísticas y strategy del generador (\textsf{domain/service/*}).
    \item Integración: repositorios + Room (DAO/Entities) con casos de uso (ingesta/sync, generación, export/import): \textsf{domain/repository/*}, \textsf{data/repository/*}, \textsf{domain/usecase/*}.
    \item UI/E2E: flujos críticos desde onboarding hasta generar, aplicar y exportar/importar, incluyendo widget y recordatorios.
  \end{itemize}
  (Rutas completas en el Anexo \ref{ann:trazabilidad-tecnotime}).

Cobertura objetivo. $\geq 75\%$ en módulos core (generator, parsing, sync, interop) y 100\% de flujos E2E críticos (listados abajo).

\subsection{Casos críticos de validación (E2E y funcionales)}
\begin{itemize}
  \item TC-ONB-001: Onboarding completo. Entradas/Pasos: Abrir la aplicación, luego ir a Welcome, ingresar el nombre, ajustar el formato 24h, fines de semana y tema, sincronizar carreras, activar IA (opcional), continuar a Home. Resultado esperado: Preferencias persistidas; carreras visibles; IA permanece desactivada si no se activó. Evidencia: \texttt{WelcomeScreen}, \texttt{WelcomeViewModel}; captura <TBD>.

  \item TC-ADD-002: Agregar materia con cambio de grupo. Entradas/Pasos: Seleccionar nivel, materia, grupo A (preview), confirmar, re-editar y cambiar a grupo B, confirmar. Resultado esperado: Selección final en grupo B; horarios asociados actualizados. Evidencia: \texttt{EditGroupScreen}; captura <TBD>.

  \item TC-END-003: Finalizar materia (aprobado/abandonar). Entradas/Pasos: En Home, abrir card de materia, finalizar, elegir "Aprobado" y confirmar; repetir con otra en "Abandonar". Resultado esperado: Estado actualizado; materia aprobada no ofertada en generador; abandonar libera cupo. Evidencia: \texttt{EditGroupViewModel}; captura <TBD>.

  \item TC-GEN-004: Generar sin choques (por defecto). Entradas/Pasos: En Generar: elegir 6 materias, \code{acceptConflicts=false}. Resultado esperado: 1-N horarios sin solapes; mensaje claro si no hay solución. Evidencia: \texttt{GenerateSchedulesUseCaseImpl}, \texttt{ScheduleGenerator}.

  \item TC-GEN-005: Priorizar profesores ON + choques OFF. Entradas/Pasos: Activar \code{prioritizeFavoriteTeachers=true}, \code{acceptConflicts=false}. Resultado esperado: Excluir combinaciones con choque; priorizar grupos con docentes favoritos. Evidencia: \texttt{ScheduleStrategyFactory}, \code{prioritizeGroupsByFavoriteTeacher(...)}.

  \item TC-GEN-006: Priorizar profesores ON + choques ON. Entradas/Pasos: Activar ambos: favoritos y aceptar choques. Resultado esperado: Permitir combinaciones con choque; ranking favorece favoritos; se etiqueta conflicto. Evidencia: \texttt{AcceptConflictsStrategy}, \texttt{PrioritizeTeachersStrategy}.

  \item TC-GEN-007: Obligatorias > límite. Entradas/Pasos: Marcar 8 obligatorias con \code{totalSubjectsCount=6}. Resultado esperado: Excedente pasa a opcional; se optimiza combinación final. Evidencia: \texttt{GenerateSchedulesUseCaseImpl} — \code{promoteLowestLevels(...)}.

  \item TC-EXP-008: Export imagen/PDF/Excel. Entradas/Pasos: Ejecutar export en Settings, confirmar. Resultado esperado: Archivo válido; share sheet abre con MIME correcto; bloqueo si horario vacío. Evidencia: \texttt{SettingsSendScheduleScreen}; \texttt{GenerateWeeklyScheduleImageUseCase}, \texttt{GenerateWeeklySchedulePdfUseCase}, \texttt{GenerateScheduleExcelUseCase}.

  \item TC-JSON-009: Enviar copia JSON (parcial). Entradas/Pasos: Seleccionar parcialmente las materias, exportar JSON, compartir por WhatsApp. Resultado esperado: JSON con claves esperadas; sólo materias seleccionadas; app de destino recibe. Evidencia: \texttt{GenerateShareableScheduleJsonUseCase}, \texttt{ShareableScheduleJsonGenerator}.

  \item TC-JSON-010: Importar JSON (no duplicar / otra carrera). Entradas/Pasos: Abrir JSON en TecnoTime, validar si ya inscritas, si carrera distinta, sincronizar. Resultado esperado: No duplicar inscritas/aprobadas; sugerir sync si carrera difiere; merge sin duplicados. Evidencia: \texttt{LoadShareableScheduleUseCase}, \texttt{ImportSharedScheduleUseCase}.

  \item TC-NOT-011: Recordatorios (10 min; canales). Entradas/Pasos: Habilitar clases, establecer 10 min, programar notificación. Resultado esperado: Notificación en canal correcto y anticipación solicitada; respeta desactivación. Evidencia: \texttt{TecnoTimeApp}, \texttt{NotificationServiceImpl}, \texttt{NotifyWorker}.

  \item TC-OFF-012: Offline con fallback PDF. Entradas/Pasos: Desconectar internet, generar, sincronizar por PDF local oficial. Resultado esperado: App opera con cache; si se provee PDF actual, re-ingesta manual exitosa. Evidencia: \texttt{SyncCareerFromLocalPdfUseCase}.
\end{itemize}
(Rutas completas en el Anexo \ref{ann:trazabilidad-tecnotime}).

% ============================
\section{F10: Seguridad y privacidad}

\subsection{Objetivo}
Minimizar datos y permisos y mantener el procesamiento local con controles explícitos.

\subsection{Qué se hizo}
Se restringieron datos/persistencia a lo esencial, se limitaron permisos a casos de uso y se habilitó IA en el dispositivo con activación voluntaria, sin telemetría obligatoria.

\subsection{Criterios de entrada/salida (Entry/Exit/DoD)}
\begin{itemize}
  \item Entrada: definición de datos mínimos y permisos requeridos por caso de uso.
  \item Salida/DoD: aplicación operativa con permisos mínimos; datos personales procesados localmente; IA 100\% opt-in/opt-out; sin telemetría obligatoria.
\end{itemize}
Datos mínimos (local). Sólo nombre de usuario y preferencias necesarias (24h, fines de semana, tema, notificaciones, IA). No se recolecta telemetría obligatoria ni se envían datos personales a terceros. Persistencia: Room/SharedPreferences locales (ver \Class{UserSettings}).

Permisos mínimos. \textsf{POST\_NOTIFICATIONS} (Android 13+) para recordatorios; permisos de almacenamiento/compartir sólo cuando export/import lo requiere (vía \textsf{FileProvider} e \textsf{Intent.ACTION\_SEND}). Declaración en el AndroidManifest.

IA on-device (opt-in). Descarga/uso del modelo bajo control del usuario: activar/desactivar IA, elegir medio de descarga (Wi‑Fi/datos), borrar el modelo. Sin telemetría obligatoria; procesamiento local.

% ============================
\section{F11: Release (parámetros y tamaños)}

\subsection{Objetivo}
Empaquetar y distribuir con tamaños razonables y control de calidad.

\subsection{Qué se hizo}
Se ajustaron parámetros de build, se aplicó shrink/minify en release y se estableció checklist de publicación y métricas.
Parámetros de build. Observados en \code{app/build.gradle.kts}: \code{minSdk=24}, \code{compileSdk=35}, \code{targetSdk=35}. Tipos de build: \code{release} con minify+shrink activos; \code{debug} sin shrink.

Tamaños objetivo. APK $<30$\,MB (sin IA). Modelo IA por defecto \(\sim\)229\,MB (cuantizado Q4\_K\_M; ver \textsf{ModelInitializationService.getDefaultModel()} y \textsf{sizeBytes}). Descarga opcional, fuera del APK.

Distribución. Canales internos/privados para QA y distribución (sin Play Store si así se define). Firma y notas de versión por cada entrega.

Disponibilidad y acceso. La aplicación estará publicada en Google Play el 25 de noviembre de 2025. Hasta esa fecha, se ofrece mediante el programa de pruebas de Google Play; el acceso requiere unirse como tester al grupo correspondiente. Enlace de acceso al programa de pruebas: \url{https://play.google.com/apps/testing/com.fragmind.tecnotime}.
\section{Resumen operacional por fases (F1–F11)}
\small % opcional: compactar un poco la tipografía

% -- Estilo de bullets homogéneo, simple y sin rutas largas --
% Convención: módulos en negrita, sin paths: Remoto, Local, UseCases, UI, Notif, IA, Widget
% Glosario rápido: DoD = Definition of Done (criterio de salida)

\subsection{F1: Ingesta de PDF a BD}
\begin{itemize}
  \item Objetivo (Goal): Transformar horarios oficiales (PDF) a datos locales sin saturar la fuente.
  \item Qué se hizo: Descubrimiento de URLs, descarga con rate limit, parsing tolerante, normalización, persistencia y marcas de freshness.
  \item Entradas a salidas: Del Remoto (PDF/Scraper) al Local (Room: materias, grupos, bloques).
  \item DoD: BD consistente por carrera/nivel/materia/grupo, last\_sync actualizado, reintentos acotados.
  \item Riesgo clave: Cambios de formato/ubicación del PDF, mitigado con parser flexible y fallback por PDF local.
\end{itemize}

\subsection{F2: Modelo de dominio}
\begin{itemize}
  \item Objetivo: Entidades, relaciones y mapeos consistentes para soporte de generador y export.
  \item Qué se hizo: Definición de entidades (Subject, Group, GroupSchedule, Teacher, Classroom, Settings), índices/keys y mappers.
  \item Entradas a salidas: Del Modelo (dominio) al Local (entities) y a UseCases (servicios).
  \item DoD: Unicidad en códigos; integridad referencial; mapeos ida/vuelta probados.
  \item Riesgo: Desalineación dominio/BD $\rightarrow$ mitigado con pruebas de carga y revisión cruzada.
\end{itemize}

\subsection{F3: Generador de horarios}
\begin{itemize}
  \item Objetivo: Producir Top-$N$ horarios sin choques por defecto, optimizando huecos y favoritos.
  \item Qué se hizo: Backtracking con MRV/LCV, poda temprana, composite scoring (choques/gaps/favoritos), aplicación parcial/completa.
  \item Entradas a salidas: De UseCases (Generate/Apply) a UI (configuración y resultados).
  \item DoD: $N$ candidatos válidos; flags influyen en ranking; aplicar parcial/total actualiza selección.
  \item Riesgo: Explosión combinatoria, mitigada con poda y priorización por nivel.
\end{itemize}

\subsection{F4: Prioridad sin conexión (offline-first)}
\begin{itemize}
  \item Objetivo: Operación robusta con conectividad intermitente y protección de la fuente.
  \item Qué se hizo: Cache local, freshness gate previo al generador, backoff+circuit breaker, ventana de no-sync, fallback por PDF local.
  \item Entradas a salidas: Del Local (cache) y UseCases (AutoSync) a datos frescos cuando hay red.
  \item DoD: App funcional sin internet; bloqueo de generación durante sync; reintentos espaciados.
  \item Riesgo: Saturar origen, mitigado con CB y ventanas controladas.
\end{itemize}

\subsection{F5 — Interoperabilidad (JSON + WhatsApp)}
\begin{itemize}
  \item Objetivo: Compartir e importar horarios de forma confiable entre usuarios.
  \item Qué se hizo: Export JSON total/parcial con versión; import con validación (duplicados, carrera distinta) y merge idempotente.
  \item Entradas a salidas: De UseCases (Share/Import) a UI (enviar/abrir) y Local (fusión).
  \item DoD: JSON válido; import no duplica; sugiere sincronizar carrera distinta antes de fusionar.
  \item Riesgo: Deriva de esquema: mitigado con versionado y validaciones.
\end{itemize}

\subsection{F6 — UX aplicada}
\begin{itemize}
  \item Objetivo: Reducir fricción: onboarding, ajustes, agregar/editar, vista semanal, widget.
  \item Qué se hizo: Flujos guiados, preview de grupo, selección por nivel/materia, valores por defecto sensatos, widget de acceso rápido.
  \item Entradas$\rightarrow$Salidas: UI (welcome, settings, home, add/edit, weekly) + Widget.
  \item DoD: Menos pasos para armar horario; preview consistente; widget sincronizado.
  \item Riesgo: Sobrecarga de opciones: mitigada con progressive disclosure.
\end{itemize}

\subsection{F7 — Notificaciones e IA (opt-in)}
\begin{itemize}
  \item Objetivo: Recordatorios puntuales y mensajería opcional con IA on-device.
  \item Qué se hizo: Canales separados (clases/mensajes), WorkManager, plantillas; gestión del modelo IA (descarga/borrado; Wi-Fi/datos; sin telemetría).
  \item Entradas a salidas: De Notif (canales+workers) e IA (local) a avisos y textos opcionales.
  \item DoD: Notifs antes de clase (10 min por defecto) y configurables; IA desactivada por defecto, activable y reversible.
  \item Riesgo: Tamaño de modelo/almacenamiento: mitigado con descarga bajo demanda y opción de eliminar.
\end{itemize}

\subsection{F8 — Exportaciones}
\begin{itemize}
  \item Objetivo: Compartir horario como imagen, PDF o Excel.
  \item Qué se hizo: Confirmación previa, bloqueo si horario vacío, generación por formato y share sheet.
  \item Entradas a salidas: De UseCases (Image/PDF/Excel) a UI (enviar/guardar).
  \item DoD: Archivos contienen materia, grupo, día, hora, aula, docente; MIME correcto; bloqueo sin materias.
  \item Riesgo: Permisos/espacio: mitigado con FileProvider y validaciones previas.
\end{itemize}

\subsection{F9 — Pruebas y validación}
\begin{itemize}
  \item Objetivo: Asegurar calidad funcional y de flujo extremo a extremo.
  \item Qué se hizo: Unitarias (tiempo/solapes/parsers/estrategias), integración (repos+usecases), UI/E2E (onboarding hasta generar, aplicar y exportar/importar).
  \item Entradas a salidas: De Tests (TCs) a reporte de verificación y issues cerrados.
  \item DoD: $\geq 75\%$ cobertura en módulos núcleo; E2E críticos verdes.
  \item Riesgo: Falsos positivos E2E, mitigado con datos semilla estables.
\end{itemize}

\subsection{F10 — Seguridad y privacidad}
\begin{itemize}
  \item Objetivo: Datos y permisos mínimos; control explícito del usuario.
  \item Qué se hizo: Sólo nombre y preferencias locales; permisos just-in-time; IA local sin envío de datos.
  \item Entradas$\rightarrow$Salidas: Manifest + AppInit (canales) + Settings (opt-in IA).
  \item DoD: App funciona con permisos mínimos; IA 100\% opt-in/opt-out.
  \item Riesgo: Uso involuntario de IA: mitigado con toggles y borrado del modelo.
\end{itemize}

\subsection{F11 — Release y KPIs}
\begin{itemize}
  \item Objetivo: Empaque optimizado y medición de adopción/uso.
  \item Qué se hizo: Shrink/optimize; firma y notas; umbrales (APK < 30 MB sin IA; IA ~200–230 MB opcional); KPIs (TTS, gaps/día, on-time, adopción JSON).
  \item Entradas a salidas: Del Build (release) a Artefactos y Métricas base.
  \item DoD: Compilación release estable; tamaños dentro de umbrales; KPIs medibles definidos.
  \item Riesgo: Aumento de tamaño por libs: mitigado con revisión de dependencias y splits.
\end{itemize}


\noindent \textsf{minSdk=24}, \textsf{targetSdk=35}, tamaño del APK: 23\,MB y tamaño del modelo IA: 229\,MB en la build de referencia.

% ----------------------------------------------------------------------
\section{Procedimientos operativos (antes “UC”, ahora “PR”)}
% (Hint EN: keep “UC” for SRS; in Chapter IV use “PR = Procedure”)
\subsection{PR-001 — Ingestión y normalización}
Actor: Sistema. Objetivo: BD local vigente.
Entry: conectividad o PDF local.
Exit/DoD: last\_sync actualizado, TC-004/005 verdes.

\subsection{PR-002 — Generar horarios}
Actor: Estudiante. Objetivo: obtener $N$ horarios.
Entry: selección de materias/flags.
Exit/DoD: Top-$N$ sin solapes (por defecto), aplicación parcial/completa.

\subsection{PR-003 — Configurar notificaciones}
Actor: Estudiante. Objetivo: avisos antes de clase (10\,min por defecto).

\subsection{PR-004 — Exportar/Importar}
Actor: Estudiante. Objetivo: compartir/clonar; validación de versión y carrera.

% ----------------------------------------------------------------------
\section{Métricas (KPIs) y fórmulas}
\label{sec:kpis}
\noindent TTS (Time-To-Schedule). 
$TTS = t_{\text{aplicado}} - t_{\text{click\_generar}}$ (objetivo $\leq 3$\,min para $N{=}10$).

\noindent Gaps/día.
$\displaystyle \text{Gaps}_{\text{prom}}=\frac{\sum_{d}{\text{minutos vacíos}(d)}}{\#\text{días activos}}$ (objetivo: tendencia a la baja).

\noindent On-time reminders (medible).
$\displaystyle \frac{\#\text{notifs abiertas }\geq X\text{ min antes}}{\#\text{notifs totales}}\times 100\%$ (objetivo $\geq 95\%$).

\noindent Adopción JSON (medible).
\% de export/import exitoso (objetivo $\uparrow$ en semanas pico); replicable por pruebas E2E (F5/F8).

% ----------------------------------------------------------------------
\section{Riesgos y mitigaciones}
\begin{table}[H]
  \centering
  \caption{Riesgos (P=probabilidad, I=impacto) y acciones. \\ \small{Fuente: Elaboración propia.}}
  \label{tab:cap4_riesgos}
  \small
  \begin{tabular}{|p{6.6cm}|c|c|p{7.8cm}|}
    \hline
    Riesgo & P & I & Mitigación \\
    \hline
    Cambios en layout/ubicación de PDF & M & A & Parsers tolerantes; revisión manual; import local de PDF \\
    \hline
    Caída prolongada/saturación del sitio & M & A & Backoff + CB; cache local; diferir sync (ventanas); fallback PDF \\
    \hline
    Conectividad intermitente & A & M & Offline-first; política de reintentos; freshness gate antes de generar \\
    \hline
    Dispositivos de baja gama & M & M & UI ligera; optimización de consultas; IA opcional (opt-in) \\
    \hline
    Datos desactualizados & M & M & Freshness check previo a generación; auto-sync condicionado \\
    \hline
  \end{tabular}
\end{table}

% ----------------------------------------------------------------------
\section{Checklist de release y CI/CD}
\begin{itemize}
  \item Pruebas verdes; cobertura $\geq 75\%$ en módulos críticos; flujos E2E críticos completos.
  \item Lint/análisis estático sin issues críticos.
  \item Verificar canales de notificación creados y configurados correctamente.
  \item APK $<30$\,MB (sin IA); firma válida; notas de versión; rollout por etapas.
\end{itemize}

% ----------------------------------------------------------------------
\section{Lecciones aprendidas y trabajo futuro}
Lecciones. Poda por hard constraints + MRV/LCV reduce el espacio de búsqueda sin perder calidad; JSON compartible habilita colaboración (WhatsApp-first); prioridad sin conexión reduce ansiedad por caídas; widget aumenta consultas diarias.

Próximos pasos. Filtros por aula/docente, eventos externos (tareas/exámenes), telemetría opt-in para refinar ranking, benchmark por dispositivo, más pruebas de UI en export/import.

% ----------------------------------------------------------------------
\section{SÍNTESIS}
Se ejecutó un enfoque iterativo-incremental guiado por evidencia de encuestas y entrevistas, con una arquitectura limpia priorizando la operación con prioridad sin conexión. El generador, basado en backtracking y heurísticas, respeta restricciones duras y preferencias blandas. La interoperabilidad JSON facilita la colaboración, mientras que los recordatorios con IA on-device opcional completan las funcionalidades. La cadena de requisitos a artefactos y pruebas asegura alineación entre problema, implementación y verificación.

% Nota editorial: rutas simplificadas
% Se omiten URLs de repositorio y rutas completas; se priorizan nombres de clases para legibilidad.

Notas de cierre. Los porcentajes de la encuesta fueron incorporados en el capítulo correspondiente; se fijó \(T_{fresh}{=}48\,h\) y \(W_{nosync}{=}07{:}00\text{–}12{:}30\) (ajustable). La simbología de flechas se mantuvo consistente en tablas y diagramas.

% ===================== CAPÍTULO IV (drop-in replacement) =====================
\chapter{METODOLOGÍA Y DESARROLLO DEL PROYECTO}
\label{chap:metodologia}

% --- Nota de cumplimiento (no teoría, solo aplicación) ---
% (Hint EN: "Applied chapter, no theory dump.")
\noindent Enfoque. Este capítulo documenta la aplicación de la metodología en TecnoTime: fases ejecutadas, actividades, decisiones, artefactos (con rutas de repositorio) y evidencias verificables. Se excluyen SRS y Área de Aplicación (capítulos previos).

% ---- Opcional: compact helpers (si tus normas lo permiten) ----
\setlength{\emergencystretch}{2em}
% \renewcommand{\arraystretch}{1.08}
% \setlist[itemize]{itemsep=0.25em, topsep=0.25em}
% \setlist[enumerate]{itemsep=0.35em, topsep=0.25em}
% \captionsetup{font=small,skip=6pt}

\providecommand{\Class}[1]{\textsf{#1}}
\providecommand{\ClassRef}[2]{\textsf{#2}}
\newcommand{\cfile}[1]{\texttt{#1}}
\providecommand{\cclass}[2]{\textsf{#2}}

% ----------------------------------------------------------------------
\section{Metodología adoptada}
Se aplicó un enfoque iterativo-incremental con Kanban ajustado a un desarrollador único. Cada fase cerró un incremento funcional corto con criterios de aceptación y revisión en tablero digital.

\begin{table}[H]
  \centering\small
  \caption{Aplicación de la metodología por fases. Fuente: Elaboración propia.}
  \begin{tabularx}{\linewidth}{@{}p{1.2cm}p{3.9cm}>{\RaggedRight\arraybackslash}X@{}}
    \hline
    Fase & Propósito principal & Entregables y evidencias \\
    \hline
    F1 & Incorporar horarios oficiales al sistema & Canal de descarga controlada de PDF, normalización a esquema Room y verificación de consistencia por carrera. \\
    F2 & Definir el modelo de dominio & Entidades, reglas y agregados consolidados en la capa \texttt{domain}, acompañados de diagramas de referencia. \\
    F3 & Generar horarios viables & Servicio de generación configurable, heurísticas MRV/LCV y pantallas para configurar y aplicar resultados. \\
    F4 & Priorizar el uso sin conexión & Flujos de sincronización diferida, políticas de frescura y mecanismos de recuperación manual. \\
    F5 & Intercambiar horarios & Contrato JSON interoperable, flujo de importación/exportación y asistentes para compartir mediante WhatsApp. \\
    F6 & Refinar la experiencia de usuario & Onboarding, navegación principal, widget de acceso rápido y guías contextuales. \\
    F7 & Automatizar recordatorios e IA & Notificaciones programadas, plantillas reutilizables y asistente ``Simón'' con activación voluntaria. \\
    F8 & Exportar resultados & Generadores de reporte en PDF, imagen y Excel con validación de formato. \\
    F9 & Realizar aseguramiento de calidad & Batería de pruebas instrumentadas, casos manuales priorizados y checklist de aceptación. \\
    F10 & Endurecer seguridad y privacidad & Gestión de permisos, cifrado de datos sensibles y revisión de políticas de respaldo. \\
    F11 & Preparar liberación y métricas & Empaquetado en release, tablero de indicadores (TTS, adopción) y plan de seguimiento post despliegue. \\
    \hline
  \end{tabularx}
\end{table}

\subsection{Contexto empírico}
El desarrollo partió de un prototipo de ingesta de PDF y normalización local. A partir de esa base se ejecutaron iteraciones sobre selección de materias, generación de horarios y validaciones con usuarios piloto. Las fases posteriores incorporaron intercambio JSON para envío y recepción de horarios, mejoras de experiencia de usuario (pantallas guía, configuración y widget), recordatorios y exportaciones en formatos comunes. Al cierre se integró la asistencia ``Simón'' como componente optativo y se documentaron controles de seguridad.

\subsection{Factores de decisión}
La encuesta aplicada a 125 estudiantes orientó las prioridades del tablero:
\begin{itemize}
  \item Preferencia por Android (aproximadamente 71,7\%): orientación exclusiva a componentes nativos y widget principal.
  \item Coordinación por WhatsApp (aproximadamente 80,3\%): intercambio de horarios mediante JSON y acciones directas de compartir.
  \item Sincronización automática (aproximadamente 88,4\%): verificación de frescura y actualización discreta sin intervención del usuario.
  \item Uso sin conexión (aproximadamente 77,7\%): arquitectura offline-first con caché local prioritaria.
  \item Recordatorios (aproximadamente 56,2\%): programación de notificaciones con plantillas y canales específicos.
  \item Compartir horario (aproximadamente 51,2\%): exportación a PDF, imagen y Excel con parámetros de periodo.
\end{itemize}

\subsection{Decisiones técnicas derivadas}
Los factores anteriores guiaron la priorización de las fases F4--F8 y condicionaron los artefactos liberados por incremento. Se reforzó el almacenamiento local y la sincronización diferida, se consolidó el contrato JSON con asistentes de importación y envío, se diseñaron recordatorios parametrizables y se establecieron exportaciones multiplataforma. Los artefactos y evidencias se encuentran documentados en \cite{tecnotime_repo}.

\subsection{Arquitectura de la solución}
La solución se apoya en una aplicación Android autosuficiente: la interacción sucede en la interfaz offline (Jetpack Compose), el motor de dominio coordina reglas y recordatorios, la ingesta procesa PDFs y la persistencia local resguarda horarios, configuraciones y políticas de frescura. El asistente IA opera sobre el mismo contexto y la única integración externa es el portal UMSS que publica los PDFs oficiales. La Figura \ref{fig:arquitectura_tecnotime} resume esta arquitectura offline-first.

\begin{figure}[H]
  \centering
  \includegraphics[width=1\linewidth]{images/diagrams/cap4_arq_general.png}
  \caption{Arquitectura general empleada en TecnoTime. Fuente: Elaboración propia.}
  \label{fig:arquitectura_tecnotime}
\end{figure}

% ----------------------------------------------------------------------
% (Reubicado al final del capítulo para que 4.2 inicie con F1)
% \section{Mapa de fases (F1-F11)} — contenido movido más adelante

% ============================
\section{F1: Ingestión y normalización (de PDF a BD)}
\textbf{Objetivo.} Transformar horarios oficiales (PDF) a un modelo estructurado local, controlando la carga a la fuente y asegurando la frescura de los datos.

\subsubsection*{4.2.1 Verificación de caché y decisión}
Al abrir el generador se valida conectividad y vigencia de datos. Si el contenido local está fresco, se usa cache; de lo contrario, se inicia sincronización automática.
\begin{itemize}
  \item Control en la interfaz: `GenerateScheduleViewModel` ejecuta \textsf{checkConnectivityAndSync()} y bloquea acciones durante sincronización.
  \item Conectividad: `NetworkConnectivityChecker` discrimina entre online/offline.
  \item Frescura: `AutoSyncUseCase` compara \textsf{updatedDate} remoto con \textsf{lastSyncTime} local (por carrera) y decide actualizar.
  \item Preferencias: `SyncPreferences` limita chequeos con una ventana (p. ej., 6 horas) para evitar consultas excesivas.
\end{itemize}

\begin{figure}[H]
  \centering
  \includegraphics[width=0.35\linewidth]{images/diagrams/cap4_c4_1_context.png}
  \caption{C4 Nivel 1 — Contexto del sistema para F1. Fuente: Elaboración propia.}
  \label{fig:c4_context_f1}
\end{figure}

\subsubsection*{4.2.2 Ingesta, parseo y persistencia con frescura}
Cuando se requiere actualización, el flujo descarga, extrae y parsea el PDF oficial; luego normaliza y persiste en Room. Se aplican controles de retroceso exponencial (backoff) y cortacircuito (circuit breaker) para proteger la fuente.
\begin{itemize}
  \item Descubrimiento/descarga: `ScheduleScraper` ubica recursos y `PdfDownloader` obtiene el PDF.
  \item Extracción/parseo: `PdfExtractor` obtiene texto; `PdfParser` aplica expresiones regulares tolerantes para cabeceras, grupos, días y horas, consolidando líneas y docentes.
  \item Normalización/persistencia: `ImportCareerSchedulesUseCase` proyecta entidades y persiste con índices únicos compuestos (Room) y políticas \textsf{ON CONFLICT}.
  \item Frescura: tras actualizar, se registra \textsf{lastSyncTime} por carrera y se guarda el instante del chequeo en `SyncPreferences`.
  \item Robustez: `ExponentialBackoff` y `CircuitBreaker` controlan reintentos y abren/cierran el circuito ante fallas repetidas.
  \item Respaldo: si no hay red, se permite importación manual desde PDF local y se opera totalmente en caché.
\end{itemize}

\begin{figure}[H]
  \centering
  \includegraphics[width=0.82\linewidth]{images/diagrams/cap4_c4_2_container.png}
  \caption{C4 Nivel 2 — Contenedores involucrados en F1. Fuente: Elaboración propia.}
  \label{fig:c4_container_f1}
\end{figure}

\begin{figure}[H]
  \centering
  \includegraphics[width=1\linewidth]{images/diagrams/cap4_c4_3_component_f1.png}
  \caption{C4 Nivel 3 — Componentes de ingesta y normalización (F1). Fuente: Elaboración propia.}
  \label{fig:c4_component_f1}
\end{figure}

\begin{figure}[H]
  \centering
  \includegraphics[width=1\linewidth]{images/diagrams/cap4_c4_4_code_f1.png}
  \caption{C4 Nivel 4 — Clases y casos de uso relevantes (F1). Fuente: Elaboración propia.}
  \label{fig:c4_code_f1}
\end{figure}

\paragraph{Criterios de entrada (Entry).}
Conectividad disponible o PDF local; sin sincronización en curso.

\paragraph{Criterios de salida (Exit/DoD).}
Base de datos consistente por carrera, nivel, materia, grupo y bloques; \textsf{lastSyncTime} actualizado; pruebas de ingestión y reparseo exitosas.

\paragraph{Evidencias (código).}
\begin{itemize}
  \item \Class{AutoSyncUseCase} (invocación y decisión) y \Class{SyncPreferences} (ventanas y sellos de tiempo).
  \item \Class{ScheduleScraper} (descubrimiento) y \Class{PdfDownloader}/\Class{PdfExtractor}/\Class{PdfParser} (flujo de procesamiento de PDF).
  \item \Class{ImportCareerSchedulesUseCase} (normalización/persistencia) junto con \Class{ExponentialBackoff} y \Class{CircuitBreaker} (resiliencia).
\end{itemize}

\paragraph{KPIs.}
Cache hit-rate semanal $\geq 80\%$; reintentos acotados; tiempo de ingesta por PDF dentro del objetivo operacional.

% ============================
\section{F2: Modelo de dominio (entidades, reglas, mapeo)}
\textbf{Entidades (derivadas del código).}
\begin{itemize}
  \item Materia: \cclass{domain/model/Subject.kt}{Subject} (\textsf{code}, \textsf{name}, \textsf{isElective}, \textsf{isApproved}, \textsf{isActive}).
  \item Grupo: \Class{Group} (\textsf{id}, \textsf{subjectCode}, \textsf{groupId}, \textsf{levelId}, \textsf{groupName}, \textsf{type}, \textsf{modality}, \textsf{isActive}).
  \item Bloque/Slot: \Class{GroupSchedule} (\textsf{day}, \textsf{startTime}, \textsf{endTime}, aula/docente opcionales).
  \item Carrera/Nivel: \Class{Career}, \Class{Level}; agregados: \Class{CareerWithLevels}, \Class{LevelWithSubjects}.
  \item Docente/Aula: \Class{Teacher}, \Class{Classroom}.
  \item Selección del usuario: \Class{SelectedSubject}, \Class{SelectedSubjectWithGroup}.
  \item Preferencias/Usuario: \Class{UserSettings} (24h, fines de semana, recordatorios, IA opt-in, etc.).
\end{itemize}

\textbf{Reglas e invariantes.}
\begin{itemize}
  \item Unicidad de \textsf{Subject.code}: \Class{SubjectEntity} define índice único sobre \textsf{code}.
  \item Unicidad compuesta \textsf{(subjectCode, groupId)}: \Class{GroupEntity} con índice compuesto único; FK a \Class{LevelEntity} y \Class{SubjectEntity}.
  \item Temporal: \Class{GroupScheduleEntity} asegura \textsf{startTime}–\textsf{endTime} para cada \textsf{day} y \textsf{groupId}; validación operativa start<end al normalizar/guardar.
  \item Integridad referencial: FKs en \Class{GroupScheduleEntity} hacia \Class{GroupEntity}, \Class{TeacherEntity} y \Class{ClassroomEntity} (nulables cuando no hay asignación).
\end{itemize}

\textbf{Mapeos del ingreso al dominio y al almacenamiento.}
\begin{itemize}
  \item Ingreso PDF/JSON: parsers PDF y DTO de intercambio en \cclass{domain/model/ShareableScheduleDto.kt}{ShareableScheduleDto}.
  \item Dominio: mappers (p. ej., \Class{SubjectMapper}, \Class{GroupMapper}, \Class{GroupScheduleMapper}, \Class{UserSettingsMapper}) transforman entre entidades y dominio.
  \item Interoperabilidad JSON: \cclass{domain/usecase/GenerateShareableScheduleJsonUseCase.kt}{GenerateShareableScheduleJsonUseCase}, \Class{LoadShareableScheduleUseCase}, \Class{ImportSharedScheduleUseCase}.
\end{itemize}

\textbf{Evidencias.}
\begin{itemize}
  \item Unicidad: \Class{SubjectEntity}: índice unique(\textsf{code}); \Class{GroupEntity}: índice compuesto unique(\textsf{subject\_code}, \textsf{group\_id}).
  \item Modelos: \Class{Subject}, \Class{Group}, \Class{GroupSchedule}, \Class{UserSettings}.
  \item Mapeo: \Class{GroupMapper} incluye \textsf{levelId}, \textsf{subjectCode} y \textsf{groupId} como claves de relación.
\end{itemize}

Como apoyo visual, la Figura \ref{fig:f2_modelo_dominio_er} ilustra el diagrama entidad–relación del modelo de dominio, destacando entidades operativas, claves e integridad referencial utilizadas en la implementación.

\begin{figure}[H]
  \centering
  \includegraphics[width=1\linewidth]{images/diagram_er.png}
  \caption{Modelo de dominio: diagrama entidad–relación (ER). Fuente: elaboración propia.}
  \label{fig:f2_modelo_dominio_er}
\end{figure}

% ============================
\section{F3: Generador (restricciones, heurísticas, UI)}
\textbf{Problema práctico.} Seleccionar un grupo por materia evitando choques por defecto y optimizando los huecos y la afinidad con docentes favoritos.

\textbf{Parámetros reales.}
\begin{itemize}
  \item Límite de materias por horario: \(\leq\)20 y número de horarios generados: hasta 25. Parametrización en \cclass{domain/model/ScheduleGenerationParams.kt}{ScheduleGenerationParams} (\textsf{totalSubjectsCount}, \textsf{numberOfSchedules}).
  \item Flags: \textsf{prioritizeFavoriteTeachers}, \textsf{minimizeGaps}, \textsf{acceptConflicts} (off por defecto), y \textsf{fixedGroups} para bloquear selecciones parciales.
\end{itemize}

\textbf{Estrategia operativa.}
\begin{itemize}
\item Selección incremental con poda temprana y un orden implícito MRV/LCV; se promueven los niveles bajos mediante \textsf{promoteLowestLevels(...)} y se asegura la progresión de series con \textsf{enforceSeriesProgression(...)} en \texttt{GenerateSchedulesUseCaseImpl}.
\item Priorización de grupos con docentes favoritos mediante \textsf{prioritizeGroupsByFavoriteTeacher(...)}: los favoritos sin conflicto prevalecen y, si \textsf{acceptConflicts} está habilitado, se consideran también los que generan conflicto.
\item Puntaje compuesto: \texttt{ScheduleStrategyFactory} integra \texttt{AcceptConflictsStrategy} (peso 1.0), \texttt{MinimizeGapsStrategy} (peso 1.0 cuando está activa) y \texttt{PrioritizeTeachersStrategy} (peso 2.0 cuando está activa) dentro de \texttt{CompositeStrategy}. Estas ponderaciones priorizan evitar choques (crítico), reducir huecos (confort) y favorecer docentes favoritos (preferencia fuerte cuando se habilita).
\item Top-N: \texttt{ScheduleGenerator} conserva las mejores combinaciones y penaliza la ausencia de alternativas preferidas.
\end{itemize}
(Rutas completas en el Anexo \ref{ann:trazabilidad-tecnotime}).

\textbf{Interacción de usuario (pantallas reales).}
\begin{itemize}
  \item Configuración y selección: \texttt{GenerateScheduleConfigScreen}, \texttt{GenerateScheduleViewModel} (selección de materias, opciones y lanzamiento de generación).
  \item Resultados y aplicación: \texttt{GenerateScheduleResultsScreen}, \texttt{SchedulePreview} (aplicar parcial: solo marcadas vía \textsf{fixedGroups}; o completo: todo el horario propuesto).
\end{itemize}
(Rutas completas en el Anexo \ref{ann:trazabilidad-tecnotime}).

\textbf{Diagramas.} A continuación se presentan dos vistas complementarias del proceso de generación: (i) secuencia extremo a extremo desde la UI hasta el evaluador y (ii) decisiones de poda, scoring y relajación.

\begin{figure}[H]
  \centering
  \includegraphics[width=1\linewidth]{images/diagrams/cap4_f3_secuencia_generacion.png}
  \caption{Secuencia de generación desde la configuración hasta los horarios propuestos. Fuente: Elaboración propia.}
  \label{fig:f3_secuencia_generacion}
\end{figure}

\noindent Para facilitar la lectura, la Figura~\ref{fig:f3_decisiones_generador} se organiza en dos paneles: (I) conformación del conjunto candidato (obligatorias, preferencia por docentes y orden de exploración) y (II) evaluación de combinaciones (penalización de huecos, tratamiento de choques y relajación cuando no hay opciones válidas).

\begin{figure}[H]
  \centering
  \begin{minipage}[t]{0.48\linewidth}
    \centering
    \includegraphics[width=\linewidth]{images/diagrams/cap4_f3_decisiones_generador_part1.png}
  \end{minipage}\hfill
  \begin{minipage}[t]{0.48\linewidth}
    \centering
    \includegraphics[width=\linewidth]{images/diagrams/cap4_f3_decisiones_generador_part2.png}
  \end{minipage}
  \caption{Decisiones del generador: (I) selección de obligatorias y preferencias de docentes; (II) evaluación de gaps, gestión de choques y relajación. Fuente: Elaboración propia.}
  \label{fig:f3_decisiones_generador}
\end{figure}

Entry/Exit/DoD.
Entry: datos frescos (si hay red) y selección de materias/grupos fijos.
Exit: \(N\) horarios sin solapes por defecto; aplicación parcial/completa funcional; casos TC-(TBD) ejecutados.

\textbf{KPIs.}
TTS (Time-To-Schedule) \(\leq\) 3 min para \(N{=}10\); regeneración parcial \(\leq\) 1 s por cambio.

% \begin{figure}[H]
%   \centering
%   \includegraphics[width=0.82\linewidth]{TODO}
%   \caption{Decisiones y mantenimiento de Top-N (vista simplificada).}
%   \label{fig:f3_generador}
% \end{figure}

\noindent\textbf{Criterios y pesos verificados.} La evaluación combina tres estrategias y dos penalizaciones fijas. Resumen breve:
\begin{itemize}
  \item Choques: \texttt{AcceptConflictsStrategy} (peso 1.0). Siempre activo. Si \textsf{acceptConflicts=false}, un choque descarta la combinación (infinito); si \textsf{true}, aplica penalización baja por conflicto.
  \item Huecos (gaps): \texttt{MinimizeGapsStrategy} (peso 1.0). Activo solo con \textsf{minimizeGaps=true}. Puntúa la suma de recesos entre bloques; menor es mejor.
  \item Docentes favoritos: \texttt{PrioritizeTeachersStrategy} (peso 2.0). Activo con \textsf{prioritizeFavoriteTeachers=true}. Reduce el puntaje cuando hay bloques con docente favorito.
  \item Penalización por preferidos ausentes: +1000.0 por cada código preferido no satisfecho.
  \item Penalización por grupo no favorito: +1500.0 si, existiendo grupos favoritos para una materia, el elegido no es favorito.
\end{itemize}
(Rutas completas en el Anexo \ref{ann:trazabilidad-tecnotime}).

% ============================
\section{F4: Prioridad sin conexión (frescura, políticas, fallback)}
\textbf{Frescura.} Antes de invocar el generador se valida conectividad y frescura de datos:
\begin{itemize}
  \item \Class{GenerateScheduleViewModel}: ejecuta \textsf{checkConnectivityAndSync()} al cargar; usa \Class{NetworkConnectivityChecker} y, con red disponible, dispara \Class{AutoSyncUseCase}.
  \item \Class{AutoSyncUseCase}: compara \textsf{Career.updatedDate} remoto con \textsf{lastSyncTime} local y actualiza materias mediante \Class{RefreshSubjectsForCareerUseCase}. Actualiza \textsf{SyncPreferences.setLastSyncCheck()}.
  \item \Class{SyncPreferences}: define intervalo de verificación de 6\,h y evita consultas excesivas.
  \item Durante la sincronización, \Class{GenerateScheduleViewModel} marca \textsf{isSyncing=true} y bloquea generate/regenerate para mantener consistencia.
\end{itemize}

\textbf{Políticas.} \(T_{fresh}=48\,h\): si los datos locales superan ese umbral, se prioriza sincronizar antes de generar. \(W_{nosync}=07{:}00\text{–}12{:}30\) (horario académico): se difieren sincronizaciones automáticas en ese rango para no interrumpir uso intensivo; el ajuste es configurable. Reintentos controlados con:
\begin{itemize}
  \item \Class{ExponentialBackoff}: $t_0{=}2$\,s, $t_{max}{=}1$\,h, multiplicador 2.0 con jitter.
  \item \Class{CircuitBreaker}: umbral de 5 fallos y recuperación en 10\,min; integra \Class{AutoSyncUseCase} para bloquear temporalmente el origen tras errores repetidos.
\end{itemize}

\textbf{Fallback.} Sin conectividad se opera desde caché local (Room). Si el servicio cae pero existe PDF oficial actualizado, se habilita reingesta manual vía \Class{SyncCareerFromLocalPdfUseCase}.

\textbf{Evidencias (código).}
\begin{itemize}
  \item Gate del generador: \Class{GenerateScheduleViewModel} (bandera \textsf{isSyncing}).
  \item Chequeo y sincronización: \Class{NetworkConnectivityChecker}, \Class{AutoSyncUseCase}, \Class{SyncPreferences}.
  \item Fallback manual: \Class{SyncCareerFromLocalPdfUseCase}.
\end{itemize}

\begin{figure}[H]
  \centering
  \includegraphics[width=1\linewidth]{images/diagrams/cap4_f1a_cache_decision.png}
  \caption{Gate de caché y frescura. Fuente: Elaboración propia.}
  \label{fig:f4_offline}
\end{figure}

\begin{figure}[H]
  \centering
  \includegraphics[width=1\linewidth]{images/diagrams/offline_first_sync.png}
  \caption{Sincronización con backoff y circuit breaker. Fuente: Elaboración propia.}
  \label{fig:f4_offline_sync}
\end{figure}

\begin{figure}[H]
  \centering
  \includegraphics[width=1\linewidth]{images/diagrams/cap4_f4_pdf_fallback.png}
  \caption{Fallback por PDF local. Fuente: Elaboración propia.}
  \label{fig:f4_offline_pdf}
\end{figure}

% ============================
\section{F5: Interoperabilidad (JSON + WhatsApp)}
\textbf{Exportar.} Acciones: imagen (PNG), PDF, Excel y copia JSON del horario. Se solicita confirmación previa; si no hay materias inscritas, se bloquea la exportación con mensaje. Implementación de UI en \texttt{SettingsSendScheduleScreen} y orquestación en \texttt{SettingsSendScheduleViewModel}. Generación: \texttt{GenerateWeeklyScheduleImageUseCase}, \texttt{GenerateWeeklySchedulePdfUseCase}, \texttt{GenerateScheduleExcelUseCase}, \texttt{GenerateShareableScheduleJsonUseCase}. Compartición mediante \textsf{FileProvider} + \textsf{Intent.ACTION\_SEND}. (Rutas completas en el Anexo \ref{ann:trazabilidad-tecnotime}).

\textbf{Enviar copia (JSON).} Permite selección parcial en \texttt{SettingsSendScheduleViewModel}. El generador JSON \texttt{ShareableScheduleJsonGenerator} produce claves: \textsf{meta}, \textsf{careers}, \textsf{subjects}, \textsf{groups}, \textsf{teachers}, \textsf{classrooms}, \textsf{entries}. (Rutas completas en el Anexo \ref{ann:trazabilidad-tecnotime}).

\textbf{Importar (desde WhatsApp).} Al abrir el JSON con TecnoTime: \texttt{LoadShareableScheduleUseCase} valida y carga; \texttt{ImportSharedScheduleUseCase} realiza merge sin duplicados y maneja:
\begin{itemize}
  \item Materias ya inscritas/aprobadas (no duplicar; cambio de grupo si aplica).
  \item Diferencia de carrera (\textsf{autoSyncCareers} para sincronizar primero si corresponde).
  \item Persistencia idempotente en \texttt{SelectedSubjectRepository} y \texttt{GroupScheduleRepository}.
\end{itemize}
(Rutas completas en el Anexo \ref{ann:trazabilidad-tecnotime}).

\begin{itemize}[nosep,leftmargin=*]
  \item \texttt{loadEnrolledSubjects()}
  \item \texttt{toggleSubjectSelection()}
  \item \texttt{selectAllSubjects()}
  \item \texttt{clearSelection()}
\end{itemize}

% Rutas omitidas: se referencian por nombre de clase.

\begin{figure}[H]
  \centering
  \includegraphics[width=1\linewidth]{images/diagrams/cap4_f5_share_import_flow_part1.png}
  \caption{Exportación del horario: confirmar, generar artefactos y compartir vía WhatsApp. Fuente: Elaboración propia.}
  \label{fig:f5_flow_export}
\end{figure}

\begin{figure}[H]
  \centering
  \includegraphics[width=1\linewidth]{images/diagrams/cap4_f5_share_import_flow_part2.png}
  \caption{Importación del horario: validar el JSON recibido y ejecutar el merge sin duplicados. Fuente: Elaboración propia.}
  \label{fig:f5_flow_import}
\end{figure}

\paragraph{Contrato JSON.} El intercambio define un bloque de metadatos y un cuerpo auto-contenido:
\begin{itemize}
  \item \textsf{meta}: \{\textsf{version}, \textsf{min\_supported}\}. Se usa control semántico (\textsf{1.0.0}). La aplicación acepta archivos cuya \textsf{min\_supported} $\leq$ versión actual del contrato.
  \item \textsf{careers}: carreras implicadas (código y nombre) para habilitar/sincronizar si corresponde.
  \item \textsf{subjects}, \textsf{groups}, \textsf{teachers}, \textsf{classrooms}: catálogos mínimos para enriquecer la vista previa y resolver identificadores.
  \item \textsf{entries}: lista de EnrolledSchedulePreview.
\end{itemize}
Campos mínimos obligatorios por entry para garantizar la importación: \textsf{subject.code}, \textsf{group.groupId} y \textsf{schedule} (día y franja horaria). El resto de atributos enriquecen la experiencia (docente, aula, color/emoji, notificaciones) y se consumen cuando están disponibles.

\paragraph{Entrada.} Materias inscritas disponibles; permisos de almacenamiento/compartir concedidos si aplica; JSON válido al importar.
\paragraph{Salida/DoD.} Exportaciones válidas (PNG/PDF/Excel/JSON) compartibles; importación ejecuta validaciones y merge sin duplicados; bloqueos correctos cuando no hay materias.

% ============================
\section{F6: UX aplicada: onboarding, navegación, edición y widget}
Para mejorar la legibilidad y evitar secuencias con flechas, la Tabla \ref{tab:f6_resumen_ux} resume los flujos, comportamientos esperados e implementación de componentes de interfaz asociados a F6.

\begin{table}[H]
   \centering\small
   \caption{Resumen de UX aplicada (F6). Fuente: Elaboración propia.}\label{tab:f6_resumen_ux}
   \begin{tabularx}{\linewidth}{@{}p{3.0cm}>{\RaggedRight\arraybackslash}X>{\RaggedRight\arraybackslash}X@{}}
     \hline
     Flujo & Pasos / Comportamiento & Implementación (clases principales) \\
     \hline
     Onboarding & Bienvenida; nombre de usuario; ajustes iniciales (formato 24h, fines de semana/días sin clase, tema claro/oscuro); sincronización de carreras; progreso por nivel (A, B, C, ... con porcentaje); activación opcional de Simón (IA). & \texttt{WelcomeScreen}, \texttt{WelcomeViewModel}; pantallas de ajustes y componentes comunes según corresponda. \\
     Agregar materia & Secuencia guiada: seleccionar nivel; seleccionar materia; elegir grupo con preview de días/horas; confirmar. Asignación de color/emoji automática con opción de elección manual. & \texttt{AddSubjectFlowScreen}, \texttt{AddSubjectViewModel}; \texttt{SelectLevelScreen}, \texttt{SelectSubjectScreen}, \texttt{SelectGroupScreen}; \texttt{ColorPickerScreen}, \texttt{EmojiPickerScreen}. \\
     Editar grupo & Desde el card del home: cambiar de grupo (con vista de horarios de destino) o finalizar materia con estado aprobado/abandonado (diálogo de confirmación). Edición de color/emoji disponible. & \texttt{EditGroupScreen}, \texttt{EditGroupSelectScreen}, \texttt{EditGroupViewModel}; \texttt{EditColorScreen}, \texttt{EditEmojiScreen}. \\
     Navegación & Deslizamiento entre días; vista semanal con pestañas animadas; selector de fecha en encabezado para ir a una fecha específica. & \texttt{HomeScreen} (paginación), \texttt{WeekDayTabsAnimated}. \\
     Widget & Muestra el día actual; permite desplazamiento entre días; tarjetas con información esencial (materia, grupo, hora, aula, docente). & \texttt{ScheduleAppWidgetProvider}, \texttt{ScheduleRemoteViewsService}; utilidades en \texttt{WidgetDateUtils}, preferencias en \texttt{WidgetPrefs}. \\
     \hline
   \end{tabularx}
\end{table}

\begin{table}[H]
   \centering\small
   \caption{Criterios de entrada y salida (F6). Fuente: Elaboración propia.}\label{tab:f6_entry_exit}
   \begin{tabularx}{\linewidth}{@{}p{2.8cm}>{\RaggedRight\arraybackslash}X@{}}
     \hline
     Criterio & Descripción \\
     \hline
     Entrada & Interfaz base disponible con datos mínimos para interacción guiada. \\
     Salida / DoD & Reducción de pasos en tareas críticas; preview consistente; widget operativo y sincronizado. \\
     \hline
   \end{tabularx}
\end{table}

% Rutas omitidas para simplificar; se emplean únicamente nombres de clases.

% ============================
\section{F7: Notificaciones e IA (Simón, opt-in)}
Para estandarizar la presentación y facilitar la lectura, la Tabla \ref{tab:f7_notif_ai} resume los flujos, el comportamiento esperado y los componentes clave. Se mantiene la activación voluntaria (opt-in) de IA y los canales separados para notificaciones.

\begin{table}[H]
   \centering\small
   \caption{Resumen de notificaciones e IA (F7). Fuente: Elaboración propia.}\label{tab:f7_notif_ai}
   \begin{tabularx}{\linewidth}{@{}p{3.5cm}>{\RaggedRight\arraybackslash}X>{\RaggedRight\arraybackslash}X@{}}
     \hline
     Flujo & Comportamiento / Reglas & Implementación (clases principales) \\
     \hline
     Recordatorios de clase y generales & Canales: \textsf{schedule\_channel} (alta prioridad) y \textsf{default}. Anticipación por defecto: 10 minutos (ajustable). Programación en segundo plano con WorkManager. Estilos enriquecidos (imagen, botones, corazón opcional). & \texttt{TecnoTimeApp} (creación de canales); \texttt{NotificationServiceImpl} (mostrar/programar); \texttt{NotifyWorker} (entrega diferida); \texttt{NotificationStyler} (estilos y acciones); \texttt{ShowNotificationUseCase}, \texttt{ScheduleNotificationUseCase}. \\
     IA "Simón" (opt-in, on-device) & Descarga opcional del modelo; evaluación de estado (up-to-date/needs-download); uso con reserva temporal (lease); generación de cierres para check-in y recordatorios cuando está activa. Sin telemetría obligatoria. & \texttt{ModelInitializationService} (estado); \texttt{ModelDownloader} (DownloadManager, metadatos y política Wi‑Fi/datos vía \texttt{UserSettingsRepository}); \texttt{AiModelUsageManager} (uso del modelo); \texttt{AiManagementViewModel} (UI y toggles); \texttt{NotificationActionReceiver} (acciones en notificación y generación de respuestas). \\
     Red y almacenamiento & Selección de red (Wi‑Fi/datos) para descarga; validación de espacio libre; reintentos controlados. IA desactivada por defecto (\textsf{enableAi=false}); opt-out disponible (borrado del modelo). & \texttt{AiManagementViewModel} (flujos de descarga/actualización/borrado); \texttt{UserSettingsRepository} (preferencias). \\
     \hline
   \end{tabularx}
\end{table}

\begin{figure}[H]
  \centering
  \includegraphics[width=1\linewidth]{images/diagrams/cap4_f7_secuencia_ia_recordatorios_part1.png}
  \caption{Secuencia (1/2): activación y programación del recordatorio. Fuente: Elaboración propia.}
  \label{fig:f7_notifications_ai_part1}
\end{figure}

\begin{figure}[H]
  \centering
  \includegraphics[width=1\linewidth]{images/diagrams/cap4_f7_secuencia_ia_recordatorios_part2.png}
  \caption{Secuencia (2/2): reprogramación y respuestas en notificación (Simón/Like). Fuente: Elaboración propia.}
  \label{fig:f7_notifications_ai_part2}
\end{figure}

\begin{table}[H]
   \centering\small
   \caption{Criterios de entrada y salida (F7). Fuente: Elaboración propia.}\label{tab:f7_entry_exit}
   \begin{tabularx}{\linewidth}{@{}p{3.0cm}>{\RaggedRight\arraybackslash}X@{}}
     \hline
     Criterio & Descripción \\
     \hline
     Entrada & Permisos de notificación habilitados; IA desactivada salvo activación explícita; canales creados en inicialización. \\
     Salida / DoD & Notificaciones puntuales y configurables; estilos consistentes; IA descargable y reversible (opt-out), integrada a recordatorios y mensajes motivacionales sin afectar el funcionamiento base. \\
     \hline
   \end{tabularx}
\end{table}

% ============================
\section{F8: Export (imagen, PDF, Excel)}
Para mantener consistencia con F6–F7, la Tabla \ref{tab:f8_export} estructura formatos, reglas de exportación y clases implicadas. Se usa la hoja de compartir del sistema y FileProvider para exponer archivos.

\begin{table}[H]
   \centering\small
   \caption{Resumen de exportación (F8). Fuente: Elaboración propia.}\label{tab:f8_export}
   \begin{tabularx}{\linewidth}{@{}p{3.0cm}>{\RaggedRight\arraybackslash}X>{\RaggedRight\arraybackslash}X@{}}
     \hline
     Formato & Reglas / Flujo & Implementación (clases principales) \\
     \hline
     Imagen (PNG) & Confirmación previa; bloqueo si no hay materias inscritas; generación de imagen semanal; comparte vía share sheet con MIME \textsf{image/png}. & \texttt{SettingsSendScheduleScreen} (UI/confirmación/share); \texttt{SettingsSendScheduleViewModel} (orquestación); \texttt{GenerateWeeklyScheduleImageUseCase}. \\
     PDF & Confirmación previa; bloqueo sin materias; generación de PDF semanal; comparte con MIME \textsf{application/pdf}. & \texttt{SettingsSendScheduleScreen}; \texttt{SettingsSendScheduleViewModel}; \texttt{GenerateWeeklySchedulePdfUseCase}. \\
     Excel & Confirmación previa; bloqueo sin materias; generación de hoja con tabla; comparte con MIME \textsf{application/vnd.ms-excel}. & \texttt{SettingsSendScheduleScreen}; \texttt{SettingsSendScheduleViewModel}; \texttt{GenerateScheduleExcelUseCase}. \\
     Copia (JSON) & Selección opcional de materias a incluir; confirmación; comparte con MIME \textsf{application/json}. Detalle del contrato en F5 (interoperabilidad). & \texttt{SettingsSendScheduleScreen} (diálogo de selección); \texttt{SettingsSendScheduleViewModel}; \texttt{GenerateShareableScheduleJsonUseCase} (ver F5). \\
     \hline
   \end{tabularx}
\end{table}

\begin{table}[H]
  \centering\small
  \caption{Aspectos técnicos de compartición. Fuente: Elaboración propia.}\label{tab:f8_tecnico}
  \begin{tabularx}{\linewidth}{@{}p{3.2cm}>{\RaggedRight\arraybackslash}X@{}}
    \hline
    Elemento & Descripción \\
    \hline
    Permisos & En Android < 13, se solicita almacenamiento para escritura si aplica; en versiones recientes se comparte vía URI con permisos de lectura temporales. \\
    Share sheet & Se utiliza \textsf{Intent.ACTION\_SEND} con tipo MIME acorde al archivo, y bandera \textsf{FLAG\_GRANT\_READ\_URI\_PERMISSION}. \\
    Exposición de archivos & \textsf{FileProvider} otorga URI seguras para archivos generados. \\
    Validaciones previas & Si no existen materias inscritas, no se permite exportar; se informa al usuario. \\
    \hline
  \end{tabularx}
\end{table}

\begin{figure}[H]
  \centering
  \includegraphics[width=1\linewidth]{images/diagrams/cap4_f8_export_flow_part1.png}
  \caption{Flujo (1/2): validación, selección de formato y preparación. Fuente: Elaboración propia.}
  \label{fig:f8_export_flow_part1}
\end{figure}

\begin{figure}[H]
  \centering
  \includegraphics[width=0.45\linewidth]{images/diagrams/cap4_f8_export_flow_part2.png}
  \caption{Flujo (2/2): confirmación, guardado y compartición (FileProvider + share sheet). Fuente: Elaboración propia.}
  \label{fig:f8_export_flow_part2}
\end{figure}

\begin{table}[H]
  \centering\small
  \caption{Criterios de entrada y salida (F8). Fuente: Elaboración propia.}\label{tab:f8_entry_exit}
  \begin{tabularx}{\linewidth}{@{}p{2.8cm}>{\RaggedRight\arraybackslash}X@{}}
    \hline
    Criterio & Descripción \\
    \hline
    Entry & Materias inscritas presentes; permisos concedidos cuando corresponda; espacio de almacenamiento suficiente. \\
    Exit / DoD & Archivos contienen materia, grupo, día, hora, aula y docente; MIME correcto; compartición exitosa; bloqueo adecuado ante horario vacío. \\
    \hline
  \end{tabularx}
\end{table}

% ============================
\section{F9: Pruebas y validación (QA)}
\textbf{Pirámide de pruebas.} Se aplicó el enfoque unit/integration/UI:E2E, alineado al código real:
\begin{itemize}
  \item Unitarias: utilitarios de tiempo y solapes (\texttt{Utils}), parsers PDF y mapeos (\textsf{data/remote/pdf/*}, \textsf{data/mapper/*}), heurísticas y strategy del generador (\textsf{domain/service/*}).
  \item Integración: repositorios + Room (DAO/Entities) con casos de uso (ingesta/sync, generación, export/import): \textsf{domain/repository/*}, \textsf{data/repository/*}, \textsf{domain/usecase/*}.
  \item UI/E2E: flujos críticos desde onboarding hasta generar, aplicar y exportar/importar, incluyendo widget y recordatorios.
\end{itemize}
(Rutas completas en el Anexo \ref{ann:trazabilidad-tecnotime}).

\begin{itemize}[nosep,leftmargin=*]
  \item \texttt{loadEnrolledSubjects()}
  \item \texttt{toggleSubjectSelection()}
  \item \texttt{selectAllSubjects()}
  \item \texttt{clearSelection()}
\end{itemize}

\paragraph{Cobertura objetivo.}
$\geq 75\%$ en módulos core (generator, parsing, sync, interop) y 100\% de flujos E2E críticos (listados abajo).

\paragraph{Casos críticos de validación (E2E y funcionales).}
\begin{description}[leftmargin=0cm,style=nextline]
  \item[TC-ONB-001: Onboarding completo]
  Entradas/Pasos: Abrir la aplicación, luego ir a Welcome, ingresar el nombre, ajustar el formato 24h, fines de semana y tema, sincronizar carreras, activar IA (opcional), continuar a Home.\\
  Resultado esperado: Preferencias persistidas; carreras visibles; IA permanece desactivada si no se activó.\\
  Evidencia: \texttt{WelcomeScreen}, \texttt{WelcomeViewModel}; captura <TBD>.

  \item[TC-ADD-002: Agregar materia con cambio de grupo]
  Entradas/Pasos: Seleccionar nivel, materia, grupo A (preview), confirmar, re-editar y cambiar a grupo B, confirmar.\\
  Resultado esperado: Selección final en grupo B; horarios asociados actualizados.\\
  Evidencia: \texttt{EditGroupScreen}; captura <TBD>.

  \item[TC-END-003: Finalizar materia (aprobado/abandonar)]
  Entradas/Pasos: En Home, abrir card de materia, finalizar, elegir "Aprobado" y confirmar; repetir con otra en "Abandonar".\\
  Resultado esperado: Estado actualizado; materia aprobada no ofertada en generador; abandonar libera cupo.\\
  Evidencia: \texttt{EditGroupViewModel}; captura <TBD>.

  \item[TC-GEN-004: Generar sin choques (por defecto)]
  Entradas/Pasos: En Generar: elegir 6 materias, \code{acceptConflicts=false}.\\
  Resultado esperado: 1-N horarios sin solapes; mensaje claro si no hay solución.\\
  Evidencia: \texttt{GenerateSchedulesUseCaseImpl}, \texttt{ScheduleGenerator}.

  \item[TC-GEN-005: Priorizar profesores ON + choques OFF]
  Entradas/Pasos: Activar \code{prioritizeFavoriteTeachers=true}, \code{acceptConflicts=false}.\\
  Resultado esperado: Excluir combinaciones con choque; priorizar grupos con docentes favoritos.\\
  Evidencia: \texttt{ScheduleStrategyFactory}, \code{prioritizeGroupsByFavoriteTeacher(...)}.

  \item[TC-GEN-006: Priorizar profesores ON + choques ON]
  Entradas/Pasos: Activar ambos: favoritos y aceptar choques.\\
  Resultado esperado: Permitir combinaciones con choque; ranking favorece favoritos; se etiqueta conflicto.\\
  Evidencia: \texttt{AcceptConflictsStrategy}, \texttt{PrioritizeTeachersStrategy}.

  \item[TC-GEN-007: Obligatorias > límite]
  Entradas/Pasos: Marcar 8 obligatorias con \code{totalSubjectsCount=6}.\\
  Resultado esperado: Excedente pasa a opcional; se optimiza combinación final.\\
  Evidencia: \texttt{GenerateSchedulesUseCaseImpl} — \code{promoteLowestLevels(...)}.

  \item[TC-EXP-008: Export imagen/PDF/Excel]
  Entradas/Pasos: Ejecutar export en Settings, confirmar.\\
  Resultado esperado: Archivo válido; share sheet abre con MIME correcto; bloqueo si horario vacío.\\
  Evidencia: \texttt{SettingsSendScheduleScreen}; \texttt{GenerateWeeklyScheduleImageUseCase}, \texttt{GenerateWeeklySchedulePdfUseCase}, \texttt{GenerateScheduleExcelUseCase}.

  \item[TC-JSON-009: Enviar copia JSON (parcial)]
  Entradas/Pasos: Seleccionar parcialmente las materias, exportar JSON, compartir por WhatsApp.\\
  Resultado esperado: JSON con claves esperadas; sólo materias seleccionadas; app de destino recibe.\\
  Evidencia: \texttt{GenerateShareableScheduleJsonUseCase}, \texttt{ShareableScheduleJsonGenerator}.

  \item[TC-JSON-010: Importar JSON (no duplicar / otra carrera)]
  Entradas/Pasos: Abrir JSON en TecnoTime, validar si ya inscritas, si carrera distinta, sincronizar.\\
  Resultado esperado: No duplicar inscritas/aprobadas; sugerir sync si carrera difiere; merge sin duplicados.\\
  Evidencia: \texttt{LoadShareableScheduleUseCase}, \texttt{ImportSharedScheduleUseCase}.

  \item[TC-NOT-011: Recordatorios (10 min; canales)]
  Entradas/Pasos: Habilitar clases, establecer 10 min, programar notificación.\\
  Resultado esperado: Notificación en canal correcto y anticipación solicitada; respeta desactivación.\\
  Evidencia: \texttt{TecnoTimeApp}, \texttt{NotificationServiceImpl}, \texttt{NotifyWorker}.

  \item[TC-OFF-012: Offline con fallback PDF]
  Entradas/Pasos: Desconectar internet, generar, sincronizar por PDF local oficial.\\
  Resultado esperado: App opera con cache; si se provee PDF actual, re-ingesta manual exitosa.\\
  Evidencia: \texttt{SyncCareerFromLocalPdfUseCase}.
\end{description}
(Rutas completas en el Anexo \ref{ann:trazabilidad-tecnotime}).

% ============================
\section{F10: Seguridad y privacidad}
Datos mínimos (local). Sólo nombre de usuario y preferencias necesarias (24h, fines de semana, tema, notificaciones, IA). No se recolecta telemetría obligatoria ni se envían datos personales a terceros. Persistencia: Room/SharedPreferences locales (ver \Class{UserSettings}).

Permisos mínimos. \textsf{POST\_NOTIFICATIONS} (Android 13+) para recordatorios; permisos de almacenamiento/compartir sólo cuando export/import lo requiere (vía \textsf{FileProvider} e \textsf{Intent.ACTION\_SEND}). Declaración en el AndroidManifest.

IA on-device (opt-in). Descarga/uso del modelo bajo control del usuario: activar/desactivar IA, elegir medio de descarga (Wi‑Fi/datos), borrar el modelo. Sin telemetría obligatoria; procesamiento local.

% ============================
\section{F11: Release (parámetros y tamaños)}
Parámetros de build. Observados en \code{app/build.gradle.kts}: \code{minSdk=24}, \code{compileSdk=35}, \code{targetSdk=35}. Tipos de build: \code{release} con minify+shrink activos; \code{debug} sin shrink.

Tamaños objetivo. APK $<30$\,MB (sin IA). Modelo IA por defecto \(\sim\)229\,MB (cuantizado Q4\_K\_M; ver \textsf{ModelInitializationService.getDefaultModel()} y \textsf{sizeBytes}). Descarga opcional, fuera del APK.

Distribución. Canales internos/privados para QA y distribución (sin Play Store si así se define). Firma y notas de versión por cada entrega.
\section{Resumen operacional por fases (F1–F11)}
\small % opcional: compactar un poco la tipografía

% -- Estilo de bullets homogéneo, simple y sin rutas largas --
% Convención: módulos en negrita, sin paths: Remoto, Local, UseCases, UI, Notif, IA, Widget
% Glosario rápido: DoD = Definition of Done (criterio de salida)

\subsection*{F1: Ingesta de PDF a BD}
\begin{itemize}
  \item \textbf{Objetivo (Goal):} Transformar horarios oficiales (PDF) a datos locales sin saturar la fuente.
  \item \textbf{Qué se hizo:} Descubrimiento de URLs, descarga con rate limit, parsing tolerante, normalización, persistencia y marcas de freshness.
  \item \textbf{Entradas a salidas:} Del \textbf{Remoto} (PDF/Scraper) al \textbf{Local} (Room: materias, grupos, bloques).
  \item \textbf{DoD:} BD consistente por carrera/nivel/materia/grupo, last\_sync actualizado, reintentos acotados.
  \item \textbf{Riesgo clave:} Cambios de formato/ubicación del PDF, mitigado con parser flexible y fallback por PDF local.
\end{itemize}

\subsection*{F2: Modelo de dominio}
\begin{itemize}
  \item \textbf{Objetivo:} Entidades, relaciones y mapeos consistentes para soporte de generador y export.
  \item \textbf{Qué se hizo:} Definición de entidades (Subject, Group, GroupSchedule, Teacher, Classroom, Settings), índices/keys y mappers.
  \item \textbf{Entradas a salidas:} Del \textbf{Modelo} (dominio) al \textbf{Local} (entities) y a \textbf{UseCases} (servicios).
  \item \textbf{DoD:} Unicidad en códigos; integridad referencial; mapeos ida/vuelta probados.
  \item \textbf{Riesgo:} Desalineación dominio/BD $\rightarrow$ mitigado con pruebas de carga y revisión cruzada.
\end{itemize}

\subsection*{F3: Generador de horarios}
\begin{itemize}
  \item \textbf{Objetivo:} Producir Top-$N$ horarios sin choques por defecto, optimizando huecos y favoritos.
  \item \textbf{Qué se hizo:} Backtracking con MRV/LCV, poda temprana, composite scoring (choques/gaps/favoritos), aplicación parcial/completa.
  \item \textbf{Entradas a salidas:} De \textbf{UseCases} (Generate/Apply) a \textbf{UI} (configuración y resultados).
  \item \textbf{DoD:} $N$ candidatos válidos; flags influyen en ranking; aplicar parcial/total actualiza selección.
  \item \textbf{Riesgo:} Explosión combinatoria, mitigada con poda y priorización por nivel.
\end{itemize}

\subsection*{F4: Prioridad sin conexión (offline-first)}
\begin{itemize}
  \item \textbf{Objetivo:} Operación robusta con conectividad intermitente y protección de la fuente.
  \item \textbf{Qué se hizo:} Cache local, freshness gate previo al generador, backoff+circuit breaker, ventana de no-sync, fallback por PDF local.
  \item \textbf{Entradas a salidas:} Del \textbf{Local} (cache) y \textbf{UseCases} (AutoSync) a datos frescos cuando hay red.
  \item \textbf{DoD:} App funcional sin internet; bloqueo de generación durante sync; reintentos espaciados.
  \item \textbf{Riesgo:} Saturar origen, mitigado con CB y ventanas controladas.
\end{itemize}

\subsection*{F5 — Interoperabilidad (JSON + WhatsApp)}
\begin{itemize}
  \item \textbf{Objetivo:} Compartir e importar horarios de forma confiable entre usuarios.
  \item \textbf{Qué se hizo:} Export JSON total/parcial con versión; import con validación (duplicados, carrera distinta) y merge idempotente.
  \item \textbf{Entradas a salidas:} De \textbf{UseCases} (Share/Import) a \textbf{UI} (enviar/abrir) y \textbf{Local} (fusión).
  \item \textbf{DoD:} JSON válido; import no duplica; sugiere sincronizar carrera distinta antes de fusionar.
  \item \textbf{Riesgo:} Deriva de esquema: mitigado con versionado y validaciones.
\end{itemize}

\subsection*{F6 — UX aplicada}
\begin{itemize}
  \item \textbf{Objetivo:} Reducir fricción: onboarding, ajustes, agregar/editar, vista semanal, widget.
  \item \textbf{Qué se hizo:} Flujos guiados, preview de grupo, selección por nivel/materia, valores por defecto sensatos, widget de acceso rápido.
  \item \textbf{Entradas$\rightarrow$Salidas:} \textbf{UI} (welcome, settings, home, add/edit, weekly) + \textbf{Widget}.
  \item \textbf{DoD:} Menos pasos para armar horario; preview consistente; widget sincronizado.
  \item \textbf{Riesgo:} Sobrecarga de opciones: mitigada con progressive disclosure.
\end{itemize}

\subsection*{F7 — Notificaciones e IA (opt-in)}
\begin{itemize}
  \item \textbf{Objetivo:} Recordatorios puntuales y mensajería opcional con IA on-device.
  \item \textbf{Qué se hizo:} Canales separados (clases/mensajes), WorkManager, plantillas; gestión del modelo IA (descarga/borrado; Wi-Fi/datos; sin telemetría).
  \item \textbf{Entradas a salidas:} De \textbf{Notif} (canales+workers) e \textbf{IA} (local) a avisos y textos opcionales.
  \item \textbf{DoD:} Notifs antes de clase (10 min por defecto) y configurables; IA desactivada por defecto, activable y reversible.
  \item \textbf{Riesgo:} Tamaño de modelo/almacenamiento: mitigado con descarga bajo demanda y opción de eliminar.
\end{itemize}

\subsection*{F8 — Exportaciones}
\begin{itemize}
  \item \textbf{Objetivo:} Compartir horario como imagen, PDF o Excel.
  \item \textbf{Qué se hizo:} Confirmación previa, bloqueo si horario vacío, generación por formato y share sheet.
  \item \textbf{Entradas a salidas:} De \textbf{UseCases} (Image/PDF/Excel) a \textbf{UI} (enviar/guardar).
  \item \textbf{DoD:} Archivos contienen materia, grupo, día, hora, aula, docente; MIME correcto; bloqueo sin materias.
  \item \textbf{Riesgo:} Permisos/espacio: mitigado con FileProvider y validaciones previas.
\end{itemize}

\subsection*{F9 — Pruebas y validación}
\begin{itemize}
  \item \textbf{Objetivo:} Asegurar calidad funcional y de flujo extremo a extremo.
  \item \textbf{Qué se hizo:} Unitarias (tiempo/solapes/parsers/estrategias), integración (repos+usecases), UI/E2E (onboarding hasta generar, aplicar y exportar/importar).
  \item \textbf{Entradas a salidas:} De \textbf{Tests} (TCs) a reporte de verificación y issues cerrados.
  \item \textbf{DoD:} $\geq 75\%$ cobertura en módulos núcleo; E2E críticos verdes.
  \item \textbf{Riesgo:} Falsos positivos E2E, mitigado con datos semilla estables.
\end{itemize}

\subsection*{F10 — Seguridad y privacidad}
\begin{itemize}
  \item \textbf{Objetivo:} Datos y permisos mínimos; control explícito del usuario.
  \item \textbf{Qué se hizo:} Sólo nombre y preferencias locales; permisos just-in-time; IA local sin envío de datos.
  \item \textbf{Entradas$\rightarrow$Salidas:} \textbf{Manifest} + \textbf{AppInit} (canales) + \textbf{Settings} (opt-in IA).
  \item \textbf{DoD:} App funciona con permisos mínimos; IA 100\% opt-in/opt-out.
  \item \textbf{Riesgo:} Uso involuntario de IA: mitigado con toggles y borrado del modelo.
\end{itemize}

\subsection*{F11 — Release y KPIs}
\begin{itemize}
  \item \textbf{Objetivo:} Empaque optimizado y medición de adopción/uso.
  \item \textbf{Qué se hizo:} Shrink/optimize; firma y notas; umbrales (APK < 30 MB sin IA; IA ~200–230 MB opcional); KPIs (TTS, gaps/día, on-time, adopción JSON).
  \item \textbf{Entradas a salidas:} Del \textbf{Build} (release) a \textbf{Artefactos} y \textbf{Métricas} base.
  \item \textbf{DoD:} Compilación release estable; tamaños dentro de umbrales; KPIs medibles definidos.
  \item \textbf{Riesgo:} Aumento de tamaño por libs: mitigado con revisión de dependencias y splits.
\end{itemize}


\noindent \textsf{minSdk=24}, \textsf{targetSdk=35}, tamaño del APK: 23\,MB y tamaño del modelo IA: 229\,MB en la build de referencia.

% ----------------------------------------------------------------------
\section{Procedimientos operativos (antes “UC”, ahora “PR”)}
% (Hint EN: keep “UC” for SRS; in Chapter IV use “PR = Procedure”)
\subsection*{PR-001 — Ingestión y normalización}
Actor: Sistema. Objetivo: BD local vigente.
Entry: conectividad o PDF local.
Exit/DoD: last\_sync actualizado, TC-004/005 verdes.

\subsection*{PR-002 — Generar horarios}
Actor: Estudiante. Objetivo: obtener $N$ horarios.
Entry: selección de materias/flags.
Exit/DoD: Top-$N$ sin solapes (por defecto), aplicación parcial/completa.

\subsection*{PR-003 — Configurar notificaciones}
Actor: Estudiante. Objetivo: avisos antes de clase (10\,min por defecto).

\subsection*{PR-004 — Exportar/Importar}
Actor: Estudiante. Objetivo: compartir/clonar; validación de versión y carrera.

% ----------------------------------------------------------------------
\section{Métricas (KPIs) y fórmulas}
\label{sec:kpis}
\noindent\textbf{TTS (Time-To-Schedule).} 
$TTS = t_{\text{aplicado}} - t_{\text{click\_generar}}$ (objetivo $\leq 3$\,min para $N{=}10$).

\noindent Gaps/día.
$\displaystyle \text{Gaps}_{\text{prom}}=\frac{\sum_{d}{\text{minutos vacíos}(d)}}{\#\text{días activos}}$ (objetivo: tendencia a la baja).

\noindent On-time reminders (medible).
$\displaystyle \frac{\#\text{notifs abiertas }\geq X\text{ min antes}}{\#\text{notifs totales}}\times 100\%$ (objetivo $\geq 95\%$).

\noindent Adopción JSON (medible).
\% de export/import exitoso (objetivo $\uparrow$ en semanas pico); replicable por pruebas E2E (F5/F8).

% ----------------------------------------------------------------------
\section{Riesgos y mitigaciones}
\begin{table}[H]
  \centering
  \caption{Riesgos (P=probabilidad, I=impacto) y acciones. Fuente: Elaboración propia.}
  \label{tab:cap4_riesgos}
  \small
  \begin{tabular}{|p{6.6cm}|c|c|p{7.8cm}|}
    \hline
    Riesgo & P & I & Mitigación \\
    \hline
    Cambios en layout/ubicación de PDF & M & A & Parsers tolerantes; revisión manual; import local de PDF \\
    \hline
    Caída prolongada/saturación del sitio & M & A & Backoff + CB; cache local; diferir sync (ventanas); fallback PDF \\
    \hline
    Conectividad intermitente & A & M & Offline-first; política de reintentos; freshness gate antes de generar \\
    \hline
    Dispositivos de baja gama & M & M & UI ligera; optimización de consultas; IA opcional (opt-in) \\
    \hline
    Datos desactualizados & M & M & Freshness check previo a generación; auto-sync condicionado \\
    \hline
  \end{tabular}
\end{table}

% ----------------------------------------------------------------------
\section{Checklist de release y CI/CD}
\begin{itemize}
  \item Pruebas verdes; cobertura $\geq 75\%$ en módulos críticos; flujos E2E críticos completos.
  \item Lint/análisis estático sin issues críticos.
  \item Verificar canales de notificación creados y configurados correctamente.
  \item APK $<30$\,MB (sin IA); firma válida; notas de versión; rollout por etapas.
\end{itemize}

% ----------------------------------------------------------------------
\section{Lecciones aprendidas y trabajo futuro}
Lecciones. Poda por hard constraints + MRV/LCV reduce el espacio de búsqueda sin perder calidad; JSON compartible habilita colaboración (WhatsApp-first); prioridad sin conexión reduce ansiedad por caídas; widget aumenta consultas diarias.

Próximos pasos. Filtros por aula/docente, eventos externos (tareas/exámenes), telemetría opt-in para refinar ranking, benchmark por dispositivo, más pruebas de UI en export/import.

% ----------------------------------------------------------------------
\section*{Síntesis}
Se ejecutó un enfoque iterativo-incremental guiado por evidencia de encuestas y entrevistas, con una arquitectura limpia priorizando la operación con prioridad sin conexión. El generador, basado en backtracking y heurísticas, respeta restricciones duras y preferencias blandas. La interoperabilidad JSON facilita la colaboración, mientras que los recordatorios con IA on-device opcional completan las funcionalidades. La cadena de requisitos a artefactos y pruebas asegura alineación entre problema, implementación y verificación.

% Nota editorial: rutas simplificadas
% Se omiten URLs de repositorio y rutas completas; se priorizan nombres de clases para legibilidad.

Notas de cierre. Los porcentajes de la encuesta fueron incorporados en el capítulo correspondiente; se fijó \(T_{fresh}{=}48\,h\) y \(W_{nosync}{=}07{:}00\text{–}12{:}30\) (ajustable). La simbología de flechas se mantuvo consistente en tablas y diagramas.

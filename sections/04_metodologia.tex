\chapter{METODOLOGÍA Y DESARROLLO DEL PROYECTO}
\label{chap:metodologia}

Este capítulo documenta la aplicación práctica de la metodología sobre \emph{TecnoTime}: fases ejecutadas, actividades, artefactos y evidencias en el repositorio. Se excluyen contenidos del SRS y del área de aplicación, ya tratados en capítulos previos.

\begin{figure}[H]
  \centering
  \includegraphics[width=0.8\linewidth]{images/diagrams/cap4_arq_general.png}
  \caption{Arquitectura general de referencia (presentación, dominio, datos) y servicios transversales. Fuente: elaboración propia.}
  \label{fig:cap4_arq_general}
\end{figure}

% ----------------------------------------------------------------------
\section{Metodología adoptada y fases}
Se adoptó un proceso iterativo–incremental con prototipado evolutivo, organizado en once fases: F1) Datos institucionales, F2) Modelo de dominio, F3) Generador de horarios, F4) Prioridad sin conexión, F5) Interoperabilidad (JSON), F6) UX/Onboarding/\emph{widget}, F7) Recordatorios e IA, F8) Exportaciones, F9) Pruebas y validación, F10) Seguridad y privacidad, F11) Empaque y despliegue. La priorización (Android-first, prioridad sin conexión, compartir por mensajería y recordatorios) se justificó con hallazgos de la \texttt{Encuesta.md} (n $\approx$ 125), sin análisis estadístico.

% ----------------------------------------------------------------------
\section{MAPA DE FASES (F1–F11) Y TABLA DE TRAZABILIDAD}
Para dar transparencia al proceso, se desglosan 11 fases. La Tabla~\ref{tab:cap4_trazabilidad_fases} resume entradas, salidas y módulos; las subsecciones detallan actividades, decisiones y artefactos.

\begin{table}[H]
  \centering
\caption{Trazabilidad de fases a módulos y salidas verificables. Fuente: elaboración propia.}
  \label{tab:cap4_trazabilidad_fases}
  \footnotesize
  \begin{tabular}{|p{1.0cm}|p{3.5cm}|p{5.0cm}|p{5.0cm}|}
    \hline
    \textbf{Fase} & \textbf{Objetivo} & \textbf{Entradas/Actividades} & \textbf{Salidas (artefactos/módulos)} \\
    \hline
    F1 & Ingesta PDF a BD & Descarga, parseo, normalización, \emph{freshness}, backoff/CB & \texttt{data/remote/pdf/*}, \texttt{data/local/*}, \texttt{AutoSyncUseCase} \\
    \hline
    F2 & Modelo de dominio & Entidades, reglas, claves/índices, mappers & \texttt{domain/model/*}, \texttt{data/local/entity/*}, \texttt{data/mapper/*} \\
    \hline
    F3 & Generador de horarios & Backtracking+MRV/LCV, estrategia compuesta, top-N & \texttt{ScheduleGenerator}, \texttt{GenerateSchedulesUseCaseImpl} \\
    \hline
    F4 & Offline-first & Caché Room, freshness, backoff, fallback PDF & \texttt{di/AppModule.kt}, \texttt{domain/util/*} \\
    \hline
    F5 & Interoperabilidad JSON & Share/Import, validaciones, intent filter & \texttt{GenerateShareableScheduleJsonUseCase}, \texttt{ImportSharedScheduleUseCase} \\
    \hline
    F6 & UX end-to-end & Onboarding, selección, edición segura, widget & \texttt{presentation/*}, \texttt{widget/*} \\
    \hline
    F7 & Recordatorios e IA & Canales, workers, IA on-device (opt-in) & \texttt{application/infrastructure/notification/*}, \texttt{domain/simon/*} \\
    \hline
    F8 & Exportaciones & Imagen/PDF/Excel con bloqueos y confirmaciones & Casos de uso de export, pantallas de envío \\
    \hline
    F9 & QA/Validacion & Piramide de tests, E2E manuales, no-regresion & Suite de pruebas (TC-001..016) \\
    \hline
    F10 & Seguridad/Privacidad & Min-data, permisos mínimos, IA local & Políticas en código y ajustes \\
    \hline
    F11 & Empaque/Release & SDKs, shrinking, firma, notas de version & \texttt{app/build.gradle.kts}, \textit{release checklist} \\
    \hline
  \end{tabular}
\end{table}

% ============================
\subsection{F1 — Datos institucionales: Ingestión y normalización (PDF$\rightarrow$BD)}
\textbf{Objetivo.} Transformar los horarios oficiales (PDF) en un \textbf{modelo estructurado local} robusto, evitando saturar la fuente institucional.

\paragraph{Flujo detallado.}
\begin{enumerate}
  \item \textbf{Descubrimiento de fuentes}: crawling liviano de páginas institucionales para URLs PDF por carrera/periodo.
  \item \textbf{Descarga controlada}: colas de descarga con \emph{rate limit} y \textbf{exponential backoff} ($t_{n}= \min(t_{\max}, t_{0}\cdot 2^n)$) ante fallos; \textbf{circuit breaker} (\emph{open} tras $k$ fallos consecutivos por $\Delta t$).
  \item \textbf{Parseo}: extracción de materias, códigos, niveles, grupos, docentes, aulas y bloques (día, inicio, fin); limpieza y normalización.
  \item \textbf{Persistencia}: DAOs Room con índices compuestos (\texttt{(subject\_code, group\_id)}), claves foráneas y políticas de \textit{ON CONFLICT}.
  \item \textbf{Freshness}: marcas de estampa (\textit{last\_sync}, \textit{source\_etag/last-modified}) para decidir re-sincronización.
  \item \textbf{Fallback}: \textit{import} manual de PDF oficial por carrera desde almacenamiento local si el origen está caído.
\end{enumerate}

\paragraph{Decisiones.}
Parser tolerante (variantes de cabeceras/días), \textbf{idempotencia} al re-parsar, y \textbf{trazabilidad} del origen (campos \texttt{source\_career}, \texttt{source\_period}).

\paragraph{Actividades ejecutadas.}
Descubrimiento y descarga con limitación de tasa; extracción y normalización de materias/grupos/horas/aulas; persistencia en Room; marcado de frescura; configuración de reintentos con \emph{backoff}/cortacircuito; opción de importación manual por PDF.

\paragraph{Criterios de aceptación.}
Datos persistidos por carrera/nivel/materia/grupo; fecha de última actualización visible; no hay intentos continuos ante fallos (los reintentos se limitan).

\paragraph{Riesgos y mitigaciones.}
Cambio de formato/ubicación del PDF (extractor flexible y sincronización manual); caída del origen (caché local y reintentos espaciados); saturación del servidor (cortacircuito + \emph{backoff}).

\begin{figure}[H]
  \centering
  \includegraphics[width=0.8\linewidth]{images/diagrams/cap4_f1_flujo_datos.png}
  \caption{Ingesta/normalización con verificación de frescura, backoff y cortacircuito. Fuente: elaboración propia.}
  \label{fig:cap4_f1}
\end{figure}

% ============================
\subsection{F2 — Modelado del dominio (entidades, reglas, mapeo)}
\textbf{Entidades principales.} \textit{Career, Level, Subject, Group, GroupSchedule, Teacher, Classroom, SelectedSubject, UserSettings, Reminder}.

\paragraph{Reglas de integridad.}
\begin{itemize}
  \item \textbf{Unicidad} de \texttt{Subject.code} y del par \texttt{(subject\_code, group\_id)}.
  \item \textbf{Temporal}: \texttt{startTime < endTime} para cada \textit{GroupSchedule}.
  \item \textbf{Referencial}: \textit{GroupSchedule} a \textit{Group} a \textit{Subject/Level/Career}; \textit{Group} referencia \textit{Teacher}/\textit{Classroom} (nulos permitidos).
\end{itemize}

\paragraph{Diccionario mínimo (extracto).}
\begin{itemize}
  \item \texttt{Subject}\,: \texttt{code}, \texttt{name}, \texttt{levelId}, \texttt{careerId}, \texttt{isOptional}, \texttt{isApproved}.
  \item \texttt{Group}\,: \texttt{subjectCode}, \texttt{groupId}, \texttt{teacherId?}, \texttt{classroomId?}, \texttt{status}.
  \item \texttt{GroupSchedule}\,: \texttt{groupId}, \texttt{day}, \texttt{start}, \texttt{end}.
  \item \texttt{SelectedSubject}\,: selección usuario + \texttt{color}, \texttt{emoji}.
\end{itemize}

\begin{figure}[H]
  \centering
  \includegraphics[width=0.8\linewidth]{images/diagrams/cap4_f2_modelo_dominio.png}
  \caption{Modelo de dominio (clases y relaciones principales). Fuente: elaboración propia.}
  \label{fig:cap4_f2}
\end{figure}

\paragraph{Actividades ejecutadas.}
Levantamiento de entidades y relaciones desde el código; alineación entre entidades de dominio y entidades Room; revisión de mapeadores.

\paragraph{Criterios de aceptación.}
Cobertura de casos de uso del generador y de exportación/importación; consistencia entre identificadores y relaciones (carrera–nivel–materia–grupo).

\paragraph{Riesgos y mitigaciones.}
Desalineación dominio/Room (revisión cruzada y pruebas de carga desde PDFs); ambigüedad en claves (unicidad por código y referencias cruzadas).

% ============================
\subsection{F3 — Generador de horarios (restricciones, heurísticas y UI)}
\textbf{Problema.} Seleccionar \textbf{un grupo por materia} respetando restricciones \textbf{duras} (no solapes) y optimizando \textbf{blandas}: minimizar \emph{gaps}, priorizar \textit{favorite teachers}, balance básico semanal, \textit{mandatory subjects}.

\paragraph{Parámetros de generación (UI).}
\begin{itemize}
  \item \textbf{Target de materias} (máx. 20) y \textbf{N de horarios} a producir (máx. 25).
  \item \textbf{Flags}: \texttt{acceptConflicts}, \texttt{minimizeGaps}, \texttt{prioritizeTeachers}.
  \item \textbf{Obligatorias}: lista seleccionada por nivel; si excede el target, el excedente se relaja a opcional.
\end{itemize}

\paragraph{Estrategia.}
\begin{enumerate}
  \item Orden con menos valores restantes (MRV): primero materias con menos grupos viables.
  \item Exploración de menor costo restante (LCV): probar primero los grupos con mayor promesa (docentes favoritos, huecos bajos).
  \item \textbf{Poda temprana}: descartar parcial con solape si \texttt{acceptConflicts=false}.
  \item \textbf{Evaluación compuesta}: $F(H)=\alpha\cdot\text{gaps}+\beta\cdot\text{no\_favorites}+\gamma\cdot\text{days\_spread}+\delta\cdot\text{conflicts}$.
  \item \textbf{Top-$N$}: mantener los mejores $N$ horarios; \textbf{aplicar} \textit{completo} o \textit{parcial}.
\end{enumerate}

\paragraph{Pseudocódigo.}
\begin{lstlisting}[basicstyle=\ttfamily\small]
generate(params):
  mandatory, optional <- splitByUserSelection()
  strategy <- StrategyFactory.create(params.flags, userPreferences)
  best <- TopN(params.nSchedules)
  expand <- expandCandidates(mandatory, optional, params.targetSubjects)

  backtrack(i=0, partial=[]):
    if i == len(expand):
      score <- strategy.evaluate(partial)
      best.add(partial, score)
      return
    for g in orderByLCV(expand[i], strategy):
      if params.acceptConflicts or noHardOverlap(partial, g):
        backtrack(i+1, partial + g)

  backtrack()
  return best.results()
\end{lstlisting}

\paragraph{Interacción de banderas (casuística real).}
\begin{itemize}
  \item \textbf{Priorizar docentes} sin \textbf{aceptar choques}: se eligen favoritos sólo si respetan no solape.
  \item \textbf{Ambas ON}: se admiten choques penalizados; casos extremos para el usuario experto.
  \item \textbf{Obligatorias} $>$ target: el exceso pasa a \emph{opcionales} con prioridad.
\end{itemize}

\paragraph{Criterios de aceptación.}
Sin solapamientos por defecto; generación de $N$ resultados configurables; parámetros influyen en el orden y contenido; aplicación completa/parcial actualiza selección local.

\paragraph{Riesgos y mitigaciones.}
Explosión combinatoria (poda por progresión y priorización por nivel); preferencias contradictorias (reglas de desempate y mensajes claros en UI); datos incompletos (refresco de datos antes de generar).

\paragraph{UI de resultados.}
\begin{itemize}
  \item Lista \textit{Lun..Sáb} con \textit{cards} por bloque: hora inicio/fin, materia, grupo, aula, nivel, docente.
  \item \textbf{Switch} por materia (selección parcial); al cambiar selección se re-genera la cohorte de horarios.
  \item \textbf{Acciones}: \textit{Parcial} (aplica sólo seleccionadas), \textit{Completo} (aplica todo el horario).
  \item \textbf{Vista semanal} por horario generado; navegación izq./der. entre resultados 1..N.
\end{itemize}

\begin{figure}[H]
  \centering
  \includegraphics[width=0.8\linewidth]{images/diagrams/cap4_f3_decisiones_generador.png}
  \caption{Decisiones del generador y mantenimiento de Top-N. Fuente: elaboración propia.}
  \label{fig:cap4_f3}
\end{figure}

% ============================
\subsection{F4 — Arquitectura \emph{offline-first} y sincronización responsable}
\paragraph{Objetivo.} Usabilidad plena con conectividad intermitente y \textbf{protección de la fuente} institucional.

\paragraph{Componentes.}
\begin{itemize}
  \item \textbf{Caché Room}: carreras, niveles, materias, grupos, horarios y preferencias.
  \item \textbf{Freshness Gate}: si hay red, se valida frescura \emph{antes} de generar; si \emph{stale}, \textit{sync}.
  \item \textbf{Retry/backoff} y \textbf{circuit breaker}: evitan martillar el origen cuando está lento/caído.
  \item \textbf{Fallback PDF manual}: import por carrera cuando el sitio está fuera de línea.
\end{itemize}

\begin{figure}[H]
  \centering
  \includegraphics[width=0.8\linewidth]{images/diagrams/cap4_f4_offline_first.png}
  \caption{Caché local, verificación de frescura y sincronización con backoff/cortacircuito. Fuente: elaboración propia.}
  \label{fig:cap4_f4}
\end{figure}

\paragraph{Actividades ejecutadas.}
Integración del verificador de conectividad; sincronización automática condicionada; bloqueo temporal de generación durante sincronización; aplicación de \emph{backoff} y cortacircuito.

\paragraph{Criterios de aceptación.}
Generación bloqueada mientras se sincroniza; operación con datos locales cuando no hay internet; reintentos espaciados cuando el origen falla.

\paragraph{Riesgos y mitigaciones.}
Reintentos excesivos (\emph{backoff} con límites); inconsistencias temporales (fecha de última actualización y reintento manual).

% ============================
\subsection{F5 — Interoperabilidad y colaboración (JSON, WhatsApp)}
\paragraph{Escenarios.}
\begin{itemize}
  \item \textbf{Enviar copia}: JSON total o \textbf{parcial} (selector con \textit{Limpiar/Todo}, \textit{Cancelar/Compartir}); confirmación de exportación.
  \item \textbf{Importar desde WhatsApp}: \textit{intent} para abrir con la app; lista de materias con validaciones:
  \begin{enumerate}
    \item Si la materia ya se encuentra inscrita, se bloquea su selección (mensaje “ya inscrita”).
    \item Si la carrera difiere, se propone \textbf{sincronizar} esa carrera e integrar la selección.
  \end{enumerate}
\end{itemize}

\paragraph{Esquema JSON (resumen).}
\texttt{careers}, \texttt{subjects}, \texttt{groups}, \texttt{teachers}, \texttt{classrooms}, \texttt{entries} (bloques de horario con referencias); \texttt{meta.version}, \texttt{meta.generatedAt}, \texttt{origin.career/period}. Idempotencia y compatibilidad hacia atrás (\emph{backward compatible}).

\begin{figure}[H]
  \centering
  \includegraphics[width=0.8\linewidth]{images/diagrams/cap4_f5_share_import_flow.png}
  \caption{Exportar/importar JSON con validaciones y fusión parcial/total. Fuente: elaboración propia.}
  \label{fig:cap4_f5}
\end{figure}

\paragraph{Actividades ejecutadas.}
Generación de JSON compartible; selector de materias a compartir; validación de esquema/versión; importación con fusión sin duplicados y aviso ante carrera distinta.

\paragraph{Criterios de aceptación.}
JSON válido según esquema/versión; importación rechaza entradas inválidas; fusión preserva selección existente; se solicita sincronizar carrera cuando difiere.

\paragraph{Riesgos y mitigaciones.}
Deriva del esquema (versionado y validación previa); incompatibilidades entre carreras (verificación de carrera y sincronización asistida).

% ============================
\subsection{F6 — UX aplicada: Onboarding, ajustes, agregar/editar, widget}
\paragraph{Onboarding (\textit{progressive disclosure}).}
\begin{enumerate}
  \item \textbf{Bienvenida + permisos}.
  \item \textbf{Nombre} (personalización e IA).
  \item \textbf{Ajustes generales}: formato 24/12h, mostrar fines de semana, resaltar días sin clases; \textit{tema claro/oscuro}.
  \item \textbf{Sincroniza tus carreras}: listado de $\sim$20 carreras; selección por usuario.
  \item \textbf{Revisar progreso}: \textit{cards} por nivel con círculo de progreso (\% materias marcadas).
  \item \textbf{IA Simon (opt-in)}: aviso de tamano de modelo (~ 229\,MB); elegir conexion (Datos/Wi-Fi).
  \item \textbf{Recordatorios}: \textit{anticipación} (por defecto \textbf{10 min}) y tipos (\textit{clases}/\textit{motivacionales}).
  \item \textbf{Docentes favoritos}: listado con estrella para \textit{prioritization later}.
\end{enumerate}
El \textbf{onboarding} sólo se muestra la primera vez; después la app abre en \textit{Home}.

\paragraph{Home y navegación.}
\begin{itemize}
  \item \textbf{Deslizamiento} izq./der. por días; \textbf{selector de fecha} (abre calendario para saltar a un día, p.\,ej., 22/11).
  \item \textbf{Acciones principales}: \textit{Agregar materia} y \textit{Generar horario} (botón flotante).
  \item \textbf{Botón semanal} (arriba derecha) y \textbf{Ajustes} (arriba izquierda).
\end{itemize}

\paragraph{Agregar materia (flujo guiado).}
\begin{enumerate}
  \item \textbf{Seleccionar nivel} $\rightarrow$ \textbf{Materia} (lista por nivel).
  \item \textbf{Seleccionar grupo} con \textbf{preview} de días/horas (p.\,ej., Mar 09:45–11:15; Mié 08:15–09:45); \textbf{Confirmar}.
  \item \textbf{Color} (paleta circular) y \textbf{Emoji} (incluye botón \textit{Al azar}).
  \item \textbf{Crear}: retorna al Home con las tarjetas en los días correspondientes.
\end{enumerate}

\paragraph{Edición desde Home.}
\begin{itemize}
  \item \textbf{Editar grupo}: cambiar a otro grupo (con preview y confirmar); ajustar color/emoji.
  \item \textbf{Finalizar materia}: diálogo \textit{Aprobado}/\textit{Abandonar} (impacta en generación futura).
\end{itemize}

\paragraph{Widget.}
Widget con el día actual, deslizamiento de días y \textit{cards} de próximas clases (mismos colores/emojis).

\begin{figure}[H]
  \centering
  \includegraphics[width=0.8\linewidth]{images/diagrams/cap4_f6_journey_ui.png}
  \caption{Journey de UI: bienvenida, configuración, agregar/editar y componente de acceso rápido (widget). Fuente: elaboración propia.}
  \label{fig:cap4_f6}
\end{figure}

\paragraph{Actividades ejecutadas.}
Implementación de pantallas de bienvenida y configuración; flujo jerárquico de agregar materia con vista previa; edición de grupo/color/emoji; integración del componente de acceso rápido (\emph{widget}).

\paragraph{Criterios de aceptación.}
Disminución de pasos para armar horario; vista previa antes de aplicar cambios; el componente (\emph{widget}) muestra el horario del día con navegación de días.

\paragraph{Riesgos y mitigaciones.}
Sobrecarga de opciones (onboarding progresivo y valores por defecto); inconsistencia visual (diseño consistente en Compose).

% ============================
\subsection{F7 — Recordatorios y IA on–device ("Simón")}
\paragraph{Recordatorios.}
\begin{itemize}
  \item \textbf{Anticipación}: horas/minutos configurables; por defecto 10\,min.
  \item \textbf{Canales}: \texttt{classes} (obligatorio) y \texttt{motivation} (opcional).
  \item \textbf{Programación}: \textit{WorkManager} con ventanas adecuadas y reintentos si el SO difiere tareas.
\end{itemize}

\paragraph{IA (opt-in).}
\begin{itemize}
  \item \textbf{Descarga de modelo}: el usuario elige \textit{Wi-Fi} o \textit{Datos}; indicador de progreso; opción \textit{Eliminar modelo}.
  \item \textbf{Toggles}: “Usar Simón para mis recordatorios” y “Mensajes motivacionales”.
  \item \textbf{Privacidad}: generacion local; sin envio de datos personales.
\end{itemize}

\begin{figure}[H]
  \centering
  \includegraphics[width=0.8\linewidth]{images/diagrams/cap4_f7_secuencia_ia_recordatorios.png}
  \caption{Secuencia IA/recordatorios: descarga, inicialización, programación y notificación. Fuente: elaboración propia.}
  \label{fig:cap4_f7}
\end{figure}

\paragraph{Actividades ejecutadas.}
Programación de notificaciones con WorkManager; estilos y acciones; activación/descarga del modelo IA bajo control del usuario; generación de mensajes motivacionales.

\paragraph{Criterios de aceptación.}
Notificaciones mostradas antes de la clase según anticipación; persistencia tras reinicio; IA desactivada por defecto y descargable; posibilidad de eliminar el modelo.

\paragraph{Riesgos y mitigaciones.}
Consumo de almacenamiento (descarga bajo demanda y opción de eliminación); notificaciones excesivas (configuración de umbrales y silencios).

% ============================
\subsection{F8 — Exportaciones (imagen, PDF, Excel)}
\paragraph{Reglas de flujo.}
\begin{itemize}
  \item Siempre hay \textbf{diálogo de confirmación}; si no hay materias, el flujo se \textbf{bloquea} con mensaje explicativo.
  \item \textbf{Imagen/PDF/Excel}: se genera y se entrega al \textit{Share sheet} (WhatsApp frecuente).
\end{itemize}

\begin{figure}[H]
  \centering
  \includegraphics[width=0.8\linewidth]{images/diagrams/cap4_f8_export_flow.png}
  \caption{Exportación con validaciones y confirmaciones. Fuente: elaboración propia.}
  \label{fig:cap4_f8}
\end{figure}

\paragraph{Actividades ejecutadas.}
Composición del horario a imagen/PDF/Excel; selección de destino; validación de precondiciones (materias existentes).

\paragraph{Criterios de aceptación.}
Exportación bloqueada si no hay materias; archivos contienen materia, grupo, día, hora, aula y docente.

\paragraph{Riesgos y mitigaciones.}
Falta de permisos o espacio (uso de \texttt{DocumentFile} y validaciones previas de espacio disponible).

% ============================
\subsection{F9 — Pruebas y Validación (Quality Assurance)}
\paragraph{Pirámide de testing.}
\begin{itemize}
  \item \textbf{Unitarias}: evaluaciones de \emph{gaps}/favoritos, parsing robusto, utilidades de calendario (orden, solapes).
  \item \textbf{Integración}: repositorios Room + casos de uso (generar, sincronizar, export/import).
  \item \textbf{UI/E2E}: flujos críticos (onboarding, agregar/editar, generar y aplicar, compartir/abrir JSON).
\end{itemize}

\paragraph{Casos de prueba (extracto ampliado).}
\begin{itemize}
  \item \textbf{TC-001}: \textit{gaps}=0 en horarios contiguos por día; \textbf{TC-002}: \emph{composite strategy} refleja flags.
  \item \textbf{TC-004/005}: parser soporta formatos de día y grupo alternos (Lun/Lunes; Gp/Grupo).
  \item \textbf{TC-006}: notificación respeta anticipación y canal.
  \item \textbf{TC-007}: \textit{AutoSync} honra periodicidad y marca \textit{last\_sync}.
  \item \textbf{TC-008}: vista semanal ordena por hora ascendente y respeta formato 24/12h.
  \item \textbf{TC-009/014}: \textit{round-trip} JSON y contenido correcto en PDF/imagen (cabecera, bloques).
  \item \textbf{TC-012}: codificación visual \textbf{determinística} (color/emoji) por materia.\footnote{En UI se ofrece “Al azar” para emoji; internamente se recomienda \emph{seeded hashing} para persistir identidad visual entre sesiones.}
  \item \textbf{TC-013}: cambio de grupo detecta y evita conflictos salvo \texttt{acceptConflicts=true}.
  \item \textbf{TC-016}: favoritos elevan la calificación respecto a alternativas equivalentes.
\end{itemize}

% ============================
\subsection{F10 — Seguridad y privacidad}
\begin{itemize}
  \item \textbf{Datos mínimos}: nombre (personalización), selección de materias/grupos (en local).
  \item \textbf{Permisos mínimos}: notificaciones; almacenamiento solo para exportaciones/importación.
  \item \textbf{IA local}: \textit{opt-in}, descarga controlada, eliminación disponible, sin telemetría obligatoria.
\end{itemize}

\paragraph{Actividades ejecutadas.}
Revisión de permisos estrictamente necesarios; almacenamiento en preferencias y BD locales; controles para descargar/eliminar el modelo de IA.

\paragraph{Criterios de aceptación.}
La aplicación opera con permisos mínimos; no se suben datos personales; la IA funciona enteramente en el dispositivo si está activa.

\paragraph{Riesgos y mitigaciones.}
Exposición accidental de datos (revisión de permisos y rutas de exportación); uso involuntario de IA (configuración explícita y reversible).

% ============================
\subsection{F11 — Empaque, compatibilidad y release}
\begin{itemize}
  \item \textbf{SDKs}: \texttt{minSdk=24}, \texttt{target/compileSdk=35}.
  \item \textbf{Tamaño}: binario objetivo $< 30$\,MB sin IA; modelo IA $\sim$200–230\,MB (opcional).
  \item \textbf{Optimización}: reducción de código y recursos; splits por ABI cuando aplica.
  \item \textbf{Release}: compilaciones reproducibles, firma, notas de versión, pruebas verdes y despliegue progresivo.
\end{itemize}

\paragraph{Actividades ejecutadas.}
Configuración de variantes \texttt{debug/release}; habilitación de reducción en \texttt{release}; verificación de compatibilidad de SDK.

\paragraph{Criterios de aceptación.}
Compilación exitosa en \texttt{release}; reducción de tamaño activa; compatibilidad con dispositivos API 24–35.

\paragraph{Riesgos y mitigaciones.}
Incremento de tamaño por dependencias (revisión de librerías y reducción); incompatibilidades de SDK (pruebas en dispositivos/AVDs objetivo).

% ----------------------------------------------------------------------
\section{REGLAS DE NEGOCIO, RESTRICCIONES Y SUPUESTOS}
\subsection*{Reglas de negocio (RB)}
\begin{enumerate}[label=RB-\arabic*]
  \item \texttt{startTime < endTime} para cada sesión.
  \item Unicidad de \texttt{Subject.code} y de \texttt{(subjectCode, groupId)}.
  \item \textbf{No solapes} por defecto; los choques sólo son válidos si \texttt{acceptConflicts=true}.
  \item Docente y aula pueden ser nulos y no bloquean la selección.
  \item Las \textbf{obligatorias} se intentan incluir todas; si superan el target, el excedente se relaja a opcional.
  \item Los \textbf{favoritos} aportan bonificación en la evaluación.
  \item \textbf{Freshness before generation}: si hay red, validar frescura y sincronizar antes de generar.
\end{enumerate}

\subsection*{Restricciones (RST)}
\begin{enumerate}[label=RST-\arabic*]
  \item Dependencia de estructura de PDFs institucionales; cambios requieren ajustar parsing.
  \item Requiere conectividad para sincronización; operación offline para consulta/generación local.
  \item Plataforma objetivo Android; iOS fuera de alcance inicial.
\end{enumerate}

\subsection*{Supuestos (SUP)}
\begin{enumerate}[label=SUP-\arabic*]
  \item Disponibilidad regular de PDFs por carrera/semestre.
  \item Predominancia Android en la población objetivo.
  \item Uso social de WhatsApp para intercambio de horarios (soporte JSON).
\end{enumerate}

% ----------------------------------------------------------------------
\section{CASOS DE USO (DETALLADOS)}
\subsection*{UC-001 — Scraping y Parsing de Datos Académicos}
\textbf{Actor:} Sistema. \textbf{Objetivo:} mantener datos locales actualizados.\\
\textbf{Pre:} conectividad disponible \emph{o} PDF local provisto. \textbf{Post:} BD local consistente.\\
\textbf{Flujo básico:} descubrir URLs $\rightarrow$ descargar con \emph{rate limit} $\rightarrow$ parsear $\rightarrow$ normalizar $\rightarrow$ persistir $\rightarrow$ marcar \textit{last\_sync}.\\
\textbf{Excepciones:} origen caído $\rightarrow$ \emph{backoff} y \textit{open circuit}; sugerir \textit{import} manual de PDF.

\subsection*{UC-002 — Generar Horarios Optimizados}
\textbf{Actor:} Estudiante. \textbf{Objetivo:} obtener N horarios candidatos.\\
\textbf{Pre:} selección de materias/flags. \textbf{Post:} lista de N horarios + vista semanal.\\
\textbf{Flujo básico:} setear parámetros $\rightarrow$ generar $\rightarrow$ revisar $\rightarrow$ aplicar parcial/completo.\\
\textbf{Alternos:} \texttt{acceptConflicts=true} (choques penalizados).

\subsection*{UC-003 — Configurar Notificaciones}
\textbf{Actor:} Estudiante. \textbf{Objetivo:} recibir avisos previos a clase.\\
\textbf{Flujo:} abrir ajustes $\rightarrow$ definir anticipación (default 10\,min) $\rightarrow$ guardar.

\subsection*{UC-004 — Seleccionar Carrera/Nivel/Materia/Grupo}
\textbf{Actor:} Estudiante. \textbf{Objetivo:} construir carga.\\
\textbf{Flujo:} jerarquía con \textbf{preview} de grupo $\rightarrow$ confirmar.

\subsection*{UC-005 — Exportar/Importar Horario}
\textbf{Actor:} Estudiante. \textbf{Objetivo:} compartir/clonar.\\
\textbf{Export:} Imagen/PDF/Excel o JSON (total/parcial). \textbf{Import:} abrir JSON, validar duplicados, sincronizar carrera si difiere, fusionar.

% ----------------------------------------------------------------------
\section{MÉTRICAS (KPIS) Y OBSERVABILIDAD}
\begin{itemize}
  \item \textbf{TTS (Time To Schedule)}: minutos hasta horario base sin choques.
  \item \textbf{Gaps/día}: minutos vacíos promedio por día (tras aplicar).
  \item \textbf{Adopción JSON}: \% de usuarios con export/import exitoso.
  \item \textbf{Widget usage}: aperturas del widget / aperturas totales.
  \item \textbf{On-time reminders}: \% notificaciones abiertas antes de clase.
\end{itemize}

% ----------------------------------------------------------------------
\section{RIESGOS Y MITIGACIONES}
\begin{table}[H]
  \centering
  \caption{Riesgos con probabilidad (P) e impacto (I) y acciones}
  \label{tab:cap4_riesgos}
  \small
  \begin{tabular}{|p{6.5cm}|c|c|p{7.8cm}|}
    \hline
    Riesgo & P & I & Mitigación \\
    \hline
    Cambios en PDFs (formato/ubicación) & M & A & Parser tolerante; pruebas de regresión; import manual por PDF \\
    \hline
    Caída/latencia de origen & M & M & Backoff + cortacircuito; caché local; diferir sincronización \\
    \hline
    Conectividad intermitente & A & M & Offline-first real; \emph{retry policy} y \emph{freshness gate} \\
    \hline
    Dispositivos de baja gama & M & M & Optimizar consultas y UI; IA opcional; cargas perezosas \\
    \hline
    Rechazo de IA & M & B & Mantener IA como \emph{add-on} desactivable; experiencia intacta sin IA \\
    \hline
  \end{tabular}
\end{table}

% ----------------------------------------------------------------------
\section{CHECKLIST DE RELEASE Y CI/CD}
\begin{itemize}
  \item Pruebas verdes; cobertura objetivo $>\,75\%$ en módulos críticos.
  \item Lint/estático sin \textit{critical issues}.
  \item Binario $< 30$\,MB (sin IA); notas de versión; firma y \emph{rollout} progresivo.
\end{itemize}

% ----------------------------------------------------------------------
\section{LECCIONES APRENDIDAS Y TRABAJO FUTURO}
\subsection*{Lecciones}
La \textbf{poda por hard constraints} y la \textbf{priorización por nivel} reducen drásticamente el espacio de búsqueda sin sacrificar calidad. El \textbf{JSON compartible} habilita colaboración real (\textit{WhatsApp-first}). El \textbf{widget} incrementa consultas diarias. \textbf{Offline-first} mitiga ansiedad ante caídas del origen.

\subsection*{Próximos pasos}
Filtros por aula/docente en UI, eventos externos (tareas/exámenes), telemetría \textit{opt-in} para refinar ranking de horarios, \textbf{benchmark} de rendimiento por dispositivo objetivo, y ampliación de pruebas de UI para export/import.

% ----------------------------------------------------------------------
\section*{SÍNTESIS}
Se operó con un enfoque \textbf{iterativo–incremental} enfocado en uso real, respaldado por una \textbf{arquitectura limpia (Clean Architecture)} y una infraestructura \textbf{prioridad sin conexión (offline-first)}. El núcleo algorítmico (\emph{backtracking} + heurísticas compuestas) respeta restricciones duras y preferencias del usuario. La interoperabilidad JSON y los recordatorios (con IA en dispositivo opcional) completan el conjunto funcional. La trazabilidad RQ$\rightarrow$RF$\rightarrow$UC$\rightarrow$Artefactos$\rightarrow$Tests asegura alineación entre problema, implementación y verificación.

% -------------------- NOTA DE IMÁGENES --------------------
% Las figuras referenciadas se asumen disponibles en images/diagrams/:
% cap4_arq_general.png, cap4_f1_flujo_datos.png, cap4_f2_modelo_dominio.png,
% cap4_f3_decisiones_generador.png, cap4_f4_offline_first.png,
% cap4_f5_share_import_flow.png, cap4_f6_journey_ui.png,
% cap4_f7_secuencia_ia_recordatorios.png, cap4_f8_export_flow.png

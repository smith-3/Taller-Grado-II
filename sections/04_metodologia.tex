% 04_metodologia.tex

\chapter{Metodología y Desarrollo}
\label{chap:metodologia}

\section{Metodología seleccionada}
% 4.1: Solo nombre y número de fases, sin teoría
Proceso iterativo--incremental con 6 fases:
\begin{itemize}
  \item Fase 1: Análisis y Planificación
  \item Fase 2: Diseño
  \item Fase 3: Implementación
  \item Fase 4: Pruebas
  \item Fase 5: Despliegue y CI/CD
  \item Fase 6: Mantenimiento y Mejoras
\end{itemize}

\section{Fase 1: Análisis y Planificación}
\subsection*{Objetivo}
Consolidar requisitos funcionales y no funcionales derivados del Capítulo III; definir reglas de negocio, restricciones, supuestos, casos de uso y la trazabilidad inicial.

\subsection*{Catálogo de requisitos funcionales (RF)}
% Fichas individuales por requisito funcional
% Definición de entorno para fichas (usa tabularx)
\begin{requisitoficha}
\textbf{ID} & RF-001 \\
\textbf{Título} & Scraping y parsing de datos académicos \\
\textbf{Prioridad} & Must \\
\textbf{Descripción} & Obtener carreras, niveles, materias, grupos y horarios desde el portal oficial y PDFs, estructurando datos para su uso interno. \\
\textbf{Criterios de aceptación} & \begin{itemize}[leftmargin=*]
  \item CA-001: Se extraen códigos, nombres y URL de PDF por carrera.
  \item CA-002: Del PDF se obtienen grupo, tipo, docente, día, hora y aula.
\end{itemize} \\
\textbf{Dependencias} & RB-001, RB-002, RST-001 \\
\end{requisitoficha}

\begin{requisitoficha}
\textbf{ID} & RF-002 \\
\textbf{Título} & Selección jerárquica carrera→nivel→materia→grupo \\
\textbf{Prioridad} & Must \\
\textbf{Descripción} & Permitir navegar jerárquicamente para conformar el conjunto de materias y grupos y persistir la selección. \\
\textbf{Criterios de aceptación} & \begin{itemize}[leftmargin=*]
  \item CA-003: Se muestra jerarquía y estado de disponibilidad.
  \item CA-004: La selección persiste localmente.
\end{itemize} \\
\textbf{Dependencias} & RB-003, RNF-003 \\
\end{requisitoficha}

\begin{requisitoficha}
\textbf{ID} & RF-003 \\
\textbf{Título} & Generación de horarios candidatos sin conflictos \\
\textbf{Prioridad} & Must \\
\textbf{Descripción} & Generar N horarios candidatos usando backtracking y estrategias de evaluación. \\
\textbf{Criterios de aceptación} & \begin{itemize}[leftmargin=*]
  \item CA-005: Sin solapamientos cuando “aceptar conflictos” = falso.
  \item CA-006: Resultados ordenados por puntaje (mejor primero).
\end{itemize} \\
\textbf{Dependencias} & RB-001, RB-004, RNF-002 \\
\end{requisitoficha}

\begin{requisitoficha}
\textbf{ID} & RF-004 \\
\textbf{Título} & Configuración de criterios de generación \\
\textbf{Prioridad} & Must \\
\textbf{Descripción} & Configurar minimizar huecos, minimizar días, priorizar docentes favoritos, aceptar conflictos y número de resultados. \\
\textbf{Criterios de aceptación} & \begin{itemize}[leftmargin=*]
  \item CA-007: Los parámetros influyen en la estrategia compuesta.
  \item CA-008: Los resultados reflejan los criterios seleccionados.
\end{itemize} \\
\textbf{Dependencias} & RB-005, RNF-002 \\
\end{requisitoficha}

\begin{requisitoficha}
\textbf{ID} & RF-005 \\
\textbf{Título} & Visualización semanal y detalles \\
\textbf{Prioridad} & Must \\
\textbf{Descripción} & Mostrar horario semanal por días con color/emoji por materia y detalles de grupo, aula y docente. \\
\textbf{Criterios de aceptación} & \begin{itemize}[leftmargin=*]
  \item CA-009: La vista por días ordena bloques por hora de inicio.
  \item CA-010: Colores/emojis se asocian de forma determinística por materia.
\end{itemize} \\
\textbf{Dependencias} & RNF-001 \\
\end{requisitoficha}

\begin{requisitoficha}
\textbf{ID} & RF-006 \\
\textbf{Título} & Edición manual y eventos académicos \\
\textbf{Prioridad} & Should \\
\textbf{Descripción} & Cambiar grupo, agregar/eliminar materias y crear eventos personales. \\
\textbf{Criterios de aceptación} & \begin{itemize}[leftmargin=*]
  \item CA-011: Cambiar grupo valida conflictos antes de aplicar.
  \item CA-012: Eventos se integran en la vista semanal.
\end{itemize} \\
\textbf{Dependencias} & RB-004 \\
\end{requisitoficha}

\begin{requisitoficha}
\textbf{ID} & RF-007 \\
\textbf{Título} & Exportación en múltiples formatos \\
\textbf{Prioridad} & Should \\
\textbf{Descripción} & Exportar horario a Excel, PDF, imagen y JSON. \\
\textbf{Criterios de aceptación} & \begin{itemize}[leftmargin=*]
  \item CA-013: La exportación incluye materia, grupo, día, hora, aula y docente.
  \item CA-014: El JSON preserva datos suficientes para importación.
\end{itemize} \\
\textbf{Dependencias} & RNF-004 \\
\end{requisitoficha}

\begin{requisitoficha}
\textbf{ID} & RF-008 \\
\textbf{Título} & Importación desde JSON \\
\textbf{Prioridad} & Should \\
\textbf{Descripción} & Importar horarios compartidos, validar compatibilidad y fusionar sin duplicados. \\
\textbf{Criterios de aceptación} & \begin{itemize}[leftmargin=*]
  \item CA-015: Validación de esquema y versión antes de importar.
  \item CA-016: Fusión sin duplicados con datos existentes.
\end{itemize} \\
\textbf{Dependencias} & RB-003, RNF-004 \\
\end{requisitoficha}

\begin{requisitoficha}
\textbf{ID} & RF-009 \\
\textbf{Título} & Notificaciones de clases y eventos \\
\textbf{Prioridad} & Must \\
\textbf{Descripción} & Notificaciones con anticipación configurable para clases y eventos, con ejecución en segundo plano. \\
\textbf{Criterios de aceptación} & \begin{itemize}[leftmargin=*]
  \item CA-017: Si habilitado y dentro del umbral, se muestra notificación.
  \item CA-018: Programación diferida se ejecuta en background.
\end{itemize} \\
\textbf{Dependencias} & RNF-005 \\
\end{requisitoficha}

\begin{requisitoficha}
\textbf{ID} & RF-010 \\
\textbf{Título} & Sincronización multi-dispositivo y backup \\
\textbf{Prioridad} & Should \\
\textbf{Descripción} & Sincronizar datos entre dispositivos y habilitar backups locales. \\
\textbf{Criterios de aceptación} & \begin{itemize}[leftmargin=*]
  \item CA-019: Cambios locales se reflejan en otros dispositivos.
  \item CA-020: Exportación y restauración JSON funcionan offline.
\end{itemize} \\
\textbf{Dependencias} & RST-002 \\
\end{requisitoficha}

\begin{requisitoficha}
\textbf{ID} & RF-011 \\
\textbf{Título} & Actualización automática de datos \\
\textbf{Prioridad} & Should \\
\textbf{Descripción} & Verificar periódicamente cambios en la oferta y actualizar, notificando al usuario. \\
\textbf{Criterios de aceptación} & \begin{itemize}[leftmargin=*]
  \item CA-021: Al cambiar la versión de datos, se refrescan y marca última sincronización.
  \item CA-022: Se emite notificación de actualización exitosa.
\end{itemize} \\
\textbf{Dependencias} & RB-006, RNF-002 \\
\end{requisitoficha}

\begin{requisitoficha}
\textbf{ID} & RF-012 \\
\textbf{Título} & Recordar carrera y materias aprobadas \\
\textbf{Prioridad} & Should \\
\textbf{Descripción} & Persistir carrera seleccionada y estado de aprobación para filtrar materias pendientes. \\
\textbf{Criterios de aceptación} & \begin{itemize}[leftmargin=*]
  \item CA-023: El estado de aprobación se refleja en filtros de selección.
  \item CA-024: Se recuerda la carrera activa al volver a la aplicación.
\end{itemize} \\
\textbf{Dependencias} & RB-003 \\
\end{requisitoficha}

\begin{requisitoficha}
\textbf{ID} & RF-013 \\
\textbf{Título} & Profesores favoritos y priorización \\
\textbf{Prioridad} & Could \\
\textbf{Descripción} & Permitir marcar docentes favoritos e incorporar la preferencia en la evaluación de horarios. \\
\textbf{Criterios de aceptación} & \begin{itemize}[leftmargin=*]
  \item CA-025: Penalización para docentes no favoritos cuando la opción esté activa.
\end{itemize} \\
\textbf{Dependencias} & RB-007 \\
\end{requisitoficha}

\subsection*{Requerimientos no funcionales (RNF)}
\textbf{ID:} RNF-001\\
\textbf{Categoría:} Usabilidad\\
\textbf{Criterio:} La vista semanal ordena por hora y utiliza codificación visual (colores/emojis) consistente y legible.\\

\textbf{ID:} RNF-002\\
\textbf{Categoría:} Rendimiento\\
\textbf{Criterio:} Debe generar al menos 5 horarios candidatos en menos de 5 segundos para 5–8 materias en un dispositivo Android de referencia (API 29).\\

\textbf{ID:} RNF-003\\
\textbf{Categoría:} Compatibilidad\\
\textbf{Criterio:} Compatibilidad con Android minSdk 24 y targetSdk 35.\\

\textbf{ID:} RNF-004\\
\textbf{Categoría:} Portabilidad/Interoperabilidad\\
\textbf{Criterio:} Exportación/Importación en formato JSON con esquema estable y validación de versión.\\

\textbf{ID:} RNF-005\\
\textbf{Categoría:} Fiabilidad\\
\textbf{Criterio:} Las notificaciones programadas deben ejecutarse aun con la aplicación en segundo plano o tras reinicio del dispositivo.\\

\subsection*{Reglas de negocio (RB)}
\textbf{ID:} RB-001\\ \textbf{Regla:} \texttt{startTime < endTime} para cada sesión.\\
\textbf{ID:} RB-002\\ \textbf{Regla:} Docente y aula pueden ser nulos; no bloquean el alta de horarios.\\
\textbf{ID:} RB-003\\ \textbf{Regla:} Unicidad de \texttt{Subject.code} y de \texttt{(subjectCode, groupId)}.\\
\textbf{ID:} RB-004\\ \textbf{Regla:} Prohibir solapamientos salvo \textit{aceptar conflictos}=true.\\
\textbf{ID:} RB-005\\ \textbf{Regla:} La evaluación compuesta refleja preferencias del usuario.\\
\textbf{ID:} RB-006\\ \textbf{Regla:} Verificar cambios de oferta con una periodicidad predefinida y notificar.\\
\textbf{ID:} RB-007\\ \textbf{Regla:} Priorizar docentes marcados como favoritos.\\

\subsection*{Restricciones (RST)}
\textbf{ID:} RST-001\\ \textbf{Restricción:} Dependencia de estructura de PDFs oficiales; variaciones pueden requerir ajustar parsing.\\
\textbf{ID:} RST-002\\ \textbf{Restricción:} Requiere conectividad para sincronización remota y scraping.\\
\textbf{ID:} RST-003\\ \textbf{Restricción:} Plataforma objetivo Android (sin iOS/PWA en alcance inicial).\\

\subsection*{Supuestos (SUP)}
\textbf{ID:} SUP-001\\ \textbf{Supuesto:} Disponibilidad de PDFs y página oficial con estructura estable por semestre.\\
\textbf{ID:} SUP-002\\ \textbf{Supuesto:} Adopción mayoritaria de Android en la población objetivo.\\

\subsection*{Casos de uso (UC) resumidos}
% Caso de uso (resumido)
\textbf{ID:} UC-001\\ \textbf{Actor:} Sistema\\ \textbf{Objetivo:} Scraping y parsing de datos académicos\\ \textbf{Flujo básico:} Conecta a UMSS → extrae URLs PDF → descarga → parsea → persiste\\ \textbf{Pre/Post:} Datos estructurados en BD\\

\textbf{ID:} UC-002\\ \textbf{Actor:} Estudiante\\ \textbf{Objetivo:} Generar horarios optimizados\\ \textbf{Flujo básico:} Configura parámetros → ejecuta generación → revisa candidatos → aplica\\ \textbf{Pre/Post:} Horario aplicado\\

\textbf{ID:} UC-003\\ \textbf{Actor:} Estudiante\\ \textbf{Objetivo:} Configurar notificaciones previas a clase\\ \textbf{Flujo básico:} Abre configuración → define anticipación → guarda\\ \textbf{Pre/Post:} Preferencia persistida; tareas programadas\\

\textbf{ID:} UC-004\\ \textbf{Actor:} Estudiante\\ \textbf{Objetivo:} Seleccionar carrera/nivel/materias/grupos\\ \textbf{Flujo básico:} Navega jerarquía y selecciona\\ \textbf{Pre/Post:} Materias/grupos persistidos\\

\textbf{ID:} UC-005\\ \textbf{Actor:} Estudiante\\ \textbf{Objetivo:} Visualizar y editar horario\\ \textbf{Flujo básico:} Abre pantalla principal → vista semanal → cambia grupo o añade evento\\ \textbf{Pre/Post:} Horario actualizado\\

\textbf{ID:} UC-006\\ \textbf{Actor:} Estudiante\\ \textbf{Objetivo:} Exportar/Importar horario\\ \textbf{Flujo básico:} Elige formato → exporta → comparte; importa JSON → valida → fusiona\\ \textbf{Pre/Post:} Archivo generado/horario integrado\\

\textbf{ID:} UC-007\\ \textbf{Actor:} Sistema\\ \textbf{Objetivo:} Sincronización/Auto-actualización\\ \textbf{Flujo básico:} Detecta cambios → actualiza datos → notifica\\ \textbf{Pre/Post:} Datos actualizados\\

\subsection*{Diccionario de datos esencial}
Entidades y campos esenciales: Carrera (código, nombre, URL, fecha de actualización), Nivel (id, nombre), Materia (código, nombre, estados: aprobada, optativa, activa), Grupo (materia, nivel, id de grupo, tipo, modalidad, estado), Horario de grupo (día, inicio, fin, aula, docente, estado), Docente (nombre completo, favorito), Aula (id, nombre). Reglas de unicidad y claves foráneas aseguran integridad referencial.

\subsection*{Trazabilidad inicial RQ→UC→Artefactos→Pruebas}
\begin{itemize}
  \item RF-001 → UC-001 → Artefacto: Obtención y normalización de datos → Pruebas: TC-004, TC-005
  \item RF-002 → UC-004 → Artefacto: Flujo de selección jerárquica → Pruebas: TC-010, TC-011
  \item RF-003, RF-004 → UC-002 → Artefacto: Generador y evaluación de horarios → Pruebas: TC-001, TC-002
  \item RF-005 → UC-005 → Artefacto: Visualización semanal → Pruebas: TC-008, TC-012
  \item RF-006 → UC-005 → Artefacto: Edición y eventos → Pruebas: TC-013
  \item RF-007, RF-008 → UC-006 → Artefacto: Exportación/Importación → Pruebas: TC-009, TC-014
  \item RF-009 → UC-003 → Artefacto: Notificaciones programadas e inmediatas → Pruebas: TC-006
  \item RF-010, RF-011 → UC-007 → Artefacto: Sincronización y auto-actualización → Pruebas: TC-007
  \item RF-012 → UC-004 → Artefacto: Persistencia de carrera y aprobadas → Pruebas: TC-015
  \item RF-013 → UC-002 → Artefacto: Priorización de docentes favoritos → Pruebas: TC-016
\end{itemize}

\subsection*{Catálogo de pruebas (TC) resumido}
\begin{itemize}
  \item TC-001: Evaluación de huecos mínimos devuelve 0 para horarios sin huecos.
  \item TC-002: Estrategia compuesta refleja parámetros de minimización y priorización.
  \item TC-004: Parsing de línea de horario simple extrae todos los campos.
  \item TC-005: El parser soporta formatos de grupo y día variados.
  \item TC-006: Notificación inmediata/programada respeta anticipación configurada.
  \item TC-007: Auto-actualización respeta periodicidad y actualiza marcadores.
  \item TC-008: La vista semanal ordena por hora y por día.
  \item TC-009: Exportación/Importación JSON mantiene integridad (round-trip).
  \item TC-010: Selección jerárquica persiste decisión de carrera.
  \item TC-011: Selección de nivel/materia/grupo refleja disponibilidad.
  \item TC-012: Visualización muestra color/emoji determinístico por materia.
  \item TC-013: Cambio de grupo detecta y evita conflictos.
  \item TC-014: Exportación PDF/imagen contiene cabecera y bloques correctos.
  \item TC-015: Filtrado por materias aprobadas excluye asignaturas completadas.
  \item TC-016: Priorización de docentes favorece favoritos en los resultados.
\end{itemize}

\subsection*{Entregables de la fase}
\begin{itemize}
  \item Catálogo RF/RNF/RB/RST/SUP.
  \item Casos de uso (UC-001..UC-007).
  \item Trazabilidad inicial RQ→UC→Artefactos→Pruebas.
\end{itemize}

\section{Fase 2: Diseño}
\subsection*{Objetivo}
Definir arquitectura, modelos de datos, flujos de UI y mapeo RF→componentes con base en ADRs y artefactos existentes.

% (Sin listado de entradas; a continuación se documenta el diseño aplicado)

\subsection*{Decisiones arquitectónicas}
\begin{itemize}
\item Clean Architecture con capas de Presentación, Dominio y Datos.
\item Estrategias de evaluación con patrón Strategy y combinación compuesta según preferencias.
\item Persistencia relacional con relaciones many-to-many y repositorios.
\item Inyección de dependencias y ejecución en segundo plano para tareas periódicas.
\end{itemize}

\subsection*{Modelos y flujos}
\begin{itemize}
\item Modelo ER: Carrera, Materia, Nivel, Grupo, Horario de Grupo, Docente, Aula y tablas de asociación.
\item Flujos UI: bienvenida, configuración inicial, selección jerárquica, generación y resultados.
\end{itemize}

\subsection*{Mapeo RF→Componentes}
\begin{itemize}
  \item RF-001 → Servicios de obtención y parsing de datos.
  \item RF-002, RF-012 → Repositorios y pantallas de selección.
  \item RF-003, RF-004, RF-013 → Generador de horarios y estrategias de evaluación.
  \item RF-005, RF-006 → Vista semanal y componentes de edición/calendario.
  \item RF-007, RF-008 → Módulos de exportación e importación.
  \item RF-009 → Servicio de notificaciones y tareas programadas.
  \item RF-010, RF-011 → Casos de uso de sincronización automática y preferencias.
\end{itemize}

\subsection*{Entregables}
\begin{itemize}
  \item Diagrama de arquitectura y capas.
  \item Decisiones de diseño y motivaciones.
  \item Modelo ER y relaciones clave.
\end{itemize}

\section{Fase 3: Implementación}
\subsection*{Objetivo}
Construir los módulos priorizando los RF Must/Should y habilitar evidencias verificables.

\subsection*{Funcionalidades por RF}
\begin{itemize}
  \item RF-001 Scraping/Parsing: Implementado.
  \item RF-002 Selección jerárquica: Implementado.
  \item RF-003/004 Generación y criterios: Implementado.
  \item RF-005 Visualización semanal: Implementado.
  \item RF-006 Edición/eventos: Parcial (pendiente consolidar comandos de edición y eventos avanzados).
  \item RF-007/008 Exportación/Importación: Parcial (pendiente validación de compatibilidad entre versiones).
  \item RF-009 Notificaciones: Implementado.
  \item RF-010/011 Sincronización/AutoSync: Implementado.
  \item RF-012 Recordar carrera/aprobadas: Implementado.
  \item RF-013 Profesores favoritos: Implementado.
\end{itemize}

\subsection*{Cobertura RF: estado}
\begin{itemize}
  \item Implementado: RF-001, RF-002, RF-003, RF-004, RF-005, RF-009, RF-010, RF-011, RF-012, RF-013
  \item Parcial: RF-006, RF-007, RF-008 \,(ver acciones siguientes)
\end{itemize}

\subsection*{Entregables}
\begin{itemize}
  \item Módulos de obtención de datos, generación de horarios, notificaciones y sincronización operativos.
\end{itemize}

\section{Fase 4: Pruebas}
\subsection*{Estrategia y cobertura}
\begin{itemize}
  \item Pirámide de testing con énfasis en unitarios de algoritmos, integración de repositorios y pruebas de UI en flujos críticos.
  \item Cobertura orientada a criterios de aceptación (CA) y casos de uso (UC) prioritarios.
\end{itemize}

\subsection*{Matriz RF→TC (criterios de aceptación)}
\begin{itemize}
  \item RF-001 → TC-004, TC-005: Parsing correcto y robusto ante variaciones.
  \item RF-003 → TC-001: Cálculo de huecos; TC-002: Estrategia compuesta correcta.
  \item RF-004 → TC-002: Parámetros reflejados en la evaluación.
  \item RF-005 → TC-008: Orden por hora en vista semanal; TC-012: codificación visual consistente.
  \item RF-009 → TC-006: Notificaciones inmediatas/programadas respetan anticipación.
  \item RF-010/011 → TC-007: Auto-actualización periódica con notificación.
  \item RF-007/008 → TC-009, TC-014: Round-trip JSON y contenido correcto en exportación.
\end{itemize}

\subsection*{Resultados y hallazgos}
\begin{itemize}
  \item Algoritmos: estrategias y backtracking verificados con pruebas unitarias.
  \item Parsing: pruebas de resistencia ante variaciones de formato.
  \item UI: plan de pruebas para selección y visualización completado; ejecución en progreso.
  \item Exportación/Importación: verificación de integridad y compatibilidad en ejecución.
\end{itemize}

\section{Fase 5: Despliegue y CI/CD}
\subsection*{Pipeline y distribución}
\begin{itemize}
  \item Configuración de build (min/target SDK, flavors, ofuscación y reducción de recursos).
  \item Integración continua: ejecución de pruebas unitarias, de integración y de UI.
  \item Publicación: empaquetado de APK/AAB y despliegue en canales de distribución.
\end{itemize}

\subsection*{Checklist de release}
\begin{itemize}
  \item Todos los tests pasan; cobertura mínima de 75\%.
  \item Artefactos firmados y notas de versión actualizadas.
  \item Validaciones de calidad estática sin advertencias críticas.
\end{itemize}

\subsection*{Entregables}
\begin{itemize}
  \item Pipeline CI/CD operativo con políticas de calidad y publicación.
\end{itemize}

\section{Fase 6: Mantenimiento y Mejoras}
\subsection*{Retroalimentación e iteraciones}
\begin{itemize}
  \item Próximas mejoras: sincronización periódica en background y configuración de intervalo.
  \item Deuda técnica: fortalecer pruebas de UI en selección/visualización; completar pruebas de exportación/importación.\\ Acción: implementar \textit{TC-008}, \textit{TC-009}.
  \item Riesgos operativos: cambios en formatos PDF (RST-001) → mantener parser y pruebas de regresión.
\end{itemize}

\subsection*{Próximos pasos}
\begin{itemize}
  \item Benchmark de rendimiento en dispositivo objetivo (RNF-002).
  \item Telemetría de generación y scraping para monitoreo y mejora continua.
\end{itemize}

\section*{Cierre del capítulo}
Se ejecutó un proceso iterativo–incremental en seis fases, generando un catálogo completo y trazable de requisitos, decisiones de diseño coherentes, implementación de componentes clave y una estrategia de pruebas y despliegue integrada. La preparación para el Capítulo V contempla síntesis de logros por fase, hallazgos de pruebas y recomendaciones de mejora continua.

% FIN DE 04_metodologia.tex

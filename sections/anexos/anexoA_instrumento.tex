\noindent Este anexo documenta el cuestionario completo aplicado para la validación con usuarios. El formulario original permanece disponible en línea en \href{https://docs.google.com/forms/d/e/1FAIpQLSeOcCIylkkdKV0ngACov69p9WMiCIVS2e3FRMKTxzrtodeZWw/viewform?usp=sharing&ouid=105927098657367867554}{Google Forms}. Todas las preguntas fueron marcadas como opcionales salvo la aceptación del consentimiento informado.

\begin{enumerate}[label=\arabic*.]
  \item \textbf{Elige el estilo de Simón}.\\
  Opciones: Simón 1v; Simón 2v.

  \item \textbf{¿Cuál es tu programa de grado?}\\
  Opciones: Ingeniería de Sistemas; Ingeniería en Informática; Ingeniería Civil; Ingeniería Electrónica; Ingeniería de Alimentos; Ingeniería Química; Ingeniería Matemática; Otro.

  \item \textbf{¿En qué semestre o nivel te encuentras actualmente?}\\
  Opciones: 1er; 2do; 3ro; 4to; 5to; 6to; 7mo; Más de 7mo.

  \item \textbf{¿Cuál es el sistema operativo principal de tu teléfono?}\\
  Opciones: Android; iOS; Otro.

  \item \textbf{¿Con qué frecuencia utilizas el sistema Capuchino?}\\
  Opciones: Solo durante la inscripción; Varias veces por semestre; Raramente; Nunca lo he usado.

  \item \textbf{¿Qué es lo que más te gusta del sistema Capuchino?}\\
  Respuesta abierta.

  \item \textbf{¿Qué es lo que menos te gusta del sistema Capuchino?}\\
  Respuesta abierta.

  \item \textbf{Si pudieras mejorar algo en Capuchino, ¿qué sería?}\\
  Respuesta abierta.

  \item \textbf{¿Desde qué dispositivo accedes habitualmente a Capuchino?}\\
  Opciones: Laptop; Teléfono Android; iPhone; Tablet.

  \item \textbf{¿Qué tan cómodo es usar Capuchino desde tu teléfono?}\\
  Escala Likert (1--4): Cómodo; Algo cómodo; Incómodo; No lo uso en mi teléfono.

  \item \textbf{¿Qué tan frustrante es seleccionar varias veces (carrera $\rightarrow$ nivel $\rightarrow$ materia $\rightarrow$ grupo) cada vez que ingresas?}\\
  Escala Likert (1--5), donde 1 es ``Nada frustrante'' y 5 es ``Muy frustrante''.

  \item \textbf{¿Utilizas WhatsApp u otros grupos para confirmar horarios de clases o información de docentes?}\\
  Opciones: Sí; No.

  \item \textbf{¿Te gustaría tener una aplicación móvil similar a Capuchino pero más rápida y fácil de usar?}\\
  Opciones: Sí; No; Depende.

  \item \textbf{¿Qué características te gustaría que incluyera la aplicación?}\\
  Selección múltiple: Horarios actualizados automáticamente; Recordar materias aprobadas; Recordatorios de clases; Asistente motivacional; Modo sin conexión; Compartir mi horario con mis compañeros; Otro.

  \item \textbf{¿Qué tan útil sería recibir notificaciones antes de que comiencen tus clases?}\\
  Escala Likert (1--5), donde 1 es ``Nada útil'' y 5 es ``Muy útil''.

  \item \textbf{¿Te gustaría que la aplicación te motivara o acompañara con mensajes personalizados según tu carrera?}\\
  Opciones: Sí; No; Tal vez.

  \item \textbf{¿Te gustaría recibir notificaciones personalizadas de un asistente virtual (llamado Simón)?}\\
  Opciones: Sí; No; Tal vez.

  \item \textbf{¿Qué tan importante es para ti que la aplicación funcione sin conexión?}\\
  Escala Likert (1--5), donde 1 es ``Nada importante'' y 5 es ``Muy importante''.

  \item \textbf{Si existiera una aplicación que gestionara tu horario y te recordara tus clases, ¿la usarías regularmente?}\\
  Opciones: Sí; Tal vez; No.

  \item \textbf{¿Qué nombre suena mejor para esta aplicación?}\\
  Opciones: Simón; MiHorario FCyT; FCyT Planner; Otro.

  \item \textbf{¿Algún comentario o sugerencia adicional?}\\
  Respuesta abierta.

  \item \textbf{Consentimiento informado}.\\
  ``La información recopilada será utilizada exclusivamente para fines académicos relacionados con un proyecto de grado en la FCyT--UMSS. Todas las respuestas son confidenciales y anónimas. ¿Acepta participar en esta encuesta?''\\
  Opciones: Sí; No.
\end{enumerate}

\chapter{ÁREA DE APLICACIÓN}
\label{chap:area_aplicacion}

Este capítulo describe el contexto real de uso de la solución: el dominio institucional, la población objetivo, los procesos actuales (AS-IS) y los procesos propuestos (TO-BE), así como la información y restricciones propias del área. No se detalla la metodología ni el diseño técnico; el propósito es situar el problema en su entorno operativo y establecer límites y criterios de evaluación vinculados al uso real.

\section{DOMINIO Y CONTEXTO INSTITUCIONAL}
La Facultad de Ciencias y Tecnología (FCyT) de la UMSS publica semestralmente los horarios oficiales en formato PDF, organizados por carrera y periodo. Durante la inscripción, el uso de herramientas web como Capuchino es intensivo; fuera de esa etapa, los estudiantes siguen necesitando: (i) consultar horarios personales con rapidez, (ii) ajustar materias ante cambios, y (iii) coordinar con pares mediante mensajería.

\section{POBLACIÓN OBJETIVO Y ACTORES}
\textbf{Actor principal}: estudiantes de la FCyT de distintos semestres y turnos que construyen, consultan y ajustan sus horarios.  
\textbf{Actores secundarios}: personal administrativo que publica/actualiza los horarios y docentes/tutores cuya asignación incide en las preferencias estudiantiles.

\subsection*{Evidencia cuantitativa (encuesta, n=126)}
La encuesta aplicada a estudiantes de FCyT confirma:
\begin{itemize}
  \item \textbf{Dispositivo}: Android domina (94\%); laptop apenas 26\%.  
  \item \textbf{Uso de Capuchino}: 48\% solo en inscripción, 44\% varias veces por semestre.  
  \item \textbf{Coordinación social}: 80\% valida información por WhatsApp.  
  \item \textbf{Frustración por repetir pasos}: 60\% califica 3–5 (alta).  
  \item \textbf{Principales demandas}: actualización automática (88\%), modo sin conexión (78\%), recordatorios (56\%).  
\end{itemize}

\begin{figure}[H]
  \centering
  \begin{minipage}{0.4\textwidth}
    \centering
    \includegraphics[width=\textwidth]{images/diagrams/encuesta_dispositivo.png}
    \caption{Distribución de dispositivos utilizados por los estudiantes (n=126). Fuente: Elaboración propia.}
    \label{fig:encuesta_dispositivo}
  \end{minipage}
  \hfill
  \begin{minipage}{0.4\textwidth}
    \centering
    \includegraphics[width=\textwidth]{images/diagrams/encuesta_uso_capuchino.png}
    \caption{Frecuencia de uso de Capuchino por los estudiantes (n=126). Fuente: Elaboración propia.}
    \label{fig:encuesta_uso_capuchino}
  \end{minipage}
\end{figure}

\begin{figure}[H]
  \centering
  \begin{minipage}{0.4\textwidth}
    \centering
    \includegraphics[width=\textwidth]{images/diagrams/encuesta_whatsapp.png}
    \caption{Coordinación social mediante WhatsApp (n=126). Fuente: Elaboración propia.}
    \label{fig:encuesta_whatsapp}
  \end{minipage}
  \hfill
  \begin{minipage}{0.4\textwidth}
    \centering
    \includegraphics[width=\textwidth]{images/diagrams/encuesta_frustracion.png}
    \caption{Nivel de frustración por repetir pasos (n=126). Fuente: Elaboración propia.}
    \label{fig:encuesta_frustracion}
  \end{minipage}
\end{figure}

\begin{figure}[H]
  \centering
  \includegraphics[width=0.7\textwidth]{images/diagrams/encuesta_demandas.png}
  \caption{Principales demandas de los estudiantes (n=126). Fuente: Elaboración propia.}
  \label{fig:encuesta_demandas}
\end{figure}

\section{PERCEPCIONES DEL ESTADO ACTUAL (AS-IS)}
Los estudiantes valoran la rapidez de Capuchino para organizar horarios y detectar choques, pero reportan problemas recurrentes: actualización tardía de datos, interfaz obsoleta, pérdida del progreso, lentitud en inscripciones y baja personalización.  
\textbf{Figura~\ref{fig:asis_proceso}} muestra el flujo actual y las limitaciones detectadas.

\begin{figure}[H]
  \centering
  \includegraphics[width=0.35\textwidth]{images/diagrams/area_asis_proceso.png}
  \caption{Proceso actual (AS-IS): navegación y patrones de uso. Fuente: Elaboración propia.}
  \label{fig:asis_proceso}
\end{figure}

\section{DEMANDA DE SOLUCIÓN Y PROCESOS PROPUESTOS (TO-BE)}
\textbf{Visión general:} los resultados evidencian la necesidad de una aplicación móvil que reduzca pasos repetitivos, funcione sin conexión y facilite la coordinación social.  
\textbf{Figura~\ref{fig:area_tobe}} resume los flujos de uso esperados.

\begin{figure}[H]
  \centering
  \includegraphics[width=0.35\textwidth]{images/diagrams/area_tobe_proceso.png}
  \caption{Procesos propuestos (TO-BE) a nivel de uso. Fuente: Elaboración propia.}
  \label{fig:area_tobe}
\end{figure}

\textbf{Características clave de uso:}
\begin{itemize}
  \item Generación de horarios sin choques, priorizando materias obligatorias y docentes favoritos.
  \item Memoria de contexto para evitar repetir carrera/nivel/materias.
  \item Intercambio estructurado (JSON) compatible con mensajería.
  \item Consulta diaria y recordatorios antes de clase.
\end{itemize}

\section{INFORMACIÓN Y REGLAS DEL ÁREA}
\textbf{Entidades}: carrera, nivel, materia, grupo, horario, docente, aula y selección del estudiante.  
\textbf{Reglas principales}:
\begin{itemize}
  \item Sin solapamientos salvo habilitación explícita.
  \item Unicidad de códigos y consistencia jerárquica.
  \item Tolerancia a datos incompletos (aula/docente pendientes).
  \item Actualización por sincronización con la fuente oficial.
\end{itemize}

\section{DATOS Y FLUJO DE INFORMACIÓN}
Los horarios en PDF son la fuente oficial. Se descargan, extraen y almacenan localmente; se verifican y actualizan de forma controlada. El intercambio social se realiza mediante archivos estructurados (JSON) para compartir horarios.  
Véase \textbf{Figura~\ref{fig:area_flujo_datos}}.

\begin{figure}[H]
  \centering
  \includegraphics[width=0.35\textwidth]{images/diagrams/area_flujo_datos.png}
  \caption{Flujo de información en el área de aplicación. Fuente: Elaboración propia.}
  \label{fig:area_flujo_datos}
\end{figure}

\section{ALCANCES, LÍMITES Y RIESGOS}
\textbf{Alcances:} estudiantes FCyT; integración con horarios oficiales; uso sin conexión; recordatorios y consulta rápida.  
\textbf{Límites:} no reemplaza inscripción oficial ni modela capacidad de aulas; foco inicial en Android.  
\textbf{Riesgos y mitigación:}
\begin{itemize}
  \item Cambios en formato o ubicación de PDFs → proceso flexible y sincronización manual.  
  \item Retrasos oficiales → mantener caché y mostrar fecha de actualización.  
  \item Conectividad limitada → operación offline-first.  
  \item Saturación de fuente → reintentos espaciados.  
\end{itemize}

\section{DERIVACIÓN DE REQUERIMIENTOS DESDE LA ENCUESTA}
\begin{table}[H]
  \centering
  \caption{Trazabilidad de hallazgos $\rightarrow$ requerimientos del área. Fuente: Elaboración propia (encuesta n=126).}
  \label{tab:derivacion_req_area}
  \small
  \begin{tabular}{|p{5.0cm}|p{7.5cm}|p{3.5cm}|}
    \hline
    Hallazgo & Requerimiento derivado & Criterio de aceptación \\
    \hline
    94\% usa Android & Priorizar Android y vista móvil & Flujo usable sin laptop \\
    \hline
    60\% frustración por repetir pasos & Memoria de carrera/nivel/materias & Ingreso sin repetir selección \\
    \hline
    80\% coordina por WhatsApp & Intercambio estructurado (JSON) & Compartir/abrir desde mensajería \\
    \hline
    78\% valora modo sin conexión & Operación offline-first & Consulta sin red \\
    \hline
    57\% valora notificaciones & Recordatorios previos a clase & Alertas configurables \\
    \hline
    88\% solicita actualización automática & Actualización desde fuente oficial & Fecha visible de actualización \\
    \hline
  \end{tabular}
\end{table}

\section{CRITERIOS DE ÉXITO E IMPACTO ESPERADO}
Reducción de pasos para construir horarios sin choques; continuidad de uso fuera del periodo de inscripción; disminución de choques no detectados; incremento del intercambio estructurado; y mejor organización académica personal.

\section*{SÍNTESIS}
El área de aplicación se ubica entre la oferta oficial (PDFs FCyT), las prácticas actuales de consulta y la coordinación estudiantil por mensajería. Las brechas identificadas justifican una solución complementaria centrada en el estudiante, con consulta móvil eficaz, uso sin conexión y respeto por la fuente oficial.

% 03_area_aplicacion.tex

\chapter{Área de Aplicación}
\label{chap:area_aplicacion}

% Nota de edición:
% Este capítulo se estructura para cumplir Estructura.md: describir el proceso manual (AS-IS),
% el estudio de campo y la información del dominio necesaria. Los detalles técnicos profundos
% (arquitectura, librerías, patrones) se tratan en el Capítulo IV (Metodología/Desarrollo).

\section{Contexto y alcance del área}
% Describir brevemente FCyT–UMSS, publicación semestral de horarios por carrera y cómo los estudiantes
% usan Capuchino durante inscripción. Dejar claro el alcance del área: desde publicación oficial
% hasta consulta diaria del estudiante.
\textbf{TODO:} Contexto institucional (FCyT--UMSS), dinámica de inscripción, rol de Capuchino.

\section{Proceso manual actual (AS-IS: Capuchino)}
% Detallar el flujo real que realiza un estudiante usando Capuchino: bandeja de carreras,
% navegación por niveles, materias y grupos; visualización semanal a la derecha; selección/deselección.
% Incluir apoyos informales (grupos de WhatsApp) y qué produce el flujo (PDF/captura).
\subsection{Acceso y navegación}
\textbf{TODO:} Describir la ``bandeja'' con todas las carreras, selección de carrera $\rightarrow$ nivel $\rightarrow$ materia $\rightarrow$ grupo.

\subsection{Selección y visualización del horario}
\textbf{TODO:} Explicar la vista semanal (días/horas) y cómo aparecen los bloques al elegir grupos. Indicar selección/deselección.

\subsection{Apoyos informales y salida del proceso}
\textbf{TODO:} Mencionar uso de WhatsApp u otros medios para confirmar opciones. Resultado típico: PDF o captura compartible.

\subsection{Limitaciones observadas}
\begin{itemize}
  \item No responsivo en móviles; experiencia pobre en smartphone.
  \item No recuerda carrera ni materias aprobadas; se repite el proceso en cada visita.
  \item Sin notificaciones ni acompañamiento tras la inscripción (útil solo en la ventana de inscripción).
  \item Sin actualización personalizada automática para el estudiante.
\end{itemize}

\section{Problemas y oportunidades}
% Sintetizar el problema central y las oportunidades de mejora.
\textbf{TODO:} Problema central (vacío posinscripción); oportunidades (experiencia móvil, personalización, actualización, recordatorios).

\section{Estudio de campo (plan)}
% Este capítulo contiene el plan del estudio que sustenta requerimientos. Los resultados se reportarán más adelante.
\subsection{Objetivo}
\textbf{TODO:} Medir uso/satisfacción de Capuchino y necesidad de app móvil complementaria.

\subsection{Población y muestra}
\textbf{TODO:} Estudiantes FCyT (Android mayoritario); criterios de inclusión y tamaño muestral.

\subsection{Método e instrumento}
\textbf{TODO:} Encuesta estructurada (frecuencia de uso, dispositivo, satisfacción, pain points, interés en app).

\subsection{Variables y dimensiones}
\textbf{TODO:} Usabilidad móvil, persistencia de datos, necesidad de notificaciones, actualización de horarios, posinscripción.

\subsection{Consideraciones}
\textbf{TODO:} La app no reemplaza Capuchino; lo complementa. Consideraciones éticas y de anonimato.

\section{Información del dominio}
% Estructurar los datos que requiere el software, su origen y reglas de integridad.
\subsection{Entidades y catálogos}
\begin{itemize}
  \item Carreras: código, nombre, URL del PDF oficial, fecha de actualización.
  \item Niveles: identificadores jerárquicos (p.\,ej., ``A'', ``1er semestre'').
  \item Materias: código único, nombre, estados (aprobada, optativa, activa).
  \item Grupos: por materia/nivel; id de grupo, tipo (T/TP/P/L), modalidad, estado.
  \item Horarios de grupo: día, hora inicio/fin, aula, docente, estado.
  \item Docentes: nombre completo, preferencia (favorito).
  \item Aulas: identificador alfanumérico único (p.\,ej., ``B-201'').
\end{itemize}

\subsection{Reglas de negocio y restricciones}
\begin{itemize}
  \item No solapamiento en horario individual; $\text{inicio} < \text{fin}$.
  \item Unicidad: código de materia; $($materia, grupo$)$; id de aula.
  \item Consistencia jerárquica: el grupo pertenece a una materia y a un nivel de una carrera.
  \item Disponibilidad: solo grupos activos generan sesiones.
\end{itemize}

\subsection{Casos especiales}
\begin{itemize}
  \item Cambio de aula/horario (reposiciones, avisos tardíos).
  \item Datos incompletos en fuente oficial (docente/aula nulos).
  \item Conflictos aceptados explícitamente por el usuario para análisis.
\end{itemize}

\subsection{Diccionario de datos esencial}
% Enumerar campos mínimos por entidad (clave para implementación en Cap. IV).
\textbf{TODO:} Completar con campos pactados por entidad (Carrera, Materia, Grupo, Horario, Docente, Aula) y su fuente.

\section{Requerimientos funcionales (derivados)}
\begin{itemize}
  \item Consulta \emph{offline} del horario y datos seleccionados.
  \item Recordar carrera y materias aprobadas; filtrado de materias pendientes.
  \item Generación de horarios candidatos sin conflictos; criterios (minimizar huecos, priorizar docentes).
  \item Notificaciones previas a clases y gestión de eventos académicos.
  \item Asistente ``Simón'' (IA) para notificaciones motivacionales/personalizadas. % Detalle técnico en Cap. IV
  \item Actualización automática desde la fuente oficial y alerta de cambios.
  \item Exportación/compartición (PDF/imagen/JSON) e importación de horarios compartidos.
\end{itemize}

\section{Requerimientos no funcionales}
\begin{itemize}
  \item Usabilidad móvil y accesibilidad; compatibilidad Android (API objetivo y mínima).
  \item Operación \emph{offline} confiable y rendimiento en dispositivo.
  \item Privacidad y manejo responsable de datos del estudiante.
\end{itemize}

\section{Alcances y exclusiones}
\begin{itemize}
  \item Alcances: estudiantes FCyT; clases y exámenes oficiales; plataforma Android.
  \item Exclusiones: inscripción directa; reemplazo de Capuchino; iOS/PWA inicial; funciones administrativas.
\end{itemize}

\section{Supuestos y riesgos}
\begin{itemize}
  \item Supuestos: disponibilidad y estructura estable de PDFs oficiales; adopción Android en la comunidad.
  \item Riesgos: cambios de formato; retrasos de publicación; datos incompletos; aceptación del componente de IA.
\end{itemize}

\section{Métricas e impacto esperado}
% Proponer indicadores para evaluar beneficios.
\textbf{TODO:} Tiempo de armado de horario, reducción de conflictos, tasa de asistencia (vía recordatorios), satisfacción de usabilidad.

\section{Trazabilidad}
% Relacionar problemas observados con requerimientos y métricas.
\textbf{TODO:} Tabla o texto que mapee: problema $\rightarrow$ requerimiento $\rightarrow$ métrica.

\section{Referencias y anexos}
% Recordatorio de incluir citas a documentos oficiales (PDFs/portal) y anexar capturas/diagramas si corresponde.
\textbf{TODO:} Citas y anexos conforme a normas de la FCyT.

% FIN DE 03_area_aplicacion.tex

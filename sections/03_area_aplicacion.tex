\chapter{MARCO DE APLICACIÓN}
\label{chap:area_aplicacion}

Este capítulo describe el área de aplicación del proyecto, contextualizando el problema que TecnoTime busca resolver en el ámbito de la gestión de horarios académicos de la Facultad de Ciencias y Tecnología. Se presenta el contexto institucional, se identifican los actores involucrados, se describen las herramientas actuales y sus limitaciones, y se fundamenta la necesidad de una solución complementaria.

\section{Contexto institucional de la Facultad de Ciencias y Tecnología}

La Facultad de Ciencias y Tecnología (FCyT) de la UMSS alberga ocho carreras de ingeniería: Informática, Sistemas, Civil, Electromecánica, Electrónica, Química, Alimentos y Licenciatura en Física. Cada carrera organiza su oferta académica en niveles diferenciados por turno (mañana, tarde, noche) y modalidad (presencial, semipresencial) \cite{fcyt_umss}.

Los horarios académicos se publican semestralmente mediante el Sistema Académico de Gestión de Aulas y Ambientes (SAGAA) en formato PDF, como muestra la Figura \ref{fig:portal_sagaa}. Este formato estático genera dificultades en la consulta dinámica, actualización y personalización de información \cite{horarios_fcyt}.

\begin{figure}[H]
  \centering
  \includegraphics[width=0.85\textwidth]{images/sagaa_horarios_pdf.png}
  \caption{Portal SAGAA de horarios de la FCyT}
  \label{fig:portal_sagaa}
  \small{Fuente: \cite{horarios_fcyt}}
\end{figure}

\section{Problemática en la gestión de horarios universitarios}

El sistema actual presenta limitaciones significativas. Los horarios se distribuyen exclusivamente en formato PDF, impidiendo búsqueda dinámica, personalización e integración con herramientas digitales. Los cambios no siempre se comunican efectivamente a los estudiantes afectados.

Según encuesta aplicada a 126 estudiantes \cite{encuesta_fcyt_2025}, el 94\% utiliza dispositivos Android como plataforma principal para consultas académicas. Esta preferencia contrasta con la ausencia de aplicaciones nativas. El 80\% valida cambios mediante grupos de WhatsApp, evidenciando dispersión entre canales formales e informales.

La falta de herramientas móviles adecuadas obliga a repetir procesos de consulta. El 60\% califica como alta su frustración por repetir pasos y perder progreso entre sesiones. Los principales problemas incluyen dispersión de información entre PDF, web y mensajería; falta de funcionamiento offline; dificultad para consolidar horarios personales; y ausencia de recordatorios automáticos.

\section{Actores involucrados y afectación}

La problemática afecta principalmente a tres actores: los estudiantes, quienes requieren consulta frecuente y detección de conflictos; los docentes, afectados por la desinformación estudiantil ante cambios; y el personal administrativo, limitado por la imposibilidad de notificaciones automáticas al publicar actualizaciones.

\section{Herramientas actuales y limitaciones}

Los estudiantes utilizan principalmente el portal SAGAA para acceder a PDFs oficiales y el sistema web Cappuchino para organizar horarios.

\subsection{Portal SAGAA}

El portal SAGAA constituye la fuente oficial de horarios. Los estudiantes deben navegar por carpetas, identificar el PDF correspondiente, descargarlo y buscar manualmente su información entre múltiples páginas. Este proceso se repite cada vez que se requiere verificar cambios, generando pérdida de tiempo y posibles errores por consulta de versiones desactualizadas.

\subsection{Sistema Cappuchino}

Cappuchino es un sistema web desarrollado por SCESI que permite organizar horarios y detectar choques durante el periodo de inscripción \cite{cappuccino_umss}. La Figura \ref{fig:cappuchino_interfaz} muestra la interfaz principal de esta herramienta.

\begin{figure}[H]
  \centering
  \includegraphics[width=0.95\textwidth]{images/cappuchino_web.png}
  \caption{Interfaz web del sistema Cappuchino}
  \label{fig:cappuchino_interfaz}
  \small{Fuente: \cite{cappuccino_umss}}
\end{figure}

Si bien es valorada por su rapidez, presenta limitaciones: funcionamiento óptimo limitado al periodo de inscripción, ausencia de modo sin conexión, falta de personalización post-inscripción y ausencia de recordatorios. Según la encuesta \cite{encuesta_fcyt_2025}, el 48\% la utiliza únicamente durante inscripción, evidenciando que no cubre necesidades de consulta continua.

\subsection{Flujo actual y puntos de fricción}

El proceso típico involucra consultar SAGAA o Cappuchino, generar un horario base, validar con PDFs actualizados, verificar en WhatsApp modificaciones de última hora, ajustar manualmente selecciones y registrar en calendarios personales. La Figura \ref{fig:asis_proceso} resume este proceso, evidenciando los múltiples pasos y herramientas requeridos.

\begin{figure}[H]
  \centering
  \includegraphics[width=0.5\textwidth]{images/diagrams/area_asis_proceso.png}
  \caption{Proceso actual (AS-IS) de gestión de horarios}
  \label{fig:asis_proceso}
  \small{Fuente: \cite{cappuccino_umss}}
\end{figure}

\section{Necesidad de una solución complementaria}

La brecha entre necesidades estudiantiles y capacidades actuales justifica una aplicación móvil complementaria. Los datos evidencian demandas concretas: 88\% solicita actualización automática, 78\% valora modo offline, 56\% requiere recordatorios y 51\% desea compartición estructurada \cite{encuesta_fcyt_2025}.

La solución no pretende reemplazar sistemas oficiales, sino complementarlos facilitando consulta rápida, personalización, uso offline y coordinación social. Los beneficios incluyen reducción de tiempo en consultas repetitivas, disminución de choques no detectados, mejora en coordinación estudiantil y continuidad de uso fuera del periodo de inscripción.

\section{Visión de la solución esperada}

TecnoTime se concibe como herramienta complementaria que permita gestionar horarios eficientemente. Las funcionalidades esperadas incluyen: generación automática sin choques, consulta rápida con vista diaria y semanal, funcionamiento offline, recordatorios configurables, intercambio estructurado con compañeros y sincronización con fuente oficial.

% \section{Alcances y límites}

% El alcance se circunscribe a estudiantes de la FCyT. La solución se enfoca en gestión personal de horarios, integrándose con PDFs oficiales del portal SAGAA, priorizando funcionamiento offline y habilitando intercambio entre estudiantes.

% Los límites son claros: no reemplaza el sistema oficial de inscripción, no modela capacidad de aulas ni cupos en tiempo real, no gestiona trámites administrativos y se enfoca inicialmente en Android. Los principales riesgos incluyen cambios en formato de PDFs, retrasos en publicación oficial y conectividad limitada, mitigados mediante proceso flexible de ingesta, caché local y arquitectura offline-first.

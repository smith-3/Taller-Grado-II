\chapter{MARCO DE APLICACIÓN}
\label{chap:area_aplicacion}

Este capítulo describe el entorno operativo de la Facultad de Ciencias y Tecnología (FCyT) de la Universidad Mayor de San Simón (UMSS), donde se desarrolla el proceso de gestión de horarios académicos. Se detalla el contexto institucional, los actores que intervienen en el proceso, las herramientas utilizadas actualmente y las reglas de negocio que rigen la planificación académica.

\section{Contexto institucional de la Facultad de Ciencias y Tecnología}

La Facultad de Ciencias y Tecnología (FCyT) de la UMSS alberga una amplia oferta académica que incluye las carreras de Alimentos, Biología, Biotecnología, Civil, Eléctrica, Electromecánica, Electrónica, Energía, Física, Industrial, Informática, Matemáticas, Mecánica, Química y Sistemas. Estas se imparten en diversos niveles y planes, abarcando ingenierías y licenciaturas, organizando su oferta académica \cite{fcyt_umss}.

Los horarios académicos se publican semestralmente mediante el Sistema Académico de Gestión de Aulas y Ambientes (SAGAA) en formato PDF. La Figura \ref{fig:portal_sagaa} muestra la interfaz de publicación de estos documentos.

\begin{figure}[H]
  \centering
  \includegraphics[width=0.85\textwidth]{images/sagaa_horarios_pdf.png}
  \caption{Portal SAGAA de horarios de la FCyT}
  \label{fig:portal_sagaa}
  \small{Fuente: \cite{horarios_fcyt}}
\end{figure}

\section{Problemática en el proceso actual}

El proceso de gestión y consulta de horarios presenta características específicas que impactan significativamente a la comunidad universitaria. La información oficial se distribuye exclusivamente en archivos estáticos (PDF), lo que implica que cualquier actualización, corrección de aulas o cambio de docentes depende enteramente de la republicación manual de estos documentos. Esta falta de dinamismo genera una brecha temporal entre la realidad operativa y la información disponible para el estudiante.

Según una encuesta aplicada a 126 estudiantes de la facultad \cite{encuesta_fcyt_2025}, el 94\% de la población estudiantil utiliza dispositivos móviles con sistema operativo Android para sus actividades académicas. En cuanto a la comunicación de cambios en la planificación, el 80\% de los encuestados indica que valida la información a través de grupos de mensajería instantánea (WhatsApp), lo que denota un flujo de información híbrido y poco eficiente que combina canales oficiales estáticos con canales informales dinámicos pero no verificados.

Los estudiantes reportan que el proceso de consolidación de su horario personal es manual y propenso a errores, requiriendo la consulta simultánea de múltiples fuentes dispersas: los PDFs oficiales para verificar aulas, el sistema web de inscripción para registrar materias y los grupos de comunicación para confirmar la asistencia del docente. Esta fragmentación dificulta la obtención de una visión unificada y confiable de sus actividades académicas diarias.

\section{Actores involucrados}

En el entorno de la gestión de horarios intervienen los siguientes actores principales:

\begin{itemize}
    \item Estudiantes: Son los usuarios finales de la información. Deben consultar la oferta académica, planificar su semestre evitando conflictos de horario y mantenerse informados sobre cambios de aulas o docentes.
    \item Personal Administrativo (Jefaturas de Carrera y OPA): Responsables de la planificación, asignación de ambientes y publicación de los horarios oficiales en el portal SAGAA.
\end{itemize}

\section{Herramientas actuales}

Actualmente, el proceso se apoya en dos herramientas institucionales principales: el portal SAGAA y el sistema Cappuchino.

\subsection{Portal SAGAA}

El portal SAGAA (Sistema Académico de Gestión de Aulas y Ambientes) constituye el repositorio oficial de la información académica de la facultad. Su función principal en este contexto es el almacenamiento centralizado y la distribución pública de los horarios en formato PDF. Para acceder a la información, los estudiantes deben navegar a través de una estructura jerárquica, seleccionando primero su facultad, luego su carrera y finalmente el nivel o semestre correspondiente.

Los documentos alojados en este portal representan la "versión oficial" de la planificación académica. Contienen la información definitiva y detallada sobre las materias habilitadas, los grupos disponibles, los docentes asignados y la distribución de aulas. Sin embargo, al tratarse de un repositorio de archivos, no ofrece funcionalidades interactivas de búsqueda, filtrado o notificación automática de cambios.

Aunque la gran mayoría de los estudiantes utiliza herramientas de terceros para visualizar esta información de manera más amigable, existe un grupo de usuarios que, por desconocimiento de alternativas o preferencia por la fuente oficial, accede directamente a estos archivos \cite{encuesta_fcyt_2025}. La Figura \ref{fig:horario_pdf_informatica} muestra un ejemplo del formato de estos documentos, correspondiente a la carrera de Ingeniería Informática.

\begin{figure}[H]
  \centering
  \includegraphics[width=0.95\textwidth]{images/sagaa_horario_informatica.png}
  \caption{Ejemplo de horario en formato PDF (Ingeniería Informática)}
  \label{fig:horario_pdf_informatica}
  \small{Fuente: \cite{horarios_fcyt}}
\end{figure}

La estructura de estos documentos presenta la información en tablas con columnas definidas para el Nivel, Materia (Código y Nombre), Grupo, Tipo, Docente, Día, Hora y Aula. El formato de hora utilizado es numérico continuo, omitiendo los dos puntos separadores (por ejemplo, '945-1115' representa el intervalo de 09:45 a 11:15). De igual manera, los días de la semana se representan mediante abreviaturas de dos letras: LU (Lunes), MA (Martes), MI (Miércoles), JU (Jueves), VI (Viernes) y SA (Sábado).

Un aspecto crítico para la planificación es la modalidad de inscripción de ciertas asignaturas, donde la componente teórica y práctica se gestionan como grupos independientes. En estos casos, el estudiante debe inscribirse obligatoriamente en un grupo de teoría y seleccionar por separado uno de los múltiples grupos de laboratorio disponibles, asegurando que no existan conflictos temporales entre ambos.

Los códigos de tipo de clase permiten distinguir la naturaleza de la actividad:
\begin{itemize}
    \item T (Teoría): Clases exclusivamente teóricas enfocadas en conceptos y principios.
    \item P (Práctica): Clases prácticas o de laboratorio con aplicación directa.
    \item TP (Teoría-Práctica): Clases mixtas que combinan ambos enfoques en un solo grupo.
\end{itemize}

\subsection{Herramientas de Apoyo a la Planificación}

Como complemento a los canales oficiales, han surgido herramientas desarrolladas por la comunidad estudiantil para facilitar la organización académica. La más destacada es Cappuchino, un sistema web creado por la Sociedad Científica de Estudiantes de Sistemas e Informática (SCESI) \cite{cappuccino_umss}. Su funcionalidad principal es asistir a los estudiantes durante el periodo de inscripción, permitiendo seleccionar materias y visualizar posibles choques de horario.

Esta herramienta se ha consolidado como un recurso esencial en la etapa de planificación. A través de una interfaz gráfica interactiva, los estudiantes pueden simular su carga académica agregando asignaturas y probando diferentes combinaciones de grupos. El sistema facilita la detección de conflictos temporales mediante una representación visual por colores, lo que permite a los usuarios optimizar su distribución de tiempo antes de proceder al registro oficial. La Figura \ref{fig:cappuchino_interfaz} presenta la interfaz de usuario de este sistema.

\begin{figure}[H]
  \centering
  \includegraphics[width=0.95\textwidth]{images/cappuchino_web.png}
  \caption{Interfaz web del sistema Cappuchino}
  \label{fig:cappuchino_interfaz}
  \small{Fuente: \cite{cappuccino_umss}}
\end{figure}

\subsection{Portal WebSIS}

El Portal WebSIS es el sistema oficial de información universitaria de la UMSS, utilizado por todas las facultades, no solo la FCyT. A diferencia de SAGAA y Cappuchino, que funcionan como herramientas de planificación previa, el WebSIS es la plataforma donde se formaliza la inscripción. Una vez inscrito, el estudiante puede visualizar su horario oficial, el cual refleja únicamente las materias y grupos registrados \cite{websis_horario}.

\subsection{Flujo de información actual}

El flujo de información comienza con la publicación de los PDFs en SAGAA. Posteriormente, los estudiantes utilizan Cappuchino para planificar su inscripción. Una vez finalizada la etapa de inscripción, la validación de horarios y aulas durante el semestre se realiza consultando nuevamente los PDFs o mediante comunicación informal entre pares.

\section{Reglas de Negocio y Restricciones}

El funcionamiento académico de la FCyT se rige por un conjunto de reglas y restricciones que definen la estructura de los horarios.

\subsection{Horarios y Turnos}

Las actividades académicas se desarrollan bajo las siguientes pautas temporales:

\begin{itemize}
    \item Días de clase: Las actividades regulares se programan de lunes a viernes. Los días sábados se utilizan ocasionalmente para exámenes o clases de recuperación, dependiendo de la materia.
    \item Turnos: La jornada académica se divide en tres turnos:
    \begin{itemize}
        \item Mañana: Comprende desde las 06:45 hasta las 12:45.
        \item Tarde: Comprende desde las 12:45 hasta las 18:45.
        \item Noche: Comprende desde las 18:45 hasta las 21:45.
    \end{itemize}
    \item Periodos: Las clases se organizan en periodos estándar de 1 hora y 30 minutos (ej. 06:45 - 08:15, 08:15 - 09:45).
\end{itemize}

\subsection{Estructura Académica}

La oferta se organiza por carreras, niveles (semestres) y grupos. Cada materia puede tener múltiples grupos (ej. Grupo 1, Grupo 2), cada uno con su propio docente y horario asignado. Existen materias compartidas entre diferentes carreras (materias de servicio), lo que implica que estudiantes de distintas carreras pueden compartir el mismo grupo y horario.

\subsection{Publicación de Información}


\section{Perfil Tecnológico y Hábitos de los Estudiantes}

Para comprender mejor las necesidades de la población estudiantil, se realizó una encuesta a 126 estudiantes de la facultad \cite{encuesta_fcyt_2025}. Los resultados obtenidos delinean un perfil tecnológico claro y evidencian hábitos específicos en el manejo de la información académica.

\subsection{Prevalencia de Dispositivos Móviles}

El ecosistema móvil es predominante entre los estudiantes. El 94\% de los encuestados utiliza dispositivos con sistema operativo Android como su herramienta principal. Este dato es fundamental para definir la plataforma objetivo de cualquier solución tecnológica propuesta, asegurando el mayor alcance posible dentro de la comunidad universitaria.

\subsection{Canales de Comunicación y Verificación}

Ante la naturaleza estática de los horarios en PDF, los estudiantes han adoptado canales informales para validar la información. El 80\% de los participantes indica que utiliza grupos de WhatsApp para confirmar horarios, aulas y cambios de último momento. Esto refleja una necesidad crítica de inmediatez y confirmación social que los canales oficiales actuales no satisfacen plenamente.

\subsection{Necesidades de Conectividad y Notificaciones}

La infraestructura de conectividad en el campus y la dinámica estudiantil generan requerimientos específicos:
\begin{itemize}
    \item Acceso Offline: Un 77.7\% de los estudiantes considera vital contar con un "Modo sin conexión". La capacidad de consultar el horario sin depender de una conexión a internet activa es una de las características más demandadas, dada la inestabilidad de la red en ciertas zonas de la facultad.
    \item Notificaciones: Existe una alta valoración hacia la recepción de alertas automáticas. Más del 50\% de los encuestados califica como muy útil recibir notificaciones antes del inicio de clases, lo que sugiere que una herramienta proactiva aportaría valor significativo a su gestión del tiempo.
\end{itemize}


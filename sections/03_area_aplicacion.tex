\chapter{MARCO DE APLICACIÓN}
\label{chap:area_aplicacion}

Este capítulo describe el contexto real de uso de la solución: el dominio institucional, la población objetivo, los procesos actuales (AS-IS) y los procesos propuestos (TO-BE), así como la información y restricciones propias del área. No se detalla la metodología ni el diseño técnico; el propósito es situar el problema en su entorno operativo y establecer límites y criterios de evaluación vinculados al uso real.

\section{Dominio y contexto institucional}
La Facultad de Ciencias y Tecnología (FCyT) de la UMSS publica semestralmente los horarios oficiales en formato PDF, organizados por carrera y periodo. Durante la inscripción, el uso de herramientas web como Cappuchino es intensivo; fuera de esa etapa, los estudiantes siguen necesitando: (i) consultar horarios personales con rapidez, (ii) ajustar materias ante cambios y (iii) coordinar con pares mediante mensajería.

\section{Población objetivo y actores}
Actor principal: estudiantes de la FCyT de distintos semestres y turnos que construyen, consultan y ajustan sus horarios.
Actores secundarios: personal administrativo que publica/actualiza los horarios y docentes/tutores cuya asignación incide en las preferencias estudiantiles.

\subsection{Evidencia cuantitativa (encuesta, n=126)}
Muestreo y aplicación: se utilizó una muestra por conveniencia de 126 estudiantes de la FCyT. El instrumento se aplicó en línea mediante Google Forms y la participación fue voluntaria. El instrumento completo y los resultados tabulados se presentan en anexos (ver \ref{ann:instrumento} y \ref{ann:historias}).

La encuesta confirma:
\begin{itemize}
  \item Dispositivo: Android domina (94\%); laptop apenas 26\%.
  \item Uso de Cappuchino: 48\% solo en inscripción, 44\% varias veces por semestre.
  \item Coordinación social: 80\% valida información por WhatsApp.
  \item Frustración por repetir pasos: 60\% califica 1-2-3 (alta).
  \item Principales demandas: actualización automática (88\%), modo sin conexión (78\%), recordatorios (56\%).
\end{itemize}

\begin{figure}[H]
  \centering
  \begin{minipage}{0.4\textwidth}
    \centering
    \includegraphics[width=\textwidth]{images/diagrams/encuesta_dispositivo.png}
    \caption[Distribución de dispositivos utilizados por los estudiantes (n=126)]{Distribución de dispositivos utilizados por los estudiantes (n=126). \\ \small{Fuente: Elaboración propia.}}
    \label{fig:encuesta_dispositivo}
  \end{minipage}
  \hfill
  \begin{minipage}{0.4\textwidth}
    \centering
    \includegraphics[width=\textwidth]{images/diagrams/encuesta_uso_capuchino.png}
\caption[Frecuencia de uso de Cappuchino por los estudiantes (n=126)]{Frecuencia de uso de Cappuchino por los estudiantes (n=126). \\ \small{Fuente: Elaboración propia.}}
    \label{fig:encuesta_uso_capuchino}
  \end{minipage}
\end{figure}

\begin{figure}[H]
  \centering
  \begin{minipage}{0.4\textwidth}
    \centering
    \includegraphics[width=\textwidth]{images/diagrams/encuesta_whatsapp.png}
    \caption[Coordinación social mediante WhatsApp (n=126)]{Coordinación social mediante WhatsApp (n=126). \\ \small{Fuente: Elaboración propia.}}
    \label{fig:encuesta_whatsapp}
  \end{minipage}
  \hfill
  \begin{minipage}{0.4\textwidth}
    \centering
    \includegraphics[width=\textwidth]{images/diagrams/encuesta_frustracion.png}
    \caption[Nivel de frustración por repetir pasos (n=126)]{Nivel de frustración por repetir pasos (n=126). \\ \small{Fuente: Elaboración propia.}}
    \label{fig:encuesta_frustracion}
  \end{minipage}
\end{figure}

\begin{figure}[H]
  \centering
  \includegraphics[width=0.7\textwidth]{images/diagrams/encuesta_demandas.png}
  \caption[Principales demandas de los estudiantes (n=126)]{Principales demandas de los estudiantes (n=126). \\ \small{Fuente: Elaboración propia.}}
  \label{fig:encuesta_demandas}
\end{figure}

\section{Percepciones del estado actual (AS-IS)}
Los estudiantes valoran la rapidez de Cappuchino para organizar horarios y detectar choques, pero reportan problemas recurrentes: actualización tardía de datos, interfaz obsoleta, pérdida del progreso, lentitud en inscripciones y baja personalización.  
Figura~\ref{fig:asis_proceso} muestra el flujo actual y las limitaciones detectadas.

Historia de uso actual (AS-IS).
De manera típica, el estudiante: (1) abre Cappuchino y genera un horario base para su carrera/semestre; (2) consulta PDFs oficiales de horarios por materia y grupo para validar o actualizar; (3) contrasta en canales de WhatsApp cambios de última hora (docentes, aulas, grupos cerrados); (4) reescribe o ajusta a mano su selección para evitar choques; (5) repite la verificación ante cada ajuste o noticia; y (6) comparte capturas/archivos con compañeros para coordinar. En periodos de alta demanda, el ciclo se repite varias veces al día.

Puntos de dolor reportados.
\begin{itemize}
  \item Repetición de pasos y pérdida de progreso entre sesiones o dispositivos.
  \item Dispersión de la información entre PDF, web y mensajería; riesgo de desactualización.
  \item Falta de funcionamiento sin conexión en momentos críticos (aula, pasillos, movilidad).
  \item Dificultad para consolidar un horario personal sin choques al cambiar un único grupo.
  \item Fricción para compartir e importar horarios en un formato estructurado.
\end{itemize}

\begin{figure}[H]
  \centering
  \includegraphics[width=0.35\textwidth]{images/diagrams/area_asis_proceso.png}
  \caption[Proceso actual (AS-IS): navegación y patrones de uso]{Proceso actual (AS-IS): navegación y patrones de uso. \\ \small{Fuente: Elaboración propia.}}
  \label{fig:asis_proceso}
\end{figure}

\section{Demanda de solución y procesos propuestos (TO-BE)}
Visión general: los resultados evidencian la necesidad de una aplicación móvil que reduzca pasos repetitivos, funcione sin conexión y facilite la coordinación social.
Figura~\ref{fig:area_tobe} resume los flujos de uso esperados.

\begin{figure}[H]
  \centering
  \includegraphics[width=0.35\textwidth]{images/diagrams/area_tobe_proceso.png}
  \caption[Procesos propuestos (TO-BE) a nivel de uso]{Procesos propuestos (TO-BE) a nivel de uso. \\ \small{Fuente: Elaboración propia.}}
  \label{fig:area_tobe}
\end{figure}

Características clave de uso:
\begin{itemize}
  \item Generación de horarios sin choques, priorizando materias obligatorias y docentes favoritos.
  \item Memoria de contexto para evitar repetir carrera/nivel/materias.
  \item Intercambio estructurado en JSON (formato de intercambio de datos basado en texto) compatible con mensajería.
  \item Consulta diaria y recordatorios antes de clase.
\end{itemize}

\section{Información y reglas del área}
Entidades: carrera, nivel, materia, grupo, horario, docente, aula y selección del estudiante.
Reglas principales:
\begin{itemize}
  \item Sin solapamientos salvo habilitación explícita.
  \item Unicidad de códigos y consistencia jerárquica.
  \item Tolerancia a datos incompletos (aula/docente pendientes).
  \item Actualización por sincronización con la fuente oficial.
\end{itemize}

\section{Datos y flujo de información}
Los horarios en PDF son la fuente oficial. Se descargan, extraen y almacenan localmente; se verifican y actualizan de forma controlada. El intercambio social se realiza mediante archivos estructurados (JSON) para compartir horarios.  
Véase Figura~\ref{fig:area_flujo_datos}.

\begin{figure}[H]
  \centering
  \includegraphics[width=0.35\textwidth]{images/diagrams/area_flujo_datos.png}
  \caption[Flujo de información en el área de aplicación]{Flujo de información en el área de aplicación. \\ \small{Fuente: Elaboración propia.}}
  \label{fig:area_flujo_datos}
\end{figure}

\section{Alcances, límites y riesgos}
En este apartado se precisan los alcances y límites desde el punto de vista del área de aplicación, complementando los alcances generales presentados en el Capítulo~1.
Alcances: estudiantes FCyT; integración con horarios oficiales; uso sin conexión; recordatorios y consulta rápida.
Límites: no reemplaza inscripción oficial ni modela capacidad de aulas; foco inicial en Android.
Riesgos y mitigación:
\begin{itemize}
  \item Cambios en formato o ubicación de PDFs: proceso flexible y sincronización manual.
  \item Retrasos oficiales: mantener caché y mostrar fecha de actualización.
  \item Conectividad limitada: operación offline-first.
  \item Saturación de fuente: reintentos espaciados.
\end{itemize}

\section{Derivación de requerimientos desde la encuesta}
\begin{table}[H]
  \centering
  \caption[Trazabilidad de hallazgos a requerimientos del área]{Trazabilidad de hallazgos a requerimientos del área. \\ \small{Fuente: Elaboración propia (encuesta n=126).}}
  \label{tab:derivacion_req_area}
  \small
  \begin{tabular}{|p{5.0cm}|p{7.5cm}|p{3.5cm}|}
    \hline
    Hallazgo & Requerimiento derivado & Criterio de aceptación \\
    \hline
    94\% usa Android & Priorizar Android y vista móvil & Flujo usable sin laptop \\
    \hline
    60\% frustración por repetir pasos & Memoria de carrera/nivel/materias & Ingreso sin repetir selección \\
    \hline
    80\% coordina por WhatsApp & Intercambio estructurado en JSON (formato de intercambio de datos basado en texto) & Compartir/abrir desde mensajería \\
    \hline
    78\% valora modo sin conexión & Operación offline-first (arquitectura que prioriza el funcionamiento sin conexión) & Consulta sin red \\
    \hline
    57\% valora notificaciones & Recordatorios previos a clase & Alertas configurables \\
    \hline
    88\% solicita actualización automática & Actualización desde fuente oficial & Fecha visible de actualización \\
    \hline
  \end{tabular}
\end{table}

\section{Criterios de éxito e impacto esperado}
Reducción de pasos para construir horarios sin choques; continuidad de uso fuera del periodo de inscripción; disminución de choques no detectados; incremento del intercambio estructurado; y mejor organización académica personal.

\section{SÍNTESIS}
El área de aplicación se ubica entre la oferta oficial (PDFs FCyT), las prácticas actuales de consulta y la coordinación estudiantil por mensajería. Las brechas identificadas justifican una solución complementaria centrada en el estudiante, con consulta móvil eficaz, uso sin conexión y respeto por la fuente oficial.

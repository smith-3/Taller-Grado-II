% 03_area_aplicacion.tex

\chapter{Área de Aplicación}
\label{chap:area_aplicacion}

\section{Contexto Institucional}
La Facultad de Ciencias y Tecnología de la Universidad Mayor de San Simón (UMSS) es la unidad académica responsable de formar profesionales en 19 carreras, entre ellas Ingeniería Informática e Ingeniería de Sistemas. Cada semestre, la FCyT coordina la asignación de horarios para miles de estudiantes de pregrado y posgrado mediante el sistema web Cappuccino, desarrollado por la Sociedad Científica de Estudiantes de Sistemas e Informática . Aunque Cappuccino proporciona una vista general de los horarios, adolece de varias deficiencias: su interfaz no es responsiva, obliga a una conexión constante y carece de mecanismos de alerta o personalización, lo que disminuye la eficiencia del proceso y la satisfacción de los usuarios .

\section{Usuarios Objetivo}
Los destinatarios de TecnoTime son los estudiantes regulares de la FCyT, típicamente de 18 a 25 años, con alta dependencia de dispositivos Android (versión mínima Android 10, API 29; ideal API 34). Estos usuarios requieren:
\begin{itemize}
  \item Consulta offline: acceso a sus horarios sin conexión, para zonas de baja cobertura o interrupciones de red .
  \item Configuración personalizada: selección dinámica de materias y grupos según la inscripción o retiro semestral .
  \item Alertas previas: notificaciones móviles configurables antes de cada clase o examen .
  \item Interfaz nativa: experiencia fluida y coherente con los lineamientos de diseño de Android moderno (Jetpack Compose + MVVM) .
\end{itemize}

\section{Problemática Actual}
Un diagnóstico de la experiencia de los estudiantes evidencia los siguientes problemas críticos:
\begin{enumerate}
  \item Dependencia de conexión: la versión web requiere Internet permanente, lo que impide la consulta en aulas sin cobertura o durante cortes programados .
  \item Ausencia de notificaciones: no existen alertas automáticas para el inicio de clases, cambios de salón o recordatorios de exámenes .
  \item Interfaz no adaptativa: el sistema no está optimizado para pantallas pequeñas, forzando zoom y desplazamientos constantes .
  \item Falta de optimización: no se facilitan sugerencias de secuencia de clases que minimicen tiempos muertos o desplazamientos .
  \item Exportación limitada: no permite generar PDF, imágenes o respaldos compartibles de los horarios, lo que dificulta la colaboración entre compañeros .
\end{enumerate}
Estos déficits no solo impacto en la productividad, sino que aumentan el riesgo de errores al inscribir materias en conflicto.

\section{Justificación de la Solución}
TecnoTime se propone como solución integral a las carencias identificadas, ofreciendo:
\begin{itemize}
  \item Operación offline: almacenamiento local en Room/SQLite para consulta continua sin red .
  \item Notificaciones configurables: alertas programables con antelación ajustable para cada evento académico .
  \item Interfaz nativa Android: Jetpack Compose y MVVM garantizan una navegación fluida y coherente con las guías de Material Design .
  \item Sugerencias inteligentes: algoritmos basados en grafos que optimizan la secuencia de clases, reduciendo huecos y desplazamientos .
  \item Exportación versátil: generación de horarios en PDF, imagen o JSON para compartir y respaldar fácilmente .
\end{itemize}
Con estas capacidades, TecnoTime incrementa la autonomía del estudiante y disminuye el tiempo dedicado a la configuración manual.

\section{Impacto Esperado}
La adopción de TecnoTime en la FCyT-UMSS busca:
\begin{itemize}
  \item Reducir en un 50 \% el tiempo de configuración manual de horarios.
  \item Disminuir choques de materias durante el proceso de inscripción.
  \item Aumentar la puntualidad y asistencia mediante alertas oportunas.
  \item Mejorar la satisfacción estudiantil, evaluable en encuestas de usabilidad.
  \item Facilitar la colaboración y el intercambio de horarios entre pares.
\end{itemize}

\section{Entorno Técnico}
El sistema se compone de:
\begin{description}
  \item[\normalfont Cliente Android:] desarrollo en Kotlin con Jetpack Compose y arquitectura MVVM; persistencia local con Room/SQLite .
  \item[\normalfont API RESTful:] servidor en NestJS (Node.js), con MongoDB para datos centrales y Redis para cacheo de consultas frecuentes .
  \item[\normalfont Scraping de horarios:] módulo en backend que emplea Axios y Cheerio para extraer y procesar PDF desde el portal FCyT, garantizando datos actualizados .
  \item[\normalfont Comunicación:] Retrofit en el cliente para consumo de JSON y sincronización bidireccional eficiente.
\end{description}

\section{Alcance y Limitaciones}
De conformidad con la guía de la FCyT-UMSS, esta versión de TecnoTime se circunscribe a:
\begin{itemize}
  \item Cobertura: únicamente las 19 carreras de la FCyT-UMSS .
  \item Plataforma: dispositivos Android (API 29+); no incluye iOS ni PWA.
  \item Eventos: gestión exclusiva de clases y exámenes; sin actividades personales.
  \item Distribución: entrega interna; no se publicará en Google Play.
  \item Sincronización externa: sin integración con calendarios externos en esta fase.
\end{itemize}
Este enfoque asegura un producto enfocado, estable y alineado con las normas de presentación de la FCyT .

% FIN DE 03_area_aplicacion.tex

\chapter{ÁREA DE APLICACIÓN}
\label{chap:area_aplicacion}

Este capítulo describe el contexto real de uso de la solución propuesta: el dominio institucional, la población objetivo, los procesos actuales (estado actual, AS-IS) y los procesos propuestos (estado futuro, TO-BE), así como los datos y restricciones propias del área. No se detalla la metodología ni el diseño técnico; esos contenidos se desarrollan en capítulos posteriores. La intención es situar el problema en su entorno operativo y establecer límites y criterios de evaluación vinculados al uso real.

\section{DOMINIO Y CONTEXTO INSTITUCIONAL}

La Facultad de Ciencias y Tecnología (FCyT) de la Universidad Mayor de San Simón (UMSS) administra semestralmente la oferta académica de aproximadamente veinte carreras de grado. Los horarios oficiales se publican en el portal institucional principalmente en formato PDF, organizados por carrera y periodo. 

Durante los periodos de inscripción, la comunidad estudiantil utiliza de forma intensiva herramientas web como Capuchino para consultar la oferta de grupos, verificar choques y construir horarios preliminares. Fuera de estas ventanas, el uso disminuye, pero persiste la necesidad de:
\begin{itemize}
  \item Consultar el horario personal de manera rápida en el día a día.
  \item Ajustar materias y grupos ante cambios institucionales o personales.
  \item Coordinar con pares mediante canales informales (principalmente WhatsApp).
\end{itemize}

En este escenario, \emph{TecnoTime} se plantea como una aplicación móvil nativa Android, con prioridad sin conexión (\emph{offline-first}), diseñada para complementar el ecosistema actual. Su ámbito cubre desde la preparación previa al proceso de inscripción hasta el seguimiento cotidiano del horario del estudiante durante el semestre.

\begin{figure}[H]
  \centering
  \includegraphics[width=0.8\textwidth]{images/diagrams/area_actores.png}
  \caption{Mapa conceptual de actores y entorno de operación. Fuente: elaboración propia.}
  \label{fig:area_actores}
\end{figure}

% ----------------------------------------------------------------------
\section{POBLACIÓN OBJETIVO Y ACTORES}

\subsection*{Actores principales y secundarios}

\textbf{Actor principal}: estudiantes de la FCyT, de distintos semestres, turnos y carreras, que construyen, consultan y ajustan sus horarios.

\textbf{Actores secundarios}:
\begin{itemize}
  \item Personal administrativo responsable de publicar y actualizar los horarios oficiales en PDF.
  \item Docentes y tutores, cuya asignación a grupos influye en las preferencias de los estudiantes.
\end{itemize}

\subsection*{Evidencia cuantitativa del contexto}

A partir de una encuesta aplicada a estudiantes de FCyT (n $\approx$ 125) se obtienen hallazgos relevantes para el área de aplicación:
\begin{itemize}
  \item Predominio de dispositivos Android (por encima del 90\%), con presencia minoritaria de iOS.
  \item Uso de Capuchino concentrado en el periodo de inscripción: cerca de la mitad reporta usarlo sólo en inscripciones y una porción similar varias veces por semestre.
  \item Más del 80\% utiliza grupos de WhatsApp para compartir y confirmar información de horarios, grupos y docentes.
  \item Principales demandas para una futura aplicación móvil:
    \begin{itemize}
      \item Horarios actualizados automáticamente.
      \item Funcionamiento sin conexión.
      \item Recordatorios de clases.
      \item Capacidad de compartir el horario con compañeros.
    \end{itemize}
\end{itemize}

\begin{figure}[H]
  \centering
  \includegraphics[width=0.7\textwidth]{images/diagrams/area_graficos_encuesta.png}
  \caption{Resumen gráfico de resultados de la encuesta (uso, dispositivo y demandas). Fuente: encuesta aplicada a estudiantes de la FCyT (n $\approx$ 125).}
  \label{fig:area_graficos_encuesta}
\end{figure}

\begin{table}[H]
  \centering
  \caption{Actores y necesidades clave en el área de aplicación. Fuente: elaboración propia.}
  \label{tab:actores_necesidades}
  \small
  \begin{tabular}{|p{3.2cm}|p{4.2cm}|p{4.2cm}|p{4.2cm}|}
    \hline
    Actor & Objetivos & Problemas actuales & Aporte potencial de TecnoTime \\
    \hline
    Estudiante FCyT &
    Construir horario sin choques; consultar fácilmente; adaptarse a cambios &
    Repetir selección carrera/nivel/materia; caídas en inscripción; dependencia total de conexión; dificultad para compartir &
    Generación de horarios sin choques por defecto; caché local; compartir/importar horario; recordatorios y componente de acceso rápido (widget) \\
    \hline
    Docente/Tutor &
    Coordinar mejor con estudiantes &
    Comunicación fragmentada; cambios informales &
    Horarios más claros; menor ambigüedad en grupos y horarios \\
    \hline
    Administrativo &
    Difundir oferta vigente oportunamente &
    Cambios de última hora; formato rígido en PDF &
    Integración desde PDFs; actualización automática hacia estudiantes sin sobrecarga manual \\
    \hline
  \end{tabular}
\end{table}

% ----------------------------------------------------------------------
\section{PROCESO ACTUAL (AS-IS): CAPUCHINO + PDFS + WHATSAPP}

\subsection{Acceso y navegación}
En el flujo actual, el estudiante:
\begin{enumerate}
  \item Ingresa al sitio web (por ejemplo, Capuchino) desde un navegador.
  \item Selecciona secuencialmente carrera, nivel, materia y grupo.
  \item Repite el proceso para cada modificación o comparación.
\end{enumerate}

Esta navegación jerárquica implica múltiples pasos y resulta poco cómoda en pantallas pequeñas. La herramienta no recuerda el contexto del usuario entre sesiones.

\subsection{Selección y visualización del horario}
El sistema muestra los grupos seleccionados en una vista de horario semanal. El estudiante:
\begin{itemize}
  \item Activa/desactiva grupos para evitar choques.
  \item Exporta su horario mediante captura de pantalla o PDF.
\end{itemize}

\begin{figure}[H]
  \centering
  \includegraphics[width=0.3\textwidth]{images/diagrams/area_asis_proceso.png}
  \caption{Esquema del proceso AS-IS con Capuchino y PDFs. Fuente: elaboración propia.}
  \label{fig:asis_proceso}
\end{figure}

\subsection{Apoyos informales}
La coordinación y validación se completa en canales no oficiales:
\begin{itemize}
  \item Envío de capturas de pantalla.
  \item Preguntas en grupos de WhatsApp sobre choques, docentes y aulas.
\end{itemize}

\subsection{Limitaciones identificadas}
\begin{itemize}
  \item Experiencia deficiente en dispositivos móviles (interfaz no adaptada).
  \item No se recuerdan carrera, historial de materias aprobadas ni preferencias.
  \item Enfoque limitado a la etapa de inscripción; el horario resultante no se integra a la rutina diaria.
  \item Dependencia de conexión constante para consultar la oferta.
  \item Actualizaciones tardías o inconsistentes en horarios y aulas percibidas por los estudiantes.
\end{itemize}

Estas brechas definen el \emph{área de oportunidad} para una solución complementaria.

% ----------------------------------------------------------------------
\section{PROCESOS PROPUESTOS (TO-BE) CON TECNOTIME}

\subsection{Visión general}
\emph{TecnoTime} aborda los puntos críticos del estado actual (AS-IS) mediante procesos diseñados para:
\begin{itemize}
  \item Reducir pasos repetitivos al construir y ajustar horarios.
  \item Mantener el horario accesible y útil durante todo el semestre.
  \item Integrarse con prácticas reales: uso de WhatsApp, consulta móvil rápida, trabajo sin conexión.
\end{itemize}

\begin{figure}[H]
  \centering
  \includegraphics[width=0.3\textwidth]{images/diagrams/area_tobe_proceso.png}
  \caption{Proceso TO-BE con TecnoTime: generación, sincronización y uso cotidiano. Fuente: elaboración propia.}
  \label{fig:area_tobe}
\end{figure}

\subsection{Flujos típicos del usuario}

\paragraph{Construcción de horario sin choques.}
El estudiante selecciona materias candidatas; la aplicación genera propuestas sin choques por defecto, con posibilidad de:
\begin{itemize}
  \item Marcar docentes favoritos.
  \item Minimizar tiempos muertos.
  \item Fijar materias obligatorias.
  \item Aceptar choques sólo si se configura explícitamente.
\end{itemize}

\paragraph{Importar y compartir horarios (JSON).}
El horario puede:
\begin{itemize}
  \item Exportarse como imagen, PDF o en formato JSON para intercambio.
  \item Compartirse por mensajería (por ejemplo, WhatsApp).
  \item Importarse desde un JSON recibido para clonar total o parcialmente la configuración.
\end{itemize}

\paragraph{Gestión dinámica de materias.}
El usuario puede:
\begin{itemize}
  \item Agregar materias manualmente (nivel $\rightarrow$ materia $\rightarrow$ grupo con vista previa).
  \item Cambiar de grupo y registrar el estado final de la materia (aprobada o abandonada).
\end{itemize}

\paragraph{Recordatorios y componente de acceso rápido (\emph{widget}).}
\begin{itemize}
  \item Configurar recordatorios con anticipación ajustable para próximas clases.
  \item Consultar el horario del día mediante un componente de acceso rápido (\emph{widget}) en la pantalla principal.
  \item Activar de forma opcional un asistente (Simón) basado en inteligencia artificial (IA) para mensajes motivacionales contextualizados.
\end{itemize}

\begin{figure}[H]
  \centering
  \includegraphics[width=0.95\textwidth]{images/diagrams/area_user_journey.png}
  \caption{Journey resumido del estudiante con TecnoTime. Fuente: elaboración propia.}
  \label{fig:area_journey}
\end{figure}

\begin{table}[H]
  \centering
  \caption{Escenarios frecuentes en el TO-BE. Fuente: elaboración propia.}
  \label{tab:escenarios}
  \small
  \begin{tabular}{|p{3.2cm}|p{3.6cm}|p{6.0cm}|p{3.0cm}|}
    \hline
    Escenario & Disparador & Pasos clave & Resultado \\
    \hline
    Armar horario & Inicio de semestre & Seleccionar materias, generar propuestas, elegir opción & Horario base sin choques \\
    \hline
    Importar JSON & Mensaje recibido & Abrir JSON, revisar materias, aplicar parcial/total & Horario alineado con pares \\
    \hline
    Agregar materia & Cambio de carga & Nivel $\rightarrow$ materia $\rightarrow$ grupo (preview) & Materia añadida al horario \\
    \hline
    Editar grupo & Cambio de preferencia & Seleccionar nuevo grupo, validar choques & Grupo actualizado \\
    \hline
    Recordatorios & Preferencia & Fijar anticipación, activar/desactivar IA & Avisos antes de cada clase \\
    \hline
    Consulta rápida & Día a día & Ver componente de acceso rápido (widget) con horario vigente & Información inmediata \\
    \hline
  \end{tabular}
\end{table}

% ----------------------------------------------------------------------
\section{INFORMACIÓN DEL DOMINIO Y REGLAS DEL ÁREA}

\subsection{Entidades y catálogos relevantes}

En el área de aplicación se identifican entidades clave:
\begin{itemize}
  \item \textbf{Carrera}: identificador, nombre, enlace al PDF oficial, fecha de actualización.
  \item \textbf{Nivel}: identificador jerárquico (p.\,ej., ``Nivel A'', ``1er semestre'').
  \item \textbf{Materia}: código único, nombre, tipo (obligatoria/optativa), estado (aprobada, pendiente).
  \item \textbf{Grupo}: número o código, tipo de clase (teoría/práctica/laboratorio), modalidad, estado.
  \item \textbf{Horario de grupo}: día, hora de inicio, hora de fin, aula, docente.
  \item \textbf{Docente}: nombre completo, posible marcación como favorito.
  \item \textbf{Aula}: identificador alfanumérico (por ejemplo, ``B-201'').
  \item \textbf{Selección del estudiante}: materias y grupos elegidos, color, emoji, preferencias asociadas.
\end{itemize}

\subsection{Reglas de negocio y casos especiales}

\begin{itemize}
  \item Para el horario individual, no se permiten solapamientos, salvo que el usuario habilite explícitamente la aceptación de choques.
  \item Unicidad de códigos de materia y consistencia jerárquica entre carrera, nivel, materia y grupo.
  \item Consideración de datos incompletos: grupos con aula o docente pendientes pueden existir y actualizarse posteriormente.
  \item Cambios tardíos de horario o aula deben poder reflejarse mediante actualizaciones desde la fuente oficial o sincronización manual de PDFs.
\end{itemize}

Estas reglas describen el comportamiento esperado en el área de aplicación; su formalización operativa se desarrolla en el capítulo de requisitos.

% ----------------------------------------------------------------------
\section{DATOS Y FLUJO DE INFORMACIÓN EN EL ÁREA}

La fuente autorizada de información son los horarios oficiales en PDF publicados por la FCyT. A partir de ello, el área de aplicación considera:

\begin{itemize}
  \item \textbf{Ingreso de datos}: descarga de PDFs y extracción estructurada de la oferta de materias, grupos y bloques.
  \item \textbf{Almacenamiento local}: mantenimiento de un caché persistente en el dispositivo del estudiante.
  \item \textbf{Actualización}: verificación de frescura de datos y sincronización responsable (sin saturar el servidor), con reintentos espaciados.
  \item \textbf{Intercambio social}: uso de archivos JSON autocontenidos para compartir horarios entre estudiantes, compatibles con importación selectiva.
\end{itemize}

\begin{figure}[H]
  \centering
  \includegraphics[width=0.3\textwidth]{images/diagrams/area_flujo_datos.png}
  \caption{Flujo de datos en el área de aplicación: fuente oficial, caché local e intercambio social. Fuente: elaboración propia.}
  \label{fig:area_flujo_datos}
\end{figure}

% ----------------------------------------------------------------------
\section{ALCANCES, LÍMITES Y RIESGOS DEL ÁREA}

\subsection*{Alcances}
\begin{itemize}
  \item Estudiantes de la FCyT que requieren construir, mantener y consultar su horario.
  \item Uso de dispositivos Android como plataforma principal.
  \item Integración con horarios oficiales publicados en PDF.
  \item Soporte a intercambio de horarios mediante archivos JSON y medios de mensajería.
  \item Recordatorios de clases, preferencias de usuario y consulta rápida mediante componente de acceso rápido (widget).
\end{itemize}

\subsection*{Límites}
\begin{itemize}
  \item No reemplaza los sistemas oficiales de inscripción ni la responsabilidad institucional sobre los horarios.
  \item No se modela capacidad de aulas ni asignación global a nivel institucional.
  \item No se garantiza cobertura de horarios de verano/invierno si no se dispone de fuente oficial fiable.
  \item Soporte inicial enfocado en Android; otras plataformas quedan como trabajo futuro.
\end{itemize}

\subsection*{Riesgos y mitigaciones}

\begin{table}[H]
  \centering
  \caption{Riesgos del área de aplicación y estrategias de mitigación. Fuente: elaboración propia.}
  \label{tab:riesgos_area}
  \small
  \begin{tabular}{|p{6.5cm}|p{9.5cm}|}
    \hline
    Riesgo & Mitigación propuesta \\
    \hline
    Cambios en el formato o ubicación de los PDFs oficiales &
    Diseñar un proceso de extracción flexible y actualizable; permitir sincronización manual desde nuevos PDFs. \\
    \hline
    Retrasos o errores en la publicación de horarios &
    Mostrar fecha de última actualización; permitir que el usuario mantenga su caché local mientras se resuelve la inconsistencia. \\
    \hline
    Conectividad intermitente o limitada &
    Mantener operación con prioridad sin conexión (\emph{offline-first}) con datos locales y sincronización diferida. \\
    \hline
    Saturación de la fuente oficial &
    Implementar estrategias de reintento con espera progresiva y límites de frecuencia para acceso automatizado. \\
    \hline
    Baja adopción del componente de IA &
    Mantener Simón como módulo opcional, transparente y controlado por el usuario, sin afectar el núcleo de la aplicación. \\
    \hline
  \end{tabular}
\end{table}

% ----------------------------------------------------------------------
\section{CRITERIOS DE ÉXITO E IMPACTO ESPERADO}

Dentro del área de aplicación, el impacto esperado de \emph{TecnoTime} incluye:
\begin{itemize}
  \item Reducción del tiempo y pasos necesarios para construir un horario sin choques respecto al proceso AS-IS.
  \item Mayor continuidad de uso más allá del periodo de inscripción (consulta diaria efectiva).
  \item Disminución de errores por choques no detectados, gracias a validaciones automáticas.
  \item Uso extendido de mecanismos de compartición estructurada (archivos JSON) en lugar de sólo capturas.
  \item Aprovechamiento de recordatorios y del componente de acceso rápido (widget) para mejorar la organización académica personal.
\end{itemize}

Estos criterios orientan la evaluación posterior del sistema y vinculan directamente el área de aplicación con los requisitos y la solución propuesta.

% ----------------------------------------------------------------------
\section*{SÍNTESIS}

El área de aplicación de \emph{TecnoTime} se sitúa en la intersección entre la oferta académica oficial de la FCyT–UMSS, las prácticas actuales de consulta mediante Capuchino y PDFs, y los canales informales de coordinación estudiantil. La existencia de picos de uso, repetición de pasos, dependencia de conexión y ausencia de herramientas móviles personalizadas conforma un espacio claro para una solución complementaria: una aplicación Android con prioridad sin conexión (\emph{offline-first}), capaz de generar y mantener horarios personalizados, integrarse con la dinámica social de los estudiantes y apoyarse en datos oficiales sin sustituir a los sistemas institucionales. Este contexto delimita, justifica y orienta los requisitos que se desarrollan en los capítulos siguientes.

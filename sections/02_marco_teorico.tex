\chapter{MARCO TEÓRICO}

Este capítulo presenta la base conceptual que sustenta el diseño de TecnoTime, una aplicación Android nativa para la gestión de horarios en la FCyT–UMSS. Se formaliza el Academic Timetabling Problem (ATP) como Constraint Satisfaction Problem (CSP), se revisan enfoques de solución y se introducen principios de arquitectura offline-first, la interoperabilidad basada en JSON (colaboración por share intent, p. ej., WhatsApp), el modelo de notificaciones y la personalización con on-device AI (Simón), además de consideraciones de UX y propiedades no funcionales. Todos los conceptos aquí expuestos corresponden a funcionalidades efectivamente implementadas en el proyecto.

% ----------------------------------------------------------------------
\section{FUNDAMENTOS Y DEFINICIÓN DEL PROBLEMA}
La planificación de horarios académicos (academic timetabling) consiste en asignar clases (asignaturas y grupos) a franjas temporales y espacios cumpliendo hard constraints (no solapamiento, disponibilidad) y optimizando soft constraints pertinentes al caso de uso individual (minimizar gaps y priorizar docentes favoritos). El ATP es típicamente NP-hard; la complejidad crece exponencialmente con materias y grupos, por lo que se emplean búsqueda con poda y evaluación multi–criterio \cite{babaei2015-survey,kristiansen2013-survey}.

En el contexto local, los horarios oficiales de la FCyT se publican en formato PDF \cite{horarios_fcyt}, lo que limita la personalización móvil, la verificación automática de conflictos y el envío de alertas. Estas restricciones motivan una solución móvil nativa, robusta sin conexión y centrada en el usuario.

% ----------------------------------------------------------------------
\section{FORMALIZACIÓN PARA UN USUARIO (SELECCIÓN DE GRUPOS)}
El caso de estudio se centra en un estudiante que elige, por materia, un único grupo, respetando factibilidad temporal y preferencias.

\subsection*{Modelo y variables}
\begin{itemize}
  \item Conjuntos: materias $C$; para cada $c\in C$, grupos ofrecidos $G(c)$; días $W$; intervalos temporales $\mathcal{I}$.
  \item Variable binaria: $y_{c,g}=1$ si se elige el grupo $g\in G(c)$ para la materia $c$; $0$ en caso contrario.
  \item Horarios: cada par $(c,g)$ tiene bloques $\text{sched}(c,g)\subseteq \mathcal{I}$ (día, inicio, fin), extraídos del PDF.
\end{itemize}

\noindent\textbf{Restricciones duras}
\begin{align}
  &\forall c\in C:\ \sum_{g\in G(c)} y_{c,g} = 1 \quad \text{(una opción por materia)} \\
  &\forall (c_1,g_1)\neq(c_2,g_2):\ \text{overlap}\big(\text{sched}(c_1,g_1),\text{sched}(c_2,g_2)\big) \Rightarrow y_{c_1,g_1}+y_{c_2,g_2} \le 1 \quad \text{(sin choques)}
\end{align}

\noindent\textbf{Restricciones blandas y objetivo}
Se minimizan gaps intra–día y se favorecen docentes marcados como favoritos. El filtrado por días permitidos se modela como preferencia previa a la evaluación (no como penalización). Se adopta una función multi–criterio por penalizaciones \cite{deb2001-moo}:
\begin{equation}
  \min_{y}\; F(y)= \sum_{k} w_k\, \phi_k(y),\quad w_k\ge 0
\end{equation}

\subsection*{Notación}
\begin{table}[H]
   \centering
   \small
   \begin{tabular}{>{\bfseries}m{2.6cm} m{7.6cm} m{3cm}}
     \hline
     Símbolo & Significado & Dominio \\
     \hline
     $C,G,W,\mathcal{I}$ & Materias, grupos, días, intervalos & Finito \\
     $y_{c,g}$ & Selección de grupo para $c$ & $\{0,1\}$ \\
     $\text{sched}(c,g)$ & Bloques (día,inicio,fin) del grupo & $\mathcal{I}$ \\
     $\text{overlap}(\cdot)$ & Predicado de solape temporal & Boole \\
     $\phi_k(y)$ & Penalización del criterio $k$ & $\mathbb{R}_{\ge 0}$ \\
     $w_k$ & Peso del criterio $k$ & $\mathbb{R}_{\ge 0}$ \\
     \hline
   \end{tabular}
\caption{Notación para el modelo de selección de grupos. Fuente: Elaboración propia.}
\end{table}

\subsection*{Restricciones (hard/soft)}
\begin{table}[H]
   \centering
   \small
   \begin{tabular}{>{\bfseries}m{1.9cm} m{7.9cm} m{3.5cm}}
     \hline
     Tipo & Descripción & Fuente de dato \\
     \hline
     Dura & Elegir exactamente un grupo por materia & Oferta académica \\
     Dura & Prohibir solapes temporales & Bloques por grupo \\
     Blanda & Minimizar gaps (minutos vacíos) & Horario ordenado \\
     Blanda & Favorecer docentes favoritos & Preferencias usuario \\
     Blanda & Días permitidos (filtro) & Preferencias del usuario \\
     \hline
   \end{tabular}
\caption{Clasificación de restricciones para el usuario. Fuente: Elaboración propia.}
\end{table}

\begin{figure}[H]
  \centering
  \includegraphics[width=0.8\linewidth]{images/diagrams/restricciones_piramide.png}
  \caption{Pirámide conceptual: base de hard constraints, cima de soft constraints. Fuente: Elaboración propia.}
\end{figure}

% ----------------------------------------------------------------------
\section{REPRESENTACIÓN CONCEPTUAL DEL DOMINIO}
El modelo conceptual habilita la expresión de restricciones y evaluación de soluciones: Carrera, Nivel, Asignatura, Grupo, GroupSchedule (bloque día/inicio/fin), Docente, Aula, Selección del Estudiante y Preferencias. El diseño lógico busca coherencia y consultas eficientes (normalización hasta 3FN, integridad referencial e índices compuestos) \cite{chen1976-er,codd1971-3nf}. En Android, Room proporciona DAOs tipadas y validación de esquema \cite{room_db}.

% (Figura suprimida para evitar duplicación de la pirámide)

% ----------------------------------------------------------------------
\section{ENFOQUES DE SOLUCIÓN AL ATP}
El ATP puede abordarse con enfoques exactos, heurísticos y metaheurísticos \cite{babaei2015-survey,kristiansen2013-survey}. La elección depende del tamaño del problema y de los requisitos de tiempo de respuesta en el dispositivo.

\begin{table}[H]
  \centering
  \small
  \begin{tabular}{>{\bfseries}m{3.0cm} m{3.8cm} m{4.0cm} m{3.3cm}}
    \hline
    Enfoque & Ejemplos & Pros & Contras \\
    \hline
    Exacto & ILP/CP, B\&B & Óptimo garantizado; trazabilidad & Costoso; poco práctico on-device \\
    Heurístico & Greedy, constructivo & Rápido; simple; controlado & Puede quedar lejos del óptimo \\
    Metaheurístico & GA, SA, Tabu & Buena calidad promedio & Configuración; tiempo mayor \\
    \hline
  \end{tabular}
  \caption{Resumen de enfoques para el ATP. Fuente: Elaboración propia.}
\end{table}

En TecnoTime se favorece un heurístico con búsqueda con poda y evaluación multi–criterio para un usuario individual, priorizando tiempos de respuesta y simplicidad de mantenimiento en Android.

% ----------------------------------------------------------------------
\section{FUNDAMENTOS ALGORÍTMICOS PARA LA GENERACIÓN}
La construcción de horarios se aborda como búsqueda combinatoria con evaluación multi–objetivo. Dos componentes son clave: (i) exploración de combinaciones y (ii) función de evaluación.

\subsection{Exploración: búsqueda con retroceso (backtracking)}
El backtracking recorre combinaciones de grupos por materia y poda ramas que violan hard constraints. Con ordenamiento heurístico (MRV/LCV) y poda temprana se reduce el espacio efectivo \cite{golomb1965-backtrack}.

\begin{lstlisting}[language={},basicstyle=\ttfamily\small]
func generarHorarios(candidatos, eval, N):
  mejores = []
  def bt(i, parcial):
    if i == len(candidatos):
      score = eval(parcial)
      insertar_topN(mejores, (parcial, score), N)
      return
    for opcion in ordenar(candidatos[i]):      # LCV
      if cumple_hard(parcial, opcion):         # poda
        bt(i+1, parcial + opcion)
  bt(0, [])
  return extraer(mejores)
\end{lstlisting}

\begin{figure}[H]
  \centering
  \includegraphics[width=1.0\linewidth]{images/diagrams/generador_decisiones.png}
  \caption{Decisiones del generador: choques, gaps y docentes favoritos. Fuente: Elaboración propia.}
\end{figure}

\subsection{Evaluación: heurísticas y multi–criterio}
La evaluación asigna una puntuación a cada horario según soft constraints: minimizar gaps y favorecer docentes favoritos; los días preferidos se consideran como filtro de candidatos antes de evaluar. Un enfoque práctico es una combinación lineal de penalizaciones (estrategia compuesta) \cite{deb2001-moo}.

\begin{table}[H]
   \centering
   \small
   \begin{tabular}{>{\bfseries}m{3.6cm} m{7.0cm} m{3.0cm}}
     \hline
     Métrica & Descripción & Observación \\
     \hline
     Choques & Solapes entre bloques (si no se permiten, invalida) & Dura \\
     Gaps & Minutos vacíos entre clases consecutivas por día & Menor es mejor \\
     Favoritos & Bonificación por incluir docentes marcados & Preferencias \\
     Días activos & Penalización por dispersión innecesaria & Balance \\
     \hline
   \end{tabular}
  \caption{Métricas consideradas por la evaluación compuesta. Fuente: Elaboración propia.}
\end{table}

\subsection{Complejidad}
El backtracking tiene complejidad $O(b^d)$, con $b$ el factor de ramificación (grupos por materia) y $d$ la profundidad (materias). La poda por hard constraints y el ordenamiento heurístico reducen el costo esperado \cite{babaei2015-survey,kristiansen2013-survey,golomb1965-backtrack}.

% ----------------------------------------------------------------------
\section{ARQUITECTURAS MÓVILES OFFLINE-FIRST}
Offline-first prioriza la experiencia sin conexión, con consistencia eventual y verificación de data freshness antes de sincronizar. Para evitar saturación del origen institucional se emplean exponential backoff y circuit breaker. La resiliencia se completa con fallback a importación de PDF oficial compartido por la comunidad (cuando el sitio esté caído).

\begin{itemize}
  \item \textbf{Cache local}: persistencia de oferta académica y selecciones.
  \item \textbf{Freshness check}: validación y auto-sync programado.
  \item \textbf{Backoff/Circuit breaker}: protección ante alta latencia o caída.
  \item \textbf{Fallback PDF}: importación manual por carrera cuando sea necesario.
\end{itemize}

\begin{figure}[H]
  \centering
  \begin{minipage}{0.4\linewidth}
    \centering
    \includegraphics[width=\linewidth]{images/diagrams/offline_first_flujo.png}
    \caption*{Flujo rápido: cache y frescura}
  \end{minipage}\hfill
  \begin{minipage}{0.49\linewidth}
    \centering
    \includegraphics[width=\linewidth]{images/diagrams/offline_first_sync.png}
    \caption*{Sincronización: backoff, CB y fallback PDF}
  \end{minipage}
  \caption{Arquitectura offline-first dividida en vista rápida y de sincronización. Fuente: Elaboración propia.}
\end{figure}

% ----------------------------------------------------------------------
\section{INTEROPERABILIDAD Y COLABORACIÓN (JSON + WHATSAPP)}
Para el intercambio social de horarios (p.\,ej., por WhatsApp) se usa un JSON shareable que describe materias seleccionadas, grupos, bloques y metadatos. Validaciones efectivas en la importación:
\begin{itemize}
  \item Evitar duplicados si ya existe la materia/grupo.
  \item Si el JSON proviene de otra carrera, solicitar sincronización previa.
  \item Permitir importación parcial (sólo ítems seleccionados).
\end{itemize}

\begin{figure}[H]
  \centering
  \includegraphics[width=.3\linewidth]{images/diagrams/json_intercambio_esquema.png}
  \caption{Esquema conceptual de intercambio de horarios en JSON. Fuente: Elaboración propia.}
\end{figure}

% ----------------------------------------------------------------------
\section{NOTIFICACIONES ACADÉMICAS EN ANDROID}
Las notificaciones locales recuerdan próximas clases con anticipación configurable: se programan en segundo plano, utilizan canales de notificación y respetan políticas de background scheduling (Workers, alarmas exactas si aplica).

\begin{itemize}
  \item \textbf{Recordatorios}: hora de inicio, anticipación (lead time) y canal.
  \item \textbf{Background}: planificación diferida y reprogramación ante cambios.
  \item \textbf{Canales}: categorización y control de importancia por el usuario.
\end{itemize}

% ----------------------------------------------------------------------
\section{PERSONALIZACIÓN CON IA ON-DEVICE (SIMÓN)}
La personalización on-device permite contenido contextual sin comprometer privacidad ni depender de red. El modelo se descarga de forma opcional; el usuario controla tipo de conexión, activación y borrado. Trade-offs: espacio en disco y consumo energético versus engagement y utilidad.

% ----------------------------------------------------------------------
\section{UX Y EXPERIENCIA DE USO}
Se privilegia un onboarding breve y progresivo para captar nombre, carrera(s), ajustes básicos y docentes favoritos; se incorporan confirmaciones explícitas en operaciones críticas (p.\,ej., desincronizar carrera); se ofrece vista previa al elegir grupo (días/horas) para reducir errores; y se incluye un widget para consulta rápida del día. Las preferencias de presentación (24/12h, fines de semana, días sin clases) no afectan el cómputo, pero mejoran la legibilidad.

% ----------------------------------------------------------------------
\section{PROPIEDADES NO FUNCIONALES}
\begin{itemize}
  \item \textbf{Rendimiento}: respuesta fluida y generación en tiempos adecuados para un usuario individual.
  \item \textbf{Tamaño de app}: objetivo $<30$ MB sin IA; on-device model como descarga opcional.
  \item \textbf{Resiliencia}: exponential backoff, circuit breaker, bloqueo de reintentos, fallback por PDF.
  \item \textbf{Mantenibilidad}: separación por capas (Clean Architecture, MVVM, Repository, Strategy) \cite{martin2017-cleanarch,gamma1995-patterns,fowler2002-peaa}.
  \item \textbf{Compatibilidad}: foco en Android por predominio de uso; iOS minoritario.

% ----------------------------------------------------------------------
\section*{Trazabilidad con la implementación}
Para asegurar consistencia entre teoría y práctica, se listan conceptos y su correspondencia con módulos implementados en el proyecto:
\begin{itemize}
  \item Modelo de dominio (Carrera, Materia, Grupo, Horario, Docente, Aula, Selección): persistencia con Room (DAOs y entidades) y relaciones tipadas.
  \item Generación de horarios: retroceso con poda, detección de choques, penalización de gaps y priorización de docentes favoritos; filtrado previo por días permitidos.
  \item Intercambio JSON: generación de archivo compartible y flujo de importación desde share intent.
  \item Sincronización offline-first: scraping de oferta, verificación de frescura, exponential backoff, circuit breaker y fallback de PDF local.
  \item Notificaciones: programación diferida y plantillas enriquecidas; widget de pantalla de inicio para consulta diaria.
  \item IA en dispositivo (Simón): cliente nativo de LLaMA, descarga opcional del modelo y control de ejecución local.
\end{itemize}
\end{itemize}

% ----------------------------------------------------------------------
\section{OBTENCIÓN DE DATOS: SCRAPING Y PARSING DE PDF}
Dado que la FCyT publica horarios en PDF \cite{horarios_fcyt}, se requiere extracción y normalización para poblar el modelo local: seguimiento de enlaces, descarga de PDFs, parsing de tablas y reglas de limpieza/validación. Aspectos teóricos: límites éticos/legales del scraping, robustez ante cambios de formato, trazabilidad de fuente \cite{mitchell2018-scraping,smith2007-tesseract}.

% (Figura de pipeline suprimida para evitar duplicación y exceso de tamaño)

% ----------------------------------------------------------------------
\section{ESTADO DEL ARTE Y REFERENTES}
A nivel local, el sistema Cappuccino ofrece organización web de horarios \cite{cappuccino_umss}. En el ámbito de apps móviles, Smart Timetable integra recordatorios y otras utilidades \cite{smart_timetable}. Soluciones institucionales como UniTime o FET abordan el timetabling a escala amplia (capacidad de aulas, múltiples estudiantes) con alcance distinto al enfoque personal \cite{unitime-ref,fet-ref}. Este panorama valida la combinación de: optimización centrada en el usuario, offline-first, notificaciones, y colaboración por JSON.

% ----------------------------------------------------------------------
\section*{Glosario EN–ES}
\begin{itemize}
  \item \textbf{Timetabling}: planificación de horarios académicos.
  \item \textbf{CSP (Constraint Satisfaction Problem)}: problema de satisfacción de restricciones.
  \item \textbf{Hard/Soft constraint}: restricción dura/blanda.
  \item \textbf{Offline-first}: diseño que prioriza operación sin conexión.
  \item \textbf{Exponential backoff}: espera incremental ante fallos.
  \item \textbf{Circuit breaker}: interruptor lógico para cortar reintentos.
  \item \textbf{On-device AI}: IA ejecutada en el dispositivo.
  \item \textbf{Share intent}: intención de compartir en Android.
  \item \textbf{Data freshness}: actualidad de los datos.
\end{itemize}

% ----------------------------------------------------------------------
\section{SÍNTESIS Y MAPA CONCEPTUAL}
TecnoTime formaliza el ATP como CSP individual, con hard constraints (sin choques) y soft constraints (gaps y favoritos), resuelto mediante retroceso con evaluación multi–criterio. La arquitectura offline-first aporta resiliencia (verificación de frescura, exponential backoff, circuit breaker y fallback por PDF). La interoperabilidad mediante JSON respalda la colaboración por mensajería; notificaciones y widget fomentan el uso cotidiano; la on-device AI (Simón) habilita personalización con control del usuario.

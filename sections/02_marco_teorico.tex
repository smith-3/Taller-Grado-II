\chapter{MARCO TEÓRICO}
\label{chap:marco-teorico}

Este capítulo presenta la base conceptual que sustenta \textit{TecnoTime}, una aplicación de gestión personal de horarios académicos. Se formaliza el \emph{Academic Timetabling Problem} (ATP) como un \emph{Constraint Satisfaction Problem} (CSP), se revisan enfoques de solución y se introducen principios teóricos conexos (arquitecturas \emph{offline-first}/\emph{local-first}, interoperabilidad basada en JSON, nociones generales de notificación, y criterios de calidad de software). El objetivo es proveer el \textbf{andamiaje teórico} (\emph{theoretical backbone}) que justifica el diseño posterior sin entrar en detalles de implementación.

% ---------------------------------------------------------
\section{Fundamentos del timetabling y el CSP}
\label{sec:fundamentos}

\subsection{Definiciones básicas}
El \textbf{timetabling académico} consiste en asignar eventos (clases) a recursos (aulas, docentes) y a ubicaciones temporales (días y franjas) cumpliendo \emph{restricciones duras} (factibilidad) y optimizando \emph{restricciones blandas} (calidad de uso) \cite{babaei2015-survey,kristiansen2013-survey}. 

En términos de CSP, se modela con:
\begin{itemize}
  \item \textbf{Variables}: decisiones de asignación (p.\,ej., selección de grupo por asignatura).
  \item \textbf{Dominios}: valores posibles por variable (grupos, bloques, etc.).
  \item \textbf{Restricciones}: predicados que limitan combinaciones válidas (solapes, disponibilidad).
\end{itemize}

\subsection{Complejidad}
El ATP es típicamente \textbf{NP-difícil} (\emph{NP-hard}); la complejidad crece rápidamente con el número de asignaturas, grupos y recursos, lo que motiva el uso de \emph{búsqueda con poda}, heurísticas y métodos multicriterio para alcanzar soluciones de buena calidad en tiempos razonables \cite{babaei2015-survey,kristiansen2013-survey}.

% ---------------------------------------------------------
\section{Formalización para un usuario (selección de grupos)}
\label{sec:formalizacion-usuario}

Se considera el caso de uso individual: un estudiante elige, para cada asignatura, un único grupo respetando factibilidad temporal y preferencias.

\subsection{Conjuntos, variables y datos}
Sean:
\begin{itemize}
  \item $C$: conjunto de \textbf{asignaturas} (materias).
  \item Para cada $c\in C$, $G(c)$: conjunto de \textbf{grupos} ofrecidos.
  \item $W$: conjunto de \textbf{días} de la semana; $\mathcal{I}$: conjunto de \textbf{intervalos} (día, inicio, fin).
  \item $y_{c,g}\in\{0,1\}$: variable binaria que indica si se elige el grupo $g\in G(c)$ para la asignatura $c$.
  \item $\text{sched}(c,g)\subseteq \mathcal{I}$: bloques temporales del grupo $g$ en $c$.
  \item $\text{fav}(c,g)\in\{0,1\}$: indicador de \textbf{docente favorito}; $D\subseteq W$ días \textbf{permitidos} por preferencia.
  \item $M\subseteq C$: subconjunto de \textbf{asignaturas obligatorias} (requisito curricular).
\end{itemize}

\subsection{Restricciones duras}
\begin{align}
  &\forall c\in C:\ \sum_{g\in G(c)} y_{c,g} = 1
  \quad\text{(exactamente un grupo por asignatura)} \label{eq:uno-por-materia} \\
  &\forall (c_1,g_1)\neq(c_2,g_2):\
    \text{overlap}\!\left(\text{sched}(c_1,g_1),\text{sched}(c_2,g_2)\right)
    \Rightarrow y_{c_1,g_1}+y_{c_2,g_2}\le 1
  \quad\text{(prohibición de solapes)} \label{eq:sin-choques} \\
  &\forall c\in M:\ \sum_{g\in G(c)} y_{c,g} = 1
  \quad\text{(cumplir asignaturas obligatorias)} \label{eq:obligatorias}
\end{align}

\paragraph{Variantes de factibilidad.}
En escenarios avanzados, puede \emph{permitirse} solape controlado como decisión explícita del usuario. Teóricamente, esto se representa cambiando \eqref{eq:sin-choques} por una \emph{penalización} en la función objetivo (ver multicriterio). Asimismo, el filtro de días $D$ puede imponerse como restricción ($\text{sched}(c,g)\subseteq D$) o como preferencia (penalización por violación).

\subsection{Función objetivo y evaluación multicriterio}
Se minimiza una suma ponderada de penalizaciones \cite{deb2001-moo}:
\begin{equation}
  \min_{y}\; F(y)= 
  w_{\text{gap}}\,\phi_{\text{gap}}(y)
  + w_{\text{fav}}\,\phi_{\text{fav}}(y)
  + w_{\text{dias}}\,\phi_{\text{dias}}(y)
  + w_{\text{solape}}\,\phi_{\text{solape}}(y),
  \quad w_k\ge 0
  \label{eq:objetivo}
\end{equation}
donde $\phi_{\text{gap}}$ mide \textbf{tiempos muertos intrajornada}, $\phi_{\text{fav}}$ \textbf{bonifica} docentes favoritos, $\phi_{\text{dias}}$ penaliza \textbf{dispersión en días no preferidos} y $\phi_{\text{solape}}$ castiga \textbf{choques} si se permiten. Para comparabilidad se recomienda \textbf{normalizar} cada $\phi_k$ en $[0,1]$ (o por máximos empíricos) y ajustar pesos $w_k$ según relevancia del usuario (\emph{user-centric weighting}).

\subsection{Notación resumida}
\begin{table}[htbp]
  \centering
  \small
  \begin{tabular}{>{\bfseries}m{2.8cm} m{8.5cm} m{2.5cm}}
    \hline
    Símbolo & Significado & Dominio \\
    \hline
    $C,\,G(c),\,W,\,\mathcal{I}$ & Conjuntos de materias, grupos, días e intervalos & Finito \\
    $y_{c,g}$ & Decisión de selección de grupo para $c$ & $\{0,1\}$ \\
    $\text{sched}(c,g)$ & Bloques (día, inicio, fin) del grupo & $\mathcal{I}$ \\
    $\phi_k,\,w_k$ & Penalizaciones y pesos en \eqref{eq:objetivo} & $\mathbb{R}_{\ge 0}$ \\
    $M$ & Asignaturas obligatorias & $\mathcal{P}(C)$ \\
    \hline
  \end{tabular}
  \caption{Notación utilizada en el modelo para un usuario. \emph{Fuente: elaboración propia}.}
  \label{tab:notacion}
\end{table}

% ---------------------------------------------------------
\section{Estado del arte de enfoques de solución}
\label{sec:estado-del-arte}

Las estrategias para timetabling se agrupan en \textbf{exactas}, \textbf{heurísticas} y \textbf{metaheurísticas} \cite{babaei2015-survey,kristiansen2013-survey}.

\begin{table}[htbp]
  \centering
  \small
  \begin{tabular}{>{\bfseries}m{3.2cm} m{4.2cm} m{4.6cm} m{2.8cm}}
    \hline
    Enfoque & Ejemplos típicos & Ventajas & Consideraciones \\
    \hline
    Exacto & ILP/CP, Branch \& Bound & Óptimo garantizado, trazabilidad & Coste alto; escalado limitado \\
    Heurístico & Greedy, constructivo, reparación & Rápido, simple, control local & Sensible a orden y reglas \\
    Metaheurístico & GA, SA, Búsqueda Tabú, VNS & Alta calidad promedio & Configuración y tiempo \\
    \hline
  \end{tabular}
  \caption{Panorama de enfoques para el ATP. \emph{Fuente: elaboración propia}.}
  \label{tab:enfoques}
\end{table}

\paragraph{Criterios de elección.} En el contexto \emph{on-device} (usuario individual), prima el \textbf{tiempo de respuesta} y la \textbf{simplicidad de configuración}; por ello suelen preferirse heurísticas con poda y evaluaciones multicriterio ligeras. Para evaluación comparativa, la literatura usa conjuntos de prueba como \emph{ITC-2007} (curriculum-based CTT), útiles para conceptos y métricas \emph{(benchmarks and metrics)}.

% ---------------------------------------------------------
\section{Fundamentos algorítmicos (búsqueda y evaluación)}
\label{sec:fundamentos-algoritmicos}

\subsection{Búsqueda con retroceso y poda}
La \textbf{búsqueda con retroceso} (\emph{backtracking}) explora combinaciones de grupos por asignatura y poda ramas que violan restricciones duras. Heurísticas de ordenación (p.\,ej., \emph{MRV}/\emph{LCV}) y \emph{poda temprana} reducen el espacio efectivo de búsqueda. La complejidad peor caso es $O(b^d)$, con $b$ el factor de ramificación y $d$ la profundidad (número de asignaturas).

\subsection{Evaluación multicriterio}
La función \eqref{eq:objetivo} combina penalizaciones normalizadas. Dos aspectos prácticos:
\begin{enumerate}
  \item \textbf{Escalamiento y normalización}: evita que una métrica domine por magnitud.
  \item \textbf{Ajuste de pesos}: puede fijarse por defecto y permitir \emph{tuning} explícito (sin cambiar el modelo).
\end{enumerate}
Alternativamente, se puede abordar como problema \textbf{multiobjetivo} con \emph{frentes de Pareto} (\emph{Pareto fronts}); al sintetizar para el usuario, suele preferirse una \textbf{función agregada} por su interpretabilidad \cite{deb2001-moo}.

% ---------------------------------------------------------
\section{Principios teóricos: offline-first y local-first}
\label{sec:offline-first}

\textbf{Offline-first} y \textbf{local-first} priorizan disponibilidad sin red, resiliencia y control del usuario sobre sus datos. A nivel conceptual:
\begin{itemize}
  \item \textbf{Caché local y fuente de verdad} (\emph{single source of truth} local) con sincronización diferida.
  \item \textbf{Consistencia eventual} y \textbf{detección de frescura} (\emph{data freshness checks}) antes de propagar cambios.
  \item \textbf{Estrategias de robustez}: \emph{exponential backoff}, \emph{circuit breakers}, reintentos acotados.
  \item \textbf{Resolución de conflictos} basada en marcas de tiempo o políticas deterministas \emph{(conflict-free policies)}.
\end{itemize}
Estos principios justifican decisiones de diseño orientadas a \textbf{disponibilidad}, \textbf{tolerancia a fallos} y \textbf{privacidad} (procesamiento local). % (Conceptual, sin entrar en frameworks específicos)

% ---------------------------------------------------------
\section{Interoperabilidad y colaboración basada en JSON}
\label{sec:interoperabilidad-json}

Para intercambio de información entre usuarios (\emph{social sharing}) se adopta un formato \textbf{JSON} conforme al estándar \cite{rfc8259-json}. A nivel teórico:
\begin{itemize}
  \item \textbf{Esquema lógico} claro (identificadores de asignaturas, grupos, bloques).
  \item \textbf{Validación}: integridad de referencias, ausencia de duplicados.
  \item \textbf{Portabilidad}: independencia del medio de transporte (mensajería, correo, etc.).
\end{itemize}

\begin{figure}[htbp]
  \centering
  % Placeholder gráfico sin dependencia de archivos externos
  \fbox{\begin{minipage}[c][42mm][c]{0.85\linewidth}\centering
    \textit{Esquema conceptual de intercambio JSON (materias, grupos, bloques, metadatos)}\\[2mm]
    \small Fuente: elaboración propia
  \end{minipage}}
  \caption{Intercambio conceptual de horarios en JSON.}
  \label{fig:json-schema}
\end{figure}

% ---------------------------------------------------------
\section{Obtención desde PDF institucional y limpieza de datos}
\label{sec:pdf-parsing}

Cuando las ofertas oficiales se publican en \textbf{PDF}, la teoría de extracción de información contempla:
\begin{itemize}
  \item \textbf{Detección de tablas} y \textbf{parsing estructurado}, con reglas de \emph{post-procesamiento} para normalizar filas/columnas.
  \item \textbf{OCR de respaldo} en documentos no textuales o mixtos (p.\,ej., con Tesseract \cite{smith2007-tesseract}).
  \item \textbf{Trazabilidad de la fuente} y \textbf{ética de extracción} (respeto a cambios, identificación de versión, citas).
\end{itemize}
Estas prácticas soportan la \textbf{calidad de datos} previa al modelado (consistencia, completitud, ausencia de duplicados).

% ---------------------------------------------------------
\section{Propiedades de calidad y criterios no funcionales}
\label{sec:prop-no-func}

Desde una perspectiva teórica (alineable con marcos como ISO/IEC 25010), se consideran:
\begin{itemize}
  \item \textbf{Rendimiento y eficiencia}: tiempos de respuesta adecuados para uso interactivo (\emph{interactive latency}).
  \item \textbf{Confiabilidad y resiliencia}: recuperación ante fallos transitorios; degradación controlada.
  \item \textbf{Mantenibilidad}: modularidad, bajo acoplamiento, alta cohesión; separación de responsabilidades.
  \item \textbf{Compatibilidad y portabilidad}: independencia de formato (p.\,ej., JSON), exportabilidad de datos del usuario.
  \item \textbf{Usabilidad}: claridad de representación del horario; reducción de errores de selección (\emph{error prevention}).
\end{itemize}

% ---------------------------------------------------------
\section{Síntesis integradora}
\label{sec:sintesis}

El ATP se modela como un CSP con restricciones duras (factibilidad: sin solapes, obligatoriedad curricular) y blandas (calidad de uso: intrajornada, docentes favoritos, días preferidos). La solución favorece \textbf{búsqueda con poda} más \textbf{evaluación multicriterio} por su equilibrio entre \emph{calidad} y \emph{tiempo de respuesta} en el contexto individual. Los principios \textbf{offline/local-first} justifican disponibilidad y robustez; la \textbf{interoperabilidad JSON} habilita colaboración y portabilidad de horarios. Este marco teórico provee los fundamentos para el diseño y evaluación de la solución propuesta.

% ---------------------------------------------------------
\section*{Glosario EN–ES}
\begin{itemize}
  \item \textbf{Timetabling}: planificación de horarios.
  \item \textbf{CSP (Constraint Satisfaction Problem)}: problema de satisfacción de restricciones.
  \item \textbf{Restricción dura/blanda}: condición de factibilidad/criterio de calidad.
  \item \textbf{Multicriterio}: combinación de varias métricas en la evaluación.
  \item \textbf{Offline-first / Local-first}: prioridad de operación sin red y datos locales como fuente primaria.
  \item \textbf{Data freshness}: actualidad/validación de los datos.
  \item \textbf{Pareto front}: conjunto de soluciones no dominadas en optimización multiobjetivo.
\end{itemize}

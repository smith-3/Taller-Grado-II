\chapter{CONCLUSIONES Y RECOMENDACIONES}

% Evita overfull en párrafos largos
\setlength{\emergencystretch}{2em}
\renewcommand{\arraystretch}{1.15}

\section{CONCLUSIONES}

\begin{enumerate}
  \item La aplicación móvil \textbf{TecnoTime} responde a una necesidad concreta de los estudiantes de la FCyT--UMSS: organizar horarios académicos de forma clara, accesible y personalizable, superando limitaciones del sistema web Cappuccino \cite{cappuccino_umss}.
  \item El desarrollo se realizó en entorno Android con arquitectura \textbf{MVVM} y principios de \textbf{Clean Architecture} \cite{martin2017-cleanarch}, logrando modularidad, escalabilidad y mantenibilidad.
  \item La aplicación consume \textbf{PDFs oficiales} publicados por la FCyT \cite{horarios_fcyt} mediante extracción y análisis robustos, sin requerir un servicio intermedio.
  \item El \textbf{modelo de datos} normalizado integra carreras, niveles, materias, grupos, horarios, docentes y aulas, lo que habilita validaciones y consultas coherentes.
  \item El uso de \textbf{Room} \cite{room_db} permite operación con prioridad sin conexión (\textit{offline-first}) y sincronización eficiente, lo que favorece la experiencia de uso.
  \item TecnoTime ofrece funcionalidades avanzadas: generación de horarios (sin choques por defecto), heurísticas de optimización (huecos, favoritos), notificaciones, exportación (PDF/imagen/Excel) y personalización visual; alineadas con prácticas de aplicaciones de referencia como \textit{Smart Timetable} \cite{smart_timetable}.
  \item El sistema promueve \textbf{independencia operativa} (menor dependencia de infraestructura institucional en tiempo real) y \textbf{personalización} del entorno académico.
  \item Se cumplieron los objetivos planteados, entregando una herramienta funcional, adaptable y centrada en la organización académica del estudiante.
\end{enumerate}

\section{EVIDENCIAS DE CUMPLIMIENTO DE OBJETIVOS}

\begin{table}[htbp]
\centering
\caption{Objetivos generales y específicos con evidencias. Fuente: elaboración propia.}
\label{tab:5-1}
\small
\begin{tabular}{|p{0.28\textwidth}|p{0.42\textwidth}|p{0.22\textwidth}|}
\hline
\textbf{Objetivo} & \textbf{Evidencias} & \textbf{Fuente} \\
\hline
Objetivo General: Desarrollar una aplicación móvil para la gestión de horarios académicos en la FCyT--UMSS. 
& Repositorio con código (\textit{TecnoTime}); pruebas unitarias/integración (TC--001..016); preferencia reportada por Android. 
& \cite{horarios_fcyt,cappuccino_umss,smart_timetable} \\
\hline
OE1: Organización personalizada de horarios exclusivamente académicos (FCyT). 
& Selección de materias/grupos con vista previa; widget de consulta; sin mezcla con eventos personales. 
& \cite{horarios_fcyt} \\
\hline
OE2: Ajuste manual según inscripción; widget; guardado local. 
& Flujo nivel$\rightarrow$materia$\rightarrow$grupo con \textit{preview}; widget en inicio; persistencia Room. 
& \cite{room_db} \\
\hline
OE3: Descarga e intercambio en múltiples formatos. 
& Exportación a imagen/PDF/Excel/JSON; importación JSON (total/parcial); validación de duplicados; uso social (WhatsApp). 
& \cite{smart_timetable} \\
\hline
OE4: Sugerencias optimizadas reduciendo huecos. 
& Backtracking con MRV/LCV y evaluación compuesta (huecos, favoritos, días) manteniendo Top--N. 
& \cite{babaei2015-survey,kristiansen2013-survey} \\
\hline
OE5: UX móvil clara e intuitiva. 
& Onboarding progresivo; vista semanal/diaria; deslizamiento por días; formato 24/12\,h; personalización visual. 
& \cite{smart_timetable} \\
\hline
\end{tabular}
\end{table}

\section{APORTES DEL PROYECTO}

\begin{table}[htbp]
\centering
\caption{Aportes a módulos y su impacto. Fuente: elaboración propia.}
\label{tab:5-2}
\small
\begin{tabular}{|p{0.28\textwidth}|p{0.42\textwidth}|p{0.22\textwidth}|}
\hline
\textbf{Aporte} & \textbf{Módulos/Capacidades} & \textbf{Impacto esperado/observado} \\
\hline
Aplicación móvil \textit{TecnoTime} 
& F1: Ingesta PDF$\rightarrow$BD; F2: Modelo de dominio; F3: Generador (backtracking+heurísticas); F4: Prioridad sin conexión (caché, frescura, backoff/cortacircuito); F5: Interoperabilidad JSON; F6: UX integral; F7: Recordatorios \& IA; F8: Exportaciones; F9: Aseguramiento de calidad; F10: Seguridad/Privacidad; F11: Empaque/Release. 
& Reducción del tiempo para construir horarios sin choques (TTS $< 5$ min); disminución de huecos diarios; adopción de intercambio JSON; resiliencia en modo sin conexión. \\
\hline
\end{tabular}
\end{table}

\section{LIMITACIONES Y MITIGACIONES}

\begin{table}[htbp]
\centering
\caption{Limitaciones del proyecto, mitigaciones y estado. Fuente: elaboración propia.}
\label{tab:5-3}
\small
\begin{tabular}{|p{0.36\textwidth}|p{0.44\textwidth}|p{0.12\textwidth}|}
\hline
\textbf{Limitación} & \textbf{Mitigación} & \textbf{Estado} \\
\hline
Enfoque en carreras de FCyT--UMSS (no otras facultades/universidades). 
& Diseño modular para extensión futura (URLs y parsers por unidad académica). 
& Mitigado \\
\hline
Disponibilidad sólo en Android; sin iOS/escritorio. 
& Predominancia Android; app nativa optimizada. 
& Mitigado \\
\hline
No publicación en Play Store (distribución interna). 
& Gestión directa por desarrolladores/facultad. 
& Mitigado \\
\hline
Exclusión de eventos personales. 
& Enfoque académico evita dilución funcional. 
& Mitigado \\
\hline
Dependencia de estructura/ubicación de PDFs institucionales. 
& Parser tolerante; pruebas de regresión; importación manual de PDF. 
& Implementado \\
\hline
Conectividad requerida solo para sincronizar (consulta y generación en modo sin conexión habilitadas). 
& Caché local; verificación de frescura; \textit{backoff} exponencial y cortacircuito; mecanismo de respaldo (\textit{fallback}) por PDF. 
& Implementado \\
\hline
\end{tabular}
\end{table}

\section{RECOMENDACIONES}

\begin{enumerate}
  \item Integrar sincronización opcional con calendarios externos (por ejemplo, Google Calendar) manteniendo control de privacidad.
  \item Ampliar personalización visual (paletas, etiquetas, emojis, estilos) por materia/grupo.
  \item Permitir agregar eventos no académicos (reuniones, prácticas) sin perder el foco académico.
  \item Parametrizar orígenes PDF para facilitar extensión a otras unidades académicas usando la misma arquitectura \cite{horarios_fcyt}.
  \item Mantener el parser actualizado ante cambios de formato/ubicación en la web oficial \cite{horarios_fcyt}.
  \item Ejecutar ciclos de pruebas de UX al cierre de cada semestre para recoger retroalimentación y ajustar.
  \item Continuar documentando código, estructuras y flujos clave para facilitar contribuciones y evolución del proyecto.
\end{enumerate}

\section{TRABAJO FUTURO}

\begin{enumerate}
  \item Extender a iOS/escritorio (evaluar frameworks multiplataforma).
  \item Integrar IA avanzada para sugerencias basadas en historial/afinidad y patrones de uso.
  \item Colaboración en tiempo real para horarios grupales (proyectos/equipos).
  \item Notificaciones con contexto (ubicación, clima) preservando privacidad.
  \item Explorar integración con sistemas institucionales (backend oficial) para sincronización directa.
  \item Estudios longitudinales de impacto (asistencia, gestión del tiempo).
  \item Versión web ligera para consulta desde navegadores.
  \item Gamificación (logros, rachas) para fomentar uso sostenido.
\end{enumerate}

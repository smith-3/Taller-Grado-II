\chapter{CONCLUSIONES Y RECOMENDACIONES}

% Afinar cortes de línea y tablas
\setlength{\emergencystretch}{2em}
\renewcommand{\arraystretch}{1.15}

\section{Conclusiones}

\noindent Al concluir el desarrollo de TecnoTime, se presentan las siguientes conclusiones derivadas de los desafíos técnicos enfrentados, las soluciones implementadas y los aprendizajes obtenidos durante el proceso.

\begin{enumerate}
\setlength\itemsep{0.45em}\setlength\parskip{0pt}\setlength\parsep{0pt}

\item \textbf{Desarrollo de aplicación móvil para organización de horarios académicos.} El mayor desafío fue adaptar el algoritmo de generación a preferencias heterogéneas de estudiantes. Inicialmente se consideró grafos para ``el mejor'' horario, pero usuarios mostraron preferencias variadas (días libres vs.\,compactos vs.\,docentes). Se adoptó estrategia multicriterio generando múltiples opciones configurables. La validación iterativa con usuarios evitó soluciones técnicamente correctas pero poco útiles.

\item \textbf{Organización de carga académica mediante selección guiada.} En fases finales surgió problema crítico: materias con componente teórico y laboratorio (p.\,ej., Física I) no se distinguían en el sistema web de la facultad. Se extendió el modelo de datos tratándolos como materias independientes con vinculación lógica, modificando el parser de PDFs. La arquitectura limpia facilitó esta extensión tardía sin comprometer componentes existentes.

\item \textbf{Ajuste manual de horarios y persistencia de preferencias.} El scraping de PDFs generaba errores por símbolos especiales y falta de datos de docentes, rompiendo integridad referencial. Se implementó formateo riguroso con validación y contingencias inteligentes reutilizando docentes de semestres anteriores. Separar teórico/laboratorio requirió migraciones complejas tardías. La limpieza de datos resultó tan crítica como el diseño del esquema.

\item \textbf{Facilitar descarga e intercambio de horarios.} La generación de imagen y PDF renderiza HTML del horario en componente web interno; Excel se genera directamente sin renderizado visual. El desafío fue alto consumo de memoria al crear imágenes grandes (~28 MB), causando errores en dispositivos de gama baja. Se solucionó liberando memoria tras comprimir, calculando dimensiones dinámicamente y ejecutando en hilo principal con control de cancelación.

\item \textbf{Generación de horarios optimizados.} El enfoque inicial con grafos mostró limitaciones: complejidad elevada, única solución generada y falta de escalabilidad a múltiples criterios. Se implementó backtracking heurístico con poda inteligente y evaluación multicriterio, generando top-N soluciones. Heurísticas eficientes con criterios claros superaron el enfoque teóricamente óptimo en escenarios reales.

\item \textbf{Optimización de experiencia móvil.} Las primeras versiones generaban confusión sobre funcionalidades y flujos. El feedback reveló mal entendimiento del usuario. Se implementó onboarding interactivo, rediseño de pantallas basado en testing con usuarios reales, y tooltips contextuales en puntos críticos. La interfaz intuitiva requiere iteración con usuarios reales, no diseño único basado en suposiciones del desarrollador.

\end{enumerate}

\vspace{0.5em}
\noindent El desarrollo de TecnoTime demostró que construir software útil requiere validación continua con usuarios, resiliencia ante datos imperfectos, arquitectura flexible que permita incorporar cambios significativos, e iteración constante basada en feedback real. Las suposiciones iniciales sobre preferencias de horarios y flujos de UI fueron desafiadas por la realidad.

\noindent Los desafíos enfrentados, desde algoritmos inadecuados hasta problemas de scraping y UI confusa, fueron oportunidades de aprendizaje que fortalecieron tanto el producto final como las competencias técnicas adquiridas.

\section{Recomendaciones}

\noindent A partir de las limitaciones identificadas durante el desarrollo y el feedback de los usuarios, se presentan las siguientes recomendaciones orientadas a extender la funcionalidad de TecnoTime y mejorar su utilidad en el contexto académico real.

\begin{enumerate}
\setlength\itemsep{0.45em}\setlength\parskip{0pt}\setlength\parsep{0pt}

\item \textbf{Edición manual de horarios generados.} Los docentes frecuentemente cambian grupos o salones sin que esto se refleje en los PDFs oficiales. Se recomienda implementar funcionalidad de edición que permita al estudiante modificar horarios generados: cambiar grupos, docentes, aulas y horarios de clases específicas. Esto evitaría que el usuario deba regenerar todo el horario por un único cambio y aumentaría la utilidad real de la aplicación ante inconsistencias en datos oficiales.

\item \textbf{Integración de actividades personales y extracurriculares.} TecnoTime actualmente solo gestiona horarios académicos formales, pero los estudiantes tienen otras actividades: deportes, reuniones estudiantiles, trabajos, compromisos personales. Se recomienda permitir agregar eventos personalizados con título, horario, recurrencia y ubicación, visualizándolos junto a clases académicas. Esto transformaría la aplicación en un gestor integral del tiempo del estudiante.

\item \textbf{Sistema de notas y recordatorios académicos.} La aplicación carece de funcionalidad para registrar fechas de exámenes, entregas de trabajos o recordatorios académicos vinculados a materias. Se recomienda implementar un módulo de notas asociadas a cada materia, permitiendo agregar eventos importantes (exámenes, laboratorios, presentaciones) con notificaciones configurables. Esta funcionalidad complementaría la organización temporal con gestión de tareas académicas.

\item \textbf{Extensión a otras facultades y universidades.} La arquitectura actual depende del formato específico de PDFs de la FCyT. Para extender TecnoTime a otras facultades o universidades, se requiere que las instituciones publiquen horarios en formato estructurado similar o proporcionen endpoint oficial. De no ser posible, sería necesario desarrollar parsers específicos por facultad, aumentando significativamente el esfuerzo de mantenimiento. La estandarización de formatos institucionales facilitaría esta expansión.

\item \textbf{Despliegue institucional y mantenimiento de datos.} Se recomienda formalizar un canal de distribución oficial (intranet/tienda institucional), firmar digitalmente versiones, y establecer monitoreo del formato y ubicación de PDFs antes de cada periodo de inscripciones. Un endpoint oficial de la universidad reemplazaría la ingesta desde PDF, mejorando confiabilidad y eliminando la fragilidad del scraping.

\item \textbf{Notificaciones inteligentes y funcionalidades avanzadas.} Como extensión futura, podrían explorarse notificaciones inteligentes basadas en IA para recordatorios contextuales, detección de conflictos en tiempo real, o sugerencias de redistribución de carga académica. Sin embargo, estas mejoras son secundarias frente a las necesidades puntuales de edición manual, actividades personales y sistema de notas identificadas como prioritarias.

\end{enumerate}

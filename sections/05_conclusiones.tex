\chapter{CONCLUSIONES Y RECOMENDACIONES}

% Afinar cortes de línea y tablas
\setlength{\emergencystretch}{2em}
\renewcommand{\arraystretch}{1.15}

\begin{enumerate}
\setlength\itemsep{0.45em}\setlength\parskip{0pt}\setlength\parsep{0pt}

\item (OG) Se comprobó el objetivo general: TecnoTime satisface la necesidad de organizar horarios académicos de forma clara, accesible y personalizable para estudiantes de la FCyT-UMSS, reduciendo pasos y errores frente al flujo web vigente (p.\,ej., Cappuccino \cite{cappuccino_umss}) y favoreciendo su uso cotidiano más allá del periodo de inscripción.

\item (OE4) La generación por defecto de horarios sin choques, junto con la evaluación multicriterio (huecos, docentes favoritos y días activos), produjo soluciones de calidad en tiempos compatibles con la ejecución on-device. Ello permitió mantener el enfoque en el caso de uso individual, respetando restricciones duras y optimizando preferencias blandas.

\item (OE3) La exportación/importación en formatos imagen/PDF/Excel y el archivo estructurado (JSON) habilitaron coordinación entre pares por mensajería con menor riesgo de transcripción y mejor trazabilidad. Esta interoperabilidad facilita reproducir, comparar y adoptar configuraciones de horario en contextos reales de curso.

\item (OE1) La organización exclusivamente académica se garantizó mediante selección guiada de materias/grupos, evitando mezclar eventos personales. Las vistas semanal y diaria, junto con la vista previa por grupo, contribuyeron a una lectura rápida y consistente del horario.

\item (OE2) La continuidad de uso se sostuvo con persistencia local y memoria de contexto, mitigando la frustración por repetir carrera-nivel-materia. El funcionamiento con prioridad sin conexión aseguró acceso al horario y ajustes esenciales en escenarios de conectividad limitada.

\item (OE5) La experiencia de uso móvil resultó clara e intuitiva: onboarding progresivo, preferencias de presentación (formato 24/12\,h, visibilidad por día) y componente de acceso rápido favorecieron la consulta diaria. Estas decisiones de UX se alinean con las necesidades observadas en el área de aplicación.

\item (Arquitectura \& Datos) La adopción de MVVM y principios de arquitectura limpia \cite{martin2017-cleanarch}, junto con la ingesta de PDFs oficiales \cite{horarios_fcyt} y el almacenamiento local con Room \cite{room_db}, permitieron operación con prioridad sin conexión, verificación de frescura y menor dependencia de infraestructura en tiempo real, mejorando mantenibilidad y escalabilidad.

\item (Impacto y transferencia) La solución reduce el tiempo para construir horarios sin choques, disminuye huecos diarios y normaliza el intercambio estructurado entre estudiantes. El diseño modular y el uso de formatos abiertos sientan bases para su extensión a otras unidades académicas con mínimos cambios.

\item (Límites reconocidos) La disponibilidad actual se centra en Android y depende de la estructura y publicación de PDFs institucionales; se mitigó con caché local, sincronización resiliente (espera exponencial y cortacircuito lógico) e importación manual de PDF. Se recomienda continuar validando en nuevos periodos y revisar sistemáticamente cambios de formato y ubicación de la fuente oficial.
\end{enumerate}
\chapter{CONCLUSIONES Y RECOMENDACIONES}

% Afinar cortes de línea y tablas
\setlength{\emergencystretch}{2em}
\renewcommand{\arraystretch}{1.15}

\noindent A continuación se presentan las conclusiones principales, alineadas explícitamente con el objetivo general (OG) y los objetivos específicos (OE1–OE5) planteados en el Capítulo 1.

\begin{enumerate}
\setlength\itemsep{0.45em}\setlength\parskip{0pt}\setlength\parsep{0pt}

\item (OG) Se cumplió el objetivo general: TecnoTime permite organizar y visualizar horarios académicos de forma clara, accesible y personalizable para estudiantes de la FCyT-UMSS, reduciendo pasos y errores frente al flujo web vigente (p.\,ej., Cappuchino \cite{cappuccino_umss}) y manteniendo utilidad más allá del periodo de inscripción.

\item (OE1) Se cumplió el objetivo de organizar exclusivamente la carga académica mediante selección guiada de materias y grupos, evitando eventos personales. Las vistas semanal/diaria y la vista previa por grupo ofrecen lectura rápida y consistente.

\item (OE2) Se cumplió el objetivo de ajuste manual y persistencia: el usuario puede seleccionar/cambiar grupos, conservar preferencias y consultar un widget (componente visual que se muestra en la pantalla principal del dispositivo); la memoria de contexto reduce repetición de carrera–nivel–materia.

\item (OE3) Se cumplió el objetivo de facilitar descarga e intercambio: exportación a imagen/PDF/Excel y archivo estructurado en JSON (formato de intercambio de datos basado en texto) habilitan coordinación por mensajería con menor riesgo de transcripción y mejor trazabilidad.

\item (OE4) Se cumplió el objetivo de sugerir horarios optimizados: la generación por defecto sin choques, con evaluación multicriterio (huecos, docentes favoritos y días activos), produce soluciones de calidad en tiempos compatibles con ejecución en el dispositivo.

\item (OE5) Se cumplió el objetivo de optimizar la experiencia móvil: onboarding progresivo, preferencias de presentación (formato 24/12\,h, visibilidad por día) y componente de acceso rápido favorecen la consulta diaria y la adopción.
\end{enumerate}

\section{Recomendaciones}
\begin{itemize}
  \item Despliegue institucional: formalizar un canal de distribución (intranet/tienda) y firmar versiones; evaluar un endpoint (punto final de una API) oficial para horarios a fin de reemplazar la ingesta desde PDF.
  \item Mantenimiento de datos: hacer seguimiento de cambios en el formato/ubicación de los PDFs; programar validaciones previas a inscripciones; mantener fallback por PDF local.
  \item Portabilidad: explorar extensión a iOS y otras facultades; considerar un modo “lectura” web para consulta rápida.
  \item Métricas opt-in: incorporar telemetría voluntaria para mejorar ranking de generador, detectar fallas de sincronización y medir adopción sin comprometer privacidad.
  \item Rendimiento y QA: ampliar pruebas de UI para export/import; realizar benchmarks por dispositivo; asegurar TTS \(\leq\) objetivos definidos.
  \item Seguridad y privacidad: revisar periódicamente permisos, almacenamiento y cifrado de datos sensibles; mantener IA en el dispositivo como activación voluntaria, con control de descarga/borrado.
  \item Documentación y soporte: mantener actualizados los manuales de instalación y de usuario (ver Anexos A–B) y una guía de preguntas frecuentes.
\end{itemize}

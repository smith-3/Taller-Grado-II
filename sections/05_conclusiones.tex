\chapter{CONCLUSIONES Y RECOMENDACIONES}

% Afinar cortes de línea y tablas
\setlength{\emergencystretch}{2em}
\renewcommand{\arraystretch}{1.15}

\section{Conclusiones}

\noindent En este capítulo se presentan las conclusiones obtenidas a partir del desarrollo del proyecto, en función del objetivo general y los objetivos específicos planteados.

\begin{itemize}
\setlength\itemsep{0.45em}\setlength\parskip{0pt}\setlength\parsep{0pt}

\item Al desarrollar el sistema de personalización, se identificó que los estudiantes poseen preferencias de organización contradictorias (días libres vs. horarios compactos), lo que impedía ofrecer una solución única. Este obstáculo se superó implementando una estrategia multicriterio que permite al usuario configurar sus propias prioridades, cumpliendo así el objetivo de flexibilidad.

\item Durante la implementación del módulo de ajuste manual, surgió la dificultad de garantizar el acceso rápido a la información del horario sin obligar al usuario a navegar constantemente por la aplicación. Se solucionó este problema mediante el desarrollo de un widget de escritorio que persiste la información seleccionada, facilitando la consulta inmediata.

\item La funcionalidad de descarga e intercambio de horarios enfrentó fallos técnicos en dispositivos de gama media debido al alto consumo de memoria al generar imágenes de alta resolución. Esta limitación se resolvió optimizando el motor de renderizado y la gestión de memoria, logrando que la exportación sea eficiente y accesible para todos los dispositivos.

\item Para la sugerencia de horarios optimizados, el enfoque inicial basado en grafos resultó ineficiente para manejar la complejidad de múltiples restricciones simultáneas. Se optó por reemplazarlo con un algoritmo de backtracking heurístico, el cual demostró ser capaz de minimizar los tiempos de espera entre clases de manera efectiva.

\item En la optimización de la interfaz de usuario, las pruebas iniciales revelaron que los flujos de navegación resultaban confusos para los estudiantes, dificultando el uso de la herramienta. Se abordó este problema mediante un rediseño iterativo y la incorporación de un onboarding interactivo, asegurando una experiencia de usuario clara e intuitiva.

\item La ingesta de datos académicos presentó el mayor desafío técnico debido a la inconsistencia y falta de estructura en los archivos PDF de la facultad. Fue necesario reestructurar el parser y el modelo de datos para manejar estas irregularidades, permitiendo distinguir correctamente entre componentes teóricos y prácticos.

\item Aunque el proyecto enfrentó desafíos significativos relacionados con la calidad de los datos y la diversidad de dispositivos, la aplicación móvil desarrollada logró gestionar los horarios de manera eficiente. La validación con usuarios confirmó que las soluciones técnicas adoptadas permitieron superar estas barreras, cumpliendo satisfactoriamente el objetivo general del proyecto.

\end{itemize}

\section{Recomendaciones}

\noindent A partir de las limitaciones identificadas durante el desarrollo y el feedback de los usuarios, se presentan las siguientes recomendaciones orientadas a extender la funcionalidad de TecnoTime y mejorar su utilidad en el contexto académico real.

\begin{itemize}
\setlength\itemsep{0.45em}\setlength\parskip{0pt}\setlength\parsep{0pt}

\item Se recomienda implementar una funcionalidad de edición manual profunda que permita modificar horarios ya generados (cambiar grupos o aulas específicas). Esto cubriría los casos donde los docentes realizan cambios informales no reflejados en los datos oficiales, aumentando la utilidad de la aplicación ante la realidad operativa de la facultad.

\item Se sugiere integrar la gestión de actividades personales y extracurriculares. Permitir a los estudiantes agregar eventos propios (deportes, trabajo) junto con su carga académica transformaría la aplicación en un gestor integral del tiempo, cubriendo una necesidad que excede el ámbito puramente académico.

\item Se recomienda desarrollar un módulo de notas y recordatorios académicos vinculados a las materias. La capacidad de registrar fechas de exámenes y entregas complementaría la organización temporal, ofreciendo una herramienta más completa para el seguimiento del rendimiento académico.

\item Para extender el sistema a otras facultades, se recomienda estandarizar los formatos de entrada de datos. La dependencia actual del formato específico de los PDFs de la FCyT es una limitante; se sugiere promover la implementación de servicios web institucionales (API) que provean la información de horarios de manera estructurada y confiable.

\item Se recomienda a la institución formalizar un canal de distribución oficial y establecer protocolos de verificación de los archivos de horarios. Mantener una ubicación y formato consistentes para los documentos fuente antes de cada periodo de inscripciones es fundamental para garantizar la disponibilidad y confiabilidad del servicio para los estudiantes.

\item Se sugiere explorar en trabajos futuros la incorporación de notificaciones inteligentes basadas en inteligencia artificial. Estas podrían ofrecer recordatorios contextuales o sugerencias de redistribución de carga académica en tiempo real, elevando el nivel de asistencia que la aplicación proporciona al estudiante.

\end{itemize}

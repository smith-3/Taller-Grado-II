\chapter{CONCLUSIONES Y RECOMENDACIONES}

\emergencystretch=1em

\section{Conclusiones}

\begin{enumerate}
  \item La aplicación móvil TecnoTime responde a una necesidad concreta de los estudiantes de la Facultad de Ciencias y Tecnología de la UMSS: organizar de forma clara, accesible y personalizada sus horarios académicos, a diferencia de soluciones previas como Cappuccino UMSS \cite{cappuccino_umss}.

  \item El desarrollo completo se realizó utilizando exclusivamente Android Studio \cite{android_studio}, bajo una arquitectura MVVM y principios de Clean Architecture \cite{pedreira2021, medina2014}, logrando una solución modular, escalable y fácil de mantener.

  \item La app se alimenta directamente de los PDFs oficiales de horarios publicados en el sitio web de la FCyT \cite{horarios_fcyt}, aplicando técnicas de extracción mediante web scraping y procesamiento por expresiones regulares, sin necesidad de un backend intermedio.

  \item El modelo de datos está cuidadosamente diseñado y normalizado, abarcando carreras, niveles, materias, grupos, horarios, docentes, aulas y eventos académicos, lo que permite funcionalidades sólidas y coherentes.

  \item La incorporación de una base de datos local (Room) \cite{room_db} permite acceso completamente offline y sincronización eficiente, mejorando la experiencia del usuario.

  \item TecnoTime ofrece funcio\-nalidades avanzadas como: generación automática de horarios, selección de materias, sugerencias opti\-mizadas, notificaciones, exportación a PDF e imagen, edición de colores y emojis por grupo, y más. Estas funciones se alinean con las mejores prácticas observadas en aplicaciones de refe\-rencia como Smart Timetable \cite{smart_timetable}.

  \item El sistema mantiene un enfoque de inde\-pendencia total, donde los estudiantes no dependen de validaciones externas ni de infraestructura ins\-titucional, y pueden perso\-nalizar su entorno académico según sus necesidades, aspecto también valorado en estudios sobre aplicaciones móviles universitarias \cite{campushome_apps, lideres_apps}.

  \item Se cumplió plenamente con los objetivos planteados, entregando una herramienta funcional, adaptable y enfocada en mejorar la organización académica del estudiante universitario.
\end{enumerate}

\section{Recomendaciones}

\begin{enumerate}
  \item Se recomienda implementar una funcionalidad para sincronizar eventos con Google Calendar \cite{google_calendar}, permitiendo al usuario integrar su vida académica con otros calendarios personales.

  \item Se sugiere ampliar las opciones de perso\-nalización, permitiendo mayor variedad de colores, etiquetas, emojis, y estilos visuales para los grupos y materias, adaptándose a las pre\-ferencias individuales de cada usuario.

  \item Podría incluirse la posi\-bilidad de agregar eventos no académicos, como reuniones, prácticas, reco\-rdatorios personales, manteniendo el foco en la planificación del tiempo dentro del entorno universitario.

  \item En caso de querer extender la aplicación a otras facultades o universidades, se recomienda crear una tabla de facultades y carreras, y establecer una es\-tructura común de URLs o sitios donde cada unidad aca\-démica publique sus propios archivos PDF de horarios \cite{fcyt_umss}.

  \item Es im\-portante mantener actualizada la lógica de extracción de datos (scraping) ante posibles cambios en el formato o ubi\-cación de los horarios oficiales en la página web de la FCyT \cite{horarios_fcyt}.

  \item Se aconseja seguir realizando pruebas de ex\-pe\-rien\-cia de usua\-rio (UX) y adaptaciones según la re\-tro\-ali\-men\-ta\-ción, especialmente al finalizar cada semestre académico, cuando los horarios cambian y los estudiantes dependen más de la herramienta.

  \item Finalmente, se recomienda continuar docu\-mentando claramente el código fuente, las es\-tructuras de datos y los flujos principales, para faci\-litar la contribución de otros desa\-rrolladores y la evo\-lución natural del proyecto.
\end{enumerate}
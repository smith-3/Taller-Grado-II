\chapter{CONCLUSIONES Y RECOMENDACIONES}

% Afinar cortes de línea y tablas
\setlength{\emergencystretch}{2em}
\renewcommand{\arraystretch}{1.15}

\section{Conclusiones}

\noindent Al concluir el desarrollo de TecnoTime, se presentan las siguientes conclusiones derivadas de los desafíos técnicos enfrentados, las soluciones implementadas y los aprendizajes obtenidos durante el proceso, alineadas con el objetivo general (OG) y los objetivos específicos (OE1–OE5).

\begin{enumerate}
\setlength\itemsep{0.45em}\setlength\parskip{0pt}\setlength\parsep{0pt}

\item \textbf{Desarrollo de aplicación móvil para organización de horarios académicos.} El mayor desafío fue adaptar el algoritmo de generación a preferencias heterogéneas de estudiantes. Inicialmente se consideró grafos para ``el mejor'' horario, pero usuarios mostraron preferencias variadas (días libres vs.\,compactos vs.\,docentes). Se adoptó estrategia multicriterio generando múltiples opciones configurables. La validación iterativa con usuarios evitó soluciones técnicamente correctas pero poco útiles.

\item \textbf{Organización de carga académica mediante selección guiada.} En fases finales surgió problema crítico: materias con componente teórico y laboratorio (p.\,ej., Física I) no se distinguían en el sistema web de la facultad. Se extendió el modelo de datos tratándolos como materias independientes con vinculación lógica, modificando el parser de PDFs. La arquitectura limpia facilitó esta extensión tardía sin comprometer componentes existentes.

\item \textbf{Ajuste manual de horarios y persistencia de preferencias.} El scraping de PDFs generaba errores por símbolos especiales y falta de datos de docentes, rompiendo integridad referencial. Se implementó formateo riguroso con validación y contingencias inteligentes reutilizando docentes de semestres anteriores. Separar teórico/laboratorio requirió migraciones complejas tardías. La limpieza de datos resultó tan crítica como el diseño del esquema.

\item \textbf{Facilitar descarga e intercambio de horarios.} La generación de imagen y PDF renderiza HTML del horario en componente web interno; Excel se genera directamente sin renderizado visual. El desafío fue alto consumo de memoria al crear imágenes grandes (~28 MB), causando errores en dispositivos de gama baja. Se solucionó liberando memoria tras comprimir, calculando dimensiones dinámicamente y ejecutando en hilo principal con control de cancelación.

\item \textbf{Generación de horarios optimizados.} El enfoque inicial con grafos mostró limitaciones: complejidad elevada, única solución generada y falta de escalabilidad a múltiples criterios. Se implementó backtracking heurístico con poda inteligente y evaluación multicriterio, generando top-N soluciones. Heurísticas eficientes con criterios claros superaron el enfoque teóricamente óptimo en escenarios reales.

\item \textbf{Optimización de experiencia móvil.} Las primeras versiones generaban confusión sobre funcionalidades y flujos. El feedback reveló mal entendimiento del usuario. Se implementó onboarding interactivo, rediseño de pantallas basado en testing con usuarios reales, y tooltips contextuales en puntos críticos. La interfaz intuitiva requiere iteración con usuarios reales, no diseño único basado en suposiciones del desarrollador.

\end{enumerate}

\vspace{0.5em}
\noindent El desarrollo de TecnoTime demostró que construir software útil requiere validación continua con usuarios, resiliencia ante datos imperfectos, arquitectura flexible que permita incorporar cambios significativos, e iteración constante basada en feedback real. Las suposiciones iniciales sobre preferencias de horarios y flujos de UI fueron desafiadas por la realidad.

\noindent Los desafíos enfrentados, desde algoritmos inadecuados hasta problemas de scraping y UI confusa, fueron oportunidades de aprendizaje que fortalecieron tanto el producto final como las competencias técnicas adquiridas.

\section{Recomendaciones}
\begin{itemize}
  \item Despliegue institucional: formalizar un canal de distribución (intranet/tienda) y firmar versiones; evaluar un endpoint (punto final de una API) oficial para horarios a fin de reemplazar la ingesta desde PDF.
  \item Mantenimiento de datos: hacer seguimiento de cambios en el formato/ubicación de los PDFs; programar validaciones previas a inscripciones; mantener fallback por PDF local.
  \item Portabilidad: explorar extensión a iOS y otras facultades; considerar un modo “lectura” web para consulta rápida.
  \item Métricas opt-in: incorporar telemetría voluntaria para mejorar ranking de generador, detectar fallas de sincronización y medir adopción sin comprometer privacidad.
  \item Rendimiento y QA: ampliar pruebas de UI para export/import; realizar benchmarks por dispositivo; asegurar TTS \(\leq\) objetivos definidos.
  \item Seguridad y privacidad: revisar periódicamente permisos, almacenamiento y cifrado de datos sensibles; mantener IA en el dispositivo como activación voluntaria, con control de descarga/borrado.
  \item Documentación y soporte: mantener actualizados los manuales de instalación y de usuario (ver Anexos A–B) y una guía de preguntas frecuentes.
\end{itemize}

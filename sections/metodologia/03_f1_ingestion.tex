% ============================
\section{F1: Ingestión y normalización (de PDF a BD)}

Esta fase establece la adquisición y transformación de datos desde los PDF oficiales de la UMSS hacia una base de datos local. Dada la variabilidad del formato fuente, esta etapa es crítica para la calidad de la información.

Se implementó un flujo ETL (Extracción, Transformación y Carga) autónomo en el dispositivo, optimizado para recursos limitados y conectividad intermitente.

\subsection{Objetivo}
Transformar horarios oficiales (PDF) a un modelo estructurado, validando la integridad referencial de materias y docentes. Se minimiza el impacto en los servidores mediante caché y sincronización diferencial.

\subsubsection{4.2.1 Verificación de caché y decisión}
Al abrir el generador se valida conectividad y vigencia de datos. Si el contenido local está fresco, se usa cache; de lo contrario, se inicia sincronización automática.
\begin{itemize}
  \item Control en la interfaz: GenerateScheduleViewModel ejecuta checkConnectivityAndSync() y bloquea acciones durante sincronización.
  \item Conectividad: NetworkConnectivityChecker discrimina entre online/offline.
  \item Frescura: AutoSyncUseCase compara updatedDate remoto con lastSyncTime local (por carrera) y decide actualizar.
  \item Preferencias: SyncPreferences limita chequeos con una ventana (p. ej., 6 horas) para evitar consultas excesivas.
\end{itemize}

Para visualizar el entorno en el que opera esta fase, la Figura \ref{fig:c4_context_f1} presenta el diagrama de contexto C4 (Nivel 1). En este diagrama se observa cómo el Sistema de Horarios interactúa con el estudiante y con los sistemas externos de la universidad (Portal Web), destacando los límites del sistema y sus principales dependencias externas para la obtención de datos.

\begin{figure}[H]
  \centering
  \includegraphics[width=0.35\linewidth]{images/diagrams/cap4_c4_1_context.png}
  \caption[C4 Nivel 1: Contexto del sistema para F1]{C4 Nivel 1: Contexto del sistema para F1.}
  \label{fig:c4_context_f1}
  \small{Fuente: Elaboración propia.}
\end{figure}

\subsubsection{4.2.2 Ingesta, análisis y persistencia con frescura}
Ante una actualización, el flujo descarga el PDF oficial, extrae su contenido y lo analiza sintácticamente para estructurar entidades clave como materias, grupos y horarios. Posteriormente, estos datos se normalizan y persisten en la base de datos local Room, garantizando la integridad y consistencia del modelo de dominio.

Para proteger la estabilidad de la fuente externa, se aplican patrones de resiliencia robustos. El retroceso exponencial gestiona los reintentos ante fallos transitorios de red, mientras que el cortacircuito previene la saturación del servidor interrumpiendo temporalmente las solicitudes tras detectar errores repetidos.
\begin{itemize}
  \item Descubrimiento/descarga: ScheduleScraper ubica recursos y PdfDownloader obtiene el PDF.
  \item Extracción y análisis: PdfExtractor obtiene texto; PdfParser (componente que analiza y transforma datos de entrada) aplica expresiones regulares tolerantes para cabeceras, grupos, días y horas, consolidando líneas y docentes.
  \item Normalización/persistencia: ImportCareerSchedulesUseCase proyecta entidades y persiste con índices únicos compuestos (Room) y políticas ON CONFLICT.
  \item Frescura: tras actualizar, se registra lastSyncTime por carrera y se guarda el instante del chequeo en SyncPreferences.
  \item Robustez: ExponentialBackoff y CircuitBreaker controlan reintentos y abren/cierran el circuito ante fallas repetidas.
  \item Respaldo: si no hay red, se permite importación manual desde PDF local y se opera totalmente en caché.
\end{itemize}

Descendiendo al siguiente nivel de abstracción, la Figura \ref{fig:c4_container_f1} (C4 Nivel 2) muestra los contenedores principales. Se destaca la aplicación móvil como el contenedor central que orquesta la lógica, interactuando con la base de datos local (Room) para la persistencia y con el sistema de archivos para la gestión de los PDFs descargados.

\begin{figure}[H]
  \centering
  \includegraphics[width=1\linewidth]{images/diagrams/cap4_c4_2_container.png}
  \caption[C4 Nivel 2: Contenedores involucrados en F1]{C4 Nivel 2: Contenedores involucrados en F1.}
  \label{fig:c4_container_f1}
  \small{Fuente: Elaboración propia.}
\end{figure}

Adentrándonos en la estructura interna de la aplicación, la Figura \ref{fig:c4_component_f1} (C4 Nivel 3) desglosa los componentes específicos de la fase de ingestión. Se pueden identificar el ScheduleScraper encargado del descubrimiento de recursos, el PdfDownloader para la descarga segura, y los componentes de procesamiento PdfExtractor y PdfParser que transforman el documento en datos estructurados.

\begin{figure}[H]
  \centering
  \includegraphics[width=1\linewidth]{images/diagrams/cap4_c4_3_component_f1.png}
  \caption[C4 Nivel 3: Componentes de ingesta y normalización (F1)]{C4 Nivel 3: Componentes de ingesta y normalización (F1).}
  \label{fig:c4_component_f1}
  \small{Fuente: Elaboración propia.}
\end{figure}

Finalmente, la Figura \ref{fig:c4_code_f1} (C4 Nivel 4) presenta el diagrama de clases detallado, mostrando las relaciones entre las clases de implementación, los casos de uso como AutoSyncUseCase e ImportCareerSchedulesUseCase, y las entidades de datos, proporcionando una vista técnica precisa de cómo se implementa la lógica de sincronización y persistencia.

\begin{figure}[H]
  \centering
  \includegraphics[width=1\linewidth]{images/diagrams/cap4_c4_4_code_f1.png}
  \caption[C4 Nivel 4: Clases y casos de uso relevantes (F1)]{C4 Nivel 4: Clases y casos de uso relevantes (F1).}
  \label{fig:c4_code_f1}
  \small{Fuente: Elaboración propia.}
\end{figure}

\subsection{Resultados y verificación}
Esta fase estableció los cimientos para la adquisición de datos. Se verificó la consistencia de la base de datos por carrera, nivel, materia y grupo, asegurando que el lastSyncTime se actualice correctamente tras una ingesta exitosa. Las pruebas confirmaron la robustez del proceso ante fallos de red mediante los mecanismos de backoff y circuit breaker.

\begin{enumerate}
  \item \textbf{Evidencias (código):}
  \begin{itemize}
    \item Caso de uso de sincronización automática (AutoSyncUseCase) y preferencias de sincronización (SyncPreferences) para ventanas y sellos de tiempo.
    \item Rastreador de horarios (ScheduleScraper) y componentes de adquisición/procesamiento de PDF (PdfDownloader/PdfExtractor/PdfParser).
    \item Importación por carrera (ImportCareerSchedulesUseCase) y componentes de resiliencia (ExponentialBackoff, CircuitBreaker).
  \end{itemize}

  \item \textbf{KPIs verificados:}
  \begin{itemize}
    \item Cache hit-rate semanal $\geq 80\%$.
    \item Reintentos acotados dentro de los parámetros del backoff.
    \item Tiempo de ingesta por PDF dentro del objetivo operacional.
  \end{itemize}
\end{enumerate}

% ----------------------------------------------------------------------
\section{Procedimientos operativos}

Los procedimientos operativos documentan los flujos de trabajo principales de la aplicación desde la perspectiva del sistema, describiendo las secuencias de acciones, validaciones y transformaciones que se ejecutan para completar cada operación crítica.

\subsection{PR-001: Ingestión y normalización}
Actor: Sistema. Objetivo: BD local vigente.
\begin{enumerate}
  \item \textbf{Precondiciones}
  \begin{itemize}
    \item Conectividad o PDF local.
  \end{itemize}
  \item \textbf{Postcondiciones}
  \begin{itemize}
    \item last\_sync actualizado, TC-004/005 verdes.
  \end{itemize}
\end{enumerate}

\subsection{PR-002: Generar horarios}
Actor: Estudiante. Objetivo: obtener $N$ horarios.
\begin{enumerate}
  \item \textbf{Precondiciones}
  \begin{itemize}
    \item Selección de materias/flags.
  \end{itemize}
  \item \textbf{Postcondiciones}
  \begin{itemize}
    \item Top-$N$ sin solapes (por defecto), aplicación parcial/completa.
  \end{itemize}
\end{enumerate}

\subsection{PR-003: Configurar notificaciones}
Actor: Estudiante. Objetivo: avisos antes de clase (10\,min por defecto).

\subsection{PR-004: Exportar/Importar}
Actor: Estudiante. Objetivo: compartir/clonar; validación de versión y carrera.

% ----------------------------------------------------------------------
\section{Estudio de campo y levantamiento de requerimientos}

Esta sección documenta el proceso completo de investigación empírica realizado para fundamentar el desarrollo de TecnoTime. El objetivo principal fue validar las hipótesis iniciales sobre las dificultades que enfrentan los estudiantes en la gestión de sus horarios y obtener datos cuantitativos que respalden las decisiones de diseño.

A través de este estudio, se buscó no solo identificar problemas, sino también descubrir patrones de comportamiento y preferencias tecnológicas que permitieran orientar la construcción de una solución verdaderamente útil y adoptable.

El proceso abarca desde el diseño del instrumento de recolección hasta la derivación de requerimientos funcionales trazables, asegurando que cada funcionalidad del sistema responda a una necesidad real detectada en la población estudiantil.

\subsection{Diseño del instrumento de recolección}

Para fundamentar el desarrollo de TecnoTime en necesidades reales de los estudiantes de la FCyT, se diseñó un cuestionario estructurado orientado a capturar patrones de uso de herramientas actuales, puntos de dolor en la gestión de horarios y expectativas sobre funcionalidades deseadas en una aplicación móvil complementaria.

El instrumento se organizó en cuatro bloques temáticos: caracterización del encuestado (carrera, semestre, dispositivos utilizados), uso de herramientas actuales (Cappuchino, PDFs oficiales, canales de comunicación), problemática percibida (frustración, repetición de pasos, dificultades de coordinación) y demandas específicas (actualización automática, modo sin conexión, recordatorios, compartir horarios).

El cuestionario fue aplicado mediante Google Forms durante el periodo académico 2025-I, con participación voluntaria y anónima. El instrumento completo se presenta en el Anexo \ref{ann:instrumento}, mientras que los resultados tabulados se encuentran documentados en la subsección \ref{sec:historias_usuario} de este capítulo y en formato digital \cite{encuesta_fcyt_2025}.

\subsection{Aplicación y características de la muestra}

Se aplicó un muestreo por conveniencia dirigido a estudiantes de la Facultad de Ciencias y Tecnología de la UMSS, obteniendo un total de 126 respuestas válidas. La muestra incluyó estudiantes de diferentes carreras, niveles académicos y turnos, reflejando la diversidad de la población objetivo.

El perfil de los encuestados abarca desde estudiantes de primeros semestres hasta niveles avanzados, con predominio de las carreras de Ingeniería Informática e Ingeniería de Sistemas. La participación fue voluntaria y la tasa de respuesta completa alcanzó el 95\% de los formularios iniciados, evidenciando el interés de los estudiantes en la problemática abordada.

\subsection{Resultados cuantitativos de la encuesta}

Los resultados de la encuesta aplicada a 126 estudiantes de la FCyT se presentan organizados por dimensiones temáticas, comenzando por el análisis de dispositivos utilizados y continuando con patrones de uso, coordinación social y demandas específicas.

El análisis de estos datos permite identificar las brechas existentes entre las herramientas actuales y las expectativas de los estudiantes. A continuación, se detallan los hallazgos más relevantes que sustentan los requerimientos del sistema.

\begin{enumerate}[label=\textbf{\arabic*.}]
  \item \textbf{Dispositivos utilizados}

  El 94\% de los encuestados utiliza Android como herramienta principal, frente a un 26\% con acceso regular a computadora portátil. Esta marcada preferencia móvil justifica la prioridad del desarrollo nativo para Android, como se ilustra en la Figura \ref{fig:encuesta_dispositivo}.

  \begin{figure}[H]
    \centering
    \includegraphics[width=0.6\textwidth]{images/diagrams/encuesta_dispositivo.png}
    \caption[Dispositivo utilizado para ver horarios (n=126)]{Dispositivo utilizado para ver horarios (n=126).}
    \label{fig:encuesta_dispositivo}
    \small{Fuente: \cite{encuesta_fcyt_2025}.}
  \end{figure}

  \item \textbf{Patrones de consulta de horarios}

  El uso se concentra en periodos específicos: 48\% consulta solo en inscripciones y 44\% varias veces al semestre. Apenas un 8\% accede diariamente, lo que evidencia que las herramientas actuales no fomentan una consulta continua, tal como se muestra en la Figura \ref{fig:encuesta_uso_capuchino}.

  \begin{figure}[H]
    \centering
    \includegraphics[width=0.6\textwidth]{images/diagrams/encuesta_uso_capuchino.png}
    \caption[Frecuencia de acceso a la información de horarios (n=126)]{Frecuencia de acceso a la información de horarios (n=126).}
    \label{fig:encuesta_uso_capuchino}
    \small{Fuente: \cite{encuesta_fcyt_2025}.}
  \end{figure}

  \item \textbf{Coordinación social mediante WhatsApp}

  El 80\% de los encuestados valida información de horarios y coordina cambios mediante grupos de WhatsApp, confirmando que la mensajería instantánea se ha convertido en un canal informal pero crítico para la gestión de información académica. Este hallazgo fundamenta la necesidad de habilitar intercambio estructurado de horarios compatible con plataformas de mensajería, como se detalla en la Figura \ref{fig:encuesta_whatsapp}.

  \begin{figure}[H]
    \centering
    \includegraphics[width=0.5\textwidth]{images/diagrams/encuesta_whatsapp.png}
    \caption[Coordinación social mediante WhatsApp (n=126)]{Coordinación social mediante WhatsApp (n=126).}
    \label{fig:encuesta_whatsapp}
    \small{Fuente: \cite{encuesta_fcyt_2025}.}
  \end{figure}

  \item \textbf{Frustración con herramientas actuales}

  El 60\% de los estudiantes reporta niveles altos de frustración (1-3) con las herramientas actuales. Este dato subraya la urgencia de mejorar la experiencia de usuario mediante memoria de contexto y persistencia, evitando así la repetición de pasos en cada consulta, tal como se refleja en la Figura \ref{fig:encuesta_frustracion}.

  \begin{figure}[H]
    \centering
    \includegraphics[width=0.7\textwidth]{images/diagrams/encuesta_frustracion.png}
    \caption[Nivel de frustración con las herramientas actuales (n=126)]{Nivel de frustración con las herramientas actuales (n=126).}
    \label{fig:encuesta_frustracion}
    \small{Fuente: \cite{encuesta_fcyt_2025}.}
  \end{figure}

  \item \textbf{Principales demandas identificadas}

  Las demandas con mayor valoración por parte de los estudiantes incluyen: actualización automática de horarios (88\%), modo sin conexión (78\%), recordatorios antes de clase (56\%) y funcionalidad de compartir horarios (51\%). Estas prioridades guiaron la definición de las fases correspondientes del desarrollo, resumidas en la Figura \ref{fig:encuesta_demandas} (F4, F5, F7 y F8).

  \begin{figure}[H]
    \centering
    \includegraphics[width=0.7\textwidth]{images/diagrams/encuesta_demandas.png}
    \caption[Principales demandas de los estudiantes (n=126)]{Principales demandas de los estudiantes (n=126).}
    \label{fig:encuesta_demandas}
    \small{Fuente: \cite{encuesta_fcyt_2025}.}
  \end{figure}
\end{enumerate}

\subsection{Derivación de requerimientos funcionales}

Los hallazgos cuantitativos de la encuesta se tradujeron en requerimientos funcionales específicos mediante un proceso de trazabilidad que vincula cada demanda identificada con una funcionalidad concreta de la aplicación y un criterio de aceptación verificable, como se presenta en el Cuadro \ref{tab:derivacion_req}.

\begin{table}[H]
  \centering
  \small
  \begin{tabular}{|p{5.0cm}|p{7.5cm}|p{3.5cm}|}
    \hline
    Hallazgo & Requerimiento derivado & Criterio de aceptación \\
    \hline
    94\% usa Android & Priorizar Android y vista móvil & Flujo usable sin laptop \\
    \hline
    60\% frustración por repetir pasos & Memoria de carrera/nivel/materias & Ingreso sin repetir selección \\
    \hline
    80\% coordina por WhatsApp & Intercambio estructurado en JSON & Compartir/abrir desde mensajería \\
    \hline
    78\% valora modo sin conexión & Operación offline-first & Consulta sin red \\
    \hline
    57\% valora notificaciones & Recordatorios previos a clase & Alertas configurables \\
    \hline
    88\% solicita actualización automática & Actualización desde fuente oficial & Fecha visible de actualización \\
    \hline
  \end{tabular}
  \caption[Trazabilidad de hallazgos a requerimientos funcionales]{Trazabilidad de hallazgos a requerimientos funcionales.}
  \label{tab:derivacion_req}
  \small{Fuente: Elaboración propia basada en \cite{encuesta_fcyt_2025}.}
\end{table}

\subsection{Proceso propuesto (TO-BE)}

Con base en las demandas identificadas y los requerimientos derivados, se diseñó un proceso mejorado que aborda las limitaciones del flujo actual.

El proceso propuesto con TecnoTime simplifica la gestión de horarios mediante: generación automática de combinaciones sin choques, memoria de contexto para evitar repetición de selecciones, intercambio estructurado compatible con mensajería, consulta diaria sin conexión y recordatorios configurables antes de clase. La Figura \ref{fig:area_tobe} ilustra este nuevo flujo.

\begin{figure}[H]
  \centering
  \includegraphics[width=0.35\textwidth]{images/diagrams/area_tobe_proceso.png}
  \caption[Procesos propuestos (TO-BE) a nivel de uso]{Procesos propuestos (TO-BE) a nivel de uso con TecnoTime.}
  \label{fig:area_tobe}
  \small{Fuente: Elaboración propia.}
\end{figure}

El proceso propuesto reduce significativamente los pasos necesarios para consultar y actualizar horarios, elimina la dispersión de información entre múltiples canales y habilita funcionalidades de coordinación social mediante formatos estructurados.

\subsection{Historias de usuario derivadas}\label{sec:historias_usuario}

A partir de los requerimientos funcionales y el proceso TO-BE, se definieron 63 historias de usuario. Para su documentación se adoptó un formato estándar que facilita la comprensión y validación:

\begin{itemize}
    \item Estructura de la Historia: Sigue la plantilla ``Como [rol], necesito [funcionalidad], para [beneficio]'', identificando al actor, su objetivo y el valor obtenido.
    \item Criterios de Aceptación (G/W/T): Se utiliza el formato \textit{Given-When-Then} (Dado que - Cuando - Entonces). Describe el escenario inicial (\textit{Given}), la acción del usuario (\textit{When}) y el resultado esperado (\textit{Then}), sirviendo como base para las pruebas.
    \item Priorización (MoSCoW): Clasificación de importancia: \textit{Must} (Debe tener), \textit{Should} (Debería tener), \textit{Could} (Podría tener) y \textit{Won't} (No tendrá por ahora).
\end{itemize}

Las historias se organizaron en las siguientes áreas funcionales:

\begin{itemize}[nosep]
  \item Onboarding y configuración inicial de perfil
  \item Gestión de carreras, niveles y seguimiento de progreso académico
  \item Navegación en la pantalla principal, vista semanal y widget
  \item Administración de materias y grupos
  \item Gestión de profesores y favoritos
  \item Generador de horarios con restricciones y optimización
  \item Exportación, importación e intercambio de horarios
  \item Recordatorios automatizados y asistente IA ``Simón''
  \item Operación sin conexión, actualización automática y resiliencia
  \item Soporte, información y preferencias del usuario
  \item Seguridad, privacidad, rendimiento y accesibilidad
\end{itemize}

A continuación, se presentan las tablas detalladas con el conjunto completo de historias de usuario (US-01 a US-63).

\noindent Las historias de usuario se han organizado agrupándolas por área funcional para facilitar su lectura y comprensión. Esta estructura modular permite visualizar claramente el alcance de cada componente del sistema a través de las tablas presentadas en las siguientes secciones.

En cada una de las tablas presentadas, se desglosan los elementos fundamentales que componen la historia de usuario. Se identifica explícitamente el Rol del actor que interactúa con el sistema, la Necesidad puntual o funcionalidad que requiere, y el Beneficio directo que esta acción le aporta, asegurando así que cada requerimiento tenga un valor claro para el usuario final.

Para garantizar la verificabilidad de cada historia, se incluyen los Criterios de Aceptación abreviados como G/W/T. Esta notación corresponde al formato Given-When-Then (Dado que / Cuando / Entonces), el cual describe el contexto inicial, la acción a realizar y el resultado esperado, sirviendo como base directa para las pruebas de aceptación. Asimismo, se asigna una Prioridad utilizando el método MoSCoW, clasificando cada historia según su criticidad para el lanzamiento.

Finalmente, cabe destacar que, con el objetivo de optimizar el espacio en el documento y presentar la información de manera tabular y concisa, se han utilizado descripciones breves en las tablas. El detalle narrativo completo y las especificaciones técnicas profundas se mantienen en la documentación extendida del proyecto.

\medskip

\subsection{Onboarding \& Perfil}

Este grupo de historias aborda la primera experiencia del usuario al abrir la aplicación, incluyendo la configuración inicial de preferencias personales, el registro del nombre para mensajes locales y la personalización de la interfaz, tal como se detalla en el Cuadro \ref{tab:hu_onboarding}.

\begin{table}[H]
\small
\centering
\begin{tabularx}{\linewidth}{@{}P{1.15cm}P{2.3cm}XP{2.3cm}XP{1.2cm}@{}}
\toprule
ID & Rol & Necesidad & Beneficio & Criterios G/W/T & Prioridad \\
\midrule
US-01 & Estudiante & Ver bienvenida y permisos & Entender la app antes de usar & G: primera apertura; W: ingreso; T: intro + permisos.\\[-0.4em]
 & & & & G: rechazo; W: continuo; T: explica funciones limitadas. & Must \\
US-02 & Estudiante & Ingresar nombre de usuario & Personalizar mensajes/notif. & G: primera ejecución; W: guardo; T: se usa en UI/IA local. & Must \\
US-03 & Estudiante & Configurar 12/24h, fines, días sin clase, tema & Visualizar horario personalizado & G: ajusto switch; W: cambio; T: ejemplo reacciona.\\[-0.4em]
 & & & & G: cambio tema; W: selecciono; T: UI cambia y persiste. & Must \\
US-04 & Estudiante & Onboarding único & Evitar repeticiones & G: terminé onboarding; W: reabro; T: ingreso directo al Home. & Must \\
\bottomrule
\end{tabularx}
\caption{Historias de usuario: Onboarding y Perfil}\label{tab:hu_onboarding}
\small{Fuente: Elaboración propia basada en \cite{encuesta_fcyt_2025}}
\end{table}

\medskip

\subsection{Carreras, Niveles y Progreso}

Estas historias definen la gestión de la información académica base, permitiendo al estudiante sincronizar su carrera, visualizar su avance por niveles, marcar materias aprobadas y gestionar la fuente de datos oficial, como se muestra en el Cuadro \ref{tab:hu_carreras_1}.

\begin{table}[H]
\small
\centering
\begin{tabularx}{\linewidth}{@{}P{1.15cm}P{2.3cm}XP{2.3cm}XP{1.2cm}@{}}
\toprule
ID & Rol & Necesidad & Beneficio & Criterios G/W/T & Prioridad \\
\midrule
US-05 & Estudiante & Sincronizar carreras & Cargar horarios y materias & G: lista carreras; W: selecciono; T: queda sincronizada. & Must \\
US-06 & Estudiante & Ver niveles con \% & Entender avance & G: entro carrera; W: abro niveles; T: cards con anillo y \%. & Should \\
US-07 & Estudiante & Marcar materia aprobada/no & Reflejar estado real & G: no aprobada; W: toco; T: pasa a ``aprobada''.\\[-0.4em]
 & & & & G: aprobada; W: revierto; T: diálogo + ``no aprobada''. & Must \\
\bottomrule
\end{tabularx}
\caption{Historias de usuario: Carreras, niveles y progreso (Parte 1)}\label{tab:hu_carreras_1}
\small{Fuente: Elaboración propia basada en \cite{encuesta_fcyt_2025}}
\end{table}

\begin{table}[H]
Complementariamente, se aborda la gestión avanzada de la fuente de datos, reconociendo que la información oficial puede variar o no estar siempre disponible. Se incluyen mecanismos para validar la integridad de los datos comparándolos con el documento PDF original proporcionado por la facultad.

\par\vspace{0.5em}

Además, se dota al sistema de flexibilidad ante contingencias. La capacidad de cargar manualmente archivos de horarios y la opción de desvincular una carrera errónea aseguran que el estudiante pueda mantener su planificación actualizada incluso en situaciones de falta de conectividad. Estas funcionalidades avanzadas se detallan en el Cuadro \ref{tab:hu_carreras_2}.

\medskip

\small
\centering
\begin{tabularx}{\linewidth}{@{}P{1.15cm}P{2.3cm}XP{2.3cm}XP{1.2cm}@{}}
\toprule
ID & Rol & Necesidad & Beneficio & Criterios G/W/T & Prioridad \\
\midrule
US-08 & Estudiante & Abrir PDF origen & Validar fuente oficial & G: card carrera; W: toco PDF; T: abre/descarga documento. & Should \\
US-09 & Estudiante & Desincronizar carrera & Limpiar malla errónea & G: carrera sync; W: desincronizo; T: diálogo crítico + limpieza. & Must \\
US-10 & Estudiante & Subir PDF manual & Actualizar sin web & G: en carrera; W: ``Sincronizar PDF''; T: parsea y actualiza. & Should \\
\bottomrule
\end{tabularx}
\caption{Historias de usuario: Carreras, niveles y progreso (Parte 2)}\label{tab:hu_carreras_2}
\small{Fuente: Elaboración propia basada en \cite{encuesta_fcyt_2025}}
\end{table}

\medskip

\subsection{Home, Navegación y Widget}

Se describen las funcionalidades relacionadas con la navegación principal y el acceso rápido a la información, incluyendo la vista diaria de clases, el desplazamiento entre fechas y el uso de widgets en la pantalla de inicio, detalladas en el Cuadro \ref{tab:hu_home}.

\begin{table}[H]
\small
\centering
\begin{tabularx}{\linewidth}{@{}P{1.15cm}P{2.3cm}XP{2.3cm}XP{1.2cm}@{}}
\toprule
ID & Rol & Necesidad & Beneficio & Criterios G/W/T & Prioridad \\
\midrule
US-11 & Estudiante & Ver cards del día & Planificar jornada & G: Home; W: abro; T: cards con hora/materia/grupo/aula. & Must \\
US-12 & Estudiante & Navegar por swipe & Revisar semana rápido & G: Home; W: deslizo; T: cambia día. & Should \\
US-13 & Estudiante & Selector de fecha & Saltar a día específico & G: toco fecha; W: elijo; T: Home muestra ese día. & Should \\
US-14 & Estudiante & Vista semanal compacta & Visualizar distribución completa & G: Home; W: botón semanal; T: resume semana activa. & Could \\
US-15 & Estudiante & Widget con días & Consultar sin abrir app & G: widget; W: deslizo; T: materias visibles cambian. & Should \\
\bottomrule
\end{tabularx}
\caption{Historias de usuario: Home, navegación y widget}\label{tab:hu_home}
\small{Fuente: Elaboración propia basada en \cite{encuesta_fcyt_2025}}
\end{table}

\medskip

\subsection{Materias \& Grupos}

\begin{table}[H]
El núcleo de la planificación académica reside en la selección de asignaturas. Se detalla el flujo de ``inscripción virtual'', un proceso guiado que permite al estudiante explorar la oferta académica filtrada por su nivel y seleccionar las materias que desea cursar.

\par\vspace{0.5em}

Este proceso se enriquece con información detallada para la toma de decisiones. Antes de confirmar una inscripción, el sistema presenta una vista previa completa de los grupos disponibles, incluyendo docentes y horarios. El Cuadro \ref{tab:hu_materias_1} detalla este flujo central de inscripción.

\medskip

\small
\centering
\begin{tabularx}{\linewidth}{@{}P{1.15cm}P{2.3cm}XP{2.3cm}XP{1.2cm}@{}}
\toprule
ID & Rol & Necesidad & Beneficio & Criterios G/W/T & Prioridad \\
\midrule
US-16 & Estudiante & Agregar materia guiada & Construir horario & G: pulso FAB; W: sigo flujo; T: materia creada (nivel→emoji). & Must \\
US-17 & Estudiante & Seleccionar nivel & Elegir contenedor correcto & G: pantalla nivel; W: elijo; T: vuelve con nivel marcado. & Must \\
US-18 & Estudiante & Seleccionar materia & Escoger código correcto & G: lista materias; W: toco; T: queda asignada. & Must \\
US-19 & Estudiante & Ver preview de grupo & Decidir con info completa & G: lista grupos; W: elijo; T: muestra días/horas/docente antes de confirmar. & Must \\
\bottomrule
\end{tabularx}
\caption{Historias de usuario: Materias y grupos (Parte 1)}\label{tab:hu_materias_1}
\small{Fuente: Elaboración propia basada en \cite{encuesta_fcyt_2025}}
\end{table}

\begin{table}[H]
Más allá de la simple selección, se centra en la personalización y gestión continua de las materias. Se introducen capacidades para asignar colores y emojis a cada asignatura, facilitando su identificación visual rápida en el calendario y reduciendo la carga cognitiva del usuario.

\par\vspace{0.5em}

Finalmente, se contempla el ciclo de vida completo de la cursada. Las opciones para editar detalles de una materia ya inscrita o finalizarla (marcarla como aprobada o abandonada) permiten que el horario evolucione junto con el semestre del estudiante. Estas opciones se abordan en el Cuadro \ref{tab:hu_materias_2}.

\medskip

\small
\centering
\begin{tabularx}{\linewidth}{@{}P{1.15cm}P{2.3cm}XP{2.3cm}XP{1.2cm}@{}}
\toprule
ID & Rol & Necesidad & Beneficio & Criterios G/W/T & Prioridad \\
\midrule
US-20 & Estudiante & Personalizar color/emoji & Identificar visualmente & G: pantalla color/emoji; W: guardo; T: cards reflejan estilo. & Could \\
US-21 & Estudiante & Editar desde card & Ajustar sin recrear & G: toco card; W: ``Editar''; T: cambio grupo/color/emoji. & Should \\
US-22 & Estudiante & Finalizar materia & Impactar generaciones futuras & G: opción finalizar; W: elijo estado; T: se registra (aprobado/abandonar). & Should \\
\bottomrule
\end{tabularx}
\caption{Historias de usuario: Materias y grupos (Parte 2)}\label{tab:hu_materias_2}
\small{Fuente: Elaboración propia basada en \cite{encuesta_fcyt_2025}}
\end{table}

\medskip

\subsection{Profesores \& Favoritos}

\begin{table}[H]
La elección de docentes es uno de los factores cualitativos más influyentes en la satisfacción del estudiante. Las historias formalizan esta preferencia, permitiendo al usuario marcar profesores específicos como ``favoritos'' dentro de la aplicación.

\par\vspace{0.5em}

Esta acción no es meramente informativa, sino que tiene un impacto funcional directo. La lista de favoritos actúa como un criterio heurístico de alto peso en los algoritmos de generación automática de horarios. Las historias enfocadas en esta preferencia docente se detallan en el Cuadro \ref{tab:hu_profes}.

\medskip

\small
\centering
\begin{tabularx}{\linewidth}{@{}P{1.15cm}P{2.3cm}XP{2.3cm}XP{1.2cm}@{}}
\toprule
ID & Rol & Necesidad & Beneficio & Criterios G/W/T & Prioridad \\
\midrule
US-23 & Estudiante & Marcar docentes favoritos & Priorizarlos al generar & G: lista docentes; W: toco estrella; T: queda favorito. & Should \\
\bottomrule
\end{tabularx}
\caption{Historias de usuario: Profesores y favoritos}\label{tab:hu_profes}
\small{Fuente: Elaboración propia basada en \cite{encuesta_fcyt_2025}}
\end{table}

\medskip

\subsection{Generador de Horarios}

\begin{table}[H]
El generador automático de horarios es una herramienta de potencia combinatoria diseñada para simplificar problemas complejos de planificación. Se definen los parámetros de entrada que el usuario puede configurar para guiar este proceso, como el número deseado de materias y la cantidad de propuestas a generar.

\par\vspace{0.5em}

Además, se establecen las preferencias de optimización iniciales. El estudiante puede indicar si desea priorizar la inclusión de sus docentes favoritos o si prefiere minimizar los tiempos muertos (recesos) entre clases. En el Cuadro \ref{tab:hu_generador_1} se establecen estos parámetros iniciales.

\medskip

\small
\centering
\begin{tabularx}{\linewidth}{@{}P{1.15cm}P{2.3cm}XP{2.3cm}XP{1.2cm}@{}}
\toprule
ID & Rol & Necesidad & Beneficio & Criterios G/W/T & Prioridad \\
\midrule
US-24 & Estudiante & Abrir generador & Configurar parámetros & G: FAB; W: ``Generar''; T: veo opciones. & Must \\
US-25 & Estudiante & Definir \# materias/propuestas & Controlar resultado & G: tarjeta parámetros; W: ajusto; T: respeta [1..20], [1..25]. & Must \\
US-26 & Estudiante & Priorizar profesores & Favorecer favoritos & G: opción ON; W: genero; T: intenta incluir favoritos. & Should \\
US-27 & Estudiante & Minimizar recesos & Reducir tiempos muertos & G: opción ON; W: genero; T: se optimizan huecos. & Should \\
\bottomrule
\end{tabularx}
\caption{Historias de usuario: Generador de horarios (Parte 1)}\label{tab:hu_generador_1}
\small{Fuente: Elaboración propia basada en \cite{encuesta_fcyt_2025}}
\end{table}

\begin{table}[H]
Se profundiza en las reglas de negocio ``duras'' que gobiernan la validez de los horarios generados. Se detalla el manejo de conflictos de horario (choques), permitiendo al usuario decidir si acepta solapes bajo ciertas condiciones o si requiere una planificación libre de conflictos.

\par\vspace{0.5em}

Asimismo, se aborda la lógica de priorización académica. El sistema debe respetar las materias marcadas como obligatorias, asegurando su inclusión en todas las propuestas, y aplicar criterios pedagógicos como favorecer asignaturas de niveles inferiores. El Cuadro \ref{tab:hu_generador_2} detalla estas reglas de negocio.

\medskip

\small
\centering
\begin{tabularx}{\linewidth}{@{}P{1.15cm}P{2.3cm}XP{2.3cm}XP{1.2cm}@{}}
\toprule
ID & Rol & Necesidad & Beneficio & Criterios G/W/T & Prioridad \\
\midrule
US-28 & Estudiante & Aceptar choques & Permitir casos extremos & G: opción ON; W: genero; T: propone horarios con solapes controlados. & Could \\
US-29 & Estudiante & Coherencia favoritos/choques & Evitar inconsistencias & G: fav ON, choques OFF; W: genero; T: sin choques (ON/ON sí permite). & Must \\
US-30 & Estudiante & Marcar materias obligatorias & Forzar inclusión & G: marco oblig.; W: genero; T: respeta máximos/convierte excedentes. & Must \\
US-31 & Estudiante & Favorecer niveles bajos & Progresión académica & G: config por defecto; W: genero; T: evita combinaciones incoherentes. & Should \\
\bottomrule
\end{tabularx}
\caption{Historias de usuario: Generador de horarios (Parte 2)}\label{tab:hu_generador_2}
\small{Fuente: Elaboración propia basada en \cite{encuesta_fcyt_2025}}
\end{table}

\begin{table}[H]
Una vez generadas las propuestas, la experiencia de selección es clave. Se describe la interfaz de exploración, que permite al estudiante navegar ágilmente entre múltiples opciones mediante gestos de deslizamiento y comparar visualmente la distribución semanal de cada alternativa.

\par\vspace{0.5em}

Para refinar la búsqueda, se incluyen herramientas de iteración. La capacidad de ``fijar'' ciertas materias o grupos que agradan al usuario permite regenerar el resto del horario respetando esas decisiones parciales. El Cuadro \ref{tab:hu_generador_3} describe estas herramientas de interfaz.

\medskip

\small
\centering
\begin{tabularx}{\linewidth}{@{}P{1.15cm}P{2.3cm}XP{2.3cm}XP{1.2cm}@{}}
\toprule
ID & Rol & Necesidad & Beneficio & Criterios G/W/T & Prioridad \\
\midrule
US-32 & Estudiante & Deslizar entre propuestas & Comparar rápido & G: propuestas listas; W: hago swipe; T: cambia 1..N. & Must \\
US-33 & Estudiante & Vista semanal propuesta & Validar a alto nivel & G: propuesta; W: botón semanal; T: muestra semana. & Could \\
US-34 & Estudiante & Fijar materias con switches & Regenerar respetando & G: propuesta; W: activo switch; T: nuevas propuestas las respetan. & Should \\
US-35 & Estudiante & Cargar completo/parcial & Construir horario final & G: propuesta; W: ``Completo/Parcial''; T: aplica selección en Home. & Must \\
\bottomrule
\end{tabularx}
\caption{Historias de usuario: Generador de horarios (Parte 3)}\label{tab:hu_generador_3}
\small{Fuente: Elaboración propia basada en \cite{encuesta_fcyt_2025}}
\end{table}

\medskip

\subsection{Exportar / Importar / Compartir}

\begin{table}[H]
La planificación académica es frecuentemente una actividad social y colaborativa. Se aborda la necesidad de compartir el horario generado, definiendo formatos visuales de alta fidelidad como imágenes y documentos PDF, ideales para su difusión en redes sociales o impresión física.

\par\vspace{0.5em}

Para usuarios avanzados o necesidades de interoperabilidad, se incluyen formatos estructurados. La exportación a hojas de cálculo (Excel) o archivos de datos (JSON) permite manipular la información en herramientas externas. El Cuadro \ref{tab:hu_export_1} se centra en estas capacidades de exportación.

\medskip

\small
\centering
\begin{tabularx}{\linewidth}{@{}P{1.15cm}P{2.3cm}XP{2.3cm}XP{1.2cm}@{}}
\toprule
ID & Rol & Necesidad & Beneficio & Criterios G/W/T & Prioridad \\
\midrule
US-36 & Estudiante & Exportar como imagen & Compartir fácilmente & G: hay materias; W: ``Enviar $\rightarrow$ Imagen''; T: confirmación + share sheet. & Should \\
US-37 & Estudiante & Exportar como PDF & Imprimir/archivar & Igual que imagen con ``Enviar $\rightarrow$ PDF''. & Should \\
US-38 & Estudiante & Exportar como Excel & Manipular en tabla & Igual que imagen con ``Enviar $\rightarrow$ Excel''. & Could \\
US-39 & Estudiante & Enviar copia JSON & Compartir selectivamente & G: ``Enviar copia''; W: uso Todo/Limpiar/Compartir; T: JSON según selección. & Must \\
\bottomrule
\end{tabularx}
\caption{Historias de usuario: Exportar, importar y compartir (Parte 1)}\label{tab:hu_export_1}
\small{Fuente: Elaboración propia basada en \cite{encuesta_fcyt_2025}}
\end{table}

\begin{table}[H]
Complementando la exportación, se enfoca en la portabilidad y recuperación de datos. Se especifican los mecanismos para importar horarios desde archivos externos, facilitando la transferencia de la planificación entre dispositivos o la restauración de copias de seguridad.

\par\vspace{0.5em}

Este proceso incluye validaciones de integridad críticas. El sistema debe ser capaz de detectar duplicados, verificar la estructura de los archivos importados y ofrecer opciones inteligentes para fusionar la información nueva con la existente. El Cuadro \ref{tab:hu_export_2} especifica estos mecanismos de importación.

\medskip

\small
\centering
\begin{tabularx}{\linewidth}{@{}P{1.15cm}P{2.3cm}XP{2.3cm}XP{1.2cm}@{}}
\toprule
ID & Rol & Necesidad & Beneficio & Criterios G/W/T & Prioridad \\
\midrule
US-40 & Estudiante & Bloquear exportar sin materias & Evitar archivos vacíos & G: 0 materias; W: exporto; T: muestra aviso y detiene acción. & Must \\
US-41 & Estudiante & Importar JSON (app) & Añadir materias propias & G: ``Añadir horario''; W: elijo JSON; T: lista materias y evita duplicados. & Must \\
US-42 & Estudiante & Abrir JSON externo & Importar directo & G: recibo JSON; W: ``Abrir con TecnoTime''; T: muestra UI importación. & Must \\
US-43 & Estudiante & Importar otra carrera & Usarla si la sincronizo & G: JSON externa; W: importo; T: pregunta “¿Sincronizar?” y continúa si acepto. & Should \\
\bottomrule
\end{tabularx}
\caption{Historias de usuario: Exportar, importar y compartir (Parte 2)}\label{tab:hu_export_2}
\small{Fuente: Elaboración propia basada en \cite{encuesta_fcyt_2025}}
\end{table}

\medskip

\subsection{Recordatorios \& IA ``Simón''}

Se definen las funcionalidades de asistencia proactiva, abarcando la configuración de recordatorios automatizados previos a las clases y la interacción opcional con el asistente de inteligencia artificial en el dispositivo, las cuales se presentan en el Cuadro \ref{tab:hu_recordatorios}.

\begin{table}[H]
\small
\centering
\begin{tabularx}{\linewidth}{@{}P{1.15cm}P{2.3cm}XP{2.3cm}XP{1.2cm}@{}}
\toprule
ID & Rol & Necesidad & Beneficio & Criterios G/W/T & Prioridad \\
\midrule
US-44 & Estudiante & Activar/desactivar notificaciones & Ajustar flujo de avisos & G: ajustes recordatorio; W: apago switch; T: oculta anticipación. & Must \\
US-45 & Estudiante & Definir tiempo de anticipación & Recibir avisos oportunos & G: notificaciones ON; W: elijo tiempo; T: se guarda y aplica. & Must \\
US-46 & Estudiante & Activar IA (descarga opcional) & Controlar datos/espacio & G: activo IA; W: elijo Wi-Fi/datos; T: confirma tamaño e inicia descarga. & Should \\
US-47 & Estudiante & Probar/eliminar IA & Gestionar almacenamiento & G: modelo listo; W: ``Probar''; T: respuesta. G: ``Eliminar''; W: confirmo; T: borra modelo. & Should \\
US-48 & Estudiante & Mensajes motivacionales & Controlar tono & G: IA disponible; W: cambio switch; T: respeta preferencia. & Could \\
\bottomrule
\end{tabularx}
\caption{Historias de usuario: Recordatorios e IA}\label{tab:hu_recordatorios}
\small{Fuente: Elaboración propia basada en \cite{encuesta_fcyt_2025}}
\end{table}

\medskip

\subsection{Offline, Actualizaciones \& Resiliencia}

Historias técnicas orientadas a garantizar la disponibilidad del sistema sin conexión a internet, la consistencia de los datos y la robustez ante fallos en la comunicación, descritas en el Cuadro \ref{tab:hu_offline}.

\begin{table}[H]
\small
\centering
\begin{tabularx}{\linewidth}{@{}P{1.15cm}P{2.3cm}XP{2.3cm}XP{1.2cm}@{}}
\toprule
ID & Rol & Necesidad & Beneficio & Criterios G/W/T & Prioridad \\
\midrule
US-49 & Estudiante & Usar sin conexión (lectura) & No depender de la red & G: sin conexión; W: abro; T: horario y funciones locales disponibles. & Must \\
US-50 & Estudiante & Validar datos antes de generar & Evitar info obsoleta & G: hay internet; W: genero; T: se verifica/actualiza antes. & Should \\
US-51 & Sistema & Aplicar backoff progresivo & No saturar servidor & G: 2 fallos; W: reintento; T: aumenta espera y limita intentos. & Must \\
US-52 & Sistema & Bloqueo temporal tras fallos & Evitar tormenta de peticiones & G: timeouts repetidos; W: nueva llamada; T: bloqueo temporal + mensaje. & Must \\
US-53 & Estudiante & Auto-actualización periódica & Mantener datos al día & G: ciclo programado; W: hay internet; T: sincroniza carrera/malla/modelo. & Should \\
\bottomrule
\end{tabularx}
\caption{Historias de usuario: Offline, actualizaciones y resiliencia}\label{tab:hu_offline}
\small{Fuente: Elaboración propia basada en \cite{encuesta_fcyt_2025}}
\end{table}

\medskip

\subsection{Soporte, Información y Preferencias}

\begin{table}[H]
El soporte y la transparencia son vitales para la confianza del usuario. Se incluyen funcionalidades para acceder a información sobre el proyecto, notas de la versión y canales de contacto, fomentando una comunidad de usuarios informada y participativa.

\par\vspace{0.5em}

Asimismo, se garantiza la consistencia de la experiencia a través de la persistencia de preferencias. Configuraciones globales como el tema visual (claro/oscuro) o el formato de hora se guardan localmente. En el Cuadro \ref{tab:hu_soporte} se detallan estas funcionalidades transversales.

\medskip

\small
\centering
\begin{tabularx}{\linewidth}{@{}P{1.15cm}P{2.3cm}XP{2.3cm}XP{1.2cm}@{}}
\toprule
ID & Rol & Necesidad & Beneficio & Criterios G/W/T & Prioridad \\
\midrule
US-54 & Estudiante & Ver versión/novedades/soporte & Informarme y colaborar & G: menú inferior; W: toco opción; T: abre Store/correo/changelog. & Could \\
US-55 & Estudiante & Tema claro/oscuro persistente & Consistencia visual & G: elijo tema; W: reinicio; T: se mantiene. & Must \\
US-56 & Estudiante & Formato 12/24h persistente & Evitar confusiones & G: cambio formato; W: guardo; T: se refleja globalmente. & Must \\
\bottomrule
\end{tabularx}
\caption{Historias de usuario: Soporte, información y preferencias}\label{tab:hu_soporte}
\small{Fuente: Elaboración propia basada en \cite{encuesta_fcyt_2025}}
\end{table}

\medskip

\subsection{Seguridad, Privacidad, Rendimiento y Accesibilidad}

\begin{table}[H]
La seguridad y el rendimiento son pilares técnicos del proyecto. Se establecen requisitos estrictos de privacidad, asegurando que los datos personales identificables (PII) nunca abandonen el dispositivo del usuario, cumpliendo con principios de privacidad por diseño.

\par\vspace{0.5em}

En paralelo, se abordan las restricciones de hardware comunes en el segmento estudiantil. Se definen métricas de rendimiento para garantizar una interfaz fluida en dispositivos de gama media y se optimiza el tamaño de la aplicación. El Cuadro \ref{tab:hu_seguridad_1} establece estos requisitos técnicos críticos.

\medskip

\small
\centering
\begin{tabularx}{\linewidth}{@{}P{1.15cm}P{2.3cm}XP{2.3cm}XP{1.2cm}@{}}
\toprule
ID & Rol & Necesidad & Beneficio & Criterios G/W/T & Prioridad \\
\midrule
US-57 & Estudiante & Privacidad del nombre & No exponer PII & G: ingreso nombre; W: uso app; T: no se envía a servidores. & Must \\
US-58 & Estudiante & App < 30 MB (sin IA) & Instalar con poco espacio & G: build; W: compilo; T: APK/Bundle < 30 MB. & Should \\
US-59 & Estudiante & Buen rendimiento en gama media & Uso cómodo & G: listas/scroll; W: navego; T: FPS aceptable sin ANR. & Should \\
US-60 & Estudiante & Accesibilidad básica & Comprender opciones & G: tema claro/oscuro; W: interactúo; T: contraste/tamaño legible. & Could \\
\bottomrule
\end{tabularx}
\caption{Historias de usuario: Seguridad, privacidad, rendimiento y accesibilidad (Parte 1)}\label{tab:hu_seguridad_1}
\small{Fuente: Elaboración propia basada en \cite{encuesta_fcyt_2025}}
\end{table}

\begin{table}[H]
Finalmente, se centra en la usabilidad bajo condiciones de error. Se especifica la necesidad de mensajes de fallo claros y accionables, que guíen al usuario hacia una solución en lugar de simplemente reportar un problema técnico.

\par\vspace{0.5em}

También se considera la adecuación cultural del software. La localización correcta de formatos de fecha, número y moneda, junto con el uso de un lenguaje apropiado para la región (ES-BO), asegura que la aplicación se sienta natural. El Cuadro \ref{tab:hu_seguridad_2} aborda esta experiencia de usuario.

\medskip

\small
\centering
\begin{tabularx}{\linewidth}{@{}P{1.15cm}P{2.3cm}XP{2.3cm}XP{1.2cm}@{}}
\toprule
ID & Rol & Necesidad & Beneficio & Criterios G/W/T & Prioridad \\
\midrule
US-61 & Estudiante & Mensajes de error claros & Saber qué hacer & G: falla servidor; W: reintento; T: causa + alternativas offline/PDF. & Must \\
US-62 & Estudiante & Manejo de vacíos/estados límite & Evitar confusiones & G: sin materias; W: voy al Home; T: mensaje ``No hay materias'' + CTA agregar/generar. & Must \\
US-63 & Estudiante & Localización ES/formatos regionales & Coherencia cultural & G: región ES-BO; W: render; T: fechas/números correctos. & Should \\
\bottomrule
\end{tabularx}
\caption{Historias de usuario: Seguridad, privacidad, rendimiento y accesibilidad (Parte 2)}\label{tab:hu_seguridad_2}
\small{Fuente: Elaboración propia basada en \cite{encuesta_fcyt_2025}}
\end{table}


% ----------------------------------------------------------------------

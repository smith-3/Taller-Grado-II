\noindent Las historias de usuario se agrupan por área funcional y se presentan en tablas compactas. Cada fila incluye rol, necesidad, beneficio, criterios clave (G/W/T) y prioridad MoSCoW. Para ahorrar espacio se usan descripciones breves; los detalles narrativos permanecen en los capítulos correspondientes. La columna G/W/T resume el criterio Given–When–Then (Dado que / Cuando / Entonces) usado para describir escenarios de prueba.

\medskip

\subsection{Onboarding \& Perfil}

Este grupo de historias aborda la primera experiencia del usuario al abrir la aplicación, incluyendo la configuración inicial de preferencias personales, el registro del nombre para mensajes locales y la personalización de la interfaz.

\begin{table}[H]
\small
\centering
\begin{tabularx}{\linewidth}{@{}P{1.15cm}P{2.3cm}XP{2.3cm}XP{1.2cm}@{}}
\toprule
ID & Rol & Necesidad & Beneficio & Criterios G/W/T & Prioridad \\
\midrule
US-01 & Estudiante & Ver bienvenida y permisos & Entender la app antes de usar & G: primera apertura; W: ingreso; T: intro + permisos.\\[-0.4em]
 & & & & G: rechazo; W: continuo; T: explica funciones limitadas. & Must \\
US-02 & Estudiante & Ingresar mi nombre & Personalizar mensajes/notif. & G: primera ejecución; W: guardo; T: se usa en UI/IA local. & Must \\
US-03 & Estudiante & Configurar 12/24h, fines, días sin clase, tema & Ver horario a mi gusto & G: ajusto switch; W: cambio; T: ejemplo reacciona.\\[-0.4em]
 & & & & G: cambio tema; W: selecciono; T: UI cambia y persiste. & Must \\
US-04 & Estudiante & Onboarding único & Evitar repeticiones & G: terminé onboarding; W: reabro; T: ingreso directo al Home. & Must \\
\bottomrule
\end{tabularx}
\caption{Historias de usuario — Onboarding y Perfil}\label{tab:hu_onboarding}
\small{Fuente: Elaboración propia basada en \cite{encuesta_fcyt_2025}}
\end{table}

\medskip

\subsection{Carreras, Niveles y Progreso}

Estas historias definen la gestión de la información académica base, permitiendo al estudiante sincronizar su carrera, visualizar su avance por niveles, marcar materias aprobadas y gestionar la fuente de datos oficial.

\begin{table}[H]
\small
\centering
\begin{tabularx}{\linewidth}{@{}P{1.15cm}P{2.3cm}XP{2.3cm}XP{1.2cm}@{}}
\toprule
ID & Rol & Necesidad & Beneficio & Criterios G/W/T & Prioridad \\
\midrule
US-05 & Estudiante & Sincronizar carreras & Cargar horarios y materias & G: lista carreras; W: selecciono; T: queda sincronizada. & Must \\
US-06 & Estudiante & Ver niveles con \% & Entender avance & G: entro carrera; W: abro niveles; T: cards con anillo y \%. & Should \\
US-07 & Estudiante & Marcar materia aprobada/no & Reflejar estado real & G: no aprobada; W: toco; T: pasa a ``aprobada''.\\[-0.4em]
 & & & & G: aprobada; W: revierto; T: diálogo + ``no aprobada''. & Must \\
US-08 & Estudiante & Abrir PDF origen & Validar fuente oficial & G: card carrera; W: toco PDF; T: abre/descarga documento. & Should \\
US-09 & Estudiante & Desincronizar carrera & Limpiar malla errónea & G: carrera sync; W: desincronizo; T: diálogo crítico + limpieza. & Must \\
US-10 & Estudiante & Subir PDF manual & Actualizar sin web & G: en carrera; W: ``Sincronizar PDF''; T: parsea y actualiza. & Should \\
\bottomrule
\end{tabularx}
\caption{Historias de usuario — Carreras, niveles y progreso}\label{tab:hu_carreras}
\small{Fuente: Elaboración propia basada en \cite{encuesta_fcyt_2025}}
\end{table}

\medskip

\subsection{Home, Navegación y Widget}

Se describen las funcionalidades relacionadas con la navegación principal y el acceso rápido a la información, incluyendo la vista diaria de clases, el desplazamiento entre fechas y el uso de widgets en la pantalla de inicio.

\begin{table}[H]
\small
\centering
\begin{tabularx}{\linewidth}{@{}P{1.15cm}P{2.3cm}XP{2.3cm}XP{1.2cm}@{}}
\toprule
ID & Rol & Necesidad & Beneficio & Criterios G/W/T & Prioridad \\
\midrule
US-11 & Estudiante & Ver cards del día & Planificar jornada & G: Home; W: abro; T: cards con hora/materia/grupo/aula. & Must \\
US-12 & Estudiante & Navegar por swipe & Revisar semana rápido & G: Home; W: deslizo; T: cambia día. & Should \\
US-13 & Estudiante & Selector de fecha & Saltar a día específico & G: toco fecha; W: elijo; T: Home muestra ese día. & Should \\
US-14 & Estudiante & Vista semanal compacta & Visualizar distribución completa & G: Home; W: botón semanal; T: resume semana activa. & Could \\
US-15 & Estudiante & Widget con días & Consultar sin abrir app & G: widget; W: deslizo; T: materias visibles cambian. & Should \\
\bottomrule
\end{tabularx}
\caption{Historias de usuario — Home, navegación y widget}\label{tab:hu_home}
\small{Fuente: Elaboración propia basada en \cite{encuesta_fcyt_2025}}
\end{table}

\medskip

\subsection{Materias \& Grupos}

Este conjunto detalla el flujo central de inscripción y gestión de materias, desde la selección guiada de niveles y grupos hasta la personalización visual y la administración del estado de las asignaturas.

\begin{table}[H]
\small
\centering
\begin{tabularx}{\linewidth}{@{}P{1.15cm}P{2.3cm}XP{2.3cm}XP{1.2cm}@{}}
\toprule
ID & Rol & Necesidad & Beneficio & Criterios G/W/T & Prioridad \\
\midrule
US-16 & Estudiante & Agregar materia guiada & Construir horario & G: pulso FAB; W: sigo flujo; T: materia creada (nivel→emoji). & Must \\
US-17 & Estudiante & Seleccionar nivel & Elegir contenedor correcto & G: pantalla nivel; W: elijo; T: vuelve con nivel marcado. & Must \\
US-18 & Estudiante & Seleccionar materia & Escoger código correcto & G: lista materias; W: toco; T: queda asignada. & Must \\
US-19 & Estudiante & Ver preview de grupo & Decidir con info completa & G: lista grupos; W: elijo; T: muestra días/horas/docente antes de confirmar. & Must \\
US-20 & Estudiante & Personalizar color/emoji & Identificar visualmente & G: pantalla color/emoji; W: guardo; T: cards reflejan estilo. & Could \\
US-21 & Estudiante & Editar desde card & Ajustar sin recrear & G: toco card; W: ``Editar''; T: cambio grupo/color/emoji. & Should \\
US-22 & Estudiante & Finalizar materia & Impactar generaciones futuras & G: opción finalizar; W: elijo estado; T: se registra (aprobado/abandonar). & Should \\
\bottomrule
\end{tabularx}
\caption{Historias de usuario — Materias y grupos}\label{tab:hu_materias}
\small{Fuente: Elaboración propia basada en \cite{encuesta_fcyt_2025}}
\end{table}

\medskip

\subsection{Profesores \& Favoritos}

Historias enfocadas en la preferencia por docentes específicos, permitiendo al estudiante marcarlos como favoritos para influir posteriormente en la generación automática de horarios.

\begin{table}[H]
\small
\centering
\begin{tabularx}{\linewidth}{@{}P{1.15cm}P{2.3cm}XP{2.3cm}XP{1.2cm}@{}}
\toprule
ID & Rol & Necesidad & Beneficio & Criterios G/W/T & Prioridad \\
\midrule
US-23 & Estudiante & Marcar docentes favoritos & Priorizarlos al generar & G: lista docentes; W: toco estrella; T: queda favorito. & Should \\
\bottomrule
\end{tabularx}
\caption{Historias de usuario — Profesores y favoritos}\label{tab:hu_profes}
\small{Fuente: Elaboración propia basada en \cite{encuesta_fcyt_2025}}
\end{table}

\medskip

\subsection{Exportar / Importar / Compartir}

Se abordan las capacidades de interoperabilidad y portabilidad de la información, permitiendo compartir horarios en múltiples formatos y transferir la planificación completa entre dispositivos mediante archivos estructurados.

\begin{table}[H]
\small
\centering
\begin{tabularx}{\linewidth}{@{}P{1.15cm}P{2.3cm}XP{2.3cm}XP{1.2cm}@{}}
\toprule
ID & Rol & Necesidad & Beneficio & Criterios G/W/T & Prioridad \\
\midrule
US-36 & Estudiante & Exportar como imagen & Compartir fácilmente & G: hay materias; W: ``Enviar $\rightarrow$ Imagen''; T: confirmación + share sheet. & Should \\
US-37 & Estudiante & Exportar como PDF & Imprimir/archivar & Igual que imagen con ``Enviar $\rightarrow$ PDF''. & Should \\
US-38 & Estudiante & Exportar como Excel & Manipular en tabla & Igual que imagen con ``Enviar $\rightarrow$ Excel''. & Could \\
US-39 & Estudiante & Enviar copia JSON & Compartir selectivamente & G: ``Enviar copia''; W: uso Todo/Limpiar/Compartir; T: JSON según selección. & Must \\
US-40 & Estudiante & Bloquear exportar sin materias & Evitar archivos vacíos & G: 0 materias; W: exporto; T: muestra aviso y detiene acción. & Must \\
US-41 & Estudiante & Importar JSON (app) & Añadir materias propias & G: ``Añadir horario''; W: elijo JSON; T: lista materias y evita duplicados. & Must \\
US-42 & Estudiante & Abrir JSON externo & Importar directo & G: recibo JSON; W: ``Abrir con TecnoTime''; T: muestra UI importación. & Must \\
US-43 & Estudiante & Importar otra carrera & Usarla si la sincronizo & G: JSON externa; W: importo; T: pregunta “¿Sincronizar?” y continúa si acepto. & Should \\
\bottomrule
\end{tabularx}
\caption{Historias de usuario — Exportar, importar y compartir}\label{tab:hu_export}
\small{Fuente: Elaboración propia basada en \cite{encuesta_fcyt_2025}}
\end{table}

\medskip

\subsection{Generador de Horarios}

Estas historias especifican el funcionamiento del motor de generación automática, incluyendo la configuración de restricciones, la aplicación de criterios de optimización y la exploración de las propuestas de horario resultantes.

\begin{table}[H]
\small
\centering
\begin{tabularx}{\linewidth}{@{}P{1.15cm}P{2.3cm}XP{2.3cm}XP{1.2cm}@{}}
\toprule
ID & Rol & Necesidad & Beneficio & Criterios G/W/T & Prioridad \\
\midrule
US-24 & Estudiante & Abrir generador & Configurar parámetros & G: FAB; W: ``Generar''; T: veo opciones. & Must \\
US-25 & Estudiante & Definir \# materias/propuestas & Controlar resultado & G: tarjeta parámetros; W: ajusto; T: respeta [1..20], [1..25]. & Must \\
US-26 & Estudiante & Priorizar profesores & Favorecer favoritos & G: opción ON; W: genero; T: intenta incluir favoritos. & Should \\
US-27 & Estudiante & Minimizar recesos & Reducir tiempos muertos & G: opción ON; W: genero; T: se optimizan huecos. & Should \\
US-28 & Estudiante & Aceptar choques & Permitir casos extremos & G: opción ON; W: genero; T: propone horarios con solapes controlados. & Could \\
US-29 & Estudiante & Coherencia favoritos/choques & Evitar inconsistencias & G: fav ON, choques OFF; W: genero; T: sin choques (ON/ON sí permite). & Must \\
US-30 & Estudiante & Marcar materias obligatorias & Forzar inclusión & G: marco oblig.; W: genero; T: respeta máximos/convierte excedentes. & Must \\
US-31 & Estudiante & Favorecer niveles bajos & Progresión académica & G: config por defecto; W: genero; T: evita combinaciones incoherentes. & Should \\
US-32 & Estudiante & Deslizar entre propuestas & Comparar rápido & G: propuestas listas; W: hago swipe; T: cambia 1..N. & Must \\
US-33 & Estudiante & Vista semanal propuesta & Validar a alto nivel & G: propuesta; W: botón semanal; T: muestra semana. & Could \\
US-34 & Estudiante & Fijar materias con switches & Regenerar respetando & G: propuesta; W: activo switch; T: nuevas propuestas las respetan. & Should \\
US-35 & Estudiante & Cargar completo/parcial & Construir horario final & G: propuesta; W: ``Completo/Parcial''; T: aplica selección en Home. & Must \\
\bottomrule
\end{tabularx}
\caption{Historias de usuario — Generador de horarios}\label{tab:hu_generador}
\small{Fuente: Elaboración propia basada en \cite{encuesta_fcyt_2025}}
\end{table}

\medskip

\subsection{Recordatorios \& IA ``Simón''}

Se definen las funcionalidades de asistencia proactiva, abarcando la configuración de recordatorios automatizados previos a las clases y la interacción opcional con el asistente de inteligencia artificial en el dispositivo.

\begin{table}[H]
\small
\centering
\begin{tabularx}{\linewidth}{@{}P{1.15cm}P{2.3cm}XP{2.3cm}XP{1.2cm}@{}}
\toprule
ID & Rol & Necesidad & Beneficio & Criterios G/W/T & Prioridad \\
\midrule
US-44 & Estudiante & Activar/desactivar notificaciones & Ajustar mi flujo & G: ajustes recordatorio; W: apago switch; T: oculta anticipación. & Must \\
US-45 & Estudiante & Definir tiempo de anticipación & Recibir avisos oportunos & G: notificaciones ON; W: elijo tiempo; T: se guarda y aplica. & Must \\
US-46 & Estudiante & Activar IA (descarga opcional) & Controlar datos/espacio & G: activo IA; W: elijo Wi-Fi/datos; T: confirma tamaño e inicia descarga. & Should \\
US-47 & Estudiante & Probar/eliminar IA & Gestionar almacenamiento & G: modelo listo; W: ``Probar''; T: respuesta. G: ``Eliminar''; W: confirmo; T: borra modelo. & Should \\
US-48 & Estudiante & Mensajes motivacionales & Controlar tono & G: IA disponible; W: cambio switch; T: respeta preferencia. & Could \\
\bottomrule
\end{tabularx}
\caption{Historias de usuario — Recordatorios e IA}\label{tab:hu_recordatorios}
\small{Fuente: Elaboración propia basada en \cite{encuesta_fcyt_2025}}
\end{table}

\medskip

\subsection{Offline, Actualizaciones \& Resiliencia}

Historias técnicas orientadas a garantizar la disponibilidad del sistema sin conexión a internet, la consistencia de los datos y la robustez ante fallos en la comunicación con el servidor.

\begin{table}[H]
\small
\centering
\begin{tabularx}{\linewidth}{@{}P{1.15cm}P{2.3cm}XP{2.3cm}XP{1.2cm}@{}}
\toprule
ID & Rol & Necesidad & Beneficio & Criterios G/W/T & Prioridad \\
\midrule
US-49 & Estudiante & Usar sin conexión (lectura) & No depender de la red & G: sin conexión; W: abro; T: horario y funciones locales disponibles. & Must \\
US-50 & Estudiante & Validar datos antes de generar & Evitar info obsoleta & G: hay internet; W: genero; T: se verifica/actualiza antes. & Should \\
US-51 & Sistema & Aplicar backoff progresivo & No saturar servidor & G: 2 fallos; W: reintento; T: aumenta espera y limita intentos. & Must \\
US-52 & Sistema & Bloqueo temporal tras fallos & Evitar tormenta de peticiones & G: timeouts repetidos; W: nueva llamada; T: bloqueo temporal + mensaje. & Must \\
US-53 & Estudiante & Auto-actualización periódica & Mantener datos al día & G: ciclo programado; W: hay internet; T: sincroniza carrera/malla/modelo. & Should \\
\bottomrule
\end{tabularx}
\caption{Historias de usuario — Offline, actualizaciones y resiliencia}\label{tab:hu_offline}
\small{Fuente: Elaboración propia basada en \cite{encuesta_fcyt_2025}}
\end{table}

\medskip

\subsection{Soporte, Información y Preferencias}

Se agrupan aquí las funcionalidades transversales de soporte al usuario y persistencia de configuraciones globales, asegurando una experiencia consistente y acceso a información del proyecto.

\begin{table}[H]
\small
\centering
\begin{tabularx}{\linewidth}{@{}P{1.15cm}P{2.3cm}XP{2.3cm}XP{1.2cm}@{}}
\toprule
ID & Rol & Necesidad & Beneficio & Criterios G/W/T & Prioridad \\
\midrule
US-54 & Estudiante & Ver versión/novedades/soporte & Informarme y colaborar & G: menú inferior; W: toco opción; T: abre Store/correo/changelog. & Could \\
US-55 & Estudiante & Tema claro/oscuro persistente & Consistencia visual & G: elijo tema; W: reinicio; T: se mantiene. & Must \\
US-56 & Estudiante & Formato 12/24h persistente & Evitar confusiones & G: cambio formato; W: guardo; T: se refleja globalmente. & Must \\
\bottomrule
\end{tabularx}
\caption{Historias de usuario — Soporte, información y preferencias}\label{tab:hu_soporte}
\small{Fuente: Elaboración propia basada en \cite{encuesta_fcyt_2025}}
\end{table}

\medskip

\subsection{Seguridad, Privacidad, Rendimiento y Accesibilidad}

Este bloque final establece los requisitos no funcionales críticos convertidos en historias verificables, priorizando la privacidad de datos, el rendimiento en dispositivos de gama media y la claridad en la comunicación de errores.

\begin{table}[H]
\small
\centering
\begin{tabularx}{\linewidth}{@{}P{1.15cm}P{2.3cm}XP{2.3cm}XP{1.2cm}@{}}
\toprule
ID & Rol & Necesidad & Beneficio & Criterios G/W/T & Prioridad \\
\midrule
US-57 & Estudiante & Privacidad del nombre & No exponer PII & G: ingreso nombre; W: uso app; T: no se envía a servidores. & Must \\
US-58 & Estudiante & App < 30 MB (sin IA) & Instalar con poco espacio & G: build; W: compilo; T: APK/Bundle < 30 MB. & Should \\
US-59 & Estudiante & Buen rendimiento en gama media & Uso cómodo & G: listas/scroll; W: navego; T: FPS aceptable sin ANR. & Should \\
US-60 & Estudiante & Accesibilidad básica & Comprender opciones & G: tema claro/oscuro; W: interactúo; T: contraste/tamaño legible. & Could \\
US-61 & Estudiante & Mensajes de error claros & Saber qué hacer & G: falla servidor; W: reintento; T: causa + alternativas offline/PDF. & Must \\
US-62 & Estudiante & Manejo de vacíos/estados límite & Evitar confusiones & G: sin materias; W: voy al Home; T: mensaje ``No hay materias'' + CTA agregar/generar. & Must \\
US-63 & Estudiante & Localización ES/formatos regionales & Coherencia cultural & G: región ES-BO; W: render; T: fechas/números correctos. & Should \\
\bottomrule
\end{tabularx}
\caption{Historias de usuario — Seguridad, privacidad, rendimiento y accesibilidad}\label{tab:hu_seguridad}
\small{Fuente: Elaboración propia basada en \cite{encuesta_fcyt_2025}}
\end{table}

% ============================
\section{F5: Interoperabilidad (JSON + WhatsApp)}

Esta fase habilita el intercambio estructurado de horarios entre estudiantes, implementando un contrato JSON interoperable y flujos de exportación/importación compatibles con plataformas de mensajería instantánea.

\subsection{Objetivo}
Permitir compartir e importar horarios entre estudiantes de forma interoperable y segura.

\subsection{Implementación de la interoperabilidad}
Se añadieron exportaciones (PNG/PDF/Excel/JSON), un contrato JSON auto-contenido y flujos de envío/recepción por mensajería con validaciones e importación idempotente.

\subsection{Exportar}
Acciones: imagen (PNG), PDF, Excel y copia JSON del horario. Se solicita confirmación previa; si no hay materias inscritas, se bloquea la exportación con mensaje. Implementación de UI en SettingsSendScheduleScreen y orquestación en SettingsSendScheduleViewModel. Generación: GenerateWeeklyScheduleImageUseCase, GenerateWeeklySchedulePdfUseCase, GenerateScheduleExcelUseCase, GenerateShareableScheduleJsonUseCase. Compartición mediante FileProvider + Intent.ACTION\_SEND. (Rutas completas en el Anexo \ref{ann:trazabilidad-tecnotime}).

\subsection{Enviar copia (JSON)}
Permite selección parcial en SettingsSendScheduleViewModel. El generador JSON ShareableScheduleJsonGenerator produce claves: meta, careers, subjects, groups, teachers, classrooms, entries. (Rutas completas en el Anexo \ref{ann:trazabilidad-tecnotime}).

\subsection{Importar (desde WhatsApp)}
Al abrir el JSON con TecnoTime: LoadShareableScheduleUseCase valida y carga; ImportSharedScheduleUseCase realiza merge sin duplicados y maneja:
\begin{itemize}
  \item Materias ya inscritas/aprobadas (no duplicar; cambio de grupo si aplica).
  \item Diferencia de carrera (autoSyncCareers para sincronizar primero si corresponde).
  \item Persistencia idempotente en SelectedSubjectRepository y GroupScheduleRepository.
\end{itemize}
(Rutas completas en el Anexo \ref{ann:trazabilidad-tecnotime}).

\begin{itemize}[nosep,leftmargin=*]
  \item loadEnrolledSubjects()
  \item toggleSubjectSelection()
  \item selectAllSubjects()
  \item clearSelection()
\end{itemize}

% Rutas omitidas: se referencian por nombre de clase.

El proceso de compartir horarios se visualiza en la Figura \ref{fig:f5_flow_export}. Este diagrama detalla el flujo de exportación, desde la selección de materias hasta la generación del archivo JSON y su envío a través de la hoja de compartir del sistema.

\begin{figure}[H]
  \centering
  \includegraphics[width=1\linewidth]{images/diagrams/cap4_f5_share_import_flow_part1.png}
\caption[Exportación del horario en la aplicación]{Exportación del horario en la aplicación.}
  \label{fig:f5_flow_export}
  \small{Fuente: Elaboración propia.}
\end{figure}

Complementariamente, la Figura \ref{fig:f5_flow_import} ilustra el flujo inverso de importación. Se muestra cómo la aplicación procesa un archivo JSON recibido, valida su contenido, detecta posibles duplicados y fusiona la información con el horario existente del usuario.

\begin{figure}[H]
  \centering
  \includegraphics[width=1\linewidth]{images/diagrams/cap4_f5_share_import_flow_part2.png}
\caption[Importación del horario en la aplicación]{Importación del horario en la aplicación.}
  \label{fig:f5_flow_import}
  \small{Fuente: Elaboración propia.}
\end{figure}

El intercambio define un bloque de metadatos y un cuerpo auto-contenido:
\begin{itemize}
  \item meta: \{version, min\_supported\}. Se usa control semántico (1.0.0). La aplicación acepta archivos cuya min\_supported $\leq$ versión actual del contrato.
  \item careers: carreras implicadas (código y nombre) para habilitar/sincronizar si corresponde.
  \item subjects, groups, teachers, classrooms: catálogos mínimos para enriquecer la vista previa y resolver identificadores.
  \item entries: lista de EnrolledSchedulePreview.
\end{itemize}
Campos mínimos obligatorios por entry para garantizar la importación: subject.code, group.groupId y schedule (día y franja horaria). El resto de atributos enriquecen la experiencia (docente, aula, color/emoji, notificaciones) y se consumen cuando están disponibles.

\subsection{Resultados y verificación}
La interoperabilidad se validó mediante el intercambio exitoso de horarios entre dispositivos con diferentes configuraciones de carrera. Se confirmó que el contrato JSON es robusto ante cambios menores de versión y que el proceso de importación maneja correctamente la fusión de datos sin generar duplicados, sugiriendo la sincronización de carrera cuando es necesario.

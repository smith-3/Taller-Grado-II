\section{Resumen operacional por fases (F1–F11)}
Esta sección consolida la información operativa de cada fase del desarrollo, presentando de forma sintética los artefactos generados, las decisiones técnicas adoptadas y los criterios de aceptación verificados en cada incremento.

% Resumen compacto de cada fase con artefactos, decisiones y criterios de aceptación.
% Facilita la revisión rápida de todo el desarrollo sin repetir detalles ya documentados.

% ============================


\subsection{F1: Ingesta de PDF a BD}
\begin{itemize}
  \item Objetivo (Goal): Transformar horarios oficiales (PDF) a datos locales sin saturar la fuente.
  \item Actividades realizadas: Descubrimiento de URLs, descarga con rate limit, parsing tolerante, normalización, persistencia y marcas de freshness.
  \item Entradas a salidas: Del Remoto (PDF/Scraper) al Local (Room: materias, grupos, bloques).
  \item DoD: BD consistente por carrera/nivel/materia/grupo, last\_sync actualizado, reintentos acotados.
  \item Riesgo clave: Cambios de formato/ubicación del PDF, mitigado con parser flexible y fallback por PDF local.
\end{itemize}

\subsection{F2: Modelo de dominio}
La Fase 2 define el modelo de dominio que representa conceptualmente el problema de gestión de horarios académicos y establece las reglas de integridad que gobiernan las relaciones entre entidades. Este modelo sirve como contrato entre las diferentes capas de la aplicación y como fuente de verdad para la lógica de negocio.

La claridad y completitud del modelo de dominio facilita la comunicación entre desarrolladores, simplifica el mantenimiento del código y reduce la probabilidad de inconsistencias en los datos. Los elementos principales son:

\begin{itemize}
  \item Objetivo: Entidades, relaciones y mapeos consistentes para soporte de generador y export.
  \item Actividades realizadas: Definición de entidades (Subject, Group, GroupSchedule, Teacher, Classroom, Settings), índices/keys y mappers.
  \item Entradas a salidas: Del Modelo (dominio) al Local (entities) y a UseCases (servicios).
  \item DoD: Unicidad en códigos; integridad referencial; mapeos ida/vuelta probados.
  \item Riesgo: Desalineación dominio/BD $\rightarrow$ mitigado con pruebas de carga y revisión cruzada.
\end{itemize}

\subsection{F3: Generador de horarios}
La Fase 3 implementa el núcleo algorítmico de TecnoTime: el generador de horarios optimizados que resuelve el problema de satisfacción de restricciones mediante búsqueda con retroceso y heurísticas de poda. Esta fase representa el mayor desafío técnico del proyecto, requiriendo el balance entre exhaustividad de la búsqueda, tiempo de ejecución y calidad de las soluciones.

El generador debe producir múltiples alternativas viables en tiempos compatibles con la ejecución en dispositivo móvil, considerando restricciones duras como no solapamiento y restricciones blandas como minimización de huecos. Los aspectos clave son:

\begin{itemize}
  \item Objetivo: Producir Top-$N$ horarios sin choques por defecto, optimizando huecos y favoritos.
  \item Actividades realizadas: Backtracking con MRV/LCV, poda temprana, composite scoring (choques/gaps/favoritos), aplicación parcial/completa.
  \item Entradas a salidas: De UseCases (Generate/Apply) a UI (configuración y resultados).
  \item DoD: $N$ candidatos válidos; flags influyen en ranking; aplicar parcial/total actualiza selección.
  \item Riesgo: Explosión combinatoria, mitigada con poda y priorización por nivel.
\end{itemize}

\subsection{F4: Prioridad sin conexión (offline-first)}
La Fase 4 establece la arquitectura offline-first que prioriza el almacenamiento local y garantiza la funcionalidad de la aplicación independientemente de la conectividad de red. Esta decisión arquitectónica responde directamente a las necesidades identificadas en la encuesta, donde el 78% de estudiantes valoró el funcionamiento sin conexión.

La implementación incluye políticas de frescura de datos, sincronización diferida en segundo plano y mecanismos de recuperación ante fallos de red. Los componentes clave son:

\begin{itemize}
  \item Objetivo: Operación robusta con conectividad intermitente y protección de la fuente.
  \item Actividades realizadas: Cache local, freshness gate previo al generador, backoff+circuit breaker, ventana de no-sync, fallback por PDF local.
  \item Entradas a salidas: Del Local (cache) y UseCases (AutoSync) a datos frescos cuando hay red.
  \item DoD: App funcional sin internet; bloqueo de generación durante sync; reintentos espaciados.
  \item Riesgo: Saturar origen, mitigado con CB y ventanas controladas.
\end{itemize}

\subsection{F5: Interoperabilidad (JSON + WhatsApp)}
La Fase 5 habilita la interoperabilidad entre instancias de TecnoTime mediante un contrato JSON estandarizado que permite compartir horarios a través de canales de mensajería como WhatsApp. Esta funcionalidad responde a la necesidad de coordinación social identificada en la encuesta, donde el 80% de estudiantes utiliza WhatsApp para validar cambios en horarios.

El diseño del formato de intercambio balancea completitud de información, compacidad del mensaje y facilidad de validación. Los elementos son:

\begin{itemize}
  \item Objetivo: Compartir e importar horarios de forma confiable entre usuarios.
  \item Actividades realizadas: Export JSON total/parcial con versión; import con validación (duplicados, carrera distinta) y merge idempotente.
  \item Entradas a salidas: De UseCases (Share/Import) a UI (enviar/abrir) y Local (fusión).
  \item DoD: JSON válido; import no duplica; sugiere sincronizar carrera distinta antes de fusionar.
  \item Riesgo: Deriva de esquema: mitigado con versionado y validaciones.
\end{itemize}

\subsection{F6: UX aplicada}
La Fase 6 refina la experiencia de usuario mediante la implementación de patrones de UX móvil que facilitan la adopción y el uso cotidiano de la aplicación. Esta fase incluye el diseño del flujo de onboarding que guía a nuevos usuarios a través de la configuración inicial, la implementación del widget de pantalla principal que proporciona acceso inmediato al horario actual, y la optimización de la navegación para minimizar el número de toques necesarios para completar tareas comunes.

El diseño de UX se validó mediante pruebas con usuarios piloto que identificaron puntos de fricción y oportunidades de mejora. Los aspectos principales son:

\begin{itemize}
  \item Objetivo: Reducir fricción: onboarding, ajustes, agregar/editar, vista semanal, widget.
  \item Actividades realizadas: Flujos guiados, preview de grupo, selección por nivel/materia, valores por defecto sensatos, widget de acceso rápido.
  \item Entradas$\rightarrow$Salidas: UI (welcome, settings, home, add/edit, weekly) + Widget.
  \item DoD: Menos pasos para armar horario; preview consistente; widget sincronizado.
  \item Riesgo: Sobrecarga de opciones: mitigada con progressive disclosure.
\end{itemize}

\subsection{F7: Notificaciones e IA (opt-in)}
La Fase 7 integra dos funcionalidades opcionales que enriquecen la experiencia del usuario sin comprometer la simplicidad de la aplicación base: el sistema de notificaciones programables para recordatorios de clases y el asistente de inteligencia artificial ``Simón'' que opera completamente en el dispositivo. Ambas funcionalidades siguen un modelo opt-in donde el usuario debe activarlas explícitamente, respetando principios de privacidad y control del usuario.

El asistente IA utiliza modelos cuantizados en formato GGUF que permiten la inferencia local sin envío de datos a servidores externos, alineándose con los requisitos de privacidad del proyecto. Los componentes son:

\begin{itemize}
  \item Objetivo: Recordatorios puntuales y mensajería opcional con IA on-device.
  \item Actividades realizadas: Canales separados (clases/mensajes), WorkManager, plantillas; gestión del modelo IA (descarga/borrado; Wi-Fi/datos; sin telemetría).
  \item Entradas a salidas: De Notif (canales+workers) e IA (local) a avisos y textos opcionales.
  \item DoD: Notifs antes de clase (10 min por defecto) y configurables; IA desactivada por defecto, activable y reversible.
  \item Riesgo: Tamaño de modelo/almacenamiento: mitigado con descarga bajo demanda y opción de eliminar.
\end{itemize}

\subsection{F8: Exportaciones}
La Fase 8 implementa las capacidades de exportación de horarios en múltiples formatos (PDF, imagen, Excel) que facilitan la distribución y el uso de los horarios fuera de la aplicación. Esta funcionalidad responde a la necesidad identificada en la encuesta donde el 51% de estudiantes solicitó capacidades de compartir horarios.

La generación de cada formato requiere transformaciones específicas: el PDF utiliza bibliotecas de renderizado vectorial, la imagen captura la vista de Compose como bitmap, y el Excel estructura los datos en hojas de cálculo con formato condicional. La implementación garantiza que los horarios exportados sean legibles, completos y mantengan la información de contexto necesaria. Los elementos son:

\begin{itemize}
  \item Objetivo: Compartir horario como imagen, PDF o Excel.
  \item Actividades realizadas: Confirmación previa, bloqueo si horario vacío, generación por formato y share sheet.
  \item Entradas a salidas: De UseCases (Image/PDF/Excel) a UI (enviar/guardar).
  \item DoD: Archivos contienen materia, grupo, día, hora, aula, docente; MIME correcto; bloqueo sin materias.
  \item Riesgo: Permisos/espacio: mitigado con FileProvider y validaciones previas.
\end{itemize}

\subsection{F9: Pruebas y validación}
La Fase 9 establece el aseguramiento de calidad mediante una batería de pruebas que validan tanto la funcionalidad individual de componentes como los flujos de extremo a extremo que atraviesan múltiples capas de la aplicación. Las pruebas unitarias verifican la lógica de negocio aislada, las pruebas de integración validan la interacción entre repositorios y casos de uso, y las pruebas de UI confirman que los flujos críticos funcionan correctamente desde la perspectiva del usuario.

La cobertura de pruebas se enfoca en los módulos de mayor riesgo: el generador de horarios, los parsers de PDF, y la lógica de sincronización offline. La estructura de pruebas es:

\begin{itemize}
  \item Objetivo: Asegurar calidad funcional y de flujo extremo a extremo.
  \item Actividades realizadas: Unitarias (tiempo/solapes/parsers/estrategias), integración (repos+usecases), UI/E2E (onboarding hasta generar, aplicar y exportar/importar).
  \item Entradas a salidas: De Tests (TCs) a reporte de verificación y issues cerrados.
  \item DoD: $\geq 75\%$ cobertura en módulos núcleo; E2E críticos verdes.
  \item Riesgo: Falsos positivos E2E, mitigado con datos semilla estables.
\end{itemize}

\subsection{F10: Seguridad y privacidad}
La Fase 10 endurece la seguridad y privacidad de la aplicación mediante la implementación de principios de minimización de datos, control explícito del usuario y operación local por defecto. La aplicación solicita únicamente los permisos estrictamente necesarios (almacenamiento para PDFs, notificaciones opcionales) y los requiere just-in-time cuando el usuario intenta usar la funcionalidad relacionada.

El asistente de IA opera completamente en el dispositivo sin enviar datos a servidores externos, y el usuario puede desactivarlo y eliminar el modelo en cualquier momento. Estos controles de privacidad se alinean con las mejores prácticas de desarrollo móvil y las expectativas de usuarios conscientes de la privacidad. Los componentes son:

\begin{itemize}
  \item Objetivo: Datos y permisos mínimos; control explícito del usuario.
  \item Actividades realizadas: Sólo nombre y preferencias locales; permisos just-in-time; IA local sin envío de datos.
  \item Entradas$\rightarrow$Salidas: Manifest + AppInit (canales) + Settings (opt-in IA).
  \item DoD: App funciona con permisos mínimos; IA 100\% opt-in/opt-out.
  \item Riesgo: Uso involuntario de IA: mitigado con toggles y borrado del modelo.
\end{itemize}

\subsection{F11: Release y KPIs}
La Fase 11 prepara la aplicación para su liberación mediante la optimización del empaquetado, la definición de métricas de adopción y uso, y el establecimiento de procesos de despliegue continuo. La optimización incluye la aplicación de ProGuard/R8 para reducir el tamaño del APK, la separación del modelo de IA en un paquete opcional descargable bajo demanda, y la configuración de splits por densidad de pantalla.

Las métricas definidas permiten evaluar el éxito de la aplicación: tiempo promedio de consulta de horario (TTS), número de huecos por día en horarios generados, porcentaje de notificaciones entregadas a tiempo, y tasa de adopción del formato JSON para compartir. Los elementos finales son:

\begin{itemize}
  \item Objetivo: Empaque optimizado y medición de adopción/uso.
  \item Actividades realizadas: Shrink/optimize; firma y notas; umbrales (APK < 30 MB sin IA; IA ~200–230 MB opcional); KPIs (TTS, gaps/día, on-time, adopción JSON).
  \item Entradas a salidas: Del Build (release) a Artefactos y Métricas base.
  \item DoD: Compilación release estable; tamaños dentro de umbrales; KPIs medibles definidos.
  \item Riesgo: Aumento de tamaño por libs: mitigado con revisión de dependencias y splits.
\end{itemize}


\noindent minSdk=24, targetSdk=35, tamaño del APK: 23\,MB y tamaño del modelo IA: 229\,MB en la build de referencia.

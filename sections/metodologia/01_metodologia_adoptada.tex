% ----------------------------------------------------------------------
\section{Metodología adoptada: Kanban iterativo-incremental}

Para el desarrollo de TecnoTime se adoptó la metodología Kanban en su variante iterativo-incremental, ajustada a un desarrollador único. Cada fase constituyó un incremento funcional con criterios de aceptación claros y revisión continua mediante tablero digital. Esta aproximación permitió priorizar funcionalidades según la evidencia de la encuesta aplicada y adaptarse a cambios en los requerimientos durante el desarrollo, tal como se detalla en el Cuadro~\ref{tab:fases_metodologia}.



\begin{table}[H]
  \centering\small
  \begin{tabularx}{\linewidth}{@{}p{1.2cm}p{3.9cm}>{\RaggedRight\arraybackslash}X@{}}
    \hline
    Fase & Propósito principal & Entregables y evidencias \\
    \hline
    F0 & Identificar necesidades reales mediante estudio de campo & Instrumento de recolección (Google Forms), muestra de 126 estudiantes, análisis cuantitativo con hallazgos clave (94\% Android, 80\% coordina por WhatsApp, 88\% solicita actualización automática) y derivación de requerimientos funcionales trazables. \\
    F1 & Incorporar horarios oficiales al sistema & Canal de descarga controlada de PDF, normalización a esquema Room y verificación de consistencia por carrera. \\
    F2 & Definir el modelo de dominio & Entidades, reglas y agregados consolidados en la capa domain, acompañados de diagramas de referencia. \\
    F3 & Generar horarios viables & Servicio de generación configurable, heurísticas MRV/LCV y pantallas para configurar y aplicar resultados. \\
    F4 & Priorizar el uso sin conexión & Flujos de sincronización diferida, políticas de frescura y mecanismos de recuperación manual. \\
    F5 & Intercambiar horarios & Contrato JSON (formato de intercambio de datos basado en texto) interoperable, flujo de importación/exportación y asistentes para compartir mediante WhatsApp. \\
    F6 & Refinar la experiencia de usuario & Onboarding (pantallas iniciales de orientación), navegación principal, widget (componente visual que se muestra en la pantalla principal del dispositivo) de acceso rápido y guías contextuales. \\
    F7 & Automatizar recordatorios e IA & Notificaciones programadas, plantillas reutilizables y asistente ``Simón'' con activación voluntaria. \\
    F8 & Exportar resultados & Generadores de reporte en PDF, imagen y Excel con validación de formato. \\
    F9 & Realizar aseguramiento de calidad & Batería de pruebas instrumentadas, casos manuales priorizados y checklist de aceptación. \\
    F10 & Endurecer seguridad y privacidad & Gestión de permisos, cifrado de datos sensibles y revisión de políticas de respaldo. \\
    F11 & Preparar liberación y métricas & Empaquetado en release, tablero de indicadores (TTS, adopción) y plan de seguimiento post despliegue. \\
    \hline
  \end{tabularx}
  \caption[Aplicación de la metodología por fases]{Aplicación de la metodología por fases.}
  \label{tab:fases_metodologia}
  \small{Fuente: Elaboración propia.}
\end{table}

\subsection{Contexto empírico}
El desarrollo de la aplicación móvil se inició desde cero, tomando como única fuente de datos el portal web de la facultad donde se publican los horarios en formato PDF \cite{horarios_fcyt}.

El primer desafío técnico consistió en implementar un mecanismo de descarga y extracción de datos (scraping) capaz de transformar estos documentos no estructurados en información procesable.

Aunque se tomaron como referencia visual y funcional otras aplicaciones existentes en el mercado para definir el alcance inicial, todo el código fuente de TecnoTime fue escrito específicamente para este proyecto.

A partir de esta base de ingesta de datos, se ejecutaron iteraciones incrementales que añadieron funcionalidades clave: selección de materias, generación de horarios y validaciones con usuarios piloto.

Las fases posteriores incorporaron intercambio JSON para envío y recepción de horarios, mejoras de experiencia de usuario (pantallas guía, configuración y widget), recordatorios y exportaciones en formatos comunes.

Al cierre se integró la asistencia ``Simón'' como componente optativo y se documentaron controles de seguridad.

\subsection{Factores de decisión}
La encuesta aplicada a 126 estudiantes \cite{encuesta_fcyt_2025} orientó las prioridades del tablero:
\begin{itemize}
  \item Preferencia por Android (94,3\%): orientación exclusiva a componentes nativos y widget principal.
  \item Coordinación por WhatsApp (80,3\%): intercambio de horarios mediante JSON y acciones directas de compartir.
  \item Sincronización automática (88,4\%): verificación de frescura y actualización discreta sin intervención del usuario.
  \item Uso sin conexión (77,7\%): arquitectura offline-first (prioriza el funcionamiento sin conexión) con caché local prioritaria.
  \item Recordatorios (56,2\%): programación de notificaciones con plantillas y canales específicos.
  \item Compartir horario (51,2\%): exportación a PDF, imagen y Excel con parámetros de periodo.
\end{itemize}

\subsection{Decisiones técnicas derivadas}
En respuesta a estos factores, se definió una arquitectura Offline-First basada en Room como fuente de verdad, garantizando el acceso inmediato a la información sin dependencia de red. Para la interoperabilidad, se implementó un protocolo de intercambio de datos mediante archivos JSON ligeros, facilitando la coordinación externa.

El desarrollo se realizó utilizando tecnologías nativas modernas (Jetpack Compose y Kotlin) para asegurar el rendimiento y la integración con el sistema, incluyendo el uso de widgets. Finalmente, el sistema de recordatorios se construyó sobre AlarmManager y WorkManager, permitiendo notificaciones locales precisas y autónomas.

Estas decisiones definieron el alcance de los incrementos y la priorización de las historias de usuario en el tablero Kanban. Los artefactos y evidencias de código resultantes se encuentran documentados en el repositorio del proyecto \cite{tecnotime_repo}.

\subsection{Arquitectura de la solución}
La solución se apoya en una aplicación Android autosuficiente: la interacción sucede en la interfaz offline (Jetpack Compose), el motor de dominio coordina reglas y recordatorios, la ingesta procesa PDFs y la persistencia local resguarda horarios, configuraciones y políticas de frescura.

El asistente IA opera sobre el mismo contexto y la única integración externa es el portal UMSS que publica los PDFs oficiales. La Figura \ref{fig:arquitectura_tecnotime} resume esta arquitectura offline-first.

\begin{figure}[H]
  \centering
  \includegraphics[width=0.7\linewidth]{images/diagrams/cap4_arq_general.png}
  \caption[Arquitectura general empleada en TecnoTime]{Arquitectura general empleada en TecnoTime.}
  \label{fig:arquitectura_tecnotime}
  \small{Fuente: Elaboración propia.}
\end{figure}
